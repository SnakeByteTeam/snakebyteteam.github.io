\documentclass[10pt, letterpaper]{article}
\usepackage{../../../template/template}

\setDocTitle{Glossario}
\setDocData{14/02/2026}
\setDocVersion{1.0.0}
\setDocState{2}
\setDocRecipient{0}

\begin{document}

\makeTitlePage

\addChangelog{1.0.0}{14/02/2026}{\noOne}{\noOne}{F. Venzo}{\approval}
\addChangelog{0.1.0}{13/02/2026}{L. Pellizzon}{F. Pasqual}{\noOne}{\firstDraft}
\makeChangelog

\newpage

\tableofcontents

\newpage

\setcounter{secnumdepth}{0}
\setcounter{tocdepth}{2}


\section{A}

\subsection*{Agile}
Il termine si riferisce ad una famiglia di metodi di sviluppo software caratterizzati da cicli iterativi e incrementali nei quali
il lavoro viene suddiviso in piccole task a valore aggiunto il cui risultato può essere modificato in periodi successivi. I metodi Agile
nascono in risposta all'eccessiva rigidità dei modelli sequenziali e di quelli esclusivamente iterativi o esclusivamente incrementali.

\subsection*{Amazon Web Services (AWS)}
Piattaforma di servizi cloud fornita da Amazon, che include risorse per calcolo, archiviazione e distribuzione di applicazioni

\subsection*{Angular}
Framework open source sviluppato da Google che permette di creare applicazioni web dinamiche e complesse con una grande facilità. Questo framework
è basato su JavaScript ed è utilizzato da molte grandi aziende per lo sviluppo di applicazioni web.

\subsection*{API-first}
Approccio di sviluppo dove la progettazione dell'interfaccia API è prioritaria rispetto all'implementazione del software. Questa
pratica garantisce modularità e interoperabilità

\subsection*{API KNX IoT 3rd-party}
Interfaccia di programmazione che permette a software esterni di collegarsi e interagire con dispositivi domotici conformi allo standard KNX,
un protocollo di comunicazione internazionale.

\section{B}

\subsection*{Black-box}
Approccio in cui un modello o sistema è utilizzato senza conoscerne il contenuto, ovvero il suo funzionamento interno, ma solo osservando l'input e l'output.

\subsection*{Brainstorming}
È una tecnica di gruppo per generare liberamente tante idee e soluzioni creative a un problema, senza giudizio iniziale, per poi valutarle in una fase successiva

\subsection*{Branch}
Nel contesto di un sistema di controllo versione come Git, un branch (ramo) è un puntatore mobile a un commit. Consente agli sviluppatori di lavorare isolatamente su nuove
funzionalità o correzioni senza alterare il codice base principale, finché il lavoro non è pronto per essere fuso (merged).

\subsection*{Bug}
Un bug, in informatica, è un errore, un difetto software o hardware che produce un comportamento inatteso o errato.

\section{C}

\subsection*{C4 Model}
Il Modello C4 (acronimo di Context, Containers, Components, Code) è un approccio per la visualizzazione e la documentazione dell'architettura software. Ideato per
superare le ambiguità dei diagrammi UML complessi, presenta l'architettura in base ai diversi livelli del sistema, tra cui il diagramma del contesto del sistema, il diagramma
del contenitore, il diagramma dei componenti e il diagramma del codice. Il modello C4 non richiede l'uso di un linguaggio grafico o di modellazione specifico, pertanto gli
utenti possono generare diagrammi architetturali in modo molto flessibile. Riferimento: \url{https://c4model.com}

\subsection*{Capitolato}
Il capitolato è un documento che descrive in modo dettagliato le specifiche tecniche e le condizioni necessarie per lo sviluppo di un
progetto software. Viene redatto dal proponente e presentato ai fornitori o agli sviluppatori interessati a partecipare alla gara
d'apparto per la realizzazione del prodotto.

\subsection*{Chat}
Sistema di comunicazione in tempo reale che avviene tra utenti in modo testuale.

\subsection*{Cloud e DevOps}
Insieme di pratiche che uniscono lo sviluppo software e operazioni IT in ambienti cloud per migliorare automazione, integrazione e distribuzione continua.

\subsection*{Commit}
Il commit è un operazione di salvataggio all'interno di un repository Git che registra una versione specifica dei file, consentendo di tenere traccia delle modifiche
effettuate nel tempo.

\subsection*{Committente}
Il committente è l'individuo o l'organizzazione che commissiona la realizzazione di un progetto software e ne finanzia lo sviluppo. Esso è responsabile della definizione dei
requisiti e delle specifiche del progetto, oltre a valutare il prodotto finale in base ai propri obbiettivi e aspettative iniziali.

\subsection*{Container}
Un container è un'unità standard di software che impacchetta il codice e tutte le sue dipendenze in modo che l'applicazione venga eseguita in modo rapido e
affidabile da un ambiente informatico all'altro

\subsection*{Continuous Integration e Deployment (CI/CD)}
Insieme di pratiche volte ad automatizzare l'integrazione, i test e la distribuzione del software per garantire rilasci frequenti e affidabili.

\subsection*{CRUD}
Acronimo di Create, Read, Update, Delete. Rappresenta le quattro operazioni fondamentali per la gestione dei dati in un sistema informatico.

\section{D}

\subsection*{Dashboard}
Interfaccia utente grafica che presenta in modo riassuntivo e organizzato le informazioni chiave (metriche, stato, dati) relative a un'applicazione, un progetto
o un sistema. È utilizzata per il monitoraggio rapido e per supportare i processi decisionali.

\subsection*{Data Analysis}
In italiano, Analisi dei Dati, è il processo sistematico di raccolta, trasformazione e modellazione di dati grezzi per estrarre informazioni utili ed individuare
trend o modelli, a supporto del sistema decisionale. Si articola in una varietà di tecniche che vanno dall'analisi statistica al machine learning, finalizzate all'analisi
dei dati passati e attuali per prevedere le tendenze future.

\subsection*{Debugger}
È un programma che permette di analizzare il codice al fine di rilevare e diagnosticare bug, il processo avviene durante l'esecuzione del codice.

\subsection*{Deliverable}
Artefatto concreto e tangibile prodotto durante il ciclo di vita del progetto che deve essere consegnato agli stakeholder. Include documenti, software, report e
qualsiasi altro prodotto specificato nei requisiti del progetto.

\subsection*{Develop}
Nel modello di branching Gitflow, 'develop' è il ramo principale di integrazione per lo sviluppo di nuove funzionalità. Contiene la storia completa del progetto
e riceve i merge dai rami secondari prima di essere pronto per una release.

\subsection*{Discord}
Discord è una piattaforma di comunicazione digitale che integra in modo fluido chat testuali, vocali e video.

\subsection*{Docker}
Piattaforma software che consente di creare, testare e distribuire applicazioni rapidamente. Docker confeziona il software in unità standardizzate chiamate contenitori
che hanno tutto ciò di cui il software ha bisogno per essere eseguito, comprese librerie, strumenti di sistema, codice e runtime.

\section{E}

\subsection*{Express}
Framework per applicazioni Web per Node.js minimale e flessibile che semplifica la creazione di applicazioni e API lato server.

\section{F}

\subsection*{Facebook AI Similarity Search (FAISS)}
Libreria open-source sviluppata da Meta per eseguire ricerche ottimizzate e rapide di similarità in grandi collezioni di dati.

\subsection*{Feature}
Una feature (funzionalità) è un ramo di sviluppo creato per implementare un requisito o una specifica caratteristica del software. Questi rami sono isolati dal
develop e vengono fusi solo al termine del lavoro, assicurando che le modifiche non interferiscano con il codice stabile.

\subsection*{Feedback}
Indica il modo in cui l'effetto derivato dall'azione di un sistema si riflette sul sistema stesso, influenzandone o correggendone il funzionamento. Nello
scenario di progetto indica l'ottenimento di un riscontro valutativo sull'operato da parte degli Stakeholders.

\section{G}

\subsection*{Git}
Sistema di controllo di versione distribuito che permette a molteplici sviluppatori di lavorare mantenendo un registro delle modifiche che vengono apportate al
codice di uno stesso progetto.

\subsection*{GitHub}
Piattaforma di sviluppo collaborativo che si basa su Git, per il controllo delle versioni. Utilizzato da gran parte degli sviluppatori di software e team di
sviluppo per la gestione di progetti, permette la tracciabilità delle modifiche al codice sorgente e la collaborazione fra i membri del team, facilitandone
il coordinamento.

\subsection*{GitHub Projects}
Funzionalità di gestione del lavoro offerta da GitHub, che consente ai team di organizzare, pianificare e monitorare le attività relative al codice, spesso
utilizzando bacheche Kanban. È strettamente integrata con Issues e pull request del repository.

\subsection*{Google Meet}
Servizio online di videoconferenze messo a disposizione da Google.

\subsection*{Google Sheets}
Editor online di fogli di calcolo messo a disposizione dalla Google Workspace, insieme di strumenti online di produttività dell'omonima azienda.

\subsection*{Gmail}
Servizio di posta elettronica di Google che si basa sul cloud.

\section{I}

\subsection*{Infrastructure as Code (IaC)}
Pratica che prevede la gestione e configurazione dell'infrastruttura informatica tramite codice, per garantire replicabilità, tracciabilità e facilità di manutenzione.

\subsection*{Internet of Things (IoT)}
Rete di dispositivi fisici dotati di sensori, software e altre tecnologie, che consente loro di connettersi a Internet, raccogliere e scambiare dati

\subsection*{Issue}
Un GitHub Issue è una funzionalità di GitHub dedicata alla gestione delle attività e al tracciamento dei problemi all'interno di un progetto software.
Ogni issue rappresenta un elemento di lavoro o una segnalazione specifica e può essere arricchita con diverse informazioni: assegnatari, per indicare i
responsabili della risoluzione; etichette (label), per classificare e organizzare le issue; e milestone, che consentono di raggruppare più issue in base a un
obiettivo comune, facilitando il monitoraggio dell'avanzamento verso una determinata scadenza o risultato.

\subsection*{Issue Tracking System}
Un Issue Tracking System è uno strumento utilizzato per registrare, monitorare e gestire problemi, segnalazioni di bug o richieste di miglioramento all'interno
di un progetto software. Consente di tenere traccia dello stato e della priorità di ciascun elemento, facilitando la collaborazione tra i membri del team e
garantendo una gestione più efficiente del ciclo di sviluppo.

\section{J}

\subsection*{Jira}
Strumento software proprietario ampiamente utilizzato per la gestione dei progetti, l'Issue Tracking e lo sviluppo Agile. Permette ai team di pianificare,
tracciare e rilasciare software, gestendo problemi (Issue) e flussi di lavoro personalizzati.

\section{L}

\subsection*{LaTeX}
LaTeX è un linguaggio di markup e un sistema di preparazione di documenti ampiamente utilizzato per la creazione di testi scientifici, accademici e tecnici,
ma anche per la realizzazione di libri, tesi, articoli e presentazioni. Basato su TeX, il sistema di tipografia sviluppato da Donald Knuth negli anni '70,
LaTeX ne estende le funzionalità offrendo pacchetti e comandi che semplificano la scrittura, la formattazione e la gestione strutturata dei contenuti.

\section{M}

\subsection*{Marketing}
Ramo dell'economia che si dedica all'analisi e comprensione di un determinato mercato di riferimento e allo studio delle esigenze del consumatore.
In particolare, il marketing si focalizza sull'indagine dell'interazione del mercato e degli utenti di un'impresa al fine di ottimizzare le strategie
commerciali e promozionali per raggiungere gli obiettivi aziendali.

\subsection*{Merge}
Operazione chiave nei sistemi di controllo versione come Git che consiste nel combinare le modifiche apportate in due rami di sviluppo separati in un unico ramo.
Il processo sincronizza le modifiche tra i diversi flussi di lavoro, spesso a seguito di una pull request.

\subsection*{Microsoft Teams}
Piattaforma di comunicazione e collaborazione unificata che combina chat di lavoro persistente, teleconferenza, condivisione di contenuti e integrazione
delle applicazioni.

\subsection*{Microsoft Copilot}I copiloti sono interazioni di intelligenza artificiale generative basate su chat di Microsoft. Hanno funzionalità specializzate in base agli utenti e ai casi
d'uso. Alcuni Copilot si possono utilizzare come prodotto autonomo, mentre altri sono integrati in altri prodotti, servizi e dispositivi Microsoft.

\subsection*{Milestone}
Le Milestone sono strumenti di gestione dei progetti utilizzati per identificare momenti chiave lungo la timeline di un'iniziativa. Rappresentano punti di
riferimento che possono coincidere con l'avvio o la conclusione di una fase, oppure con momenti di verifica o revisione. Generalmente, le milestone non
influenzano direttamente la durata complessiva del progetto, ma servono a evidenziare i principali traguardi che ne determinano l'avanzamento e il successo.

\subsection*{Mocking}
In ingegneria del software è una tecnica che permette di isolare e testare con facilità una porzione del programma sfruttando degli oggetti (mock) che simulano
dipendenze reali come databases e APIs.

\section{N}
\subsection{NestJS}
NestJS è un framework per lo sviluppo di applicazioni server-side con Node.js, progettato per essere modulare, scalabile e facilmente manutenibile. Sfrutta pienamente
TypeScript (pur permettendo l’uso di JavaScript) e adotta principi simili a quelli di Angular, come la dependency injection, l’uso di decoratori e l’architettura modulare, 
facilitando la creazione di applicazioni organizzate e robuste.

\section{O}

\subsection*{Open Web Application Security Project (OWASP)}
Definisce l'organizzazione che promuove la sicurezza del software e che fornisce linee guida per la protezione delle applicazioni web.
\section{P}

\subsection*{Prisma ORM}
Toolkit open source per Node.js e TypeScript che semplifica l'accesso ai database.

\subsection*{Project Board}
Tabellone interattivo fornito da piattaforme di gestione di progetto come Github o Jira che riassume in maniera visiva lo stato del progetto e ne permette
la gestione.

\subsection*{Prompt Engineering}
Tecnica di progettazione delle istruzioni (prompt) per ottimizzare l'interazione e le risposte fornite dai LLM.

\subsection*{Proof of Concept (PoC)}
Nel contesto dello sviluppo software, è un metodo di verifica della realizzabilità di un prodotto, dal punto di vista concettuale e tecnologico, che avviene
attraverso lo sviluppo di una versione molto ridotta della soluzione finale.

\subsection*{PostgreSQL}
Database open source relazionale a oggetti progettato per prestazioni di livello enterprise, è apprezzato per le sue solide funzionalità e affidabilità.
Nato nel 1986 come evoluzione di Ingres, un precedente database della University of California, utilizza Structured Query Language (SQL) per query e transazioni.

\subsection*{Pull Request}
Meccanismo utilizzato nei sistemi di controllo versione come Git/GitHub che permette a uno sviluppatore di notificare ai manutentori del repository che ha
completato nuove modifiche e desidera che vengano integrate (fuse) nel ramo principale.

\subsection*{Push}
Comando Git utilizzato per trasferire i commit locali (salvataggi) da un repository locale al repository remoto, aggiornando la storia del codice condiviso
su piattaforme come GitHub. Questo rende le modifiche disponibili agli altri membri del team.

\section{R}

\subsection*{Release}
Il ramo 'release' (o l'azione di rilascio) in un sistema di controllo versione rappresenta il momento in cui il software è pronto per essere distribuito
agli utenti finali. Questo ramo viene isolato dal ramo 'develop' per permettere solo correzioni critiche prima del lancio.

\subsection*{Remediation}
Azione correttiva volta a risolvere problemi di sicurezza rilevati in un sistema software.

\subsection*{Representational State Transfer (REST)}
Architettura per la comunicazione tra sistemi distribuiti basata su protocolli HTTP, utilizzata nella creazione di API web.

\subsection*{Round Review}
Consiste in una delle tante review di un documento che avvengono in maniera iterativa da parte di verificatori, esse generano modifiche del documento
fino al raggiungimento di un risultato approvabile. (P.S Non si trovano definizioni online, non si capisce se è un processo o se è un iterazione del processo).

\section{S}

\subsection*{Software Development Kit (SDK)}
Insieme di strumenti, librerie e documentazione forniti per facilitare lo sviluppo di applicazioni che interagiscono con un determinato sistema o chiamate API.

\subsection*{Sprint}
Nel contesto Agile, uno Sprint è un breve periodo di tempo, di durata 2-4 settimane, in cui uno Scrum Team lavora per completare una determinata quantità di
lavoro.

\subsection*{Stato Avanzamento Lavori (SAL)}
Nell'ambito del project management, il SAL è una riunione periodica che viene stabilita per certificare che il progetto si stia sviluppando secondo le
aspettative di tempi, costi, e risultati, devono essere concordati gli argomenti da valutare secondo precise logiche.

\section{T}

\subsection*{TimescaleDB}
Database relazionale per sequenze di dati che cambiano nel tempo, è implementato come estensione di PostgreSQL.

\section{U}

\subsection*{UML2.5}
UML2.5 consiste nell'ultima versione stabile dell'Unified Modeling Language (UML), che aggiunge dettaglio, coerenza, completezza al linguaggio e raffina i
diagrammi di comportamento, dunque potenziando la capacità di descrivere i modelli dinamici.

\subsection*{Use Case}
Descrizione di un'azione o sequenza di eventi che un sistema (o prodotto) deve eseguire in risposta a un obiettivo specifico o a un'interazione con un utente
(o attore). È fondamentale nella fase di definizione dei requisiti di un progetto software.

\section{W}

\subsection*{Way Of Working}
Il Way of Working (WoW) descrive il modo in cui un team o un'organizzazione gestisce le proprie attività, i flussi di lavoro e il raggiungimento degli obiettivi.
Include pratiche, processi, metodologie, tecniche e convenzioni che favoriscono un lavoro efficiente e coerente. Il WoW riguarda aspetti come comunicazione,
collaborazione, gestione del tempo, uso di strumenti e processi decisionali. Un WoW ben definito migliora l'efficacia operativa, la qualità dei risultati e la
soddisfazione del team, adattandosi alla cultura e alle esigenze aziendali.

\end{document}