\documentclass[10pt, letterpaper]{article}
\usepackage[nomarginpar, margin=2.75cm, tmargin=3cm, bmargin=1.75cm]{geometry}
\usepackage{graphicx}
\usepackage[
    colorlinks=true,      
    linkcolor=black,      
    urlcolor=blue,       
    citecolor=black       
]{hyperref}
\usepackage[table, x11names]{xcolor}
\renewcommand{\contentsname}{Indice}
\usepackage{fancyhdr}

%Comandi per livello di sottosezioni = 3
\setcounter{tocdepth}{4}
\setcounter{secnumdepth}{4}
\newcommand{\trisubsection}[1]{\paragraph{#1}\mbox{}\\}
\pagestyle{fancy}
\fancyhf{}
\fancyhead[L]{SnakeByte} 
\fancyhead[R]{Way of Working}
\fancyfoot[C]{\thepage}


\begin{document}

\begin{titlepage}
    \begin{center}
        \begin{center}
            \includegraphics[width=0.6\textwidth]{./img/logo.pdf}
        \end{center}
        \vspace{4cm}
        \huge\textbf{Way of Working}\par
        \vspace{2cm}
        \large \textbf{SnakeByte} (Gruppo 1):\\
        \large Valeria Baleanu, Leonardo Pellizzon, Filippo Venzo, Giuseppe De Fina, \\
         Francesco Pasqual, Christian Libralato, Luca Granziero \\
        (2109911, 2111006, 2113705, 2113187, 2103119, 2101047, 2075512)
        \vfill
        \small
        \begin{center}
            \begin{tabular}{|c|c|c|c|}
                \hline
                \multicolumn{4}{|c|}{\textbf{Informazioni documento}} \\
                \hline
                \rowcolor{lightgray} \textbf{Versione} & \textbf{Data} & \textbf{Stato} & \textbf{Destinatari} \\
                \hline
                0.1.0 & 17/10/2025 & Verificato & SnakeByte, prof. Tullio Vardanega, prof. Riccardo Cardin \\
                \hline
            \end{tabular}
        \end{center}
        \vfill
        \large Contatti: snakebyteteam@gmail.com
    \end{center}
\end{titlepage}
\begin{center}
\begin{tabular}{|c|c|c|c|c|c|}
    \hline
    \multicolumn{6}{|c|}{\textbf{Versioni del documento}} \\
    \hline
    \rowcolor{lightgray} \textbf{Versione} & \textbf{Data} & \textbf{Autore} & \textbf{Verifica} & \textbf{Approvazione} & \textbf{Descrizione} \\
    \hline
    0.1.0 & 17/10/2025 & F. Venzo & L. Granziero & - & Prima stesura \\
    \hline
\end{tabular}    
\end{center}

\newpage

\tableofcontents

\newpage

\section{Introduzione}
\subsection{Finalità del documento}
Il presente documente intende fissare le linee guida che il gruppo \textit{SnakeByte} si impegna a rispettare ed attuare per perseguire la migliore efficenza ed efficacia nel processo di realizzazione del progetto didattico.

Il documento è strutturato secondo le norme dello Standard ISO/IEC 122007:1995 e segue quanto descritto nel \textit{Regolamento del progetto didattico (A.a. 2025/2026)}. Presenta una descrizione dei \textit{processi} del ciclo di vita del \textit{software} e delle \textit{attività} di cui sono composti.
A sua volta, ogni attività, è composta da una serie di procedure metodiche dotate di obiettivi e strumenti ben definiti.

E' importante notare che il documento in questione è in continua evoluzione fino al suo ritiro, poiché le norme contenute al suo interno vegono costantemente revisionate, ottimizzate ed aggiornate, seguendo un approccio iterativo.

Ogni attività svolta nell'interesse del progetto didattico e nei suoi materiali è regolamentata precedentemente all'esecuzione della stessa.

\subsection{Glossario}
Il documento cita alcuni termini la cui defizione può risultare ambigua. Per questo, è possibile consultare il glossario, contenuto nella documentazione del progetto \texttt{<link a pdf glossario>}, il quale contiene le definizioni di tali espressioni, che sarrano marcate da una lettera \textit{G} a pedice (?)

\subsection{Riferimenti Normativi}
\begin{itemize}
    \item \textbf{Standard ISO/IEC 122007:1995}: 
    
    \url{https://www.math.unipd.it/~tullio/IS-1/2009/Approfondimenti/ISO_12207-1995.pdf} 
    
    (consultato il 20/10/2025)
    \item \textbf{Regolamento del progetto didattico}: 
    
    \url{https://www.math.unipd.it/~tullio/IS-1/2025/Dispense/PD1.pdf} 
    
    (consultato il 20/10/2025)
\end{itemize}

\subsection{Riferimenti Informativi}
\begin{itemize}
    \item Glossario \texttt{<link a glossario>}
\end{itemize}

\section{Processi primari}
\subsection{Fornitura}
Il processo di fornitura, come specificato nello Standard ISO/IEC 122007:1995, definisce le attività dell'organizzazione che fornisce il prodotto software all'acquirente, dalla concezione fino alla consegna del prodotto. Viene istanziato conseguentemente alla redazione della \textit{Valutazioni dei capitolati} \texttt{<link al documento valutazione>}.

\subsubsection{Attività}
Il processo di fornitura si compone delle seguenti attività
\begin{itemize}
    \item \textbf{Avviamento}: revisione delle proposte dei richiedenti. Per i capitolati di maggiore interesse vengono mandate delle comunicazione via mail per eventuali approfondimenti.
    
    \item \textbf{Preparazione delle riposta}: viene scelto il capitolato per cui ci vuole candidare sulla base delle considerazioni fatte nelle fase precendete e viene preparato il documento di \textbf{candidatura} \texttt{<link doc>}.
\end{itemize}

\subsubsection{Documentazione}
La documentazione prodotta durante le attività di fornitura, la quale verrà consegnata ai committenti quali \textit{Prof.} Tullio Vardanega, \textit{Prof.} Riccardo Cardin e all'azienda proponente è la seguente 

\begin{itemize}
    \item \textbf{Valutazioni dei capitolati} contenente 
    \begin{itemize}
        \item Titolo del capitolato e nome dell'azienda proponente;
        \item Una breve descrizione del capitolato e dei suoi obiettivi;
        \item Punti di forza;
        \item Criticità.
    \end{itemize}
    \item \textbf{Candidatura} contenente
    \begin{itemize}
        \item Scandenza di consegna prevista;
        \item Preventivo dei costi totali del progetto (calcolato secondo il \textit{Regolamento del progetto didattico});
        \item Dichiarazioni di impengo in ore produttive per componente del gruppo.
    \end{itemize}
\end{itemize}

\section{Processi di supporto}
\subsection{Documentazione}

\subsubsection{Struttura dei documenti}
Le seguenti sezioni illustrano le componenti che ogni documento creato deve avere e per ogni tipologia le caratteristiche dedicate.

\trisubsection{Prima pagina}  

La prima pagina di ogni documento deve riportare, in ordine di posizionamento dall'alto verso il basso

\begin{itemize}
    \item Logo del gruppo \textit{SnakeByte};
    \item Titolo del documento;
    \item Il nome e il numero del gruppo \textit{SnakeByte};
    \item Nome e cognome di ogni componente e relativo numero di matricola UniPD;
    \item Informazioni generali del documento quali
    \begin{itemize}
        \item Versione attuale;
        \item Data di creazione della versione;
        \item Lo stato attuale;
        \item I destinatari del documento.
    \end{itemize}
    \item Contatto email del gruppo \textit{SnakeByte}
\end{itemize}

\trisubsection{Intestazione}

Ogni pagina di qualsiasi documento deve riportate come intestazione
\begin{itemize}
    \item Nome del gruppo \textit{SnakeByte}
    \item Titolo del relativo documento
\end{itemize}

\trisubsection{Registro delle modifiche}

Tutti i documenti, interni ed esterni, devono riportare a partire dalla seconda pagina il \textit{registro delle modifiche} sottoforma di tabella, la quale deve riassumemere

\begin{itemize}
    \item Versione del documento;
    \item Data di creazione;
    \item Autore della versione;
    \item Verificatore della versione;
    \item Approvatore della versione;
    \item Descrizione riassuntiva delle modifiche alla versione precedente.
\end{itemize}

Ogni modifica ad un documento scatena la crezione di una nuova versione di esso e quindi la compilazione di una nuova riga, verso il basso, della tabella.\\

\trisubsection{Indice}

Ogni documente deve riportare l'indice dove saranno elencati i titoli di tutte le sezioni e sottosezioni.

Ogni titolo deve essere provvisto di link che porta alla sezione associata all'interno dello stesso documento.\\

\textbf{Metodologie}
\begin{itemize}
    \item Questo deve essere fatto tramite il pacchetto \texttt{hyperref} fornito dal il linguaggio \LaTeX
\end{itemize}

\subsubsection{Struttura dei verbali}
I verbali sia interni che esterni, oltre alla struttura descritta nel capitolo \hyperref[sec:Struttura]{\textit{§3.1.1}}, devono essere composti dalle seguenti sezioni

\begin{itemize}
    \item \textbf{Informazioni}, contenente
    \begin{itemize}
        \item Data di svolgimento;
        \item Ora inizio;
        \item Ora fine;
        \item Modalità di svolgimento (Presenza, Online o tramite canali asincroni).
    \end{itemize}
    \item \textbf{Presenze}, contenente, in forma tabellare, le seguenti informazioni
    \begin{itemize}
        \item Nome e cognome di tutti i membri;
        \item Ruolo (ND se non definito);
        \item Presenza alla riunione.
    \end{itemize}
    \item \textbf{Ordine del giorno}, con all'interno una lista degli argomenti che vengono trattati all'interno dell'incontro;
    \item \textbf{Approfondimento} degli argomenti ordine del giorno;
    \item \textbf{Decisioni} (\hyperref[sec:TabellaDecisioni]{Sezione \textit{§3.1.2.1}});
    \item \textbf{Attività da svolgere} (\hyperref[sec:TabellaToDo]{Sezione \textit{§3.1.2.2}})
\end{itemize}

\label{sec:TabellaDecisioni}
\trisubsection{Tabella delle decisioni}
Per il tracciamento e l'organizzazione di ogni decisione presa collettivamente dal gruppo \textit{SnakeByte}, al termine di ogni verbale, deve essere presente una tabella che riporta le decisioni prese in seguito alla riunione in questione. Per ogni decisione deve essere riportato
\begin{itemize}
    \item Identificativo alfanumerico della decisioni, così composto\\ \texttt{v\{i, e\}}\_AAAA\_MM\_GG.d\texttt{<numero\_decisione>};
    \item Descrizione testuale della decisione presa;
\end{itemize}

\label{sec:TabellaToDo}
\trisubsection{Tabella delle attività da svolgere}
Per il tracciamento delle attività da eseguire, emerse durante l'incontro trattato dal verbale, deve essere presente una tabella che riporta una lista di queste, riassumendo le seguenti informazioni
\begin{itemize}
    \item Identificativo alfanumerico dell'attività, così composto\\ \texttt{v\{i, e\}}\_AAAA\_MM\_GG.a\texttt{<numero\_attività>};
    \item Descrizione testuale dell'attività;
    \item Id GitHub Issue associata all'attività (carattere \textit{"-"} se non presente).
\end{itemize}


\section{Processi organizzativi}

\subsection{Coordinamento}

\subsubsection{Riunione interna fissata}
Il gruppo \textit{SnakeByte} ha scelto di fissare un giorno all'interno della settimana lavorativa in cui si svolge una riunione in presenza a cui prendono parte i componenti del gruppo. La partecipazione dei membri è richiesta ma non tassativa.

Il giorno attuale in cui si tiene la riunione è il \textit{Lunedì} alle ore \textit{12.15} circa.

Altre eventuali riunioni possono essere fissate tramite confronto e accordo tra i componenti del gruppo su giorno e ora.

\subsubsection{Comunicazioni interne}
Per le comunicazione interne, il gruppo \textit{SnakeByte} ha individuato come mezzo per la trasmissione di informazioni in formato asincrono l'applicazione di messaggistica \textit{Whatsapp}. Mentre per le comunicazioni sincrone \textit{Discord}.

\subsubsection{Comunicazini esterne}
Per le comunicazioni esterne, il gruppo \textit{SnakeByte} ha individuato come mezzo per la trasmissione di informazioni in formato asincrono i messaggi email, tramite esclusivamente l'indirizzo ufficiale del gruppo (\textit{snakebyteteam@gmail.com}). Mentre per le comunicazioni sincrone l'applicazione di videoconferenze \textit{Google Meet}. 

\end{document}