\documentclass[10pt, letterpaper]{article}

\usepackage{../../../../template/template}

\setDocTitle{Verbale interno 15/12/2025}
\setDocVersion{0.1.0}
\setDocData{18/12/2025}
\setDocIndex{vi\_2025\_12\_15}
\setDocState{1}
\setDocRecipient{0}

% comandi extra (rimuovere questi commenti una volta approvato il documento)
%
% -- costanti --
% \noOne [-]
% \noRole [ND]
% \firstDraft [Prima stesura]
% \approval [Approvazione]
% \teamName [SnakeByte]
% \teamEmail [snakebyteteam@gmail.com]
% \teamRepo [https://github.com/SnakeByteTeam/snakebyteteam.github.io]
% \teamSite [https://snakebyteteam.github.io/]
% \recipientTeachers [prof. Vardanega Tullio, prof. Cardin Riccardo]
% \vimar [Vimar]
% \vimarspa [Vimar S.p.A.]
%
% -- variabili --
% \docIndex | \setDocIndex{nome documento senza versione}
% \docTitle | \setDocTitle{titolo documento} 
% \docData | \setDocData{data documento} [GG/MM/AAAA]
% \docVersion | \setDocVersion{versione documento}
% \docState | \setDocState{0|1|2} [Da verificare][Verificato][Approvato]
% \docRecipient [table] | \setDocRecipient{0|1|2} [SnakeByte][SnakeByte, prof][SnakeByte, prof, vimar]
% 
% -- tabelle -- [gli id sono automatici dove serve]
% \makeMeetingInfoTable{data}{ora inizio}{ora fine}{modalità}
% \addParticipant{nome}{cognome}{ruolo}{presenza} | \makeMeetingParticipantsTable
% \addTodo{id github issue}{descrizione}{assegnatario}{scadenza} | \makeTodoTable
% \addPrecedentTodo{id todo}{id github issue todo}{assegnatario todo}{data} | \makePrecedentTodoTable
% \addDecision{descrizione decisione} | \makeDecisionTable
%
% -- pagine intere --
% \makeTitlePage
% \addChangelog{versione}{data}{autore}{verificatore}{approvatore}{descrizione} | \makeTodoTable

\begin{document}

\makeTitlePage

\newpage

\addChangelog{\docVersion}{\docData}{C. Libralato}{F. Pasqual}{\noOne}{\firstDraft}
\makeChangelog

\newpage

\tableofcontents

\newpage

\makeMeetingInfoTable{15/12/2025}{17:00}{18:30}{via Discord}

\addParticipant{Christian}{Libralato}{Responsabile}{P}
\addParticipant{Giuseppe}{De Fina}{Amministratore}{P}
\addParticipant{Valeria}{Baleanu}{Progettista}{NP}
\addParticipant{Francesco}{Pasqual}{Verificatore}{P}
\addParticipant{Leonardo}{Pellizzon}{Analista}{P}
\addParticipant{Filippo}{Venzo}{Programmatore}{P}
\addParticipant{Luca}{Granziero}{Programmatore}{P}
\makeMeetingParticipantsTable

\addPrecedentTodo{vi\_2025\_12\_01.a1}{\#17}{F. Pasqual, L. Pellizzon}{10/12/2025}
\addPrecedentTodo{vi\_2025\_12\_01.a2}{\#19}{F. Pasqual, L. Pellizzon}{05/12/2025}
\addPrecedentTodo{vi\_2025\_12\_01.a3}{\#22}{C. Libralato}{14/12/2025}
\addPrecedentTodo{vi\_2025\_12\_01.a4}{\noOne}{F. Pasqual}{02/12/2025}
\addPrecedentTodo{vi\_2025\_12\_01.a5}{\noOne}{F. Pasqual}{02/12/2025}
\addPrecedentTodo{vi\_2025\_12\_01.a6}{\#5}{F. Venzo}{08/12/2025}
\addPrecedentTodo{vi\_2025\_12\_01.a7}{\noOne}{G. De Fina, V. Baleanu, F. Venzo}{07/12/2025}
\addPrecedentTodo{vi\_2025\_12\_01.a8}{\#6}{F. Venzo}{08/12/2025}
\addPrecedentTodo{vi\_2025\_12\_01.a9}{\#18}{G. De Fina, V. Baleanu, F. Venzo}{05/12/2025}
\addPrecedentTodo{vi\_2025\_12\_01.a10}{\#16}{C. Libralato}{09/12/2025}
\makePrecedentTodoTable


\section{Ordine del giorno}
\begin{itemize}
    \item \textit{Sprint Retrospective$_G$};
    \item revisione Analisi dei Requisiti;
    \item discussione Piani di Progetto e di Qualifica;
    \item ridistribuzione ore e ruoli;
    \item proposta di utilizzo di \textit{Prisma$_G$}.
\end{itemize}

\section{Approfondimento}
\subsection*{\textit{Sprint Retrospective}}
Dall'analisi del precedente periodo di avanzamento non sono emerse particolari criticità, le attività sono
state portate a termine secondo le scadenze. L'unica difficoltà incontrata è stata la coordinazione tra analisti sulle 
convenzioni da adottare riguardo la struttura dei casi d'uso, che ha complicato il processo di unificazione
dei vari lavori svolti in autonomia.

\subsection*{Revisione Analisi dei Requisiti}
È stata identificata la necessità di correggere la struttura dei casi d'uso e la gestione degli errori all'interno
dell'Analisi dei Requisiti in vista del colloquio con il professor Cardin.

\subsection*{Discussione Piani di Progetto e di Qualifica}
Durante la riunione è emersa la necessità di velocizzare la redazione del Piano di Progetto e del Piano di Qualifica.
In riferimento a quest'ultimo si è riflettuto sulle metriche da utilizzare per misurare il livello di avanzamento e la qualità. 
Sebbene queste non siano state identificate in maniera definitiva, si ritiene che il piano conterrà circa cinque metriche legate
all'avanzamento.

\subsection*{Ridistribuzione ore e ruoli}
Dalla prima stesura del Piano di Progetto è emerso, come era lecito aspettarsi, che le ore previste per i singoli ruoli
non risultano perfettamente accurate, è emerso dunque il dubbio su come gestire le ore avanzate.
È stato deciso che, in caso di sovrastime delle ore svolte da un ruolo in uno sprint, queste
verranno tracciate e, in caso di necessità, redistribuite ad altri ruoli.
Inoltre è avvenuta una variazione dei ruoli previsti per lo sprint corrente, in particolare, a
Leonardo Pellizzon è stato assegnato il ruolo Analista.

\subsection*{Proposta di utilizzo di \textit{Prisma}}
Al fine di semplificare e velocizzare l'interazione con il database è stato proposto lo strumento
\textit{Prisma}, sarà necessario discutere con la Proponente la possibilità di utlizzare la suddetta tecnologia.


\addDecision{Le ore non consumate da un ruolo in uno sprint possono essere distribuite su necessità}
\makeDecisionTable

\addTodo{\#33}{Terminazione AdR}{L. Pellizzon}{30/12/2025}
\addTodo{\#34}{Compilazione PdP}{G. De Fina}{30/12/2025}
\addTodo{\#35}{Prima Stesura PdQ}{G. De Fina}{30/12/2025}
\addTodo{\#36}{Inizio implementazione PoC}{F. Venzo, L. Granziero}{30/12/2025}
\addTodo{\#37}{Definizione diagramma Gantt}{G. De Fina}{30/12/2025}
\addTodo{\#38}{Scrittura verbale esterno 10/12/2025}{C. Libralato, F. Pasqual}{20/12/2025}
\addTodo{\#39}{Scrittura verbale interno 15/12/2025}{C. Libralato, F. Pasqual}{20/12/2025}
\makeTodoTable

\end{document}