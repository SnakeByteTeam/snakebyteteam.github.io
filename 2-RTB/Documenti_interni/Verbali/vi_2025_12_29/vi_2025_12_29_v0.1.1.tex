\documentclass[10pt, letterpaper]{article}

\usepackage{../../../../template/template}

\setDocTitle{Verbale interno 29/12/2025}
\setDocVersion{0.1.1}
\setDocData{29/12/2025}
\setDocIndex{vi\_2025\_12\_29}
\setDocState{0}
\setDocRecipient{1}

\begin{document}

\makeTitlePage

\newpage

\addChangelog{\docVersion}{02/01/2026}{L. Granziero}{F. Venzo}{\noOne}{Correzione errori}
\addChangelog{0.1.0}{\docData}{L. Granziero}{F. Venzo}{\noOne}{\firstDraft}
\makeChangelog

\newpage

\tableofcontents

\newpage

% -------------------------------------------------
% Informazioni riunione
% -------------------------------------------------
\makeMeetingInfoTable{29/12/2025}{13:20}{14:00}{via Discord}

\addParticipant{Valeria}{Baleanu}{Amministratore}{P}
\addParticipant{Leonardo}{Pellizzon}{Programmatore}{P}
\addParticipant{Luca}{Granziero}{Responsabile}{P}
\addParticipant{Francesco}{Pasqual}{Programmatore}{P}
\addParticipant{Giuseppe}{De Fina}{Verificatore}{P}
\addParticipant{Christian}{Libralato}{Programmatore}{P}
\addParticipant{Filippo}{Venzo}{Verificatore}{P}
\makeMeetingParticipantsTable


\addPrecedentTodo{vi\_2025\_12\_15.a2}{\#34}{G. de Fina}{27/12/2025}
\addPrecedentTodo{vi\_2025\_12\_15.a4}{\#36}{F. Venzo, L. Granziero}{29/12/2025}
\addPrecedentTodo{vi\_2025\_12\_15.a6}{\#38}{C. Libralato}{23/12/2025}
\addPrecedentTodo{vi\_2025\_12\_15.a7}{\#39}{C. Libralato}{23/12/2025}
\makePrecedentTodoTable


% -------------------------------------------------
% Ordine del giorno
% -------------------------------------------------
\section{Ordine del giorno}
\begin{itemize}
    \item definizione e selezione delle metriche di qualità;
    \item metriche di processo, sviluppo, documentazione e prodotto;
    \item automazione del calcolo delle metriche;
    \item gestione Gantt e monitoraggio attività;
    \item requisiti e use case$_G$;
    \item comunicazioni esterne e scadenze.
\end{itemize}

% -------------------------------------------------
% Approfondimento
% -------------------------------------------------
\section{Approfondimento}

\subsection*{Metriche di processo e fornitura}
Vengono presentate le metriche di processo per la fornitura: \textit{planned value}, \textit{earned value},
\textit{actual cost}, \textit{CPI (Cost Performance Index)} e \textit{SPI (Schedule Performance Index)}.
Viene confermata l’importanza di tali metriche per il monitoraggio di tempi e costi; alcune metriche
ridondanti, come le variance, vengono escluse.

\subsection*{Metriche per lo sviluppo}
Vengono illustrate le metriche \textit{RSI (Requirements Stability Index)}, \textit{RCV (Requirements Change Volatility)} e \textit{PAR (Planned Activities Ratio)},
ritenute utili per valutare stabilità dei requisiti e avanzamento delle attività.  
Il gruppo concorda nel mantenerle.

\subsection*{Metriche per la documentazione}
Si discute l’utilità di metriche quali indice di Gulpease, densità degli errori ortografici e indice di frammentazione.

\subsection*{Metriche di qualità di prodotto}
Vengono analizzate metriche relative ai requisiti, al \textit{code coverage} e alla manutenibilità.
Si conferma il mantenimento della complessità ciclomatica e l’eliminazione della densità dei commenti.
Resta aperta la valutazione sulla metrica di facilità di apprendimento.

\subsection*{Automazione e strumenti}
Il gruppo concorda sulla necessità di automatizzare il più possibile il calcolo delle metriche.
Viene confermato l’utilizzo di Excel per la creazione dei grafici e di strumenti online per alcune metriche di documentazione.

\subsection*{Gestione Gantt}
Vengono illustrate diverse opzioni per la gestione del Gantt: screenshot da GitHub$_G$, pacchetti \LaTeX{}$_G$ o Excel.
Il gruppo decide all'unanimità di utizzare la sezione roadmap di GitHub projects per realizzare il Gantt.

\subsection*{Requisiti e use case}
Viene stabilità una granularità dettagliata per quanto riguarda la tabella dei requisiti, identificando anche i requisiti non funzionali.

\subsection*{Comunicazioni esterne}
Viene fatta presente la creazione del diario di bordo entro la data stabilita dal prof. Vardanega con i vari progressi e dubbi emersi.

% -------------------------------------------------
% Decisioni
% -------------------------------------------------

\addDecision{Mantenere le metriche principali di processo e fornitura}
\addDecision{Mantenere indice di Gulpease ed eliminare densità commenti nel codice}
\addDecision{Utilizzare Excel per i grafici delle metriche}
\addDecision{Assegnare tre programmatori e due verificatori per il prossimo sprint}
\addDecision{Utilizzare la roadmap di GitHub  projects per il Gantt}
\makeDecisionTable

% -------------------------------------------------
% Azioni
% -------------------------------------------------

\addTodo{\#33}{Terminazione AdR}{L. Pellizzon}{30/12/2025}
\addTodo{\#35}{Prima stesura PqQ}{G. de Fina}{30/12/2025}
\addTodo{\#37}{Definizione Gantt}{C. Libralato}{ND}
\addTodo{\#55}{Aggiungere "derivable" al Glossario}{L. Granziero}{3/01/2026}
\addTodo{\#56}{Scrittura verbale interno}{L. Granziero}{3/01/2026}
\makeTodoTable

\end{document}
