\documentclass[10pt, letterpaper]{article}

\usepackage{../../../../template/template}

\setDocTitle{Verbale interno 01/12/2025}
\setDocVersion{1.0.0}
\setDocData{10/02/2026}
\setDocIndex{vi\_2025\_12\_01}
\setDocState{2}
\setDocRecipient{0}

% comandi extra (rimuovere questi commenti una volta approvato il documento)
%
% -- costanti --
% \noOne [-]
% \noRole [ND]
% \firstDraft [Prima stesura]
% \approval [Approvazione]
% \teamName [SnakeByte]
% \teamEmail [snakebyteteam@gmail.com]
% \teamRepo [https://github.com/SnakeByteTeam/snakebyteteam.github.io]
% \teamSite [https://snakebyteteam.github.io/]
% \recipientTeachers [prof. Vardanega Tullio, prof. Cardin Riccardo]
% \vimar [Vimar]
% \vimarspa [Vimar S.p.A.]
%
% -- variabili --
% \docIndex | \setDocIndex{nome documento senza versione}
% \docTitle | \setDocTitle{titolo documento} 
% \docData | \setDocData{data documento} [GG/MM/AAAA]
% \docVersion | \setDocVersion{versione documento}
% \docState | \setDocState{0|1|2} [Da verificare][Verificato][Approvato]
% \docRecipient [table] | \setDocRecipient{0|1|2} [SnakeByte][SnakeByte, prof][SnakeByte, prof, vimar]
% 
% -- tabelle -- [gli id sono automatici dove serve]
% \makeMeetingInfoTable{data}{ora inizio}{ora fine}{modalità}
% \addParticipant{nome}{cognome}{ruolo}{presenza} | \makeMeetingParticipantsTable
% \addTodo{id github issue}{descrizione}{assegnatario}{scadenza} | \makeTodoTable
% \addPrecedentTodo{id todo}{id github issue todo}{assegnatario todo}{data} | \makePrecedentTodoTable
% \addDecision{descrizione decisione} | \makeDecisionTable
%
% -- pagine intere --
% \makeTitlePage
% \addChangelog{versione}{data}{autore}{verificatore}{approvatore}{descrizione} | \makeTodoTable

\begin{document}

\makeTitlePage

\newpage

\addChangelog{\docVersion}{\docData}{\noOne}{\noOne}{F. Venzo}{Approvazione}
\addChangelog{0.1.1}{10/12/2025}{F. Pasqual}{L. Pellizzon}{\noOne}{Modifiche todo}
\addChangelog{0.1.0}{03/12/2025}{F. Pasqual}{L. Pellizzon}{\noOne}{\firstDraft}
\makeChangelog

\newpage

\tableofcontents

\newpage

\makeMeetingInfoTable{\docData}{12:30}{13:30}{via presenza}

\addParticipant{Francesco}{Pasqual}{Responsabile}{P}
\addParticipant{Christian}{Libralato}{Amministratore}{P}
\addParticipant{Luca}{Granziero}{Progettista}{P}
\addParticipant{Leonardo}{Pellizzon}{Verificatore}{P}
\addParticipant{Valeria}{Baleanu}{Analista}{P}
\addParticipant{Filippo}{Venzo}{Analista}{P}
\addParticipant{Giuseppe}{De Fina}{Analista}{P}
\makeMeetingParticipantsTable

\addPrecedentTodo{vi\_2025\_11\_25.a1}{\#1}{F. Pasqual, V. Baleanu}{27/11/2025}
\addPrecedentTodo{vi\_2025\_11\_25.a2}{\noOne}{F. Venzo, C. Libralato, L. Granziero}{27/11/2025}
\addPrecedentTodo{vi\_2025\_11\_25.a3}{\#5}{F. Venzo, C. Libralato, L. Granziero}{30/11/2025}
\addPrecedentTodo{vi\_2025\_11\_25.a4}{\#6}{F. Venzo, C. Libralato, L. Granziero}{30/11/2025}
\addPrecedentTodo{vi\_2025\_11\_25.a5}{\noOne}{F. Pasqual}{27/11/2025}
\addPrecedentTodo{vi\_2025\_11\_25.a6}{\noOne}{C. Libralato}{29/11/2025}
\addPrecedentTodo{vi\_2025\_11\_25.a7}{\noOne}{G. De Fina}{27/11/2025}
\addPrecedentTodo{vi\_2025\_11\_25.a8}{\noOne}{L. Pellizzon}{30/11/2025}
\makePrecedentTodoTable

\section{Ordine del giorno}
\begin{itemize}
    \item \textit{Sprint Retrospective$_G$};
    \item creazione e assegnazione delle attività da svolgere nel nuovo (corrente) sprint;
    \item revisione della struttura e forma degli \textit{use case$_G$};
    \item \textit{use case} autenticazione e registrazione utenti;
    \item redazione e responsabilità dei documenti di progetto: Norme di Progetto, Piano di Progetto, verbali;
    \item redazione e consegna della bozza di Analisi dei Requisiti a \vimarspa.
\end{itemize}

\section{Approfondimento}

\subsection*{\textit{Sprint Retrospective}}
Come prima cosa, il gruppo ha svolto l'attività di retrospettiva su quanto fatto nelle ultime due settimane (ultimo sprint).
La retrospettiva ha evidenziato una situazione che il gruppo ritiene abbastanza positiva, dove quasi tutte le attività programmate
sono state svolte con puntualità. \\
Sono state individuate due \textit{issue$_G$} non completate. Per gestirle, si è deciso 
di non allungare lo sprint ma di spostarle e proseguirle nello sprint successivo, contrassegnandole con dei campi 
custom e commenti interni alla \textit{issue} stessa. \\
I risultati ricavati dalla retrospettiva andranno riportati nel consuntivo di periodo all'interno del documento di Piano di Progetto,
con particolare attenzione alle conseguenze dettate dalle attività non completate. 

\subsection*{Revisione della struttura e forma degli \textit{use case}}
Gli Analisti dello sprint appena concluso hanno presentato il lavoro svolto riguardo
all'attività di Analisi dei Requisiti. 
Sono stati esposti i dubbi e le difficoltà incontrati nel definire la struttura e la forma degli \textit{use case} e 
nella definizione e gestione delle precondizioni, trigger, inclusioni ed estensioni.
A tal riguardo, verranno poste delle domande ai docenti.

\subsection*{\textit{Use case} autenticazione e registrazione utenti}
Il gruppo ha valutato le modalità di autenticazione e registrazione da mettere a disposizione dell'utente nell'applicativo web \textit{View4Life$_G$}.
E' stato discusso, in particolare per la registrazione, l'utilizzo di un codice di registrazione o una password temporanea, con preferenza per quest'ultima.
Rimane comunque aperta la possibilità di modifica futura.

\subsection*{Redazione e responsabilità dei documenti di progetto}
Il gruppo ha discusso brevemente riguardo alla produzione della documentazione.
E' stata individuata la necessità di cominciare a redarre il documento di Piano di Progetto
dove tracciare preventivi e consuntivi dei primi periodi di lavoro.
Inoltre, è emersa la necessità di aggiornare il documento di Norme di Progetto 
con una sezione dedicata all'attività di Analisi dei Requisiti. \\
Riguardo questi due documenti, il gruppo ha deciso che saranno presi in carico dell'Amministratore, in quanto
il Responsabile, figura il cui lavoro è molto costoso, dovrà redarre il verbale interno della presente riunione e
il verbale esterno relativo all'incontro \textit{SAL$_G$} tenuto recentemente con \vimarspa. \\
Così facendo, il gruppo crede di poter rispettare ore e costo del lavoro preventivato.

\subsection*{Redazione e consegna della bozza di Analisi dei Requisiti a \vimarspa}
La proponente \vimarspa \ ha richiesto una consegna anticipata della bozza del documento di Analisi dei Requisiti
per monitorare la comprensione del dominio e delle richieste del progetto da parte del team.
Il gruppo ha pertanto deciso che gli Analisti si riuniranno, preferibilmente in maniera sincrona, per redarre il documento richiesto.
Sarà poi il Responsabile a inviare il documento di bozza a \vimarspa.


\addDecision{Non si allunga lo sprint: le attività non completate vengono trasferite e gestite nel prossimo sprint}
\addDecision{Le issue non concluse saranno tracciate nel nuovo sprint tramite campi custom e commenti interni alla issue}
\addDecision{Rivedere la struttura degli \textit{use case} e chiedere chiarimenti ai docenti}
\addDecision{Password temporanea per la registrazione degli utenti}
\addDecision{Piano di Progetto e Norme di Progetto a carico dell'Amministratore}
\addDecision{Incontro tra Analisti per redarre la bozza dell'Analisi dei Requisiti}
\makeDecisionTable


\addTodo{\#17}{Stesura e verifica del verbale dell'incontro del 01/12/2025}{F. Pasqual, L. Pellizzon}{05/12/2025}
\addTodo{\#19}{Stesura, verifica e consegna del verbale esterno dell'incontro \textit{SAL}}{F. Pasqual, L. Pellizzon}{05/12/2025}
\addTodo{\#22}{Aggiornare il consuntivo di periodo nel Piano di Progetto}{C. Libralato}{08/12/2025}
\addTodo{\noOne}{Assegnare ai membri le nuove \textit{issue} per il nuovo (corrente) sprint}{F. Pasqual}{02/12/2025}
\addTodo{\noOne}{Spostare le issue non concluse nel backlog del nuovo (corrente) sprint}{F. Pasqual}{02/12/2025}
\addTodo{\#5}{Stesura \textit{use case} cruscotto}{F. Venzo}{07/12/2025}
\addTodo{\noOne}{Aggiornare struttura e forma \textit{use case}}{G. De Fina, V. Baleanu, F. Venzo}{07/12/2025}
\addTodo{\#6}{Stesura \textit{use case} autenticazione}{F. Venzo}{07/12/2025}
\addTodo{\#18}{Bozza dell'Analisi dei Requisiti}{G. De Fina, V. Baleanu, F. Venzo}{08/12/2025}
\addTodo{\#16}{Modifiche NdP: aggiungere sezione Analisi dei Requisiti}{C. Libralato}{08/12/2025}
\makeTodoTable



\end{document}