\documentclass[10pt, letterpaper]{article}

\usepackage{../../../../template/template}

\setDocTitle{Verbale interno 13/01/2026}
\setDocVersion{1.0.0}
\setDocData{13/01/2026}
\setDocIndex{vi\_2026\_01\_13}
\setDocState{1}
\setDocRecipient{0}

\begin{document}

\makeTitlePage

\newpage

\tableofcontents

\newpage

% -------------------------------------------------
% Informazioni riunione
% -------------------------------------------------
\makeMeetingInfoTable{13/01/2026}{11:00}{12:20}{via Discord}

\addParticipant{Valeria}{Baleanu}{Responsabile}{P}
\addParticipant{Leonardo}{Pellizzon}{Analista}{P}
\addParticipant{Luca}{Granziero}{Verificatore}{P}
\addParticipant{Francesco}{Pasqual}{Amministratore}{P}
\addParticipant{Giuseppe}{De Fina}{Programmatore}{P}
\addParticipant{Christian}{Libralato}{Amministratore}{P}
\addParticipant{Filippo}{Venzo}{Programmatore}{P}
\makeMeetingParticipantsTable


\addPrecedentTodo{vi\_2025\_12\_29.a2}{\#35}{V. Baleanu}{30/12/2025}
\addPrecedentTodo{vi\_2025\_12\_29.a4}{\#55}{L. Granziero}{2/01/2026}
\addPrecedentTodo{vi\_2025\_12\_15.a5}{\#56}{L. Granziero}{5/01/2026}
\makePrecedentTodoTable


% -------------------------------------------------
% Ordine del giorno
% -------------------------------------------------
\section{Ordine del giorno}
\begin{itemize}
    \item Definizione dei ruoli per lo sprint$_G$ e assegnazione finale delle responsabilità;
    \item aggiornamenti su piano di qualifica e metriche di qualità;
    \item stato di avanzamento del \textit{PoC}$_G$ e chiarimento concettuale sulla modellazione degli allarmi e delle soglie;
    \item pianificazione delle prossime attività e degli incontri con la proponente.
\end{itemize}

% -------------------------------------------------
% Approfondimento
% -------------------------------------------------
\section{Approfondimento}

\subsection*{Ridefinizione ruoli}
Si è avviata una discussione sulla necessità di ridefinire i ruoli dello sprint 
corrente, alla luce delle informazioni emerse dal cruscotto di qualità e delle 
ore risultate sottostimate nello sprint precedente. 
In particolare si è deciso di definire due amministratori (Francesco e Christian) 
e due programmatori (Filippo e Giuseppe) per permettere un avanzamento importante nella 
redazione dei documenti e nello sviluppo del \textit{Proof of Concept}.

\subsection*{Aggiornamenti sul cruscotto di qualità}
L'amministratore dello sprint precedente ha mostrato le modifiche apportate al 
documento del Piano di Qualifica, inclusi i grafici e gli indicatori di qualità, con 
la necessità di aggiornare i dati per gli sprint futuri. 

\subsection*{Stato e sviluppo del Proof of Concept}
I programmatori dello sprint precedente hanno presentato lo stato di avanzamento 
del \textit{Proof of Concept}, con particolare attenzione alla gestione di allarmi 
e soglie e allo sviluppo dell'interfaccia \textit{frontend}. 
Inoltre, i programmatori hanno illustrato la ristrutturazione della struttura 
delle cartelle del progetto per aderire meglio all'architettura esagonale.
È stata infine approfondita la distinzione concettuale tra 
“allarme” e “soglia”, evidenziando le criticità 
emerse nella fase di modellazione; si è concordato che l'allarme rappresenta 
un evento generato dal superamento di una soglia. Poiché tale conclusione 
deriva da una riflessione interna al gruppo, si è ritenuto opportuno sottoporla 
a un confronto con la proponente per una validazione definitiva durante il prossimo SAL$_G$.

\subsection*{Avvicinamento al \textit{Requirements and Technology Baseline}}
Durante la riunione è stato ufficializzato l'avvicinamento alla candidatura per 
il \textit{Requirements and Technology Baseline}. 
È stato pertanto deciso di pianificare un incontro con il prof.\ Cardin per 
essere sicuri delle ultime modifiche apportate all'Analisi dei Requisiti e di 
avviare la preparazione della presentazione relativa alle tecnologie adottate, 
da utilizzare in sede di candidatura all'RTB.

% -------------------------------------------------
% Decisioni
% -------------------------------------------------

\addDecision{Ridefinizione ruoli: doppio amministratore e doppio programmatore per lo sprint 5}
\addDecision{Richiedere un incontro straordinario con il prof. Cardin prima dell'RTB}
\addDecision{Esporre dubbi riguardanti la modellazione delle soglie e degli allarmi alla proponente nel prossimo SAL, 
previsto in data 21/01/2026}
\makeDecisionTable

% -------------------------------------------------
% Azioni
% -------------------------------------------------

\addTodo{-}{Richiedere incontro con prof. Cardin}{L. Pellizzon}{15/01/2026}
\addTodo{\#72}{Iniziare presentazione delle tecnologie per RTB}{C. Libralato, F. Pasqual, V. Baleanu}{27/01/2026}
\addTodo{\#73}{Scrittura verbale interno}{V.Baleanu}{20/01/2026}
\addTodo{\#74}{Aggiornamento cruscotto di qualità in PdQ}{C. Libralato, F. Pasqual}{25/01/2026}
\addTodo{\#75}{Dubbi sul PoC nel SAL del 21/01/2026}{C. Libralato, F. Pasqual, V. Baleanu}{21/01/2026}
\makeTodoTable

\end{document}
