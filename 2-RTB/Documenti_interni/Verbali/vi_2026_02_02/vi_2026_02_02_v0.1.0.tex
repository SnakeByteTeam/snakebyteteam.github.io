\documentclass[10pt, letterpaper]{article}

\usepackage{../../../../template/template}

\setDocTitle{Verbale interno 02/02/2026}
\setDocVersion{0.1.0}
\setDocIndex{vi\_2026\_02\_02}
\setDocData{02/02/2026}
\setDocState{1}
\setDocRecipient{1}

% comandi extra (rimuovere questi commenti una volta approvato il documento)
%
% -- costanti --
% \noOne [-]
% \noRole [ND]
% \firstDraft [Prima stesura]
% \approval [Approvazione]
% \teamName [SnakeByte]
% \teamEmail [snakebyteteam@gmail.com]
% \teamRepo [https://github.com/SnakeByteTeam/snakebyteteam.github.io]
% \teamSite [https://snakebyteteam.github.io/]
% \recipientTeachers [prof. Vardanega Tullio, prof. Cardin Riccardo]
% \vimar [Vimar]
% \vimarspa [Vimar S.p.A.]
%
% -- variabili --
% \docIndex | \setDocIndex{nome documento senza versione}
% \docTitle | \setDocTitle{titolo documento} 
% \docData | \setDocData{data documento} [GG/MM/AAAA]
% \docVersion | \setDocVersion{versione documento}
% \docState | \setDocState{0|1|2} [Da verificare][Verificato][Approvato]
% \docRecipient [table] | \setDocRecipient{0|1|2} [SnakeByte][SnakeByte, prof][SnakeByte, prof, vimar]
% 
% -- tabelle -- [gli id sono automatici dove serve]
% \makeMeetingInfoTable{data}{ora inizio}{ora fine}{modalità}
% \addParticipant{nome}{cognome}{ruolo}{presenza} | \makeMeetingParticipantsTable
% \addTodo{id github issue}{descrizione}{assegnatario}{scadenza} | \makeTodoTable
% \addPrecedentTodo{id todo}{id github issue todo}{assegnatario todo}{data} | \makePrecedentTodoTable
% \addDecision{descrizione decisione} | \makeDecisionTable
%
% -- pagine intere --
% \makeTitlePage
% \addChangelog{versione}{data}{autore}{verificatore}{approvatore}{descrizione} | \makeTodoTable

\begin{document}

\makeTitlePage

\newpage

\addChangelog{\docVersion}{\docData}{L. Pellizzon}{V. Baleanu}{\noOne}{\firstDraft}
\makeChangelog

\newpage

\tableofcontents

\newpage

\makeMeetingInfoTable{02/02/2026}{14:00}{15:00}{via \textit{Discord$_G$}}

\addParticipant{Giuseppe}{De Fina}{Amministratore}{P}
\addParticipant{Francesco}{Pasqual}{Progettista}{P}
\addParticipant{Leonardo}{Pellizzon}{Responsabile}{P}
\addParticipant{Valeria}{Baleanu}{Verificatore}{P}
\addParticipant{Filippo}{Venzo}{Amministratore}{P}
\addParticipant{Christian}{Libralato}{Programmatore}{P}
\addParticipant{Luca}{Granziero}{Programmatore}{P}
\makeMeetingParticipantsTable

\addPrecedentTodo{vi\_2025\_11\_17.a1}{\noOne}{F. Pasqual}{23/11/2025}
\addPrecedentTodo{vi\_2025\_11\_17.a2}{\#2}{G. De Fina, V. Baleanu}{23/11/2025}
\addPrecedentTodo{vi\_2025\_11\_17.a3}{\noOne}{G. De Fina}{19/11/2025}
\makePrecedentTodoTable

\section{Ordine del giorno}
\begin{itemize}
    \item Ispezione ad ampio spettro del documento Analisi dei Requisiti;
    \item rivisitazione dei ruoli dello sprint 7;
    \item stesura della presentazione per la candidatura alla revisione RTB;
\end{itemize}

\section{Approfondimento}

\subsection*{Ispezione ad ampio spettro del documento Analisi dei Requisiti}
Il gruppo ha eseguito un lavoro di \textit{walkthrough}, per quanto riguarda il documento Analisi dei Requisiti. In particolare sono state 
evidenziate alcune parti in cui la modellazione non è stata svolta correttamente.

\subsection*{Rivisitazione dei ruoli dello sprint 7}
A fronte di un evidente sovraccarico lavorativo subito dall'amministratore si è deciso di attuare una modifica al piano di assegnazione dei 
ruoli, in particolare durante lo sprint corrente, si è deciso di cambiare di ruolo G. De Fina che invece di essere progettista sarà amministratore.

\subsection*{Stesura della presentazione per la candidatura alla revisione RTB}
Il gruppo ha eseguito una prima stesura della presentazione per la candidatura alla revisione RTB che verrà presentata al prof. Cardin. La presentazione
conterrà una slide per ogni argomento:
\begin{itemize}
    \item una panoramica sull'obiettivo del capitolato;
    \item un resoconto delle tecnologie scelte;
    \item la lista dei casi d'uso che il \textit{PoC$_{G}$}.
\end{itemize}

\addDecision{Modifica del ruolo di G. De Fina da progettista ad amministratore}
\addDecision{La presentazione per la candidatura alla revisione RTB conterrà 3 }
\makeDecisionTable

\addTodo{\#86}{Stesura e verifica del verbale dell'incontro del 02/02/2026}{L. Pellizzon, L. Granziero}{12/02/2026}
\addTodo{\#72}{Terminare la presentazione per la candidatura RTB}{\teamName}{11/02/2026}

\makeTodoTable


\end{document}