\documentclass[10pt, letterpaper]{article}
\usepackage{../../../../template/template}

\setDocIndex{vi\_2025\_11\_07}
\setDocTitle{Verbale interno 07/11/2025}
\setDocData{08/11/2025}
\setDocVersion{0.1.1}
\setDocState{1}
\setDocRecipient{0}

\begin{document}

\makeTitlePage

\addChangelog{\docVersion}{\docData}{C. Libralato}{L. Pellizzon}{\noOne}{Correzione errori, aggiunta attività .a5}
\addChangelog{0.1.0}{07/11/2025}{C. Libralato}{L. Pellizzon}{\noOne}{\firstDraft}
\makeChangelog

\newpage
\tableofcontents

\newpage

\makeMeetingInfoTable{\docData}{9:30}{11:30}{via \textit{Discord$_G$}}

\addParticipant{Filippo}{Venzo}{Responsabile}{P}
\addParticipant{Christian}{Libralato}{Amministratore}{P}
\addParticipant{Leonardo}{Pellizzon}{Verificatore}{P}
\addParticipant{Valeria}{Baleanu}{Analista}{P}
\addParticipant{Giuseppe}{De Fina}{Analista}{P}
\addParticipant{Luca}{Granziero}{Analista}{P}
\addParticipant{Francesco}{Pasqual}{Analista}{P}

\makeMeetingParticipantsTable


\section{Ordine del giorno}
    \begin{itemize}
        \item Tracciamento attività svolte;
        \item nuove norme di gestione repository;
        \item allineamento prerequisiti per analisi dei requisiti;
        \item preparazione per diario di bordo;
        \item automazione nella redazione documenti.
    \end{itemize}


\section{Approfondimento}
    \subsection*{Tracciamento attività svolte}
    È stata discussa l'integrazione di una nuova tabella, all'interno dei verbali, volta al tracciamento delle attività svolte fino a quel punto. Si è giunti alla conclusione di crearla, includendo l'ID dell'attività, la \textit{GitHub$_G$} issue associata (se presente) e la data di completamento, per favorire un tracciamento indipendente dal repository gestionale di \textit{GitHub}. Inoltre non tutte le attività avranno una \textit{GitHub} issue associata, non trattandosi di azioni che vanno a modificare materiali presenti sul repository.\\ \\
    Tale tabella verrà integrata a partire dal prossimo verbale redatto.

    \subsection*{Nuove norme di gestione repository}
    È stata discussa la gestione del repository tenuta fin'ora, nella quale sono stati evidenziati, da parte di alcuni membri, una serie di problematiche tra cui la copia dell'immagine del logo del gruppo in ogni directory dei documenti e un non sufficiente utilizzo degli strumenti offerti da \textit{Git$_G$}, in particolare i branch.\\ \\
    La risoluzione di questi problemi è ancora in corso. 

    \subsection*{Allineamento prerequisiti per analisi dei requisiti}
    I membri del gruppo si sono allineati sulle conoscenze di base per lo svolgimento dell'attività di analisi dei requisiti. Le informazioni sono state ricavate dalle lezioni teoriche di Ingegneria del Software tenute dal prof. Vardanega nelle settimane precedenti.
    Sulla base di ciò gli analisti si impegneranno, in vista del primo colloquio con la Proponente, a stilare una serie di domande per definire un punto di partenza per strutturare l'analisi dei requisiti.

    \subsection*{Preparazione per diario di bordo}
    I membri del gruppo si sono confrontati e hanno concordato gli argomenti che andranno esposti al diario di bordo che si terrà in data 10/11/2025.

    \subsection*{Automazione nella redazione documenti}
    I membri del gruppo hanno concordato nell'adottare un template che permette di automatizzare una parte della redazione dei documenti. In particolare, utilizzando un  file \texttt{template.sty}\footnote{\url{https://it.overleaf.com/learn/latex/Understanding_packages_and_class_files} consultato il 7/11/2025}, è possibile rendere automatica la scrittura della parte dei documenti comune a tutti.


\addDecision{Creare tabella nei verbali per il tracciamento delle attività svolte}
\addDecision{Utilizzo \texttt{template.sty} per l'automazione parziale della redazione dei verbali}
\makeDecisionTable

\addTodo{\noOne}{Aggiungere alle Norme di Progetto la nuova tabella delle attività svolte}{F. Venzo}{10/11/2025}
\addTodo{\noOne}{Redazione diapositive per diario di bordo}{L. Pellizzon, C. Libralato}{09/11/2025}
\addTodo{\noOne}{Creazione di \texttt{template.sty}}{L. Pellizzon}{11/11/2025}
\addTodo{\noOne}{Individuare alternativa alla duplicazione del logo}{V. Baleanu}{14/11/2025}
\addTodo{\noOne}{Definizione delle domande di analisi dei requisiti per il primo colloquio}{V. Baleanu, G. De Fina, L. Granziero, F. Pasqual}{12/11/2025}

\makeTodoTable

\end{document}