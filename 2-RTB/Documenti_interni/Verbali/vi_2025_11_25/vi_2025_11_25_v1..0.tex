\documentclass[10pt, letterpaper]{article}

\usepackage{../../../../template/template}

\setDocTitle{Verbale interno 25/11/2025}
\setDocVersion{1.0.0}
\setDocIndex{vi\_2025\_11\_25}
\setDocData{10/02/2026}
\setDocState{2}
\setDocRecipient{0}

% comandi extra (rimuovere questi commenti una volta approvato il documento)
%
% -- costanti --
% \noOne [-]
% \noRole [ND]
% \firstDraft [Prima stesura]
% \approval [Approvazione]
% \teamName [SnakeByte]
% \teamEmail [snakebyteteam@gmail.com]
% \teamRepo [https://github.com/SnakeByteTeam/snakebyteteam.github.io]
% \teamSite [https://snakebyteteam.github.io/]
% \recipientTeachers [prof. Vardanega Tullio, prof. Cardin Riccardo]
% \vimar [Vimar]
% \vimarspa [Vimar S.p.A.]
%
% -- variabili --
% \docIndex | \setDocIndex{nome documento senza versione}
% \docTitle | \setDocTitle{titolo documento} 
% \docData | \setDocData{data documento} [GG/MM/AAAA]
% \docVersion | \setDocVersion{versione documento}
% \docState | \setDocState{0|1|2} [Da verificare][Verificato][Approvato]
% \docRecipient [table] | \setDocRecipient{0|1|2} [SnakeByte][SnakeByte, prof][SnakeByte, prof, vimar]
% 
% -- tabelle -- [gli id sono automatici dove serve]
% \makeMeetingInfoTable{data}{ora inizio}{ora fine}{modalità}
% \addParticipant{nome}{cognome}{ruolo}{presenza} | \makeMeetingParticipantsTable
% \addTodo{id github issue}{descrizione}{assegnatario}{scadenza} | \makeTodoTable
% \addPrecedentTodo{id todo}{id github issue todo}{assegnatario todo}{data} | \makePrecedentTodoTable
% \addDecision{descrizione decisione} | \makeDecisionTable
%
% -- pagine intere --
% \makeTitlePage
% \addChangelog{versione}{data}{autore}{verificatore}{approvatore}{descrizione} | \makeTodoTable

\begin{document}

\makeTitlePage

\newpage

\addChangelog{\docVersion}{\docData}{\noOne}{\noOne}{F. Venzo}{Approvazione}
\addChangelog{0.1.0}{25/11/2025}{F. Pasqual}{V. Baleanu}{\noOne}{\firstDraft}
\makeChangelog

\newpage

\tableofcontents

\newpage

\makeMeetingInfoTable{25/11/2025}{17:30}{19:00}{via \textit{Discord$_G$}}

\addParticipant{Giuseppe}{De Fina}{Responsabile}{P}
\addParticipant{Francesco}{Pasqual}{Amministratore}{P}
\addParticipant{Leonardo}{Pellizzon}{Progettista}{P}
\addParticipant{Valeria}{Baleanu}{Verificatore}{P}
\addParticipant{Filippo}{Venzo}{Analista}{P}
\addParticipant{Christian}{Libralato}{Analista}{P}
\addParticipant{Luca}{Granziero}{Analista}{P}
\makeMeetingParticipantsTable

\addPrecedentTodo{vi\_2025\_11\_17.a1}{\noOne}{F. Pasqual}{23/11/2025}
\addPrecedentTodo{vi\_2025\_11\_17.a2}{\#2}{G. De Fina, V. Baleanu}{23/11/2025}
\addPrecedentTodo{vi\_2025\_11\_17.a3}{\noOne}{G. De Fina}{19/11/2025}
\makePrecedentTodoTable

\section{Ordine del giorno}
\begin{itemize}
    \item Analisi dei requisiti: relazione \textit{use case$_G$} - requisiti;
    \item gestione dei requisiti opzionali;
    \item migrazione documentazione e \textit{repository} principale;
    \item sistema di \textit{project e task management};
    \item suddivisione dei compiti e preparazione presentazione per il \textit{SAL$_G$} del 27/11/2025;
    \item assegnazione del \textit{kit$_G$ hardware} bianco.
\end{itemize}

\section{Approfondimento}

\subsection*{Analisi dei requisiti: relazione \textit{use case} - requisiti}
Gli analisti hanno presentato la situazione attuale relativa all'individuazione degli \textit{use case}, sollevando questioni 
riguardo alla granularità, a eventuali duplicazioni e al livello di astrazione degli stessi. 
Il gruppo ha concordato che gli \textit{use case} devono essere astrazioni di alto livello e non includere dettagli implementativi o di \textit{UI},
mentre i dubbi sulla granularità permangono. \\
Prima di proseguire, per allinearsi nell'analisi, ogni analista dovrà produrre un \textit{draft$_G$}, anche grafico, 
della possibile \textit{user experience} nell'applicazione, indicando gli \textit{use case} che ne fanno parte e dove si presentano.

\subsection*{Gestione dei requisiti opzionali}
Il gruppo ha discusso riguardo alla considerazione dei requisiti opzionali nell'analisi dei requisiti
e nella generazione degli \textit{use case}. Tale decisione verrà presa dopo un confronto con il prof. Cardin e il responsabile di \vimarspa. \\
Inoltre, il gruppo concorda che verrà implementato il requisito opzionale riguardante l'aggiunta di un amministratore,
in modo da poter implementare un sistema di autenticazione amministrativa che sfrutti a pieno l'\textit{API KNX IoT 3rd-party$_G$} e una modalità di gestione dei permessi.

\subsection*{Migrazione documentazione e \textit{repository} principale}
Il gruppo ha deciso che la documentazione prodotta finora e contenuta nel \textit{repository} privato \textit{Sorgenti}
verrà spostata nel repository ufficiale del sito \teamRepo.
D'ora in avanti si lavorerà esclusivamente su quest'ultimo. \\
La migrazione dovrà tener conto del fatto che:
\begin{itemize}
    \item nel \textit{branch$_G$} \textit{main} dovrà comparire tutto ciò che riguarda la \textit{milestone$_G$} \textit{Candidatura};
    \item nel \textit{branch develop} dovrà comparire tutto ciò che riguarda la \textit{milestone} corrente. 
\end{itemize}

\subsection*{Sistema di \textit{project e task management}}
Il gruppo ha discusso riguardo la possibile adozione di \textit{Jira$_G$} per integrare il processo di creazione di \textit{task} e \textit{issue$_G$} finora realizzato
interamente su \textit{GitHub Projects$_G$}.
Si sono valutati pro e contro, tenendo in particolar modo conto della complessità del \textit{setup} di \textit{Jira}, della necessità di integrarlo con \textit{GitHub} e
della compatibilità con il sistema adottato per la verifica tramite \textit{pull request$_G$}.\\
A tal riguardo, il gruppo ha ancora pareri divisi e aspetta un confronto con il responsabile di progetto di \vimarspa. \\
La decisione è stata dunque posticipata.

\subsubsection*{Suddivisione dei compiti e preparazione presentazione per il \textit{SAL}}
In vista del \textit{SAL} del 27/11/2025, l'attuale responsabile G. De Fina è incaricato di creare una presentazione che mostri
gli avanzamenti del gruppo, difficoltà e dubbi incontrati nelle ultime due settimane. 
La struttura della presentazione dovrà rispettare quella concordata con \vimarspa nel primo incontro conoscitivo e predefinita 
nell'apposito \textit{template}.

\subsection*{Assegnazione del \textit{kit} bianco}
Dopo la notifica di \vimarspa della disponibilità di un secondo \textit{kit} di sensori di colore bianco,
il gruppo ha deciso che il responsabile della conservazione del \textit{kit} sarà F. Pasqual.



\addDecision{\textit{Use case} ad alto livello e indicazione chiara di requisiti obbligatori e opzionali}
\addDecision{Scelta del requisito opzionale sull'amministratore}
\addDecision{Migrazione dei documenti su \teamRepo, dismesso il \textit{repository} privato \textit{Sorgenti}}
\addDecision{Creazione di \textit{draft} di \textit{user experience} tramite \textit{use case}}
\addDecision{Il responsabile preparerà la presentazione per il \textit{SAL} del 27/11/2025}
\addDecision{F. Pasqual responsabile del secondo \textit{kit} di sensori}
\makeDecisionTable


\addTodo{\#1}{Stesura e verifica del verbale dell'incontro del 25/11/2025}{F. Pasqual, V. Baleanu}{27/11/2025}
\addTodo{\noOne}{Creazione \textit{draft} e diagrammi \textit{user experience} e \textit{use case}}{F. Venzo, C. Libralato, L. Granziero}{27/11/2025}
\addTodo{\#5}{\textit{Use case} per amministratore e relativa autenticazione}{F. Venzo, C. Libralato, L. Granziero}{30/11/2025}
\addTodo{\#6}{\textit{Use case} per gestione dei permessi}{F. Venzo, C. Libralato, L. Granziero}{30/11/2025}
\addTodo{\noOne}{Migrazione documentazione e \textit{repository}}{F. Pasqual}{27/11/2025}
\addTodo{\noOne}{Correzione dei percorsi dei file nel codice sorgente del sito}{C. Libralato}{29/11/2025}
\addTodo{\noOne}{Preparazione presentazione \textit{SAL}}{G. De Fina}{27/11/2025}
\addTodo{\noOne}{Test \textit{API KNX IoT 3rd-party $_G$}}{L. Pellizzon}{30/11/2025}

\makeTodoTable


\end{document}