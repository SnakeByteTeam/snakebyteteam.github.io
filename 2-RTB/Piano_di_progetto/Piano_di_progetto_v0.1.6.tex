\documentclass[10pt, letterpaper]{article}

\usepackage{../../template/template}
\usepackage{templatePdP}
\usepackage{spreadtab,simplekv,xpatch}

\usepackage[table]{xcolor}

% Macro sono in relazione all'ultima modifica
\setDocIndex{Piano di progetto}
\setDocTitle{Piano di progetto}
\setDocVersion{0.1.6}
\setDocState{1}
\setDocData{1/1/2026}
\setDocRecipient{1}

\setcounter{tocdepth}{4}
\setcounter{secnumdepth}{4}
\newcommand{\trisubsection}[1]{\paragraph{#1}\mbox{}\\}


\newcommand{\checkLevel}[1]{%
  \ifnum#1=0\relax
  \else\ifnum#1=1\relax
  \else\ifnum#1=2\relax
  \else
    \errmessage{Valore non valido (#1). Usare solo 0,1,2}%
  \fi\fi\fi
}

\newcommand{\convertLevelF}[1]{%
  \ifnum#1=0 Bassa\fi
  \ifnum#1=1 Media\fi
  \ifnum#1=2 Alta\fi
}
\newcommand{\convertLevelM}[1]{%
  \ifnum#1=0 Basso\fi
  \ifnum#1=1 Medio\fi
  \ifnum#1=2 Alto\fi
}

\newcounter{rtctr}
\newcommand{\newTRisk}[5]{
    \checkLevel{#4}
    \checkLevel{#5}
    \stepcounter{rtctr}
    \subsubsection{Rischio RT\thertctr : #1}
    \begin{center}
        \begin{tabularx}{0.9\textwidth}{|l|X|}%0.9 è un po meglio per l'indentazione
        \hline
        \rowcolor{lightgray}
        \textbf{Attributo} & \textbf{Valore} \\
        \hline
        Codice & RT\thertctr \label{RT\thertctr} \\
        \hline
        Nome & #1 \\
        \hline
        Descrizione & #2 \\
        \hline
        Mitigazione & #3 \\
        \hline
        Probabilità di occorrenza & \convertLevelF{#4} \\
        \hline
        Livello di impatto & \convertLevelM{#5} \\
        \hline
        \end{tabularx}
    \end{center}
}

\newcounter{rictr}
\newcommand{\newIRisk}[5]{
    \checkLevel{#4}
    \checkLevel{#5}
    \stepcounter{rictr}
    \subsubsection{Rischio RI\therictr : #1}
    \begin{center}
        \begin{tabularx}{0.9\textwidth}{ |l|X|}
        \hline
        \rowcolor{lightgray}
        \textbf{Attributo} & \textbf{Valore} \\
        \hline
        Codice & RI\therictr \label{RI\therictr} \\
        \hline
        Nome & #1 \\
        \hline
        Descrizione & #2 \\
        \hline
        Mitigazione & #3 \\
        \hline
        Probabilità di occorrenza & \convertLevelF{#4} \\
        \hline
        Livello di impatto & \convertLevelM{#5} \\
        \hline
        \end{tabularx}
    \end{center}
}

\newcounter{rcctr}
\newcommand{\newCRisk}[5]{
    \checkLevel{#4}
    \checkLevel{#5}
    \stepcounter{rcctr}
    \subsubsection{Rischio RC\thercctr : #1}
    \begin{center}
        \begin{tabularx}{0.9\textwidth}{|l|X|}
        \hline
        \rowcolor{lightgray}
        \textbf{Attributo} & \textbf{Valore} \\
        \hline
        Codice & RC\thercctr \label{RC\thercctr} \\
        \hline
        Nome & #1 \\
        \hline
        Descrizione & #2 \\
        \hline
        Mitigazione & #3 \\
        \hline
        Probabilità di occorrenza & \convertLevelF{#4} \\
        \hline
        Livello di impatto & \convertLevelM{#5} \\
        \hline
        \end{tabularx}
    \end{center}
}


\newcounter{roctr}
\newcommand{\newORisk}[5]{
    \checkLevel{#4}
    \checkLevel{#5}
    \stepcounter{roctr}
    \subsubsection{Rischio RO\theroctr : #1}
    \begin{center}
        \begin{tabularx}{0.9\textwidth}{|l|X|}
        \hline
        \rowcolor{lightgray}
        \textbf{Attributo} & \textbf{Valore} \\
        \hline
        Codice & RO\theroctr \label{RO\theroctr} \\
        \hline
        Nome & #1 \\
        \hline
        Descrizione & #2 \\
        \hline
        Mitigazione & #3 \\
        \hline
        Probabilità di occorrenza & \convertLevelF{#4} \\
        \hline
        Livello di impatto & \convertLevelM{#5} \\
        \hline
        \end{tabularx}
    \end{center}
}

\newcommand{\rif}[1]{\hyperref[#1]{§#1}}

\newcommand{\rotaz}{0}



% comandi extra (rimuovere questi commenti una volta approvato il documento)
%
% -- costanti --
% \noOne [-]
% \noRole [ND]
% \firstDraft [Prima stesura]
% \approval [Approvazione]
% \teamName [SnakeByte]
% \teamEmail [snakebyteteam@gmail.com]
% \teamRepo [https://github.com/SnakeByteTeam/snakebyteteam.github.io]
% \teamSite [https://snakebyteteam.github.io/]
% \recipientTeachers [prof. Vardanega Tullio, prof. Cardin Riccardo]
% \vimar [Vimar]
% \vimarspa [Vimar S.p.A.]
%
% -- variabili --
% \docIndex | \setDocIndex{nome documento senza versione}
% \docTitle | \setDocTitle{titolo documento} 
% \docData | \setDocData{data documento} [GG/MM/AAAA]
% \docVersion | \setDocVersion{versione documento}
% \docState | \setDocState{0|1|2} [Da verificare][Verificato][Approvato]
% \docRecipient [table] | \setDocRecipient{0|1|2} [SnakeByte][SnakeByte, prof][SnakeByte, prof, vimar]
%
% -- pagine intere --
% \makeTitlePage
% \addChangelog{versione}{data}{autore}{verificatore}{approvatore}{descrizione} | \makeTodoTable



% ============= STRUTTURA SPRINT GENERICO =============
\iffalse
%================================
\vspace{1em}
\hrule
\vspace{1em}

\subsection{SprintX (GG/MM/AAAA - GG/MM/AAAA)}
\subsubsection{Avanzamento}
\trisubsection{Attività programmate}
Si prevede l'esecuzione delle seguenti attività:
\begin{itemize}
    \item
\end{itemize}

\trisubsection{Attività svolte}
Nel corso del periodo di avanzamento sono state svolte le seguenti attività:
\begin{itemize}
    \item 
\end{itemize}

\subsubsection{Rischi}
\trisubsection{Rischi attesi}
\begin{itemize}
    \item 
\end{itemize}
\trisubsection{Rischi manifestati}
\begin{itemize}
    \item 
\end{itemize}

\subsubsection{Costi}
Le ore effettive svolte dai singoli componenti del gruppo sono di seguito riportate:

\setRuoloFilippo{\Anal}
\setRuoloChristian{\Anal}
\setRuoloValeria{\Verif}
\setRuoloGiuseppe{\Resp}
\setRuoloFrancesco{\Amm}
\setRuoloLuca{\Anal}
\setRuoloLeonardo{\Proge}

\setHPrevisteFilippo{0}
\setHPrevisteChristian{0}
\setHPrevisteValeria{0}
\setHPrevisteGiuseppe{0}
\setHPrevisteFrancesco{0}
\setHPrevisteLuca{0}
\setHPrevisteLeonardo{0}

\setHRealiFilippo{0}
\setHRealiChristian{0}
\setHRealiValeria{0}
\setHRealiGiuseppe{0}
\setHRealiFrancesco{0}
\setHRealiLuca{0}
\setHRealiLeonardo{0}

\makeTableRuoli
Si deriva il consuntivo delle ore e dei costi rispetto ai singoli ruoli:
%\makeTableDeviazioni

\subsubsection{Preventivo a finire}
Lo stato aggiornato delle risorse è calcolato nella seguente tabella, in cui i campi "iniziali" fanno riferimento all'inizio dello sprint corrente:

%\makeTablePreventivo
\fi

% ============== LEGGI =================
% Se le modifiche sulle tabelle non compaiono, elimina i file extra e fai rebuild

\begin{document}

\makeTitlePage

\addChangelog{0.1.6}{1/1/2026}{V. Baleanu}{G. De Fina}{\noOne}{Aggiunti parametri completi relativi allo Sprint 4}
\addChangelog{0.1.5}{29/12/2025}{G. De Fina}{F. Pasqual}{\noOne}{Modifica RO2 e piccole correzioni}
\addChangelog{0.1.4}{24/12/2025}{G. De Fina}{F. Pasqual}{\noOne}{Aggiunti Sprint 6, 7, 8, 9 e 10 con struttura completa}
\addChangelog{0.1.3}{23/12/2025}{G. De Fina}{F. Pasqual}{\noOne}{Redatti rischi di comunicazione e organizzativi con descrizioni e mitigazioni dettagliate}
\addChangelog{0.1.2}{21/12/2025}{G. De Fina}{F. Pasqual}{\noOne}{Aggiunti parametri completi relativi allo Sprint 3}
\addChangelog{0.1.1}{19/12/2025}{G. De Fina}{F. Pasqual}{\noOne}{Aggiunte attività previste e da svolgere, migliorate restrospettive e mitigazioni relative allo Sprint 1 e 2}
\addChangelog{0.1.0}{09/12/2025}{C. Libralato}{L. Pellizzon}{\noOne}{Prima stesura}
\makeChangelog

\newpage

\tableofcontents

\newpage

\section{Introduzione}
\subsection{Contenuto documento}
Il Piano di Progetto, nell'ambito dell'Ingegneria del Software, è un documento che ha lo scopo di definire in maniera chiara e dettagliata una struttura che funga 
da tabella di marcia per il progetto e da strumento di coordinazione per il team.

La suddetta struttura consiste in pianificazioni riguardo:
\begin{itemize}
    \item la dislocazione temporale dell'esecuzione delle attività di processo;
    \item i preventivi e le scadenze;
    \item la suddivisione di responsabilità dei ruoli;
    \item la gestione dei rischi;
    \item la gestione delle risorse;
\end{itemize}

\subsubsection{Struttura del contenuto}
Il documento è suddiviso in quattro macrosezioni:
\begin{enumerate}
    \item \textbf{Introduzione}: espone il contenuto e lo scopo del documento, oltre che i Riferimenti Normativi e Informativi.
    \item \textbf{Gestione rischi}: consiste in una classificazione e descrizione dei rischi che possono insorgere durante l'avanzamento del ciclo
    di vita del prodotto. Ad ogni rischio vengono associati comportamenti e regole sia per la sua prevenzione che per la mitigazione degli effetti in caso di occorrenza.
    \item \textbf{Previsioni a lungo termine}: presenta le stime iniziali eseguite in fase di Candidatura riguardo i costi, l'impegno orario e le responsabilità oltre che le stime future sugli
    stati di avanzamento previsti in vista delle scadenze quali RTB e PB.
    \item \textbf{Periodi di avanzamento}: elenco dei singoli periodi di avanzamento, definiti come \texorpdfstring{\textit{sprint$_G$}}{sprint} di durata pari a due settimane, di cui si analizzano i seguenti aspetti:
    \begin{itemize}
        \item Consuntivo di periodo:
        \begin{itemize}
            \item avanzamento previsto e avanzamento effettivo;
            \item rischi attesi, rischi occorsi e relativa gestione;
            \item costi attesi e costi effettivi.
        \end{itemize}
        \item Preventivo a finire:
        \begin{itemize}
            \item revisione calendario e risorse utilizzate;
            \item \texorpdfstring{\textit{sprint retrospective$_G$}}{sprint retrospective}.
        \end{itemize}
        
    \end{itemize}
\end{enumerate}


\subsection{Glossario}
All'interno del documento possono essere presenti termini il cui significato può risultare ambiguo o sconosciuto, essi sono
riportati in italico e alla loro prima occorrenza è associata una \textit{G} a pedice ad indicare la presenza del suddetto termine all'interno del \texorpdfstring{\textit{glossario$_G$}}{glossario}, che è disponibile al seguente link: \href{https://snakebyteteam.github.io/glossary.html}{Glossario}.

\subsection{Riferimenti Normativi}
\begin{itemize}
    \item \textbf{Norme di Progetto}\\
    \url{link-finale-norme-di-progetto}
\end{itemize}

\subsection{Riferimenti Formativi}
\begin{itemize}
    \item \textbf{Regolamento del progetto didattico}\\
    \url{https://www.math.unipd.it/~tullio/IS-1/2025/Dispense/PD1.pdf}
    \item \textbf{Gestione di progetto}\\
    \url{https://www.math.unipd.it/~tullio/IS-1/2025/Dispense/T04.pdf}
\end{itemize}



\section{Analisi e gestione rischi}
    Nella seguente sezione vengono analizzati i possibili rischi che potrebbero emergere durante il progetto e definite le modalità 
    per ridurne l'impatto (mitigazione). Sono state individuate quattro tipologie principali di rischio:
    \begin{itemize}
        \item \textbf{rischi tecnologici}: sono legati alla scelta e all'utilizzo delle tecnologie adottate durante lo sviluppo;
        \item \textbf{rischi individuali}: emergono da imprevisti della vita quotidiana o impegni personali;
        \item \textbf{rischi di comunicazione}: sono causati da una mancanza di comunicazione tra le figure coinvolte nel progetto, quali membri del gruppo, Proponente e professori;
        \item \textbf{rischi organizzativi}: sono legati ad un'organizzazione errata o non adatta allo svolgimento di un'attività.
    \end{itemize}
    

    \subsection{Rischi tecnologici}
        \newTRisk{Conoscenza insufficiente di una tecnologia}
        {Il progetto richiede l'utilizzo di molte tecnologie diverse che risultano spesso sconosciute
        a causa dell'inesperienza e dell'ampia offerta di strumenti nell'ambito.}
        {Dedicare ore di calendario allo studio individuale di una tecnologia e, in caso di componenti 
        del gruppo con conoscenze pregresse sull'argomento, organizzare dei momenti di allineamento
        generale.}
        {2}{2}

        \newTRisk{Incompatibilità tecnologie}
        {La scelta delle tecnologie in fase di progettazione, se non effettuata con cura, può provocare problemi
        di compatibilità in fase di sviluppo, in particolare durante la realizzazione del \textbf{Proof of Concept} (PoC$_G$).}
        {È necessario interrompere lo sviluppo delle parti che coinvolgono le tecnologie in conflitto e rielaborare la scelta progettuale 
        in maniera più attenta e motivata.}
        {0}{2} % 0 | 1 | 2

        \newTRisk{\textit{Bug$_G$} nel codice}
        {Nonostante l'utilizzo di metodi automatizzati di verifica del codice si può presentare
        un comportamento inatteso o errato dell'applicativo.}
        {Individuare la parte di codice che genera il \textit{bug} attraverso analisi del codice 
        e strumenti come \textit{debugger$_G$}. Modificare il codice per correggere il comportamento
        fallace. Verificare occorrenza del rischio \rif{RT4}.}
        {2}{1}

        \newTRisk{Test non esaustivi}
        {Casistica in cui i test automatici del codice risultano essere non esaustivi o insufficienti al
        fine di produrre un codice funzionante, robusto e manutenibile, secondo lo stato dell'arte.}
        {Effettuare un controllo dell'esaustività, della correttezza e del livello di copertura del codice dei 
        test automatici; su necessità aggiungere test o modificare quelli esistenti.}
        {1}{1}
    
    \subsection{Rischi individuali}
        \newIRisk{Impegni personali}
        {I singoli componenti del gruppo possono essere indisponibili allo svolgimento delle attività
        assegnate per un certo periodo di tempo a causa di motivi personali.}
        {Comunicare l'indisponibilità al gruppo il prima possibile, organizzare una riunione urgente per 
        effettuare una ripianificazione delle attività di periodo. In base al momento d'occorrenza, alle 
        attività in sospeso e al carico di lavoro dei componenti, decidere tra le opzioni:
        \begin{itemize}
            \item redistribuire le attività in sospeso tra i componenti;
            \item rimandare lo svolgimento delle attività in sospeso al periodo di avanzamento successivo.
        \end{itemize}}
        {0}{1}

        \newIRisk{Impegni universitari}
        {Gli impegni universitari possono ridurre il tempo disponibile per lo svolgimento delle attività,
        specialmente nel periodo della sessione, avendo effetto contemporaneamente sull'intero gruppo e
        dunque rallentando l'avanzamento.}
        {Gestire l'organizzazione del team e delle attività considerando questo rischio anticipatamente e assegnando un carico di
        lavoro appropriato per evitare cambi di programma durante il periodo d'interesse.}
        {2}{2}
    
    \subsection{Rischi di comunicazione}
        \newCRisk{Mancanza di coordinamento tra componenti}
        {La collaborazione tra i membri del gruppo può risultare inefficace a causa di una comunicazione 
        insufficiente o poco chiara, portando a duplicazione di lavoro, incomprensioni sugli obiettivi 
        o ritardi nelle attività.}
        {Stabilire canali di comunicazione chiari e utilizzarli regolarmente. Organizzare riunioni periodiche 
        di allineamento per verificare lo stato di avanzamento delle attività. Utilizzare strumenti di 
        tracciamento condivisi (\textit{GitHub Projects$_G$}) per mantenere visibilità sulle attività in corso. 
        Documentare decisioni importanti nei verbali per garantire che tutti i membri abbiano accesso alle 
        informazioni necessarie.}
        {1}{2}
        
        \newCRisk{Divergenza tra Proponente e gruppo}
        {Possono emergere incomprensioni o divergenze di visione tra il gruppo e la Proponente riguardo 
        requisiti, priorità o direzione del progetto, causando rilavorazioni o sviluppo di funzionalità 
        non conformi alle aspettative.}
        {Organizzare incontri frequenti con la Proponente (\textit{SAL$_G$} ogni due settimane) per presentare l'avanzamento 
        e raccogliere feedback continuo. Inviare documentazione intermedia (bozze) per validazione anticipata. 
        Mantenere un canale di comunicazione diretto per chiarimenti rapidi. Documentare e confermare per 
        iscritto le decisioni prese durante gli incontri tramite verbali esterni.}
        {1}{2}
        
        \newCRisk{Disparit\`a di conoscenze tra componenti}
        {I membri del gruppo possono avere livelli di conoscenza diversi su tecnologie, metodologie o 
        domini specifici del progetto, causando difficoltà nella collaborazione e nella produzione di 
        output omogenei.}
        {Organizzare sessioni di studio condiviso e momenti di allineamento quando un componente acquisisce 
        nuove competenze. Favorire la condivisione di conoscenze attraverso documentazione interna e 
        spiegazioni durante le riunioni. Assegnare compiti che permettano ai membri meno esperti di apprendere 
        affiancati da chi ha maggiore esperienza. Dedicare tempo alla revisione collaborativa del lavoro per 
        uniformare approcci e terminologia.}
        {2}{1}
        
        \newCRisk{Incomprensione requisiti}
        {I requisiti del progetto possono essere interpretati in modo errato o incompleto, portando allo 
        sviluppo di funzionalità non richieste o non conformi alle aspettative della Proponente.}
        {Analizzare approfonditamente il capitolato e preparare domande specifiche per la Proponente. 
        Validare la comprensione dei requisiti attraverso la redazione di casi d'uso dettagliati e diagrammi, 
        da sottoporre alla Proponente per conferma. In caso di dubbi, richiedere chiarimenti tempestivamente 
        senza procedere con assunzioni. Coinvolgere i docenti per validazione metodologica dell'Analisi dei 
        Requisiti.}
        {1}{2}

    \subsection{Rischi organizzativi}
        \newORisk{Carico di lavoro eccessivo}
        {Un membro del gruppo può trovarsi sovraccarico di attività a causa di una distribuzione non 
        equilibrata del lavoro o di imprevisti, causando ritardi e potenziale deterioramento della qualità 
        del lavoro prodotto.}
        {Monitorare regolarmente il carico di lavoro di ciascun membro durante le riunioni interne. 
        Redistribuire le attività in modo più equilibrato quando emergono situazioni di sovraccarico. 
        Favorire la comunicazione precoce di difficoltà nel rispettare le scadenze, permettendo 
        ripianificazioni tempestive. Mantenere un buffer di tempo nelle pianificazioni per gestire imprevisti.}
        {1}{2}
        
        \newORisk{Non conformità ai requisiti della \textit{baseline$_G$}}
        {Il gruppo potrebbe non soddisfare pienamente i livelli di qualità e completezza richiesti per la 
        baseline RTB o PB, producendo documenti o prodotti non sufficientemente dettagliati, verificati o 
        conformi agli standard richiesti, compromettendo l'approvazione della candidatura.}
        {Definire chiaramente i criteri di qualità e completezza per ciascun documento e prodotto della baseline. 
        Effettuare revisioni interne frequenti durante lo sviluppo per verificare la conformità agli standard. 
        Allocare tempo sufficiente per cicli di verifica e correzione. Consultare i docenti per validazione 
        metodologica e confronto con le aspettative. Utilizzare checklist di completezza per ogni \textit{deliverable$_G$} 
        della baseline prima della candidatura.}
        {0}{2}
        
        \newORisk{Inadempimento scadenza \textit{milestone$_G$} interne}
        {Il gruppo può non rispettare le milestone intermedie definite internamente per lo \textit{sprint$_G$}, 
        causando accumulo di lavoro arretrato e pressione crescente verso le scadenze finali.}
        {Definire milestone realistiche basate sull'esperienza degli sprint precedenti. Monitorare 
        quotidianamente l'avanzamento tramite lo strumento di tracciamento (\textit{GitHub Projects$_G$}). 
        Organizzare riunioni di verifica intermedie durante lo sprint per identificare tempestivamente 
        ritardi. Ripianificare le attività rimanenti se necessario, spostando quelle meno prioritarie allo 
        sprint successivo.}
        {2}{1}
        
        \newORisk{Inadempimento scadenza Proponente}
        {Il gruppo potrebbe non rispettare le scadenze concordate con la Proponente per consegne intermedie 
        (bozze di documenti, SAL, demo), compromettendo il rapporto di fiducia e la possibilità di ricevere 
        feedback tempestivo.}
        {Concordare scadenze realistiche con la Proponente considerando il carico di lavoro e altre priorità. 
        Pianificare il completamento delle attività con anticipo rispetto alla scadenza esterna per avere 
        margine per imprevisti. Comunicare tempestivamente alla Proponente eventuali difficoltà nel 
        rispettare una scadenza, proponendo alternative. Assegnare alta priorità alle attività con scadenze 
        esterne.}
        {1}{2}
        
        \newORisk{Inadempimento scadenza \textit{issue$_G$}}
        {Le singole \textit{issue} assegnate ai membri del gruppo possono non essere completate entro 
        i tempi previsti, causando ritardi nell'avanzamento complessivo dello sprint.}
        {Assegnare \textit{issue} con stime realistiche basate sulla complessità dell'attività. Monitorare 
        lo stato delle \textit{issue} durante lo sprint e intervenire tempestivamente in caso di blocchi. 
        Favorire la comunicazione immediata di difficoltà da parte dei membri senza timore di ripercussioni. 
        Ridistribuire le \textit{issue} tra i membri del gruppo se necessario per bilanciare il carico di 
        lavoro. Suddividere \textit{issue} particolarmente complesse in sotto-attività più gestibili per 
        facilitarne il completamento.}
        {2}{1}
        
        \newORisk{Divergenza tra fase di sviluppo e progettazione}
        {Durante lo sviluppo possono emergere problematiche che rendono la progettazione iniziale inadeguata 
        o irrealizzabile, richiedendo modifiche architetturali significative con conseguente perdita di 
        tempo e risorse.}
        {Realizzare un \textit{Proof of Concept$_G$} completo prima della progettazione definitiva per 
        validare le scelte tecnologiche e architetturali. Mantenere la progettazione sufficientemente 
        flessibile da accogliere modifiche. Documentare le decisioni progettuali e le motivazioni per 
        facilitare eventuali revisioni. Prevedere momenti di validazione della progettazione con sviluppo 
        di prototipi prima di procedere con l'implementazione completa.}
        {1}{2}
        
        \newORisk{Sovrastima costi e ore}
        {Le ore preventivate per le attività possono risultare eccessive rispetto a quelle effettivamente 
        necessarie, portando a uno spreco di risorse allocate e a una pianificazione inefficiente degli 
        sprint successivi.}
        {Analizzare sistematicamente i consuntivi degli sprint precedenti per identificare pattern di 
        sovrastima. Raffinare progressivamente le stime basandosi sui dati storici raccolti. Considerare 
        la crescente esperienza del team nelle tecnologie e nei processi quando si pianificano attività 
        simili a quelle già svolte. Evitare di sovrastimare sistematicamente per "sicurezza", puntando 
        invece a stime realistiche.}
        {2}{2}
        
        \newORisk{Sottostima costi e ore}
        {Le ore preventivate possono risultare insufficienti rispetto alla complessità effettiva delle 
        attività, causando ritardi, accumulo di lavoro arretrato e potenziale superamento del budget 
        complessivo del progetto.}
        {Analizzare attentamente la complessità delle attività prima di stimarle, considerando eventuali 
        dipendenze e rischi. Includere nelle stime il tempo per verifica, correzioni e documentazione, 
        non solo l'implementazione. Prevedere buffer di tempo per attività nuove o particolarmente 
        complesse. Monitorare costantemente il rapporto tra ore preventivate e reali, intervenendo con 
        ripianificazioni tempestive quando si rilevano sottostime significative. Utilizzare i dati degli 
        sprint precedenti per calibrare meglio le stime future.}
        {2}{2}


\section{Previsioni a lungo termine}
Durante la fase di pianificazione di progetto, in vista della candidatura, sono state effettuate delle stime riguardo la distribuzione
oraria per ogni ruolo e il costo associato, da cui si è poi ricavato il costo complessivo del progetto. 

All'interno di questo documento è possibile confrontare queste stime con il consumo effettivo delle risorse nei
singoli periodi di avanzamento.\
Dal documento Dichiarazione degli Impegni si riporta:
\begin{center}
    \begin{tabular}{|c|c|c|c|c|}
        \hline
        \rowcolor{lightgray} \textbf{Ruolo} & \textbf{Costo orario} & \textbf{Ore totali} & \textbf{Costo totale}
        \\
        \hline
        Responsabile & 30,00 \euro/h & 78 & 2.340,00\euro \\
        \hline
        Amministratore & 20,00\euro/h & 76 & 1520,00\euro\\
        \hline
        Analista & 25,00\euro/h & 86 & 2.150,00\euro \\
        \hline
        Progettista & 25,00\euro/h & 111 & 2.775,00\euro \\
        \hline
        Programmatore & 15,00\euro/h & 142 & 2.130,00\euro\\
        \hline
        Verificatore & 15,00\euro/h & 142 & 2.130,00\euro \\
        \hline
        \cellcolor{lightgray}{\textbf{Totale}} & - & 635 & 13.045,00\euro \\
        \hline
    \end{tabular}
\end{center}
Adottando la rotazione dei ruoli all'inizio di ogni ciclo di avanzamento, ossia ogni 2 settimane, e con
preventivo dei costi che ammonta ad un totale di 13.045,00 \euro, è stata fissata come scadenza ultima di consegna del 
progetto la data 18 marzo 2026.

\subsection{RTB} 
In vista della Requirement and Technology Baseline (RTB), il gruppo SnakeByte fissa come data prevista per la candidatura il
GG/MM/AAAA (?); termine entro il quale si prevede il completamento delle seguenti attività: 
\begin{itemize}
    \item stesura dell'Analisi Requisiti (AdR);
    \item stesura delle Norme di Progetto (NdP);
    \item stesura del Piano di Progetto (PdP);
    \item stesura del Piano di Qualifica (PdQ);
    \item scrittura e consegna Proof of Concept (PoC);
\end{itemize}
Al momento della candidatura, avente luogo durante lo Sprint 6 (?), in base alle previsioni sulla 
distribuzione oraria dei ruoli, si stima uno stato delle risorse corrispondente a quanto riportato:
\begin{center}
    \begin{tabular}{|c|c|c|}
        \hline
        \rowcolor{lightgray} \textbf{Ruolo} & \textbf{Ore rimanenti} & \textbf{Risorse rimanenti} 
        \\
        \hline
        Responsabile  & 0 & 0\euro\\
        \hline
        Amministratore  & 0 & 0\euro\\
        \hline
        Analista & 0 & 0\euro\\
        \hline
        Progettista & 0 & 0\euro \\
        \hline
        Programmatore & 0 & 0\euro\\
        \hline
        Verificatore & 0 & 0\euro\\
        \hline
        \cellcolor{lightgray}{\textbf{Totale}} & - & - \euro \\
        \hline
    \end{tabular}
\end{center}

\subsection{PB}
DA COMPILARE IN DETTAGLIO A SEGUITO DELLA RTB

A questo punto le uniche previsioni sono:
\begin{itemize}
    \item Consegna MVP alla Proponente;
    \item Candidatura al PB.
\end{itemize}


\section{Periodi di avanzamento}
\subsection{Pre-candidatura}
\subsubsection{Avanzamento}
Durante il periodo precedente alla candidatura il gruppo ha portato a termine le seguenti attività:
\begin{itemize}
    \item definizione del nome e del logo del gruppo;
    \item scelta dei canali di comunicazione e della frequenza degli incontri;
    \item analisi dei capitolati dei progetti di maggior interesse;
    \item creazione del repository e del sito web associato al progetto;
    \item prima stesura delle Norme di Progetto;
    \item prima definizione dei template per i documenti;
    \item contatto con le aziende per chiarire dubbi sui capitolati;
    \item stesura della documento di valutazione dei capitolati;
    \item stesura della Dichiarazione degli Impegni;
    \item stesura della Lettera di Candidatura.
\end{itemize}

\subsubsection{Rischi}
Si è manifestato il rischio \rif{RT1}, riferito allo strumento utilizzato per la redazione
dei documenti, ossia \LaTeX.

\subsubsection{Retrospettiva}
In questa prima fase di assestamento lo svolgimento delle attività è stato gestito
in maniera ancora incerta ma funzionale, senza una divisione formale dei ruoli, ed è stata caratterizzata da 
decisioni che hanno richiesto molta riflessione, tra cui le convenzioni per il versionamento dei file e la struttura dei documenti.


\subsection{Sprint 1 (03/11/2025 - 16/11/2025)}
\subsubsection{Avanzamento}
\trisubsection{Attività programmate}
All'inizio dello Sprint 1 sono state pianificate le seguenti attività:
\begin{itemize}
    \item proseguimento della stesura delle Norme di Progetto;
    \item organizzazione del primo incontro con la Proponente per allineamento iniziale sul progetto;
    \item avvio dell'attività di analisi e identificazione dei requisiti;
    \item approfondimento tecnico sulle tecnologie proposte;
    \item definizione delle prime domande da sottoporre alla Proponente nell'ambito dell'Analisi dei Requisiti;
    \item miglioramento della struttura documentale attraverso l'ottimizzazione dei template.
\end{itemize}

\trisubsection{Attività svolte}
Durante lo Sprint 1 il gruppo ha completato tutte le attività programmate:
\begin{itemize}
    \item \textbf{Norme di Progetto}: prosecuzione della redazione con particolare attenzione alle convenzioni documentali e ai processi organizzativi;
    \item \textbf{Incontro con la Proponente}: primo meeting formale per discutere gli obiettivi di progetto e chiarire le aspettative;
    \item \textbf{Analisi dei Requisiti}: avvio dell'identificazione dei casi d'uso e dei requisiti funzionali;
    \item \textbf{Approfondimento tecnologico}: sessione di studio con la Proponente sulle tecnologie da utilizzare per l'implementazione;
    \item \textbf{Template documentali}: perfezionamento della struttura dei documenti con creazione di template riutilizzabili.
\end{itemize}

\subsubsection{Rischi}
\trisubsection{Rischi attesi}
All'inizio dello sprint erano previsti i seguenti rischi:
\begin{itemize}
    \item \rif{RT1} - Conoscenza insufficiente di una tecnologia: rischio legato all'apprendimento di nuovi strumenti e tecnologie;
    \item \rif{RI2} - Impegni universitari: possibile interferenza con altri esami e attività accademiche;
    \item \rif{RO7} - Sovrastima costi e ore: difficoltà nella stima accurata delle ore necessarie per le attività.
\end{itemize}

\trisubsection{Rischi manifestati}
Durante lo Sprint 1 si sono verificati i seguenti rischi:
\begin{itemize}
    \item \rif{RO7} - Sovrastima costi e ore: è stata riscontrata una sovrastima nell'assegnazione delle ore per i ruoli di Verificatore, Responsabile e Amministratore. 
    \\La mitigazione è avvenuta attraverso una ripianificazione più accurata per gli sprint successivi, basata sull'esperienza acquisita.
\end{itemize}

\subsubsection{Costi}
Le ore effettive svolte dai singoli componenti del gruppo sono di seguito riportate:

% Inizializzazione NECESSARIA e SUFFICIENTE per la costruzione delle tabelle successive
\setRuoloFilippo{\Resp}
\setRuoloChristian{\Amm}
\setRuoloValeria{\Anal}
\setRuoloGiuseppe{\Anal}
\setRuoloFrancesco{\Anal}
\setRuoloLuca{\Anal}
\setRuoloLeonardo{\Verif}

\setHPrevisteFilippo{8}
\setHPrevisteChristian{9}
\setHPrevisteValeria{5}
\setHPrevisteGiuseppe{5}
\setHPrevisteFrancesco{5}
\setHPrevisteLuca{5}
\setHPrevisteLeonardo{8}

\setHRealiFilippo{6}
\setHRealiChristian{6}
\setHRealiValeria{5}
\setHRealiGiuseppe{5}
\setHRealiFrancesco{5}
\setHRealiLuca{5}
\setHRealiLeonardo{6}

\makeTableRuoli
Si deriva il consuntivo delle ore e dei costi rispetto ai singoli ruoli:
\makeTableDeviazioni

\subsubsection{Preventivo a finire}
Lo stato aggiornato delle risorse è calcolato nella seguente tabella, in cui i campi "iniziali" fanno riferimento all'inizio dello sprint corrente:

\makeTablePreventivo

\subsubsection{Retrospettiva}
Lo Sprint 1 ha rappresentato l'avvio operativo del progetto con l'applicazione concreta delle metodologie definite durante la Candidatura. 

Il gruppo ha completato tutte le attività programmate, dimostrando buona capacità di adattamento ai nuovi strumenti e processi. L'incontro con la Proponente ha permesso di chiarire aspetti fondamentali del progetto (soprattutto pratici) e stabilire una base solida per la collaborazione e comunicazione future.

L'analisi dei dati ha evidenziato una leggera sovrastima delle ore per i ruoli di Verificatore (-2 ore), Responsabile (-2 ore) e Amministratore (-3 ore), per un totale di 7 ore di scostamento negativo. Questo è attribuibile all'inesperienza nella pianificazione e alla fase di assestamento iniziale. Si nota tuttavia che tale sovrastima è relativamente contenuta rispetto a quanto si verificherà negli sprint successivi.

Per i prossimi sprint, il gruppo intende:
\begin{itemize}
    \item utilizzare i dati raccolti per migliorare le stime;
    \item monitorare più frequentemente l'avanzamento delle attività;
    \item calibrare meglio l'assegnazione delle ore.
\end{itemize}

%================================
\vspace{1em}
\hrule
\vspace{1em}

\subsection{Sprint 2 (17/11/2025 - 30/11/2025)}
\subsubsection{Avanzamento}
\trisubsection{Attività programmate}
All'inizio dello Sprint 2 sono state pianificate le seguenti attività:
\begin{itemize}
    \item proseguimento della stesura dell'Analisi dei Requisiti con definizione dei casi d'uso;
    \item definizione e implementazione del \textit{workflow$_G$} di versionamento tramite \textit{Git$_G$} con introduzione dei \textit{branch$_G$} \textit{feature}, \textit{develop} e \textit{release};
    \item implementazione di un processo di revisione sistematica tramite \textit{pull request$_G$};
    \item aggiornamento delle Norme di Progetto con nuova struttura per il tracciamento delle attività;
    \item valutazione e scelta della \textit{dashboard$_G$} per il tracciamento delle attività del gruppo;
    \item preparazione delle diapositive per il diario di bordo e il SAL.
\end{itemize}

\trisubsection{Attività svolte}
Durante lo Sprint 2 il gruppo ha completato le seguenti attività:
\begin{itemize}
    \item \textbf{Analisi dei Requisiti}: aggiornamento alla versione 0.1.2 con aggiunta dei casi d'uso fino a UC4 e UC4.1;
    \item \textbf{\textit{Workflow} di versionamento}: definizione e implementazione del processo di versionamento con creazione dei \textit{branch} \textit{develop} e \textit{release};
    \item \textbf{Processo di revisione}: implementazione di un sistema di revisione tramite \textit{pull request} con commenti in linea;
    \item \textbf{Norme di Progetto}: aggiornamento alla versione 0.1.5 con nuova struttura della tabella delle attività completate;
    \item \textbf{Verbali}: redazione del verbale interno del 17/11/2025 e del 25/11/2025;
    \item \textbf{Responsabilità operative}: definizione chiara delle responsabilità per revisione documenti e controllo qualità;
    \item \textbf{Presentazioni}: redazione delle diapositive per il diario di bordo del 24/11/2025 e preparazione del SAL del 27/11/2025.
\end{itemize}

\subsubsection{Rischi}
\trisubsection{Rischi attesi}
All'inizio dello sprint erano previsti i seguenti rischi:
\begin{itemize}
    \item \rif{RT1} - Conoscenza insufficiente di una tecnologia: rischio legato all'adozione di nuovi strumenti come \textit{Git} e sistemi di tracciamento;
    \item \rif{RI2} - Impegni universitari: possibile interferenza con esami e altre attività accademiche;
    \item \rif{RO7} - Sovrastima costi e ore: persistenza delle difficoltà di stima già riscontrate nello Sprint 1.
\end{itemize}

\trisubsection{Rischi manifestati}
Durante lo Sprint 2 si sono verificati i seguenti rischi:
\begin{itemize}
    \item \rif{RT1} - Conoscenza insufficiente di una tecnologia: il gruppo ha affrontato una curva di apprendimento significativa nell'adozione del nuovo \textit{workflow} \textit{Git} con gestione dei conflitti e delle \textit{pull request}. 
    \\La mitigazione è avvenuta attraverso sessioni di studio condiviso e documentazione nelle Norme di Progetto.
    \item \rif{RO7} - Sovrastima costi e ore: è stata riscontrata una lieve sovrastima per il ruolo di Analista (Christian: -3 ore). 
    \\La pianificazione sta gradualmente migliorando grazie all'esperienza acquisita.
\end{itemize}

\subsubsection{Costi}
Le ore effettive svolte dai singoli componenti del gruppo sono di seguito riportate:

\setRuoloFilippo{\Anal}
\setRuoloChristian{\Anal}
\setRuoloValeria{\Verif}
\setRuoloGiuseppe{\Resp}
\setRuoloFrancesco{\Amm}
\setRuoloLuca{\Anal}
\setRuoloLeonardo{\Proge}

\setHPrevisteFilippo{9}
\setHPrevisteChristian{9}
\setHPrevisteValeria{9}
\setHPrevisteGiuseppe{8}
\setHPrevisteFrancesco{8}
\setHPrevisteLuca{9}
\setHPrevisteLeonardo{12}

\setHRealiFilippo{9}
\setHRealiChristian{6}
\setHRealiValeria{9}
\setHRealiGiuseppe{8}
\setHRealiFrancesco{8}
\setHRealiLuca{9}
\setHRealiLeonardo{12}

\makeTableRuoli
Si deriva il consuntivo delle ore e dei costi rispetto ai singoli ruoli:
\makeTableDeviazioni

\subsubsection{Preventivo a finire}
Lo stato aggiornato delle risorse è calcolato nella seguente tabella, in cui i campi "iniziali" fanno riferimento all'inizio dello sprint corrente:

\makeTablePreventivo

\subsubsection{Retrospettiva}
Lo Sprint 2 ha segnato un importante passo avanti nell'organizzazione del gruppo con l'introduzione del \textit{workflow} \textit{Git} e del sistema di \textit{pull request}, che hanno migliorato la tracciabilità delle modifiche e il processo di revisione.

L'Analisi dei Requisiti è progredita con l'aggiunta dei primi casi d'uso, mostrando una complessità maggiore del previsto. La collaborazione tra gli analisti ha garantito coerenza nella documentazione.

La sovrastima delle ore è stata limitata a un solo componente (-3 ore), indicando un miglioramento nella capacità di pianificazione rispetto allo Sprint 1. Permangono criticità nella gestione dei conflitti \textit{Git} e nell'identificazione dei requisiti non funzionali, che saranno affrontate negli sprint successivi.


%================================
\vspace{1em}
\hrule
\vspace{1em}

\subsection{Sprint 3 (01/12/2025 - 14/12/2025)}
\subsubsection{Avanzamento}
\trisubsection{Attività programmate}
All'inizio dello Sprint 3 sono state pianificate le seguenti attività:
\begin{itemize}
    \item completamento degli \textit{use case$_G$} per amministratore e gestione permessi;
    \item proseguimento dell'Analisi dei Requisiti con redazione e invio della bozza del documento alla Proponente;
    \item test dell'\textit{API$_G$} \textit{KNX IoT 3rd party$_G$} e correzione dell'implementazione dell'autenticazione;
    \item decisione definitiva sul sistema di tracciamento delle attività di progetto;
    \item aggiornamento delle Norme di Progetto con sezione Sviluppo e Analisi dei Requisiti;
    \item configurazione del secondo \textit{kit$_G$} hardware.
\end{itemize}

\trisubsection{Attività svolte}
Durante lo Sprint 3 il gruppo ha completato le seguenti attività:
\begin{itemize}
    \item \textbf{Analisi dei Requisiti}: redazione e invio della bozza del documento alla Proponente per revisione; completamento degli \textit{use case} relativi a cruscotto, autenticazione, amministratore e gestione permessi;
    \item \textbf{Norme di Progetto}: aggiornamento alla versione 0.2.0 con aggiunta della sezione Sviluppo e della sezione Analisi dei Requisiti;
    \item \textbf{Sistema di tracciamento}: decisione finale di adottare \textit{GitHub Projects$_G$} per il tracciamento delle attività di progetto;
    \item \textbf{Verbali}: redazione del verbale interno del 01/12/2025 e del verbale esterno relativo al SAL del 27/11/2025;
    \item \textbf{Presentazioni}: preparazione e presentazione del SAL del 10/12/2025 con la Proponente;
    \item \textbf{Hardware}: ricezione e avvio della configurazione del secondo kit hardware;
    \item \textbf{Test tecnologici}: \textit{debugging$_G$} dell'autenticazione con \textit{KNX IoT 3rd party API}.
\end{itemize}

\subsubsection{Rischi}
\trisubsection{Rischi attesi}
All'inizio dello sprint erano previsti i seguenti rischi:
\begin{itemize}
    \item \rif{RT1} - Conoscenza insufficiente di una tecnologia: rischio legato all'apprendimento delle \textit{API} \textit{KNX IoT} e delle tecnologie per il PoC;
    \item \rif{RC3} - Disparità di conoscenze tra componenti: possibili difficoltà nella collaborazione asincrona per la redazione dell'Analisi dei Requisiti;
    \item \rif{RO7} - Sovrastima costi e ore: possibile persistenza delle difficoltà di stima riscontrate negli sprint precedenti.
\end{itemize}

\trisubsection{Rischi manifestati}
Durante lo Sprint 3 si sono verificati i seguenti rischi:
\begin{itemize}
    \item \rif{RC3} - Disparità di conoscenze tra componenti: la redazione asincrona della bozza dell'Analisi dei Requisiti ha evidenziato differenze terminologiche e di approccio tra gli analisti. 
    \\La mitigazione è avvenuta attraverso riunioni di allineamento tra gli analisti e revisione condivisa della struttura degli \textit{use case}.
    \item \rif{RO7} - Sovrastima costi e ore: è stata riscontrata una sovrastima nell'assegnazione delle ore per alcuni ruoli (Filippo: -2.5 ore, Valeria: -2 ore, Leonardo: -1 ora). 
    \\La mitigazione è avvenuta attraverso una ripianificazione per gli sprint successivi.
    \item \rif{RO8} (non previsto) - Sottostima costi e ore: si registra una lieve sottostima nelle ore per un ruolo (Luca: +2 ore come Progettista).
    \\La mitigazione è avvenuta attraverso una ripianificazione per gli sprint successivi.
\end{itemize}

\subsubsection{Costi}
Le ore effettive svolte dai singoli componenti del gruppo sono di seguito riportate:

\setRuoloFilippo{\Anal}
\setRuoloChristian{\Amm}
\setRuoloValeria{\Anal}
\setRuoloGiuseppe{\Anal}
\setRuoloFrancesco{\Resp}
\setRuoloLuca{\Proge}
\setRuoloLeonardo{\Verif}

\setHPrevisteFilippo{10}
\setHPrevisteChristian{7}
\setHPrevisteValeria{10}
\setHPrevisteGiuseppe{10}
\setHPrevisteFrancesco{8}
\setHPrevisteLuca{14}
\setHPrevisteLeonardo{10}

\setHRealiFilippo{7.5}
\setHRealiChristian{8}
\setHRealiValeria{8}
\setHRealiGiuseppe{10}
\setHRealiFrancesco{8}
\setHRealiLuca{8}
\setHRealiLeonardo{9}

\makeTableRuoli
Si deriva il consuntivo delle ore e dei costi rispetto ai singoli ruoli:
\makeTableDeviazioni

\subsubsection{Preventivo a finire}
Lo stato aggiornato delle risorse è calcolato nella seguente tabella, in cui i campi "iniziali" fanno riferimento all'inizio dello sprint corrente:

\makeTablePreventivo

\subsubsection{Retrospettiva}
Lo Sprint 3 ha visto un significativo avanzamento nell'Analisi dei Requisiti con la redazione della prima bozza completa del documento inviata alla Proponente per revisione. L'adozione di \textit{GitHub Projects} come sistema di tracciamento ha migliorato l'organizzazione delle attività.

La collaborazione asincrona sulla bozza del documento ha evidenziato criticità legate a differenze terminologiche e di approccio tra gli analisti, mitigate attraverso riunioni di allineamento dedicate. Sono emersi dubbi metodologici sulla corretta modellazione degli \textit{use case} (attori secondari, relazioni \textit{extend}, granularità), che hanno portato alla richiesta di un colloquio chiarificatore con il prof. Cardin.

L'analisi dei dati mostra una maggiore variabilità rispetto agli sprint precedenti: si registra una sottostima significativa per il ruolo di Progettista (+2 ore) e sovrastima per alcuni Analisti e per il Verificatore. Il gruppo sta ancora affinando la capacità di pianificazione, con particolare difficoltà nella stima delle attività di analisi.


%================================
\vspace{1em}
\hrule
\vspace{1em}

\subsection{Sprint 4 (15/12/2025 - 28/12/2025)}
\subsubsection{Avanzamento}
\trisubsection{Attività programmate}
Si prevede l'esecuzione delle seguenti attività:
\begin{itemize}
    \item Terminazione dell'Analisi dei Requisiti;
    \item continuazione stesura del Piano di Progetto;
    \item definizione metriche e prima stesura del Piano di Qualifica;
    \item inizio implementazione del \textit{Proof of Concept};
    \item definizione diagramma di Gantt;
    \item correzione dei documenti valutati su \textit{Round Review};
\end{itemize}

\trisubsection{Attività svolte}
Nel corso dello Sprint 4 sono state completate le seguenti attività:
\begin{itemize}
    \item \textbf{Analisi dei Requisiti}: il colloquio con il prof. Cardin, svoltosi in data 17/12/2025, ha evidenziato alcune criticità nell'Analisi dei Requisiti, 
    rendendo necessario l'aggiornamento del documento alla versione 1.1.0, con l'aggiunta, la cancellazione e la modifica di alcuni \textit{use case};
    \item \textbf{Piano di Progetto}: aggiornamento alla versione 0.1.5 con aggiunta della retrospettiva dello Sprint 3 e redazione dei rischi di comunicazione e organizzativi;
    \item \textbf{Norme di Progetto}: aggiornamento alla versione 0.3.0 con aggiunta delle metriche di qualità di processo e di prodotto;
    \item \textbf{Piano di Qualifica}: prima stesura e aggiornamento alla versione 0.2.0 con aggiunta delle soglie di accettabilità e ottimalità per le metriche di qualità di processo e di prodotto;
    \item \textbf{Proof of Concept}: prima implementazione del \textit{PoC}, comprensiva di una visualizzazione preliminare dei principali dispositivi all'interno della \textit{Dashboard};
    \item \textbf{Diagramma di Gantt}: scelta di basare la pianificazione sulla \textit{roadmap} di \textit{GitHub Projects}, in attesa di implementazione;
    \item \textbf{Verbali}: redazione del verbale interno del 15/12/2025 e del verbale esterno relativo al SAL del 23/12/2025. Correzione del verbale esterno del 27/11/2025 e del verbale esterno del 10/12/2025 dopo la valutazione su \textit{Round Review};
    \item \textbf{Presentazioni}: preparazione e presentazione del SAL del 23/12/2025 con la Proponente.
\end{itemize}

\subsubsection{Rischi}
\trisubsection{Rischi attesi}
All'inizio dello Sprint erano previsti i seguenti rischi:
\begin{itemize}
    \item \rif{RT1} - Conoscenza insufficiente di una tecnologia: rischio legato all'apprendimento delle tecnologie per il \textit{PoC};
    \item \rif{RI1} - Impegni personali: indisponibilità di un membro del gruppo per motivi personali;
    \item \rif{RI2} - Impegni universitari: possibile interferenza con altri esami e attività accademiche;
    \item \rif{RO7} - Sovrastima costi e ore: possibile persistenza delle difficoltà di stima riscontrate negli sprint precedenti.
\end{itemize}

\trisubsection{Rischi manifestati}
Durante lo Sprint 4 si sono verificati i seguenti rischi:
\begin{itemize}
    \item \rif{RT1} - Conoscenza insufficiente di una tecnologia: durante la prima implementazione del \textit{PoC}, i programmatori hanno affrontato una significativa curva di apprendimento nell'utilizzo di \textit{Angular} e nell'adozione del modello di architettura esagonale.
    \\La mitigazione è avvenuta attraverso sessioni di studio collaborative.
    \item \rif{RI1} - Impegni personali: per la prima settimana dello sprint un membro del gruppo non ha potuto svolgere alcuna attività per motivi personali. 
    \\Il gruppo ha dunque aggiornato le attività e le ore previste per gli sprint successivi.
    \item \rif{RO7} - Sovrastima costi e ore: è stata riscontrata una sovrastima nell'assegnazione delle ore per alcuni ruoli (Francesco: -1 ora e -15 ore per il ruolo di Progettista che in questo sprint nessuno ha svolto).
    \\La mitigazione è avvenuta attraverso una ripianificazione più accurata per gli sprint successivi.
    \item \rif{RO8} (non previsto) - Sottostima costi e ore: è stata riscontrata una sottostima nell'assegnazione delle ore per alcuni ruoli (Christian: +1 ora, Giuseppe: +4 ore, Valeria: +8 ore, Leonardo: +4 ore, Filippo: +1 ora).
    \\La mitigazione è avvenuta attraverso una ripianificazione più accurata per gli sprint successivi.
\end{itemize}

\subsubsection{Costi}
Le ore effettive svolte dai singoli componenti del gruppo sono di seguito riportate:

\setRuoloFilippo{\Progr}
\setRuoloChristian{\Resp}
\setRuoloValeria{\Amm}
\setRuoloGiuseppe{\Amm}
\setRuoloFrancesco{\Verif}
\setRuoloLuca{\Progr}
\setRuoloLeonardo{\Anal}
\setRuoloVacante{\Proge}

\setHPrevisteFilippo{9}
\setHPrevisteChristian{8}
\setHPrevisteValeria{0}
\setHPrevisteGiuseppe{8}
\setHPrevisteFrancesco{12}
\setHPrevisteLuca{7}
\setHPrevisteLeonardo{10}
\setHPrevisteVacante{15}

\setHRealiFilippo{10}
\setHRealiChristian{9}
\setHRealiValeria{8}
\setHRealiGiuseppe{12}
\setHRealiFrancesco{11}
\setHRealiLuca{7}
\setHRealiLeonardo{14}
\setHRealiVacante{0}

\makeTableRuoli
Si deriva il consuntivo delle ore e dei costi rispetto ai singoli ruoli:
\makeTableDeviazioni

\subsubsection{Preventivo a finire}
Lo stato aggiornato delle risorse è calcolato nella seguente tabella, in cui i campi "iniziali" fanno riferimento all'inizio dello sprint corrente:

\makeTablePreventivo

\subsubsection{Retrospettiva}
Durante lo Sprint 4 le attività svolte hanno permesso un avanzamento significativo su più fronti, 
in particolare nelle Norme di Progetto, nel Piano di Progetto, nell'Analisi dei Requisiti e nel \textit{Proof of Concept}. 
Lo sprint è risultato complessivamente produttivo, sebbene caratterizzato da alcune criticità di 
natura organizzativa, tecnologica e di pianificazione delle risorse.

Il colloquio con il prof. Cardin ha rappresentato una svolta nell'Analisi dei Requisiti in quanto ha 
permesso di chiarire alcuni dubbi e di individuare diverse problematiche nella definizione degli \textit{use case}, i quali 
sono stati opportunamente modificati.

La definizione delle metriche di qualità e delle soglie di accettabilità e ottimalità 
ha evidenziato un carico di lavoro superiore alle stime iniziali. Per far fronte 
a questa esigenza si è resa necessaria una riassegnazione dei ruoli: 
in particolare, Valeria è stata assegnata al ruolo di Amministratore anziché a 
quello di Progettista, per permettere una distribuzione equa delle attività tra 
i due Amministratori.

La sottostima e la sovrastima delle ore assegnate ad alcuni ruoli hanno confermato la necessità di affinare 
ulteriormente il processo di pianificazione.
Le deviazioni riscontrate saranno utilizzate come base per una 
ripianificazione più accurata degli sprint successivi.


%================================
\vspace{1em}
\hrule
\vspace{1em}

\subsection{Sprint 5 (29/12/2025 - 11/01/2025)}
\subsubsection{Avanzamento}
\trisubsection{Attività programmate}
Si prevede l'esecuzione delle seguenti attività:
\begin{itemize}
    \item Terminazione dell'Analisi dei Requisiti;
    \item continuazione implementazione del \textit{PoC};
    \item continuazione stesura del Piano di Progetto;
    \item continuazione stesura del Piano di Qualifica;
    \item correzione dei documenti valutati su \textit{Round Review};
    \item caricamento dei verbali esterni firmati dalla proponente nella \textit{repository}.
\end{itemize}

\trisubsection{Attività svolte}
Nel corso del periodo di avanzamento sono state svolte le seguenti attività:
\begin{itemize}
    \item 
\end{itemize}

\subsubsection{Rischi}
\trisubsection{Rischi attesi}
\begin{itemize}
    \item 
\end{itemize}
\trisubsection{Rischi manifestati}
\begin{itemize}
    \item 
\end{itemize}

\subsubsection{Costi}
Le ore effettive svolte dai singoli componenti del gruppo sono di seguito riportate:

\setRuoloFilippo{\Anal}
\setRuoloChristian{\Anal}
\setRuoloValeria{\Verif}
\setRuoloGiuseppe{\Resp}
\setRuoloFrancesco{\Amm}
\setRuoloLuca{\Anal}
\setRuoloLeonardo{\Proge}

\setHPrevisteFilippo{0}
\setHPrevisteChristian{0}
\setHPrevisteValeria{0}
\setHPrevisteGiuseppe{0}
\setHPrevisteFrancesco{0}
\setHPrevisteLuca{0}
\setHPrevisteLeonardo{0}

\setHRealiFilippo{0}
\setHRealiChristian{0}
\setHRealiValeria{0}
\setHRealiGiuseppe{0}
\setHRealiFrancesco{0}
\setHRealiLuca{0}
\setHRealiLeonardo{0}

\makeTableRuoli
Si deriva il consuntivo delle ore e dei costi rispetto ai singoli ruoli:
\makeTableDeviazioni

\subsubsection{Preventivo a finire}
Lo stato aggiornato delle risorse è calcolato nella seguente tabella, in cui i campi "iniziali" fanno riferimento all'inizio dello sprint corrente:

\makeTablePreventivo

\subsubsection{Retrospettiva}


%================================
\vspace{1em}
\hrule
\vspace{1em}

\subsection{Sprint 6 (12/01/2026 - 25/01/2026)}
\subsubsection{Avanzamento}
\trisubsection{Attività programmate}
Si prevede l'esecuzione delle seguenti attività:
\begin{itemize}
    \item
\end{itemize}

\trisubsection{Attività svolte}
Nel corso del periodo di avanzamento sono state svolte le seguenti attività:
\begin{itemize}
    \item 
\end{itemize}

\subsubsection{Rischi}
\trisubsection{Rischi attesi}
\begin{itemize}
    \item 
\end{itemize}
\trisubsection{Rischi manifestati}
\begin{itemize}
    \item 
\end{itemize}

\subsubsection{Costi}
Le ore effettive svolte dai singoli componenti del gruppo sono di seguito riportate:

\setRuoloFilippo{\Anal}
\setRuoloChristian{\Anal}
\setRuoloValeria{\Verif}
\setRuoloGiuseppe{\Resp}
\setRuoloFrancesco{\Amm}
\setRuoloLuca{\Anal}
\setRuoloLeonardo{\Proge}

\setHPrevisteFilippo{0}
\setHPrevisteChristian{0}
\setHPrevisteValeria{0}
\setHPrevisteGiuseppe{0}
\setHPrevisteFrancesco{0}
\setHPrevisteLuca{0}
\setHPrevisteLeonardo{0}

\setHRealiFilippo{0}
\setHRealiChristian{0}
\setHRealiValeria{0}
\setHRealiGiuseppe{0}
\setHRealiFrancesco{0}
\setHRealiLuca{0}
\setHRealiLeonardo{0}

\makeTableRuoli
Si deriva il consuntivo delle ore e dei costi rispetto ai singoli ruoli:
\makeTableDeviazioni

\subsubsection{Preventivo a finire}
Lo stato aggiornato delle risorse è calcolato nella seguente tabella, in cui i campi "iniziali" fanno riferimento all'inizio dello sprint corrente:

\makeTablePreventivo

\subsubsection{Retrospettiva}


%================================
\vspace{1em}
\hrule
\vspace{1em}

\subsection{Sprint 7 (26/01/2026 - 08/02/2026)}
\subsubsection{Avanzamento}
\trisubsection{Attività programmate}
Si prevede l'esecuzione delle seguenti attività:
\begin{itemize}
    \item
\end{itemize}

\trisubsection{Attività svolte}
Nel corso del periodo di avanzamento sono state svolte le seguenti attività:
\begin{itemize}
    \item 
\end{itemize}

\subsubsection{Rischi}
\trisubsection{Rischi attesi}
\begin{itemize}
    \item 
\end{itemize}
\trisubsection{Rischi manifestati}
\begin{itemize}
    \item 
\end{itemize}

\subsubsection{Costi}
Le ore effettive svolte dai singoli componenti del gruppo sono di seguito riportate:

\setRuoloFilippo{\Anal}
\setRuoloChristian{\Anal}
\setRuoloValeria{\Verif}
\setRuoloGiuseppe{\Resp}
\setRuoloFrancesco{\Amm}
\setRuoloLuca{\Anal}
\setRuoloLeonardo{\Proge}

\setHPrevisteFilippo{0}
\setHPrevisteChristian{0}
\setHPrevisteValeria{0}
\setHPrevisteGiuseppe{0}
\setHPrevisteFrancesco{0}
\setHPrevisteLuca{0}
\setHPrevisteLeonardo{0}

\setHRealiFilippo{0}
\setHRealiChristian{0}
\setHRealiValeria{0}
\setHRealiGiuseppe{0}
\setHRealiFrancesco{0}
\setHRealiLuca{0}
\setHRealiLeonardo{0}

\makeTableRuoli
Si deriva il consuntivo delle ore e dei costi rispetto ai singoli ruoli:
\makeTableDeviazioni

\subsubsection{Preventivo a finire}
Lo stato aggiornato delle risorse è calcolato nella seguente tabella, in cui i campi "iniziali" fanno riferimento all'inizio dello sprint corrente:

\makeTablePreventivo

\subsubsection{Retrospettiva}


%================================
\vspace{1em}
\hrule
\vspace{1em}

\subsection{Sprint 8 (09/02/2026 - 22/02/2026)}
\subsubsection{Avanzamento}
\trisubsection{Attività programmate}
Si prevede l'esecuzione delle seguenti attività:
\begin{itemize}
    \item
\end{itemize}

\trisubsection{Attività svolte}
Nel corso del periodo di avanzamento sono state svolte le seguenti attività:
\begin{itemize}
    \item 
\end{itemize}

\subsubsection{Rischi}
\trisubsection{Rischi attesi}
\begin{itemize}
    \item 
\end{itemize}
\trisubsection{Rischi manifestati}
\begin{itemize}
    \item 
\end{itemize}

\subsubsection{Costi}
Le ore effettive svolte dai singoli componenti del gruppo sono di seguito riportate:

\setRuoloFilippo{\Anal}
\setRuoloChristian{\Anal}
\setRuoloValeria{\Verif}
\setRuoloGiuseppe{\Resp}
\setRuoloFrancesco{\Amm}
\setRuoloLuca{\Anal}
\setRuoloLeonardo{\Proge}

\setHPrevisteFilippo{0}
\setHPrevisteChristian{0}
\setHPrevisteValeria{0}
\setHPrevisteGiuseppe{0}
\setHPrevisteFrancesco{0}
\setHPrevisteLuca{0}
\setHPrevisteLeonardo{0}

\setHRealiFilippo{0}
\setHRealiChristian{0}
\setHRealiValeria{0}
\setHRealiGiuseppe{0}
\setHRealiFrancesco{0}
\setHRealiLuca{0}
\setHRealiLeonardo{0}

\makeTableRuoli
Si deriva il consuntivo delle ore e dei costi rispetto ai singoli ruoli:
\makeTableDeviazioni

\subsubsection{Preventivo a finire}
Lo stato aggiornato delle risorse è calcolato nella seguente tabella, in cui i campi "iniziali" fanno riferimento all'inizio dello sprint corrente:

\makeTablePreventivo

\subsubsection{Retrospettiva}


%================================
\vspace{1em}
\hrule
\vspace{1em}

\subsection{Sprint 9 (23/02/2026 - 08/03/2026)}
\subsubsection{Avanzamento}
\trisubsection{Attività programmate}
Si prevede l'esecuzione delle seguenti attività:
\begin{itemize}
    \item
\end{itemize}

\trisubsection{Attività svolte}
Nel corso del periodo di avanzamento sono state svolte le seguenti attività:
\begin{itemize}
    \item 
\end{itemize}

\subsubsection{Rischi}
\trisubsection{Rischi attesi}
\begin{itemize}
    \item 
\end{itemize}
\trisubsection{Rischi manifestati}
\begin{itemize}
    \item 
\end{itemize}

\subsubsection{Costi}
Le ore effettive svolte dai singoli componenti del gruppo sono di seguito riportate:

\setRuoloFilippo{\Anal}
\setRuoloChristian{\Anal}
\setRuoloValeria{\Verif}
\setRuoloGiuseppe{\Resp}
\setRuoloFrancesco{\Amm}
\setRuoloLuca{\Anal}
\setRuoloLeonardo{\Proge}

\setHPrevisteFilippo{0}
\setHPrevisteChristian{0}
\setHPrevisteValeria{0}
\setHPrevisteGiuseppe{0}
\setHPrevisteFrancesco{0}
\setHPrevisteLuca{0}
\setHPrevisteLeonardo{0}

\setHRealiFilippo{0}
\setHRealiChristian{0}
\setHRealiValeria{0}
\setHRealiGiuseppe{0}
\setHRealiFrancesco{0}
\setHRealiLuca{0}
\setHRealiLeonardo{0}

\makeTableRuoli
Si deriva il consuntivo delle ore e dei costi rispetto ai singoli ruoli:
\makeTableDeviazioni

\subsubsection{Preventivo a finire}
Lo stato aggiornato delle risorse è calcolato nella seguente tabella, in cui i campi "iniziali" fanno riferimento all'inizio dello sprint corrente:

\makeTablePreventivo

\subsubsection{Retrospettiva}


%================================
\vspace{1em}
\hrule
\vspace{1em}

\subsection{Sprint 10 (09/03/2026 - 22/03/2026)}
\subsubsection{Avanzamento}
\trisubsection{Attività programmate}
Si prevede l'esecuzione delle seguenti attività:
\begin{itemize}
    \item
\end{itemize}

\trisubsection{Attività svolte}
Nel corso del periodo di avanzamento sono state svolte le seguenti attività:
\begin{itemize}
    \item 
\end{itemize}

\subsubsection{Rischi}
\trisubsection{Rischi attesi}
\begin{itemize}
    \item 
\end{itemize}
\trisubsection{Rischi manifestati}
\begin{itemize}
    \item 
\end{itemize}

\subsubsection{Costi}
Le ore effettive svolte dai singoli componenti del gruppo sono di seguito riportate:

\setRuoloFilippo{\Anal}
\setRuoloChristian{\Anal}
\setRuoloValeria{\Verif}
\setRuoloGiuseppe{\Resp}
\setRuoloFrancesco{\Amm}
\setRuoloLuca{\Anal}
\setRuoloLeonardo{\Proge}

\setHPrevisteFilippo{0}
\setHPrevisteChristian{0}
\setHPrevisteValeria{0}
\setHPrevisteGiuseppe{0}
\setHPrevisteFrancesco{0}
\setHPrevisteLuca{0}
\setHPrevisteLeonardo{0}

\setHRealiFilippo{0}
\setHRealiChristian{0}
\setHRealiValeria{0}
\setHRealiGiuseppe{0}
\setHRealiFrancesco{0}
\setHRealiLuca{0}
\setHRealiLeonardo{0}

\makeTableRuoli
Si deriva il consuntivo delle ore e dei costi rispetto ai singoli ruoli:
\makeTableDeviazioni

\subsubsection{Preventivo a finire}
Lo stato aggiornato delle risorse è calcolato nella seguente tabella, in cui i campi "iniziali" fanno riferimento all'inizio dello sprint corrente:

\makeTablePreventivo

\subsubsection{Retrospettiva}



\end{document}
