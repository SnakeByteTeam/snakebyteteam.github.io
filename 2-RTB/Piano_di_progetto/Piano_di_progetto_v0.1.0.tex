\documentclass[10pt, letterpaper]{article}

\usepackage{../../template/template}
\usepackage{templatePdP}
\usepackage{spreadtab,simplekv,xpatch}

\usepackage[table]{xcolor}

%Queste macro sono in relazione all'ultima modifica
\setDocIndex{Piano di progetto}
\setDocTitle{Piano di progetto}
\setDocVersion{0.1.0}
\setDocState{1}
\setDocData{09/12/2025}
\setDocRecipient{1}

\setcounter{tocdepth}{4}
\setcounter{secnumdepth}{4}
\newcommand{\trisubsection}[1]{\paragraph{#1}\mbox{}\\}


\newcommand{\checkLevel}[1]{%
  \ifnum#1=0\relax
  \else\ifnum#1=1\relax
  \else\ifnum#1=2\relax
  \else
    \errmessage{Valore non valido (#1). Usare solo 0,1,2}%
  \fi\fi\fi
}

\newcommand{\convertLevelF}[1]{%
  \ifnum#1=0 Bassa\fi
  \ifnum#1=1 Media\fi
  \ifnum#1=2 Alta\fi
}
\newcommand{\convertLevelM}[1]{%
  \ifnum#1=0 Basso\fi
  \ifnum#1=1 Medio\fi
  \ifnum#1=2 Alto\fi
}

\newcounter{rtctr}
\newcommand{\newTRisk}[5]{
    \checkLevel{#4}
    \checkLevel{#5}
    \stepcounter{rtctr}
    \subsubsection{Rischio RT\thertctr : #1}
    \begin{center}
        \begin{tabularx}{0.9\textwidth}{|l|X|}%0.9 è un po meglio per l'indentazione
        \hline
        \rowcolor{lightgray}
        \textbf{Attributo} & \textbf{Valore} \\
        \hline
        Codice & RT\thertctr \label{RT\thertctr} \\
        \hline
        Nome & #1 \\
        \hline
        Descrizione & #2 \\
        \hline
        Mitigazione & #3 \\
        \hline
        Probabilità di occorrenza & \convertLevelF{#4} \\
        \hline
        Livello di impatto & \convertLevelM{#5} \\
        \hline
        \end{tabularx}
    \end{center}
}

\newcounter{rictr}
\newcommand{\newIRisk}[5]{
    \checkLevel{#4}
    \checkLevel{#5}
    \stepcounter{rictr}
    \subsubsection{Rischio RI\therictr : #1}
    \begin{center}
        \begin{tabularx}{0.9\textwidth}{ |l|X|}
        \hline
        \rowcolor{lightgray}
        \textbf{Attributo} & \textbf{Valore} \\
        \hline
        Codice & RI\therictr \label{RI\therictr} \\
        \hline
        Nome & #1 \\
        \hline
        Descrizione & #2 \\
        \hline
        Mitigazione & #3 \\
        \hline
        Probabilità di occorrenza & \convertLevelF{#4} \\
        \hline
        Livello di impatto & \convertLevelM{#5} \\
        \hline
        \end{tabularx}
    \end{center}
}

\newcounter{rcctr}
\newcommand{\newCRisk}[5]{
    \checkLevel{#4}
    \checkLevel{#5}
    \stepcounter{rcctr}
    \subsubsection{Rischio RC\thercctr : #1}
    \begin{center}
        \begin{tabularx}{0.9\textwidth}{|l|X|}
        \hline
        \rowcolor{lightgray}
        \textbf{Attributo} & \textbf{Valore} \\
        \hline
        Codice & RC\thercctr \label{RC\thercctr} \\
        \hline
        Nome & #1 \\
        \hline
        Descrizione & #2 \\
        \hline
        Mitigazione & #3 \\
        \hline
        Probabilità di occorrenza & \convertLevelF{#4} \\
        \hline
        Livello di impatto & \convertLevelM{#5} \\
        \hline
        \end{tabularx}
    \end{center}
}


\newcounter{roctr}
\newcommand{\newORisk}[5]{
    \checkLevel{#4}
    \checkLevel{#5}
    \stepcounter{roctr}
    \subsubsection{Rischio RO\theroctr : #1}
    \begin{center}
        \begin{tabularx}{0.9\textwidth}{|l|X|}
        \hline
        \rowcolor{lightgray}
        \textbf{Attributo} & \textbf{Valore} \\
        \hline
        Codice & RO\theroctr \label{RO\theroctr} \\
        \hline
        Nome & #1 \\
        \hline
        Descrizione & #2 \\
        \hline
        Mitigazione & #3 \\
        \hline
        Probabilità di occorrenza & \convertLevelF{#4} \\
        \hline
        Livello di impatto & \convertLevelM{#5} \\
        \hline
        \end{tabularx}
    \end{center}
}

\newcommand{\rif}[1]{\hyperref[#1]{§#1}}

\newcommand{\rotaz}{0}



% comandi extra (rimuovere questi commenti una volta approvato il documento)
%
% -- costanti --
% \noOne [-]
% \noRole [ND]
% \firstDraft [Prima stesura]
% \approval [Approvazione]
% \teamName [SnakeByte]
% \teamEmail [snakebyteteam@gmail.com]
% \teamRepo [https://github.com/SnakeByteTeam/snakebyteteam.github.io]
% \teamSite [https://snakebyteteam.github.io/]
% \recipientTeachers [prof. Vardanega Tullio, prof. Cardin Riccardo]
% \vimar [Vimar]
% \vimarspa [Vimar S.p.A.]
%
% -- variabili --
% \docIndex | \setDocIndex{nome documento senza versione}
% \docTitle | \setDocTitle{titolo documento} 
% \docData | \setDocData{data documento} [GG/MM/AAAA]
% \docVersion | \setDocVersion{versione documento}
% \docState | \setDocState{0|1|2} [Da verificare][Verificato][Approvato]
% \docRecipient [table] | \setDocRecipient{0|1|2} [SnakeByte][SnakeByte, prof][SnakeByte, prof, vimar]
%
% -- pagine intere --
% \makeTitlePage
% \addChangelog{versione}{data}{autore}{verificatore}{approvatore}{descrizione} | \makeTodoTable



% ============= STRUTTURA SPRINT GENERICO =============
\iffalse
%================================
\vspace{1em}
\hrule
\vspace{1em}

\subsection{SprintX (GG/MM/AAAA - GG/MM/AAAA)}
\subsubsection{Avanzamento}
\trisubsection{Attività programmate}
Si prevede l'esecuzione delle seguenti attività:
\begin{itemize}
    \item
\end{itemize}

\trisubsection{Attività svolte}
Nel corso del primo periodo di avanzamento sono state svolte le seguenti attività:
\begin{itemize}
    \item 
\end{itemize}

\subsubsection{Rischi}
\trisubsection{Rischi attesi}
\begin{itemize}
    \item 
\end{itemize}
\trisubsection{Rishi manifestati}
\begin{itemize}
    \item 
\end{itemize}

\subsubsection{Costi}
Le ore effettive svolte dai singoli componenti del gruppo sono di seguito riportate:

\setRuoloFilippo{\Anal}
\setRuoloChristian{\Anal}
\setRuoloValeria{\Verif}
\setRuoloGiuseppe{\Resp}
\setRuoloFrancesco{\Amm}
\setRuoloLuca{\Anal}
\setRuoloLeonardo{\Proge}

\setHPrevisteFilippo{0}
\setHPrevisteChristian{0}
\setHPrevisteValeria{0}
\setHPrevisteGiuseppe{0}
\setHPrevisteFrancesco{0}
\setHPrevisteLuca{0}
\setHPrevisteLeonardo{0}

\setHRealiFilippo{0}
\setHRealiChristian{0}
\setHRealiValeria{0}
\setHRealiGiuseppe{0}
\setHRealiFrancesco{0}
\setHRealiLuca{0}
\setHRealiLeonardo{0}

\makeTableRuoli
Si deriva il consuntivo delle ore e dei costi rispetto ai singoli ruoli:
%\makeTableDeviazioni

\subsubsection{Preventivo a finire}
Lo stato aggiornato delle risorse è calcolato nella seguente tabella, in cui i campi "iniziali" fanno riferimento all'inizio dello sprint corrente:

%\makeTablePreventivo
\fi

% ============== LEGGI =================
% Se le modifiche sulle tabelle non compaiono, elimina i file extra e fai rebuild

\begin{document}

\makeTitlePage

\addChangelog{0.1.0}{09/12/2025}{C. Libralato}{L. Pellizzon}{\noOne}{Prima stesura}
\makeChangelog

\newpage

\tableofcontents

\newpage

\section{Introduzione}
\subsection{Contenuto documento}
Il Piano di Progetto, nell'ambito dell'Ingegneria del Software, è un documento che ha lo scopo di definire in maniera chiara e dettagliata una struttura che funga 
da tabella di marcia per il progetto e da strumento di coordinazione per il team.

La suddetta struttura consiste in pianificazioni riguardo:
\begin{itemize}
    \item la dislocazione temporale dell'esecuzione delle attività di processo;
    \item i preventivi e le scadenze;
    \item la suddivisione di responsabilità dei ruoli;
    \item gestione dei rischi;
    \item la gestione delle risorse;
\end{itemize}

\subsubsection{Struttura del contenuto}
Il documento è suddiviso in quattro macrosezioni:
\begin{enumerate}
    \item \textbf{Introduzione}: sezione introduttiva che espone il contenuto e lo scopo del documento, oltre che i Riferimenti Normativi e Informativi.
    \item \textbf{Gestione rischi}: consiste in una classificazione e descrizione dei rischi che possono insorgere durante l'avanzamento del ciclo
    di vita del prodotto. Ad ogni rischio vengono associati comportamenti e regole sia per la sua prevenzione che per la mitigazione degli effetti in caso di occorrenza.
    \item \textbf{Previsioni a lungo termine}: presenta le stime iniziali eseguite in fase di Candidatura riguardo i costi, l'impegno orario e le responsabilità oltre che le stime future sugli
    stati di avanzamento previsti in vista delle scadenze quali RTB e PB.
    \item \textbf{Periodi di avanzamento}: elenco dei singoli periodi di avanzamento, definiti come \textit{sprint$_G$} di durata pari a due settimane, di cui si analizzano i seguenti aspetti:
    \begin{itemize}
        \item Consuntivo di periodo:
        \begin{itemize}
            \item avanzamento previsto e avanzamento effettivo;
            \item rischi attesi, rischi occorsi e relativa gestione;
            \item costi attesi ed costi effettivi.
        \end{itemize}
        \item Preventivo a finire:
        \begin{itemize}
            \item revisione calendario e risorse;
            \item retrospettiva.
        \end{itemize}
        
    \end{itemize}
\end{enumerate}


\subsection{Glossario}
All'interno del documento possono essere presenti termini il cui significato può risultare ambiguo o sconosciuto, essi sono
riportati in italico e alla loro prima occorrenza è associata una \textit{G} a pedice ad indicare la presenza del suddetto termine all'interno del \textit{glossario$_G$}, che è disponibile al seguente link: \href{https://snakebyteteam.github.io/glossary.html}{Glossario}.

\subsection{Riferimenti Normativi}
\begin{itemize}
    \item \textbf{Norme di Progetto}\\
    \url{link-finale-norme-di-progetto}
\end{itemize}

\subsection{Riferimenti Formativi}
\begin{itemize}
    \item \textbf{Regolamento del progetto didattico}\\
    \url{https://www.math.unipd.it/~tullio/IS-1/2025/Dispense/PD1.pdf}
    \item \textbf{Gestione di progetto}\\
    \url{https://www.math.unipd.it/~tullio/IS-1/2025/Dispense/T04.pdf}
\end{itemize}



\section{Analisi e gestione rischi}
    Nella seguente sezione vengono analizzati i possibili rischi che potrebbero emergere durante il progetto e definite le modalità 
    per ridurne l'impatto. Sono state individuate quattro tipologie principali di rischio:
    \begin{itemize}
        \item \textbf{rischi tecnologici}: sono legati alla scelta e all'utilizzo delle tecnologie adottate durante lo sviluppo;
        \item \textbf{rischi individuali}: emergono da imprevisti della vita quotidiana o impegni personali;
        \item \textbf{rischi di comunicazione}: sono causati da una mancanza di comunicazione tra le figure coinvolte nel progetto, quali membri del gruppo, Proponente e professori;
        \item \textbf{rischi organizzativi}: sono legati ad un'organizzazione errata o non adatta allo svolgimento di un'attività.
    \end{itemize}
    

    \subsection{Rischi tecnologici}
        \newTRisk{Conoscenza insufficiente di una tecnologia}
        {Il progetto richiede l'utilizzo di molte tecnologie diverse che risultano spesso sconosciute
        a causa dell'inesperienza e dell'ampia offerta di strumenti nell'ambito.}
        {Dedicare ore di calendario allo studio individuale di una tecnologia e, in caso di componenti 
        del gruppo con conoscenze pregresse sull'argomento, organizzare dei momenti di allineamento
        generale.}
        {2}{2}

        \newTRisk{Incompatibilità tecnologie}
        {La scelta delle tecnologie in fase di progettazione, se non presa con cura, può provocare problemi
        di compatibilità in fase di sviluppo, in particolare durante la realizzazione del Proof of Concept.}
        {È necessario interrompere lo sviluppo delle parti che coinvolgono le tecnologie in conflitto e rielaborare la scelta progettuale 
        in maniera più attenta e motivata.}
        {0}{2} % 0 | 1 | 2

        \newTRisk{\textit{Bug$_G$} nel codice}
        {Nonostante l'utilizzo di metodi automatizzati di verifica del codice si può presentare
        un comportamento inatteso o errato dell'applicativo.}
        {Individuare la parte di codice che genera il \textit{bug} attraverso analisi del codice 
        e strumenti come \textit{debugger$_G$}. Modificare il codice per correggere il comportamento
        fallace. Verificare occorrenza del rishio \rif{RT4}.}
        {2}{1}

        \newTRisk{Test non esaustivi}
        {Casistica in cui i test automatici del codice risultano essere non esaustivi o insufficienti al
        fine di produrre un codice funzionante, robusto e manutenibile, secondo lo stato dell'arte.}
        {Effettuare un controllo dell'esaustività, della correttezza e del livello di copertura del codice dei 
        test automatici; su necessità aggiungere test o modificare quelli esistenti.}
        {1}{1}
    
    \subsection{Rischi individuali}
        \newIRisk{Impegni personali}
        {Le singole componenti del gruppo possono essere indisponibili allo svolgimento delle attività
        assegnate per un certo periodo di tempo a causa di motivi personali.}
        {Comunicare l'indisponibilità al gruppo il prima possibile, organizzare una riunione urgente per 
        effettuare una ripianificazione delle attività di periodo. In base al momento d'occorrenza, alle 
        attività in sospeso e al carico di lavoro dei componenti, decidere tra le opzioni:
        \begin{itemize}
            \item redistribuire le attività in sospeso tra i componenti;
            \item rimandare lo svolgimento delle attività in sospeso al periodo di avanzamento successivo.
        \end{itemize}}
        {0}{1}

        \newIRisk{Impegni universitari}
        {Gli impegni universitari possono ridurre il tempo disponibile per lo svolgimento delle attività,
        specialmente nel periodo della sessione, avendo effetto contemporaneamente sull'intero gruppo e
        dunque rallentando l'avanzamento.}
        {Gestire l'organizzazione del team e delle attività considerando questo rischio anticipatamente e assegnando un carico
        lavoro appropriato per evitare cambi di programma durante il periodo d'interesse.}
        {2}{2}
    
    \subsection{Rischi di comunicazione}
        \newCRisk{Mancanza di coordinamento tra componenti}{test}{test}{0}{0}
        \newCRisk{Divergenza tra Proponente e gruppo}{test}{test}{0}{0}
        \newCRisk{Disparità di conoscenze tra componenti}{test}{test}{0}{0}
        \newCRisk{Incomprensione requisiti}{test}{test}{0}{0}

    \subsection{Rischi organizzativi}
        \newORisk{Carico di lavoro eccessivo}{test}{test}{0}{0}
        \newORisk{Inadempimento scadenza baseline}{test}{test}{0}{0}
        \newORisk{Inadempimento scadenza milestone}{test}{test}{0}{0}
        \newORisk{Inadempimento scadenza Proponente (? boh non so)}{test}{test}{0}{0}
        \newORisk{Inadempimento scadenza issue}{test}{test}{0}{0}
        \newORisk{Divergenza tra fase di sviluppo e progettazione}{test}{test}{0}{0}
        \newORisk{Sovrastima costi e ora}{test}{test}{0}{0}
        \newORisk{Sottostima costi e ora}{test}{test}{0}{0}


\section{Previsioni a lungo termine}
Durante la fase di pianificazione di progetto, in vista della candidatura, sono state effettuate delle stime riguardo la distribuzione
oraria per ogni ruolo e il costo associato, da cui si è poi ricavato il costo complessivo del progetto. 

All'interno di questo documento è possibile confrontare queste stime con il consumo effettivo delle risorse nei
singoli periodi di avanzamento\\
Dal documento Dichiarazione degli Impegni si riporta:
\begin{center}
    \begin{tabular}{|c|c|c|c|c|}
        \hline
        \rowcolor{lightgray} \textbf{Ruolo} & \textbf{Costo orario} & \textbf{Ore totali} & \textbf{Costo totale}
        \\
        \hline
        Responsabile & 30,00 \euro/h & 78 & 2.340,00\euro \\
        \hline
        Amministratore & 20,00\euro/h & 76 & 1520,00\euro\\
        \hline
        Analista & 25,00\euro/h & 86 & 2.150,00\euro \\
        \hline
        Progettista & 25,00\euro/h & 111 & 2.775,00\euro \\
        \hline
        Programmatore & 15,00\euro/h & 142 & 2.130,00\euro\\
        \hline
        Verificatore & 15,00\euro/h & 142 & 2.130,00\euro \\
        \hline
        \cellcolor{lightgray}{\textbf{Totale}} & - & 635 & 13.045,00\euro \\
        \hline
    \end{tabular}
\end{center}
Adottando una rotazione dei ruoli avente luogo alla fine di ogni ciclo di avanzamento, ossia ogni 2 settimane, e con
preventivo dei costi che ammonta ad un totale di 13.045,00 \euro, è stata fissata come scadenza ultima di consegna del 
progetto la data 18 marzo 2026.

\subsection{RTB} 
In vista della Requirement and Technlogy Baseline (RTB), il gruppo SnakeByte fissa come data prevista per la candidatura il
GG/MM/AAAA (?); termine entro il quale si prevede la terminazione delle seguenti attività: 
\begin{itemize}
    \item stesura dell'Analisi Requisiti (AdR);
    \item stesura delle Norme di Progetto (NdP);
    \item stesura del Piano di Progetto (PdP);
    \item stesura del Piano di Qualifica (PdQ);
    \item scrittura e consegna Proof of Concept (PoC);
\end{itemize}
Al momento della candidatura, avente luogo durante lo Sprint 6 (?), in base alle previsioni sulla 
distribuzione oraria dei ruoli, si stima uno stato delle risorse corrispondente a quanto riportato:
\begin{center}
    \begin{tabular}{|c|c|c|}
        \hline
        \rowcolor{lightgray} \textbf{Ruolo} & \textbf{Ore rimanenti} & \textbf{Risorse rimanenti} 
        \\
        \hline
        Responsabile  & 0 & 0\euro\\
        \hline
        Amministratore  & 0 & 0\euro\\
        \hline
        Analista & 0 & 0\euro\\
        \hline
        Progettista & 0 & 0\euro \\
        \hline
        Programmatore & 0 & 0\euro\\
        \hline
        Verificatore & 0 & 0\euro\\
        \hline
        \cellcolor{lightgray}{\textbf{Totale}} & - & - \euro \\
        \hline
    \end{tabular}
\end{center}

\subsection{PB}
DA COMPILARE IN DETTAGLIO A SEGUITO DELLA RTB

A questo punto le uniche previsioni sono
\begin{itemize}
    \item Consegna MVP alla Proponente
    \item Condidatura al PB
\end{itemize}


\section{Periodi di avanzamento}
\subsection{Pre-candidatura}
\subsubsection{Avanzamento}
Durante il periodo precedente alla candidatura il gruppo ha portato a termine le seguenti attività:
\begin{itemize}
    \item definizione del nome e del logo del gruppo;
    \item scelta dei canali di comunicazione e della frequenza degli incontri;
    \item analisi dei capitolati dei progetti di maggior interesse;
    \item creazione del repository e del sito web associato al progetto;
    \item prima stesura delle Norme di Progetto;
    \item prima definizione dei template per i documenti;
    \item contatto con le aziende per chiarire dubbi sui capitolati;
    \item stesura della documento di valutazione dei capitolati;
    \item stesura della Dichiarazione degli Impegni;
    \item stesura della Lettera di Candidatura.
\end{itemize}

\subsubsection{Rischi}
Si è manifestato il rischio \rif{RT1}, riferito allo strumento utilizzato per la redazione
dei documenti, ossia \LaTeX.

\subsubsection{Retrospettiva}
In questa prima fase di assestamento lo svolgimento delle attività è stato gestito
in maniera ancora incerta ma funzionale, senza una divisione dei ruoli, ed è stata caratterizzata da 
decisioni che hanno richiesto molta riflessione, tra cui le convenzioni per le versioni dei file e la struttura dei documenti.


\subsection{Sprint 1 (03/11/2025 - 16/11/2025)}
\subsubsection{Avanzamento}
\trisubsection{Attività programmate}
\begin{itemize}
    \item 
\end{itemize}
\trisubsection{Attività svolte}
Nel corso del primo periodo di avanzamento sono state svolte le seguenti attività:
\begin{itemize}
    \item continuazione della scrittura delle Norme di Progetto;
    \item primo incontro con la Proponente;
    \item inizio definizione dei requisiti;
    \item incontro di approfondimento sulle tecnologie con la Proponente;
    \item definizione di prime domande per l'Analisi dei requisiti;
    \item perfezionamento struttura documenti tramite templates.
\end{itemize}

\subsubsection{Rischi}
\begin{itemize}
    \item 
\end{itemize}
\trisubsection{Rischi attesi}
\begin{itemize}
    \item 
\end{itemize}
\trisubsection{Rishi manifestati}
In questo sprint si sono verificati i rischi legati alla sovrastima delle ore (\rif{RO7}).

\subsubsection{Costi}
Le ore effettive svolte dai singoli componenti del gruppo sono di seguito riportate:

% Inizializzazione NECESSARIA e SUFFICIENTE per la costruzione delle tabelle successive
\setRuoloFilippo{\Resp}
\setRuoloChristian{\Amm}
\setRuoloValeria{\Anal}
\setRuoloGiuseppe{\Anal}
\setRuoloFrancesco{\Anal}
\setRuoloLuca{\Anal}
\setRuoloLeonardo{\Verif}

\setHPrevisteFilippo{8}
\setHPrevisteChristian{9}
\setHPrevisteValeria{5}
\setHPrevisteGiuseppe{5}
\setHPrevisteFrancesco{5}
\setHPrevisteLuca{5}
\setHPrevisteLeonardo{8}

\setHRealiFilippo{6}
\setHRealiChristian{6}
\setHRealiValeria{5}
\setHRealiGiuseppe{5}
\setHRealiFrancesco{5}
\setHRealiLuca{5}
\setHRealiLeonardo{6}

\makeTableRuoli
Si deriva il consuntivo delle ore e dei costi rispetto ai singoli ruoli:
\makeTableDeviazioni

\subsubsection{Preventivo a finire}
Lo stato aggiornato delle risorse è calcolato nella seguente tabella, in cui i campi "iniziali" fanno riferimento all'inizio dello sprint corrente:

\makeTablePreventivo

\subsubsection{Retrospettiva}
Il periodo mostra ancora gli effetti dell'assestamento iniziale e della sperimentazione con i nuovi
strumenti. È stata effettuata una sovrastima nell'assegnazione delle ore ai ruoli: Verificatore, Responsabile e Amministratore.

%================================
\vspace{1em}
\hrule
\vspace{1em}

\subsection{Sprint 2 (17/11/2025 - 30/11/2025)}
\subsubsection{Avanzamento}
\trisubsection{Attività programmate}
\begin{itemize}
    \item 
\end{itemize}

\trisubsection{Attività svolte}
Nel corso del primo periodo di avanzamento sono state svolte le seguenti attività:
\begin{itemize}
    \item 
\end{itemize}

\subsubsection{Rischi}
\trisubsection{Rischi attesi}
\begin{itemize}
    \item 
\end{itemize}
\trisubsection{Rishi manifestati}
\begin{itemize}
    \item 
\end{itemize}

\subsubsection{Costi}
Le ore effettive svolte dai singoli componenti del gruppo sono di seguito riportate:

\setRuoloFilippo{\Anal}
\setRuoloChristian{\Anal}
\setRuoloValeria{\Verif}
\setRuoloGiuseppe{\Resp}
\setRuoloFrancesco{\Amm}
\setRuoloLuca{\Anal}
\setRuoloLeonardo{\Proge}

\setHPrevisteFilippo{9}
\setHPrevisteChristian{9}
\setHPrevisteValeria{9}
\setHPrevisteGiuseppe{8}
\setHPrevisteFrancesco{8}
\setHPrevisteLuca{9}
\setHPrevisteLeonardo{12}

\setHRealiFilippo{9}
\setHRealiChristian{6}
\setHRealiValeria{9}
\setHRealiGiuseppe{8}
\setHRealiFrancesco{8}
\setHRealiLuca{9}
\setHRealiLeonardo{12}

\makeTableRuoli
Si deriva il consuntivo delle ore e dei costi rispetto ai singoli ruoli:
\makeTableDeviazioni

\subsubsection{Preventivo a finire}
Lo stato aggiornato delle risorse è calcolato nella seguente tabella, in cui i campi "iniziali" fanno riferimento all'inizio dello sprint corrente:

\makeTablePreventivo

\subsubsection{Retrospettiva}


%================================
\vspace{1em}
\hrule
\vspace{1em}

\subsection{Sprint 3 (01/12/2025 - 14/12/2025)}
\subsubsection{Avanzamento}
\trisubsection{Attività programmate}
Si prevede l'esecuzione delle seguenti attività:
\begin{itemize}
    \item
\end{itemize}

\trisubsection{Attività svolte}
Nel corso del primo periodo di avanzamento sono state svolte le seguenti attività:
\begin{itemize}
    \item 
\end{itemize}

\subsubsection{Rischi}
\trisubsection{Rischi attesi}
\begin{itemize}
    \item 
\end{itemize}
\trisubsection{Rishi manifestati}
\begin{itemize}
    \item 
\end{itemize}

\subsubsection{Costi}
Le ore effettive svolte dai singoli componenti del gruppo sono di seguito riportate:

\setRuoloFilippo{\Anal}
\setRuoloChristian{\Anal}
\setRuoloValeria{\Verif}
\setRuoloGiuseppe{\Resp}
\setRuoloFrancesco{\Amm}
\setRuoloLuca{\Anal}
\setRuoloLeonardo{\Proge}

\setHPrevisteFilippo{0}
\setHPrevisteChristian{0}
\setHPrevisteValeria{0}
\setHPrevisteGiuseppe{0}
\setHPrevisteFrancesco{0}
\setHPrevisteLuca{0}
\setHPrevisteLeonardo{0}

\setHRealiFilippo{0}
\setHRealiChristian{0}
\setHRealiValeria{0}
\setHRealiGiuseppe{0}
\setHRealiFrancesco{0}
\setHRealiLuca{0}
\setHRealiLeonardo{0}

\makeTableRuoli
Si deriva il consuntivo delle ore e dei costi rispetto ai singoli ruoli:
\makeTableDeviazioni

\subsubsection{Preventivo a finire}
Lo stato aggiornato delle risorse è calcolato nella seguente tabella, in cui i campi "iniziali" fanno riferimento all'inizio dello sprint corrente:

\makeTablePreventivo

%================================
\vspace{1em}
\hrule
\vspace{1em}

\subsection{Sprint 4 (15/12/2025 - 28/12/2025)}
\subsubsection{Avanzamento}
\trisubsection{Attività programmate}
Si prevede l'esecuzione delle seguenti attività:
\begin{itemize}
    \item terminazione Analisi Requisiti;
    \item continuazione stesura del Piano di Progetto;
    \item definizione metriche e prima stesura del Piano di Qualifica;
    \item inizio implementazione del Proof of Concept;
    \item definizione Gantt;
    \item correzioni documenti valutati Round Review;
    \item configurazione secondo kit.
\end{itemize}

\trisubsection{Attività svolte}
Nel corso del primo periodo di avanzamento sono state svolte le seguenti attività:
\begin{itemize}
    \item 
\end{itemize}

\subsubsection{Rischi}
\trisubsection{Rischi attesi}
\trisubsection{Rishi manifestati}

\subsubsection{Costi}
Le ore effettive svolte dai singoli componenti del gruppo sono di seguito riportate:

\setRuoloFilippo{\Anal}
\setRuoloChristian{\Anal}
\setRuoloValeria{\Verif}
\setRuoloGiuseppe{\Resp}
\setRuoloFrancesco{\Amm}
\setRuoloLuca{\Anal}
\setRuoloLeonardo{\Proge}

\setHPrevisteFilippo{0}
\setHPrevisteChristian{0}
\setHPrevisteValeria{0}
\setHPrevisteGiuseppe{0}
\setHPrevisteFrancesco{0}
\setHPrevisteLuca{0}
\setHPrevisteLeonardo{0}

\setHRealiFilippo{0}
\setHRealiChristian{0}
\setHRealiValeria{0}
\setHRealiGiuseppe{0}
\setHRealiFrancesco{0}
\setHRealiLuca{0}
\setHRealiLeonardo{0}

\makeTableRuoli
Si deriva il consuntivo delle ore e dei costi rispetto ai singoli ruoli:
\makeTableDeviazioni

\subsubsection{Preventivo a finire}
Lo stato aggiornato delle risorse è calcolato nella seguente tabella, in cui i campi "iniziali" fanno riferimento all'inizio dello sprint corrente:

\makeTablePreventivo

\subsubsection{Retrospettiva}


%================================
\vspace{1em}
\hrule
\vspace{1em}

\subsection{Sprint 5 (29/12/2025 - 11/01/2025)}
\subsubsection{Avanzamento}
\trisubsection{Attività programmate}
Si prevede l'esecuzione delle seguenti attività:
\begin{itemize}
    \item
\end{itemize}

\trisubsection{Attività svolte}
Nel corso del primo periodo di avanzamento sono state svolte le seguenti attività:
\begin{itemize}
    \item 
\end{itemize}

\subsubsection{Rischi}
\trisubsection{Rischi attesi}
\begin{itemize}
    \item 
\end{itemize}
\trisubsection{Rishi manifestati}
\begin{itemize}
    \item 
\end{itemize}

\subsubsection{Costi}
Le ore effettive svolte dai singoli componenti del gruppo sono di seguito riportate:

\setRuoloFilippo{\Anal}
\setRuoloChristian{\Anal}
\setRuoloValeria{\Verif}
\setRuoloGiuseppe{\Resp}
\setRuoloFrancesco{\Amm}
\setRuoloLuca{\Anal}
\setRuoloLeonardo{\Proge}

\setHPrevisteFilippo{0}
\setHPrevisteChristian{0}
\setHPrevisteValeria{0}
\setHPrevisteGiuseppe{0}
\setHPrevisteFrancesco{0}
\setHPrevisteLuca{0}
\setHPrevisteLeonardo{0}

\setHRealiFilippo{0}
\setHRealiChristian{0}
\setHRealiValeria{0}
\setHRealiGiuseppe{0}
\setHRealiFrancesco{0}
\setHRealiLuca{0}
\setHRealiLeonardo{0}

\makeTableRuoli
Si deriva il consuntivo delle ore e dei costi rispetto ai singoli ruoli:
\makeTableDeviazioni

\subsubsection{Preventivo a finire}
Lo stato aggiornato delle risorse è calcolato nella seguente tabella, in cui i campi "iniziali" fanno riferimento all'inizio dello sprint corrente:

\makeTablePreventivo



\end{document}