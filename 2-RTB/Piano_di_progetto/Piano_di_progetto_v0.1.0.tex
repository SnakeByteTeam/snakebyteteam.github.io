\documentclass[10pt, letterpaper]{article}

\usepackage{../../template/template}
\usepackage{tabularx}
\usepackage{eurosym}

%Queste macro sono in relazione all'ultima modifica
\setDocIndex{Piano di progetto}
\setDocTitle{Piano di progetto}
\setDocVersion{0.1.0}
\setDocState{1}
\setDocData{09/12/2025}
\setDocRecipient{1}

\setcounter{tocdepth}{4}
\setcounter{secnumdepth}{4}
\newcommand{\trisubsection}[1]{\paragraph{#1}\mbox{}\\}


\newcommand{\checkLevel}[1]{%
  \ifnum#1=0\relax
  \else\ifnum#1=1\relax
  \else\ifnum#1=2\relax
  \else
    \errmessage{Valore non valido (#1). Usare solo 0,1,2}%
  \fi\fi\fi
}

\newcommand{\convertLevel}[1]{%
  \ifnum#1=0 Basso\fi
  \ifnum#1=1 Medio\fi
  \ifnum#1=2 Alto\fi
}

\newcounter{rtctr}
\newcommand{\newTRisk}[5]{
    \checkLevel{#4}
    \checkLevel{#5}
    \stepcounter{rtctr}
    \subsubsection{Rischio RT\thertctr : #1}
    \begin{center}
        \begin{tabularx}{0.9\textwidth}{|l|X|}%0.9 è un po meglio per l'indentazione
        \hline
        \rowcolor{lightgray}
        \textbf{Attributo} & \textbf{Valore} \\
        \hline
        Codice & RT\thertctr \label{RT\thertctr} \\
        \hline
        Nome & #1 \\
        \hline
        Descrizione & #2 \\
        \hline
        Mitigazione & #3 \\
        \hline
        Livello di occorrenza & \convertLevel{#4} \\
        \hline
        Livello di impatto & \convertLevel{#5} \\
        \hline
        \end{tabularx}
    \end{center}
}

\newcounter{rictr}
\newcommand{\newIRisk}[5]{
    \checkLevel{#4}
    \checkLevel{#5}
    \stepcounter{rictr}
    \subsubsection{Rischio RI\therictr : #1}
    \begin{center}
        \begin{tabularx}{\textwidth}{|l|X|}
        \hline
        \rowcolor{lightgray}
        \textbf{Attributo} & \textbf{Valore} \\
        \hline
        Codice & RI\therictr \label{RI\therictr} \\
        \hline
        Nome & #1 \\
        \hline
        Descrizione & #2 \\
        \hline
        Mitigazione & #3 \\
        \hline
        Livello di occorrenza & \convertLevel{#4} \\
        \hline
        Livello di impatto & \convertLevel{#5} \\
        \hline
        \end{tabularx}
    \end{center}
}

\newcounter{rcctr}
\newcommand{\newCRisk}[5]{
    \checkLevel{#4}
    \checkLevel{#5}
    \stepcounter{rcctr}
    \subsubsection{Rischio RC\thercctr : #1}
    \begin{center}
        \begin{tabularx}{\textwidth}{|l|X|}
        \hline
        \rowcolor{lightgray}
        \textbf{Attributo} & \textbf{Valore} \\
        \hline
        Codice & RC\thercctr \label{RC\thercctr} \\
        \hline
        Nome & #1 \\
        \hline
        Descrizione & #2 \\
        \hline
        Mitigazione & #3 \\
        \hline
        Livello di occorrenza & \convertLevel{#4} \\
        \hline
        Livello di impatto & \convertLevel{#5} \\
        \hline
        \end{tabularx}
    \end{center}
}


\newcounter{roctr}
\newcommand{\newORisk}[5]{
    \checkLevel{#4}
    \checkLevel{#5}
    \stepcounter{roctr}
    \subsubsection{Rischio RO\theroctr : #1}
    \begin{center}
        \begin{tabularx}{\textwidth}{|l|X|}
        \hline
        \rowcolor{lightgray}
        \textbf{Attributo} & \textbf{Valore} \\
        \hline
        Codice & RO\theroctr \label{RO\theroctr} \\
        \hline
        Nome & #1 \\
        \hline
        Descrizione & #2 \\
        \hline
        Mitigazione & #3 \\
        \hline
        Livello di occorrenza & \convertLevel{#4} \\
        \hline
        Livello di impatto & \convertLevel{#5} \\
        \hline
        \end{tabularx}
    \end{center}
}




% comandi extra (rimuovere questi commenti una volta approvato il documento)
%
% -- costanti --
% \noOne [-]
% \noRole [ND]
% \firstDraft [Prima stesura]
% \approval [Approvazione]
% \teamName [SnakeByte]
% \teamEmail [snakebyteteam@gmail.com]
% \teamRepo [https://github.com/SnakeByteTeam/snakebyteteam.github.io]
% \teamSite [https://snakebyteteam.github.io/]
% \recipientTeachers [prof. Vardanega Tullio, prof. Cardin Riccardo]
% \vimar [Vimar]
% \vimarspa [Vimar S.p.A.]
%
% -- variabili --
% \docIndex | \setDocIndex{nome documento senza versione}
% \docTitle | \setDocTitle{titolo documento} 
% \docData | \setDocData{data documento} [GG/MM/AAAA]
% \docVersion | \setDocVersion{versione documento}
% \docState | \setDocState{0|1|2} [Da verificare][Verificato][Approvato]
% \docRecipient [table] | \setDocRecipient{0|1|2} [SnakeByte][SnakeByte, prof][SnakeByte, prof, vimar]
%
% -- pagine intere --
% \makeTitlePage
% \addChangelog{versione}{data}{autore}{verificatore}{approvatore}{descrizione} | \makeTodoTable

\begin{document}

\makeTitlePage

\addChangelog{0.1.0}{09/12/2025}{C. Libralato}{L. Pellizzon}{\noOne}{Prima stesura}
\makeChangelog

\newpage

\tableofcontents

\newpage

\section{Introduzione}
\subsection{Contenuto documento}
Il Piano di Progetto, nell'ambito dell'Ingegneria del Software, è un documento che ha lo scopo di definire in maniera chiara e dettagliata una struttura che funga 
da tabella di marcia per il progetto e da strumento di coordinazione per il team.

La suddetta struttura consiste in pianificazioni riguardo:
\begin{itemize}
    \item la dislocazione temporale dell'esecuzione delle attività di processo;
    \item i preventivi e le scadenze;
    \item la suddivisione di responsabilità dei ruoli;
    \item gestione dei rischi;
    \item la gestione delle risorse;
\end{itemize}

\subsubsection{Struttura del contenuto}
Il documento è suddiviso in quattro macrosezioni:
\begin{enumerate}
    \item \textbf{Introduzione}: sezione introduttiva che espone il contenuto e lo scopo del documento, oltre che i Riferimenti Normativi e Informativi.
    \item \textbf{Gestione rischi}: consiste in una classificazione e descrizione dei rischi che possono insorgere durante l'avanzamento del ciclo
    di vita del prodotto. Ad ogni rischio vengono associati comportamenti e regole sia per la sua prevenzione che per la mitigazione degli effetti in caso di occorrenza.
    \item \textbf{Previsioni a lungo termine}: presenta le stime iniziali eseguite in fase di Candidatura riguardo i costi, l'impegno orario e le responsabilità oltre che le stime future sugli
    stati di avanzamento previsti in vista delle scadenze quali RTB e PB.
    \item \textbf{Periodi di avanzamento}: elenco dei singoli periodi di avanzamento, definiti come \textit{sprint$_G$} di durata pari a due settimane, di cui si analizzano i seguenti aspetti:
    \begin{itemize}
        \item Consuntivo di periodo:
        \begin{itemize}
            \item avanzamento previsto e avanzamento effettivo;
            \item rischi attesi, rischi occorsi e relativa gestione;
            \item costi attesi ed costi effettivi.
        \end{itemize}
        \item Preventivo a finire:
        \begin{itemize}
            \item revisione calendario e risorse;
            \item retrospettiva.
        \end{itemize}
        
    \end{itemize}
\end{enumerate}


\subsection{Glossario}
All'interno del documento possono essere presenti termini il cui significato può risultare ambiguo o sconosciuto, essi sono
riportati in italico e alla loro prima occorrenza è associata una \textit{G} a pedice ad indicare la presenza del suddetto termine all'interno del \textit{glossario$_G$}, che è disponibile al seguente link: \href{https://snakebyteteam.github.io/glossary.html}{Glossario}.

\subsection{Riferimenti Normativi}
\begin{itemize}
    \item \textbf{Norme di Progetto}\\
    \url{link-finale-norme-di-progetto}
\end{itemize}

\subsection{Riferimenti Formativi}
\begin{itemize}
    \item \textbf{Regolamento del progetto didattico}\\
    \url{https://www.math.unipd.it/~tullio/IS-1/2025/Dispense/PD1.pdf}
    \item \textbf{Gestione di progetto}\\
    \url{https://www.math.unipd.it/~tullio/IS-1/2025/Dispense/T04.pdf}
\end{itemize}



\section{Analisi e gestione rischi}
    \subsection{Introduzione}
        Nella seguente sezione vengono analizzati i possibili rischi che potrebbero emergere durante il progetto e definite le modalità 
        per ridurne l'impatto. Sono state individuate cinque tipologie principali di rischio:
        \begin{itemize}
            \item \textbf{rischi tecnologici}: sono legati alla scelta e all'utilizzo delle tecnologie adottate durante lo sviluppo;
            \item \textbf{rischi di comunicazione}: sono causati da una mancanza di comunicazione tra parti del progetto, quali membri del gruppo, Proponente e professori;
            \item \textbf{rischi individuali}: emergono da imprevisti della vita quotidiana o impegni personali;
            \item \textbf{rischi organizzativi}: sono legati ad un'organizzazione errata o non adatta allo svolgimento di un'attività
            \item \textbf{rischi legati alle stime}: originati da previsioni che errate.
        \end{itemize}
    

    \subsection{Rischi tecnologici}
        \newTRisk{Incompatibilità tecnologie}{test}{test}{0}{0} % 0 | 1 | 2
        \newTRisk{Bug nel codice}{test}{test}{0}{0}
        \newTRisk{Test non efficaci}{test}{test}{0}{0}
    
    \subsection{Rischi individuali}
        \newIRisk{Impegni personali}{test}{test}{0}{0}
        \newIRisk{Impegni universitari}{test}{test}{0}{0}
    
    \subsection{Rischi di comunicazione}
        \newCRisk{Mancanza di coordinamento tra componenti}{test}{test}{0}{0}
        \newCRisk{Divergenza tra Proponente e gruppo}{test}{test}{0}{0}
        \newCRisk{Disparità di conoscenze tra componenti}{test}{test}{0}{0}
        \newCRisk{Incomprensione requisiti}{test}{test}{0}{0}

    \subsection{Rischi organizzativi}
        \newORisk{Carico di lavoro eccessivo}{test}{test}{0}{0}
        \newORisk{Inadempimento scadenza baseline}{test}{test}{0}{0}
        \newORisk{Inadempimento scadenza milestone}{test}{test}{0}{0}
        \newORisk{Inadempimento scadenza Proponente (? boh non so)}{test}{test}{0}{0}
        \newORisk{Inadempimento scadenza issue}{test}{test}{0}{0}
        \newORisk{Divergenza tra fase di sviluppo e progettazione}{test}{test}{0}{0}
        \newORisk{Sovrastima costi e ora}{test}{test}{0}{0}
        \newORisk{Sottostima costi e ora}{test}{test}{0}{0}


\section{Previsioni a lungo termine}

\section{Periodi di avanzamento}

\subsection{Sprint 1}
\subsubsection{Consuntivo}
\trisubsection{Avanzamento}
Lista di attività da svolgere (prese dai verbali) e attività svolgimento

\trisubsection{Rischi}
Tabella con indice rischio, descrizione contesto, mitigazione?
\begin{center}
    \begin{tabularx}{\textwidth}{|c|X|X|}
        \hline
        \rowcolor{lightgray} \textbf{Codice Richio} & \textbf{Contesto} & \textbf{Mitigazione}\\
        \hline
        RX &
        Durante la fase pre candidatura sono state cambiate le convenzioni della struttura di alcuni
        documenti già approvati blablabla è un esempio &
        Abbiamo modifciatou tute le cose e bleblebl loremi ipsum loremi ipsumloremi ipsumloremi ipsum
        loremi ipsumloremi ipsumloremi ipsumloremi ipsum
        loremi ipsum\\
        \hline
    \end{tabularx}
\end{center}

\trisubsection{Costi}
Tabella con confronto di ore previste e ora svolte
\begin{center}
    \begin{tabular}{|c|c|c|c|c|}
        \hline
        \rowcolor{lightgray} \textbf{Membro} & \textbf{Ruolo} & \textbf{Ore previste} & \textbf{Ore reali} & \textbf{Differenza} \\
        \hline
        V. Baleanu & & & & \\
        \hline
        L. Pellizzon & & & & \\
        \hline
        F. Venzo & & & & \\
        \hline
        G. De Fina & & & & \\
        \hline
        F. Pasqual & & & & \\
        \hline
        C. Libralato & & & & \\
        \hline
        L. Granziero & & & & \\
        \hline
    \end{tabular}
\end{center}

\subsubsection{Preventivo a finire}
\trisubsection{Revisione}
Aggiornamento delle ore rimanenti
\begin{center}
    \begin{tabular}{|c|c|c|c|c|c|c|c|}
        \hline
        \rowcolor{lightgray} \rotatebox[origin=c]{55}{\textbf{Ruolo}} & \rotatebox[origin=c]{55}{\textbf{Costo orario}} 
        & \rotatebox[origin=c]{55}{\textbf{Ore previste}} & \rotatebox[origin=c]{55}{\textbf{Ore effettive}} 
        & \rotatebox[origin=c]{55}{\textbf{Costo previsto}} & \rotatebox[origin=c]{55}{\textbf{Costo effettivo}} 
        & \rotatebox[origin=c]{55}{\textbf{Ore rimanenti}} & \rotatebox[origin=c]{55}{\textbf{Budget rimanente}} \\
        \hline
        Responsabile & 30 \euro/h & 0 & 0 & 0 \euro & 0 \euro & 0 &  0 \euro\\
        \hline
        Amministratore & 20 \euro/h & 0 & 0 & 0 \euro & 0 \euro & 0 &  0 \euro\\
        \hline
        Analista & 25 \euro/h & 0 & 0 & 0 \euro & 0 \euro & 0 &  0 \euro\\
        \hline
        Progettista & 25 \euro/h & 0 & 0 & 0 \euro & 0 \euro & 0 &  0 \euro\\
        \hline
        Programmatore & 15 \euro/h & 0 & 0 & 0 \euro & 0 \euro & 0 &  0 \euro\\
        \hline
        Verificatore & 15 \euro/h & 0 & 0 & 0 \euro & 0 \euro & 0 &  0 \euro\\
        \hline
    \end{tabular}
\end{center}

\trisubsection{Retrospettiva}

\subsection{Sprint2 ...}



\end{document}