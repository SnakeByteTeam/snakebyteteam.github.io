\documentclass[10pt, letterpaper]{article}
\usepackage[nomarginpar, margin=2.75cm, tmargin=3cm, bmargin=1.75cm]{geometry}
\usepackage[
    colorlinks=true,      
    linkcolor=black,      
    urlcolor=blue,       
    citecolor=black       
]{hyperref}
\usepackage{graphicx}
\usepackage[table, x11names]{xcolor}
\usepackage{tabularx}
\usepackage{lastpage} 
\renewcommand{\arraystretch}{1.2} % migliora la leggibilità
\renewcommand{\contentsname}{Indice}
\usepackage{fancyhdr}
%Comandi per livello di sottosezioni = 3
\setcounter{tocdepth}{4}
\setcounter{secnumdepth}{4}
\newcommand{\trisubsection}[1]{\paragraph{#1}\mbox{}\\}

\pagestyle{fancy}
\fancyhf{}
\fancyhead[L]{SnakeByte} 
\fancyhead[R]{Analisi dei Requisiti}
\fancyfoot[C]{Pagina \thepage\ di \pageref{LastPage}}


\begin{document}

\begin{titlepage}
    \begin{center}
        \begin{center}
            \includegraphics[width=0.6\textwidth]{./img/logo.pdf}
        \end{center}
        \vspace{4cm}
        \huge\textbf{Analisi dei requisiti}\par
        \vspace{2cm}
        \large \textbf{SnakeByte} (Gruppo 1):\\
        \large Valeria Baleanu, Leonardo Pellizzon, Filippo Venzo, Giuseppe De Fina, \\
         Francesco Pasqual, Christian Libralato, Luca Granziero \\
        (2109911, 2111006, 2113705, 2113187, 2103119, 2101047, 2075512)
        \vfill
        \small
        \begin{center}
            \begin{tabular}{|c|c|c|c|}
                \hline
                \multicolumn{4}{|c|}{\textbf{Informazioni documento}} \\
                \hline
                \rowcolor{lightgray} \textbf{Versione} & \textbf{Data} & \textbf{Stato} & \textbf{Destinatari} \\
                \hline
                0.1.2 & 24/11/2025 & Modificato & prof. Vardanega Tullio, prof. Cardin Riccardo, Vimar S.p.A. \\
                \hline
            \end{tabular}
        \end{center}
        \vfill
        \large Contatti: snakebyteteam@gmail.com
    \end{center}
\end{titlepage}

\begin{center}
    \begin{tabularx}{\textwidth}{|c|c|c|c|c|X|}
        \hline
        \multicolumn{6}{|c|}{\textbf{Registro delle modifiche}} \\
        \hline
        \rowcolor{lightgray} \textbf{Versione} & \textbf{Data} & \textbf{Autore} & \textbf{Verificatore} & \textbf{Approvatore} & \textbf{Descrizione} \\
        \hline
        0.1.2 & 24/10/2025 & F. Venzo & - & - & Aggiunta Use Cases UC4, UC4.1 \\
        \hline
        0.1.1 & 20/10/2025 & F. Venzo & V. Baleanu & - & Aggiunta Use Cases UC1, UC1.1, UC1.2, UC1.3, UC2, UC2.1, UC2.2, UC2.3, UC2.4, UC2.5, UC2.6, UC2.7, UC3 \\
        \hline
        0.1.0 & 05/10/2025 & L. Granziero & L. Pellizon & - & Prima Stesura \\
        \hline
    \end{tabularx}
\end{center}

\newpage

\tableofcontents

\newpage
\section{Introduzione}
\subsection{Finalità del documento}{
Il documento di analisi dei requisiti, in un contesto di ingegneria del software, ha lo scopo fondamentale di tradurre l'esigenza dell'utenza e degli stakeholder, in questo caso la proponente Vimar S.p.A., in una specifica completa, coerente e verificabile di requisiti, destinata a guidare le fasi di progettazione e sviluppo, verifica e infine validazione del sistema.

\noindent
\\
\textbf{Definire chiaramente “cosa” il sistema deve fare e “in quali condizioni”:}
\begin{itemize}
    \item L’analisi dei requisiti serve a identificare le funzioni (requisiti funzionali) e le qualità (requisiti non funzionali: prestazioni, usabilità, affidabilità, portabilità...) del software.
    \item Definisce i limiti del sistema e i vincoli (tecnici, di interfaccia...)
    \item Permette di evitare ambiguità o fraintendimenti sulle funzionalità richieste. 
\end{itemize}

\noindent
\textbf{Allineare tutti gli stakeholder su un linguaggio comune e condividere le aspettative:}
\begin{itemize}
    \item Questo documento funge da contratto tra cliente/committente e il team di sviluppo: specifica ciò che sarà consegnato.
    \item Aiuta a garantire che utenti, committenti, analisti, progettisti e tester abbiano la stessa comprensione del sistema.
\end{itemize}

\noindent
\textbf{Fornire una base stabile per le fasi successive del ciclo di vita del software:}
\begin{itemize}
    \item Il documento di analisi dei requisiti serve come input per la progettazione del sistema, per la pianificazione dello sviluppo e per la pianificazione dei test.
    \item Serve anche come base per la verifica e la convalida: si può usare come riferimento per capire se il prodotto finale soddisfa i requisiti richiesti. 
\end{itemize}
\noindent
\textbf{Gestire i rischi e controllare le modifiche:}
\begin{itemize}
    \item Durante l’analisi dei requisiti si identificano requisiti non realizzabili, conflitti tra requisiti, omissioni e incoerenze. Ciò consente di ridurre i rischi fin dalle prime fasi.
    \item Aiuta a limitare il fenomeno dello “scope creep” (ovvero l’aggiunta non controllata di funzionalità) e a mantenere il controllo sul cambiamento dei requisiti. 
\end{itemize}
}

\subsection{Sviluppo del documento}{
Il presente documento è stato sviluppato in modo graduale e incrementale, con lo scopo di facilitare modifiche future in base alle esigenze che verranno concordate tra il gruppo e l'azienda committente. Il documento è quindi soggetto a un processo di miglioramento continuo nel tempo.
}

\subsection{Riferimenti}

\subsubsection{Riferimenti Normativi}
\begin{itemize}
    \item \textbf{Norme di Progetto}: 
    
    \url{link-alle-norme-di-progetto} 
    
    (consultato il 30/10/2025);
    \item \textbf{Vimar View4Life Capitolato di Ingegneria del Software Università di Padova 2025 - 2026:}
    
    \url{https://www.math.unipd.it/~tullio/IS-1/2025/Progetto/C9.pdf} 
    
    (consultato il 20/10/2025).
\end{itemize}

\subsubsection{Riferimenti Informativi}
\begin{itemize}
    \item \textbf{830-1998 - IEEE Recommended Practice for Software Requirements Specifications}
    
    \url{https://ieeexplore.ieee.org/document/720574} 
    
    (consultato il 20/10/2025).

    \item \textbf{Diagrammi Use Case - Riccardo Cardin}
    
    \url{https://www.math.unipd.it/~rcardin/swea/2022/Diagrammi%20Use%20Case.pdf} 
    
    (consultato il 18/10/2025).
\end{itemize}

\section{Descrizione del prodotto}
\subsection{Prospettiva del prodotto}
La prospettiva del prodotto è un sistema domotico integrato per anziani autosufficienti che si basa su dispositivi Vimar con connessione mesh Bluetooth.
Gli obiettivi principali sono la sicurezza e il comfort delle persone occupanti, l'aumento dell'efficienza energetica della struttura e la semplificazione della gestione operativa dell'impianto elettrico.
Tali risultati vengono raggiunti attraverso la centralizzazione del controllo di illuminazione, temperatura, televisione e dispositivi di sicurezza mediante l'app View e i relativi servizi cloud, con la possibilità di controllo da remoto da parte del personale medico.

\subsection{Obiettivi del prodotto}{
Il progetto consiste nella realizzazione di una piattaforma unica \textit{View4Life} per la gestione intelligente degli impianti \textit{Smart} nelle residenze protette per anziani, sfruttando i dispositivi domotici Vimar connessi in rete \textit{mesh Bluetooth} tramite l’\textit{API KNX IoT 3rd-party$_{G}$}. Questa soluzione mira a supportare il lavoro del personale sanitario fornendo uno strumento che integri un sistema di gestione degli allarmi (come il rilevamento di cadute o presenze prolungate in determinate stanze) per garantire un intervento rapido e tempestivo. Inoltre, la piattaforma è progettata per permettere il monitoraggio del consumo energetico e la rilevazione di anomalie nell’impianto.
}

\subsection{Funzionalità del prodotto}
Dal punto di vista degli utenti del personale sanitario l'applicativo svolge le seguenti funzioni:
\begin{itemize}
    \item Visualizzazione delle informazioni generali di allarmi, statistiche ed analitiche tramite cruscotto riassuntivo (Dashboard);
    \item Possibilità di essere notificati, visualizzare e gestire gli allarmi attivi;
    \item Possibilità di visualizzare e gestire i dispositivi dei vari impianti collocati in diverse residenze;
    \item Possibilità di visualizzare statistiche, tramite grafici, sui consumi dell'impianto, sulla variazione di temperatura e sugli allarmi passati;
    \item Possibilità di ricevere consigli, basati sulle statistiche, per ridurre i consumi energetici.
\end{itemize}

\subsection{Utenza di riferimento}{
Il prodotto si rivolge principalmente a quattro categorie principali di utenti, descritte di seguito:

\begin{itemize}
    \item \textbf{Personale medico e operatori sanitari} che utilizzano l'applicazione per monitorare lo stato di ambienti e dispositivi, oltre a ricevere notifiche di allarme e gestire da remoto funzioni come temperatura, illuminazione e sicurezza delle stanze degli ospiti presenti. 
    \item \textbf{Personale amministrativo} che si fa carico della configurazione e manutenzione del sistema e del monitoraggio dei consumi elettrici, e quindi a una conseguente ottimizzazione dell'efficienza dell'impianto elettrico.
\end{itemize}
Questa sezione evidenzia come il sistema domotico proposto miri a centralizzare il controllo e la gestione degli impianti all’interno della residenza, semplificando le operazioni quotidiane del personale e migliorando la qualità della vita degli ospiti.
Attraverso l’app View e i servizi cloud Vimar, l’applicazione permette di gestire in modo integrato illuminazione, temperatura e sicurezza, contribuendo a maggior efficienza energetica, sicurezza e comfort abitativo per tutti gli utenti coinvolti.
}





\section{Casi d'uso}
Un \textit{caso d'uso$_{G}$} è la descrizione dettagliata, tramite \textit{diagramma UML$_G$} e descrizione testuale, di un insieme di scenari che hanno uno scopo comune, all'interno del Sistema, per un attore.
Permettono di comprendere al meglio le funzionalità che devono essere rese disponibili dal Sistema \textit{software}.

In particolare, le descrizioni dei casi d'uso contenute in questo documento conterranno le informazioni riportate nella seguente tabella:

\begin{center}
    \begin{tabularx}{\textwidth}{|c| >{\centering\arraybackslash}X|}
        \hline
        \rowcolor{lightgray} \textbf{Campo} & \textbf{Descrizione} \\
        \hline
        Attori & Coloro che partecipano attivamente al caso d'uso per raggiungere un preciso obiettivo  \\
        \hline
        Pre-condizioni & Condizioni che devono essere soddisfatte prima dello scenario descritto dal caso d'uso\\
        \hline
        Post-condizioni & Condizioni che risultano soddisfatte dopo il completamento dello scenario principale del caso d'uso. Se viene completato uno scenario alternativo, saranno soddisfatte le Post-condizioni di quest'ultimo \\
        \hline
        Trigger & La motivazione che porta l'utente a svolgere i passi del caso d'uso \\
        \hline
        Scenario principale & Sequenza di passi che l'utente deve seguire per completare il caso d'uso \\
        \hline
        Scenari alternativi & Scenario divergente dal principale per il verificarsi di una particolare condizione \\
        \hline
        Estensioni &  Casi d'uso ulteriori eseguiti al verificarsi di una particolare condizione nel caso d'uso primario. Modificano Scenario e Post-condizioni \\
        \hline
        Inclusioni & Casi d'uso ulteriori eseguiti al fine di completare il caso d'uso principale. Vengono eseguiti tutti incondizionatamente. \\
        \hline
        
    \end{tabularx}
\end{center}

Non tutti gli attributi sono necessari per ogni caso d'uso. Nel caso in cui un campo sia assente in un caso d'uso, allora tale sarà assente anche nella sua descrizione e nel suo diagramma UML.

\subsection{Attori}
Di seguito vengono riportati gli attori individuati 

\begin{center}
    \begin{tabularx}{\textwidth}{|c| >{\centering\arraybackslash}X|}
        \hline
        \rowcolor{lightgray} \textbf{Attore} & \textbf{Descrizione} \\
        \hline
        Utente personale sanitario & Rappresenta il personale sanitario. \\
        \hline         
    \end{tabularx}
\end{center}

\subsection{Lista dei casi d'uso}

\subsubsection{UC1: Autenticazione}

\begin{center}
    \includegraphics[width=0.6\textwidth]{./img/UC1.pdf}
\end{center}

\begin{itemize}
    \item \textbf{Attore principale}: Utente personale sanitario

    \item \textbf{Pre-condizioni}: 
        \begin{itemize}
            \item Il Sistema è attivo
            \item L'utente del personale sanitario non è autenticato nel Sistema.
        \end{itemize}

    \item \textbf{Post-condizioni}:
        \begin{itemize}
            \item L'utente del personale sanitario è autenticato nel Sistema.
        \end{itemize}

    \item \textbf{Trigger}: L'utente del personale sanitario vuole autenticarsi nel Sistema;
    
    \item \textbf{Scenario principale}:
        \begin{itemize}
            \item L'utente del personale sanitario inserisce il proprio Username;
            \item L'utente del personale sanitario inserisce la propria Password.
        \end{itemize}
    
    \item \textbf{Inclusioni}:
        \begin{itemize}
            \item \hyperref[UC1.1]{§UC1.1}
            \item \hyperref[UC1.2]{§UC1.2}
        \end{itemize}

    \item \textbf{Scenari alternativi}:
        \begin{itemize}
            \item L'utente del personale sanitario inserisce Username o Password errate.
        \end{itemize}

    \item \textbf{Estensioni}:
        \begin{itemize}
            \item \hyperref[UC1.3]{§UC1.3}
        \end{itemize}
    
\end{itemize}


\trisubsection{UC1.1 Inserimento Username} \label{UC1.1}
\texttt{<diagramma d'uso>}

\begin{itemize}
    \item \textbf{Attore principale}: Utente personale sanitario

    \item \textbf{Pre-condizioni}: 
        \begin{itemize}
            \item Il Sistema è attivo;
            \item L'utente del personale sanitario non è autenticato nel Sistema;
            \item L'utente del personale sanitario ha selezionato l'opzione di inserimento dell'Username;
            \item Il Sistema non conosce l'Username dell'utente del personale sanitario.
        \end{itemize}

    \item \textbf{Post-condizioni}:
        \begin{itemize}
            \item Il Sistema conosce l'Username dell'utente del personale sanitario.
        \end{itemize}

    \item \textbf{Trigger}: L'utente del personale sanitario vuole autenticarsi nel Sistema;
    
    \item \textbf{Scenario principale}:
        \begin{itemize}
            \item L'utente del personale sanitario inserisce il proprio Username;
        \end{itemize}
    
\end{itemize}

\trisubsection{UC1.2 Inserimento Password} \label{UC1.2}
\texttt{<diagramma d'uso>}

\begin{itemize}
    \item \textbf{Attore principale}: Utente personale sanitario

    \item \textbf{Pre-condizioni}: 
        \begin{itemize}
            \item Il Sistema è attivo;
            \item L'utente del personale sanitario non è autenticato nel Sistema;
            \item L'utente del personale sanitario ha selezionato l'opzione di inserimento della Password;
            \item Il Sistema non conosce la Password dell'utente del personale sanitario.
        \end{itemize}

    \item \textbf{Post-condizioni}:
        \begin{itemize}
            \item Il Sistema conosce la Password dell'utente del personale sanitario.
        \end{itemize}

    \item \textbf{Trigger}: L'utente del personale sanitario vuole autenticarsi nel Sistema;
    
    \item \textbf{Scenario principale}:
        \begin{itemize}
            \item L'utente del personale sanitario inserisce la propria Password;
        \end{itemize}
    
\end{itemize}

\trisubsection{UC1.3 Autenticazione fallita} \label{UC1.3}
\texttt{<diagramma d'uso>}

\begin{itemize}
    \item \textbf{Attore principale}: Utente personale sanitario

    \item \textbf{Pre-condizioni}: 
        \begin{itemize}
            \item Il Sistema è attivo;
            \item L'utente del personale sanitario non è autenticato nel Sistema;
            \item L'utente del personale sanitario ha immesso Username o Password errati;
        \end{itemize}

    \item \textbf{Post-condizioni}:
        \begin{itemize}
            \item Il Sistema segnala l'errore di autenticazione; 
            \item L'utente del personale sanitario non è autenticato.
        \end{itemize}

    \item \textbf{Trigger}: 
    \begin{itemize}
        \item L'utente del personale sanitario ha immesso Username errato;
        \item L'utente del personale sanitario ha immesso Password errata;
    \end{itemize}
    
    
    \item \textbf{Scenario principale}:
        \begin{itemize}
            \item Il Sistema non autentica l'utente del personale sanitario;
        \end{itemize}
    
\end{itemize}

\subsubsection{UC2 Visualizza Dashboard} \label{UC2}
\texttt{<diagramma d'uso>}

\begin{itemize}
    \item \textbf{Attore principale}: Utente personale sanitario

    \item \textbf{Pre-condizioni}: 
        \begin{itemize}
            \item Il Sistema è attivo;
            \item L'utente del personale sanitario è autenticato nel Sistema;
        \end{itemize}

    \item \textbf{Post-condizioni}:
        \begin{itemize}
            \item L'utente del personale sanitario visualizza la Dashboard riassuntiva.
        \end{itemize}

    \item \textbf{Trigger}: L'utente del personale sanitario vuole vedere la Dashboard riassuntiva.
    
    \item \textbf{Scenario principale}:
        \begin{itemize}
            \item L'utente del personale sanitario seleziona il menù relativo alla Dashboard. 
        \end{itemize}
    
\end{itemize}


\trisubsection{UC2.1 Visualizza sezione Allarmi attivi Dashboard} \label{UC2.1}
\texttt{<diagramma d'uso>}

\begin{itemize}
    \item \textbf{Attore principale}: Utente personale sanitario

    \item \textbf{Pre-condizioni}: 
        \begin{itemize}
            \item Il Sistema è attivo;
            \item L'utente del personale sanitario è autenticato nel Sistema;
        \end{itemize}

    \item \textbf{Post-condizioni}:
        \begin{itemize}
            \item L'utente del personale sanitario visualizza la sezione riassuntiva degli allarmi attivi nella Dashboard.
        \end{itemize}

    \item \textbf{Trigger}: L'utente del personale sanitario vuole vedere la sezione riassuntiva degli allarmi attivi nella Dashboard.
    
    \item \textbf{Scenario principale}:
        \begin{itemize}
            \item L'utente del personale sanitario visualizza la Dashboard (Vedi \hyperref[UC2]{§UC2})
            \item L'utente sanitario visualizza la sezione riassuntiva degli allarmi attivi nella Dashboard
        \end{itemize}
    
    \item \textbf{Inclusioni}
        \begin{itemize}
            \item \hyperref[UC2]{§UC2}
        \end{itemize}
    
\end{itemize}

\trisubsection{UC2.2 Visualizza sezione Statistiche allarmi Dashboard} \label{UC2.2}
\texttt{<diagramma d'uso>}

\begin{itemize}
    \item \textbf{Attore principale}: Utente personale sanitario

    \item \textbf{Pre-condizioni}: 
        \begin{itemize}
            \item Il Sistema è attivo;
            \item L'utente del personale sanitario è autenticato nel Sistema;
        \end{itemize}

    \item \textbf{Post-condizioni}:
        \begin{itemize}
            \item L'utente del personale sanitario visualizza la sezione riassuntiva delle statistiche allarmi nella Dashboard.
        \end{itemize}

    \item \textbf{Trigger}: L'utente del personale sanitario vuole vedere la sezione riassuntiva delle statistiche allarmi nella Dashboard.
    
    \item \textbf{Scenario principale}:
        \begin{itemize}
            \item L'utente del personale sanitario visualizza la Dashboard (Vedi \hyperref[UC2]{§UC2})
            \item L'utente sanitario visualizza la sezione riassuntiva degli statistiche allarmi nella Dashboard
        \end{itemize}
    
    \item \textbf{Inclusioni}
        \begin{itemize}
            \item \hyperref[UC2]{§UC2}
        \end{itemize}
\end{itemize}

\trisubsection{UC2.3 Visualizza sezione Informazioni utente Dashboard} \label{UC2.3}
\texttt{<diagramma d'uso>}

\begin{itemize}
    \item \textbf{Attore principale}: Utente personale sanitario

    \item \textbf{Pre-condizioni}: 
        \begin{itemize}
            \item Il Sistema è attivo;
            \item L'utente del personale sanitario è autenticato nel Sistema;
        \end{itemize}

    \item \textbf{Post-condizioni}:
        \begin{itemize}
            \item L'utente del personale sanitario visualizza la sezione delle informazioni utente nella Dashboard.
        \end{itemize}

    \item \textbf{Trigger}: L'utente del personale sanitario vuole vedere la sezione delle informazioni utente nella Dashboard.
    
    \item \textbf{Scenario principale}:
        \begin{itemize}
            \item L'utente del personale sanitario visualizza la Dashboard (Vedi \hyperref[UC2]{§UC2})
            \item L'utente sanitario visualizza la sezione delle informazioni utente nella Dashboard
        \end{itemize}
    
    \item \textbf{Inclusioni}
        \begin{itemize}
            \item \hyperref[UC2]{§UC2}
        \end{itemize}
\end{itemize}

\trisubsection{UC2.4 Visualizza sezione Analytics Dashboard} \label{UC2.4}
\texttt{<diagramma d'uso>}

\begin{itemize}
    \item \textbf{Attore principale}: Utente personale sanitario

    \item \textbf{Pre-condizioni}: 
        \begin{itemize}
            \item Il Sistema è attivo;
            \item L'utente del personale sanitario è autenticato nel Sistema;
        \end{itemize}

    \item \textbf{Post-condizioni}:
        \begin{itemize}
            \item L'utente del personale sanitario visualizza la sezione Analytics nella Dashboard.
        \end{itemize}

    \item \textbf{Trigger}: L'utente del personale sanitario vuole vedere la sezione Analytics nella Dashboard.
    
    \item \textbf{Scenario principale}:
        \begin{itemize}
            \item L'utente del personale sanitario visualizza la Dashboard (Vedi \hyperref[UC2]{§UC2})
            \item L'utente sanitario visualizza la sezione Analytics nella Dashboard
        \end{itemize}
    
    \item \textbf{Inclusioni}
        \begin{itemize}
            \item \hyperref[UC2]{§UC2}
        \end{itemize}
\end{itemize}

\trisubsection{UC2.5 Visualizza informazioni allarme attivo} \label{UC2.5}
\texttt{<diagramma d'uso>}

\begin{itemize}
    \item \textbf{Attore principale}: Utente personale sanitario

    \item \textbf{Pre-condizioni}: 
        \begin{itemize}
            \item Il Sistema è attivo;
            \item L'utente del personale sanitario è autenticato nel Sistema;
            \item Il Sistema ha segnalato almeno un allarme.
        \end{itemize}

    \item \textbf{Post-condizioni}:
        \begin{itemize}
            \item L'utente del personale sanitario visualizza le informazioni sull'allarme attivo.
        \end{itemize}

    \item \textbf{Trigger}: L'utente del personale sanitario vuole visualizzare la informazioni dell'allarme attivo.
    
    \item \textbf{Scenario principale}:
        \begin{itemize}
            \item L'utente del personale sanitario visualizza la Dashboard (Vedi \hyperref[UC2]{§UC2});
            \item L'utente del personale sanitario visualizza la sezione allarmi attivi nella Dashboard (Vedi \hyperref[UC2.1]{§UC2.1});
            \item L'utente del personale sanitario seleziona l'allarme attivo di cui vuole visualizzare le informazioni.
        \end{itemize}
    
    \item \textbf{Inclusioni}
        \begin{itemize}
            \item \hyperref[UC2]{§UC2};
            \item \hyperref[UC2.1]{§UC2.1}.
        \end{itemize}
\end{itemize}

\trisubsection{UC2.6 Presa in carico allarme attivo} \label{UC2.6}
\texttt{<diagramma d'uso>}

\begin{itemize}
    \item \textbf{Attore principale}: Utente personale sanitario

    \item \textbf{Pre-condizioni}: 
        \begin{itemize}
            \item Il Sistema è attivo;
            \item L'utente del personale sanitario è autenticato nel Sistema;
            \item Il Sistema ha segnalato almeno un allarme.
        \end{itemize}

    \item \textbf{Post-condizioni}:
        \begin{itemize}
            \item L'utente del personale sanitario prende in carico un allarme attivo.
        \end{itemize}

    \item \textbf{Trigger}: L'utente del personale sanitario vuole prendere in carico un allarme attivo.
    
    \item \textbf{Scenario principale}:
        \begin{itemize}
            \item L'utente del personale sanitario visualizza la Dashboard (Vedi \hyperref[UC2]{§UC2});
            \item L'utente del personale sanitario visualizza la sezione allarmi attivi nella Dashboard (Vedi \hyperref[UC2.1]{§UC2.1});
            \item L'utente del personale sanitario seleziona l'allarme attivo di cui vuole visualizzare le informazioni (Vedi \hyperref[UC2.5]{§UC2.5});
            \item L'utente del personale sanitario seleziona l'allarme attivo di cui vuole prendere carico.
        \end{itemize}
    
    \item \textbf{Inclusioni}
        \begin{itemize}
            \item \hyperref[UC2]{§UC2};
            \item \hyperref[UC2.1]{§UC2.1};
            \item \hyperref[UC2.5]{§UC2.5}.
        \end{itemize}
\end{itemize}

\trisubsection{UC2.7 Risolvi allarme attivo} \label{UC2.7}
\texttt{<diagramma d'uso>}

\begin{itemize}
    \item \textbf{Attore principale}: Utente personale sanitario

    \item \textbf{Pre-condizioni}: 
        \begin{itemize}
            \item Il Sistema è attivo;
            \item L'utente del personale sanitario è autenticato nel Sistema;
            \item Il Sistema ha segnalato almeno un allarme;
            \item L'utente del personale sanitario ha preso carico di almeno un'allarme.
        \end{itemize}

    \item \textbf{Post-condizioni}:
        \begin{itemize}
            \item L'utente del personale sanitario risolve un allarme attivo.
        \end{itemize}

    \item \textbf{Trigger}: L'utente del personale sanitario vuole risolvere un allarme attivo.
    
    \item \textbf{Scenario principale}:
        \begin{itemize}
            \item L'utente del personale sanitario visualizza la Dashboard (Vedi \hyperref[UC2]{§UC2});
            \item L'utente del personale sanitario visualizza la sezione allarmi attivi nella Dashboard (Vedi \hyperref[UC2.1]{§UC2.1});
            \item L'utente del personale sanitario seleziona l'opzione di risoluzione, nella sezione degli allarmi attivi nella Dashboard, di un'allarme a suo carico.
        \end{itemize}
    
    \item \textbf{Inclusioni}
        \begin{itemize}
            \item \hyperref[UC2]{§UC2};
            \item \hyperref[UC2.1]{§UC2.1};
        \end{itemize}
\end{itemize}

\trisubsection{UC2.8 Visualizza notifica Allarme attivo} \label{UC2.8}
\texttt{<diagramma d'uso>}

\begin{itemize}
    \item \textbf{Attore principale}: Utente personale sanitario

    \item \textbf{Pre-condizioni}: 
        \begin{itemize}
            \item Il Sistema è attivo;
            \item L'utente del personale sanitario è autenticato nel Sistema;
            \item Il Sistema ha segnalato almeno un allarme.
        \end{itemize}

    \item \textbf{Post-condizioni}:
        \begin{itemize}
            \item L'utente del personale sanitario visualizza la notifica di un allarme attivo.
        \end{itemize}

    \item \textbf{Trigger}: Il Sistema segnala, tramite notifica, l'attivazione di un allarme.
    
    \item \textbf{Scenario principale}:
        \begin{itemize}
            \item L'utente del personale sanitario visualizza la Dashboard (Vedi \hyperref[UC2]{§UC2});
            \item L'utente del personale sanitario riceve la notifica Allarme attivo;
            \item L'utente del personale sanitario visualizza la notifica Allarmo attivo.
        \end{itemize}
    
    \item \textbf{Inclusioni}
        \begin{itemize}
            \item \hyperref[UC2]{§UC2};
        \end{itemize}
\end{itemize}

\subsubsection{UC3 Visualizza Gestione allarmi} \label{UC3}
\texttt{<diagramma d'uso>}

\begin{itemize}
    \item \textbf{Attore principale}: Utente personale sanitario

    \item \textbf{Pre-condizioni}: 
        \begin{itemize}
            \item Il Sistema è attivo;
            \item L'utente del personale sanitario è autenticato nel Sistema;
        \end{itemize}

    \item \textbf{Post-condizioni}:
        \begin{itemize}
            \item L'utente del personale sanitario visualizza la sezione Gestione allarmi.
        \end{itemize}

    \item \textbf{Trigger}: L'utente del personale sanitario vuole vedere la sezione dedicata alla gestione degli allarmi.
    
    \item \textbf{Scenario principale}:
        \begin{itemize}
            \item L'utente del personale sanitario seleziona il menù relativo alla Gestione allarmi. 
        \end{itemize}
\end{itemize}

\trisubsection{UC3.1 Visualizza allarmi attivi} \label{UC3.1}
\texttt{<diagramma d'uso>}

\begin{itemize}
    \item \textbf{Attore principale}: Utente personale sanitario

    \item \textbf{Pre-condizioni}: 
        \begin{itemize}
            \item Il Sistema è attivo;
            \item L'utente del personale sanitario è autenticato nel Sistema;
        \end{itemize}

    \item \textbf{Post-condizioni}:
        \begin{itemize}
            \item L'utente del personale sanitario visualizza gli allarmi attivi nella sezione Gestione allarmi.
        \end{itemize}

    \item \textbf{Trigger}: L'utente del personale sanitario vuole vedere gli allarmi attivi nella sezione dedicata alla gestione degli allarmi.
    
    \item \textbf{Scenario principale}:
        \begin{itemize}
            \item L'utente del personale sanitario visualizza la Gestione allarmi (Vedi \hyperref[UC3]{§UC3}) 
            \item L'utente del personale sanitario visualizza gli allarmi attivi nella Gestione allarmi
        \end{itemize}

    \item \textbf{Inclusioni}
        \begin{itemize}
            \item \hyperref[UC3]{§UC3}
        \end{itemize}
\end{itemize}

\trisubsection{UC3.2 Visualizza allarmi passati} \label{UC3.2}
\texttt{<diagramma d'uso>}

\begin{itemize}
    \item \textbf{Attore principale}: Utente personale sanitario

    \item \textbf{Pre-condizioni}: 
        \begin{itemize}
            \item Il Sistema è attivo;
            \item L'utente del personale sanitario è autenticato nel Sistema;
        \end{itemize}

    \item \textbf{Post-condizioni}:
        \begin{itemize}
            \item L'utente del personale sanitario visualizza gli allarmi passati nella sezione Gestione allarmi.
        \end{itemize}

    \item \textbf{Trigger}: L'utente del personale sanitario vuole vedere gli allarmi passati nella sezione dedicata alla gestione degli allarmi.
    
    \item \textbf{Scenario principale}:
        \begin{itemize}
            \item L'utente del personale sanitario visualizza la Gestione allarmi (Vedi \hyperref[UC3]{§UC3}) 
            \item L'utente del personale sanitario visualizza gli allarmi passati nella Gestione allarmi
        \end{itemize}

    \item \textbf{Inclusioni}
        \begin{itemize}
            \item \hyperref[UC3]{§UC3}
        \end{itemize}
\end{itemize}

\trisubsection{UC3.3 Visualizza informazioni allarme passato} \label{UC3.3}
\texttt{<diagramma d'uso>}

\begin{itemize}
    \item \textbf{Attore principale}: Utente personale sanitario

    \item \textbf{Pre-condizioni}: 
        \begin{itemize}
            \item Il Sistema è attivo;
            \item L'utente del personale sanitario è autenticato nel Sistema;
        \end{itemize}

    \item \textbf{Post-condizioni}:
        \begin{itemize}
            \item L'utente del personale sanitario visualizza le informazioni di un allarme passato.
        \end{itemize}

    \item \textbf{Trigger}: L'utente del personale sanitario vuole visualizzare le informazioni di un allarme passato.
    
    \item \textbf{Scenario principale}:
        \begin{itemize}
            \item L'utente del personale sanitario visualizza la Gestione allarmi (Vedi \hyperref[UC3]{§UC3}) 
            \item L'utente del personale sanitario visualizza gli allarmi passati nella Gestione allarmi (Vedi \hyperref[UC3.2]{§UC3.2})
            \item L'utente del personale sanitario visualizza le informazioni di un allarme passato.
        \end{itemize}

    \item \textbf{Inclusioni}
        \begin{itemize}
            \item \hyperref[UC3]{§UC3}
            \item \hyperref[UC3.2]{§UC3.2}
        \end{itemize}
\end{itemize}

\subsubsection{UC 4 Visualizza Analytics} \label{UC4}
\texttt{<diagramma d'uso>}

\begin{itemize}
    \item \textbf{Attore principale}: Utente personale sanitario

    \item \textbf{Pre-condizioni}: 
        \begin{itemize}
            \item Il Sistema è attivo;
            \item L'utente del personale sanitario è autenticato nel Sistema;
        \end{itemize}

    \item \textbf{Post-condizioni}:
        \begin{itemize}
            \item L'utente del personale sanitario visualizza la sezione Analytics.
        \end{itemize}

    \item \textbf{Trigger}: L'utente del personale sanitario vuole vedere la sezione dedicata alle statistiche della piattaforma e dell'impianto
    
    \item \textbf{Scenario principale}:
        \begin{itemize}
            \item L'utente del personale sanitario seleziona il menù relativo alla sezione Analytics. 
        \end{itemize}
\end{itemize}

\trisubsection{UC 4.1 Visualizza Analytics Energia consumata}
\texttt{<diagramma d'uso>}

\begin{itemize}
    \item \textbf{Attore principale}: Utente personale sanitario

    \item \textbf{Pre-condizioni}: 
        \begin{itemize}
            \item Il Sistema è attivo;
            \item L'utente del personale sanitario è autenticato nel Sistema;
        \end{itemize}

    \item \textbf{Post-condizioni}:
        \begin{itemize}
            \item L'utente del personale sanitario visualizza il grafico dedicato all'energia consumata, nelle Analytics.
        \end{itemize}

    \item \textbf{Trigger}: L'utente del personale sanitario vuole visualizzare le Analytics realativa all'energia consumata.
    
    \item \textbf{Scenario principale}:
        \begin{itemize}
            \item L'utente del personale sanitario visualizza la sezione Analytics (Vedi \hyperref[UC4]{§UC4})
            \item L'utente del personale sanitario seleziona il grafico dedicato all'energia consumata.
        \end{itemize}

    \item \textbf{Inclusioni}
        \begin{itemize}
            \item \hyperref[UC4]{§UC4}
        \end{itemize}
\end{itemize}

\trisubsection{UC 4.2 Visualizza Analytics Anomalie di impianto}
\texttt{<diagramma d'uso>}

\begin{itemize}
    \item \textbf{Attore principale}: Utente personale sanitario

    \item \textbf{Pre-condizioni}: 
        \begin{itemize}
            \item Il Sistema è attivo;
            \item L'utente del personale sanitario è autenticato nel Sistema;
        \end{itemize}

    \item \textbf{Post-condizioni}:
        \begin{itemize}
            \item L'utente del personale sanitario visualizza il grafico dedicato alle anomalie di impianto, nelle Analytics.
        \end{itemize}

    \item \textbf{Trigger}: L'utente del personale sanitario vuole visualizzare le Analytics relative alle anomalie di impianto
    
    \item \textbf{Scenario principale}:
        \begin{itemize}
            \item L'utente del personale sanitario visualizza la sezione Analytics (Vedi \hyperref[UC4]{§UC4})
            \item L'utente del personale sanitario seleziona il grafico dedicato alle anomalie di impianto.
        \end{itemize}

    \item \textbf{Inclusioni}
        \begin{itemize}
            \item \hyperref[UC4]{§UC4}
        \end{itemize}
\end{itemize}






\end{document}