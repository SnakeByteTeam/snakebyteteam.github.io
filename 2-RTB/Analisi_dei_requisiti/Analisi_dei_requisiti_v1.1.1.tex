\documentclass[10pt, letterpaper]{article}
\usepackage[nomarginpar, margin=2.75cm, tmargin=3cm, bmargin=1.75cm]{geometry}
\usepackage[
    colorlinks=true,      
    linkcolor=black,      
    urlcolor=blue,       
    citecolor=black       
]{hyperref}
\usepackage{xltabular}
\usepackage{template}
\usepackage{float}
\usepackage{graphicx}
\usepackage{caption}
\usepackage[table, x11names]{xcolor}
\usepackage{lastpage}
\renewcommand{\arraystretch}{1.05} % migliora la leggibilità
\renewcommand{\contentsname}{Indice}
\renewcommand{\listfigurename}{Elenco delle figure}
\renewcommand{\listtablename}{Elenco delle tabelle}
\usepackage{fancyhdr}
%Comandi per livello di sottosezioni = 3
\setcounter{tocdepth}{4}
\setcounter{secnumdepth}{4}
\newcommand{\trisubsection}[1]{\paragraph{#1}\mbox{}\\}

\pagestyle{fancy}
\fancyhf{}
\fancyhead[L]{SnakeByte}
\fancyhead[R]{Analisi dei Requisiti}
\fancyfoot[C]{Pagina \thepage\ di \pageref{LastPage}}

\newcommand\Au{1} % Autenticazione
\newcommand\Aucp{2} % Autenticazione con mod. pas.

\newcommand\VaMV{3} % Visualizza account MyVimar
\newcommand\CaMV{4} % Collega account MyVimar
\newcommand\RaMV{5} % Rimuovi account MyVimar

\newcommand\Vu{6} % Visualizza utenti OS
\newcommand\Cu{7} % Creazione OS
\newcommand\Eu{8} % Eliminazione OS

\newcommand\Vr{9} % Visualizza reparti

\newcommand\Cr{10} % Creazione reparto
\newcommand\Mr{11} % Modifica nome reparto
\newcommand\Er{12} % Eliminazione reparto

\newcommand\Aaosr{13} % Aggiunta assegnazione OS - reparto
\newcommand\Raosr{14} % Rimozion assegnazione OS - reparto

\newcommand\Aaapr{15} % Aggiunta assegnazione appartamento - reparto
\newcommand\Raapr{16} % Rimozion assegnazione appartamento - reparto

\newcommand\Vd{17} % Visualizzazione Dashboard

\newcommand\Va{27} % Visualizzazione analytics

\newcommand\Vap{28} % Visualizzazione Appartamento

\newcommand\Aapp{29} % Abilita appartamento
\newcommand\Dapp{30} % Disabilita appartamento

\newcommand\Aa{31} % Aggiunta allarme
\newcommand\Mpa{32} % Modifica priorità allarme
\newcommand\Msia{33} % Modifica soglia intervento allarme
\newcommand\Moaa{34} % Modifica orario attivazione allarme
\newcommand\Moda{35} % Modifica orario disattivazione allarme
\newcommand\Aal{36} % Abilita allarme
\newcommand\Dal{37} % Disabilita allarme
\newcommand\Ea{38} % Elimina allarme

\newcommand\Vn{39} % Visualizzazione notifiche

\newcommand\Erunr{40} % Errore username non registrato o password errata
\newcommand\Erpte{41} % Errore username non registrato o password temporeanea errata
\newcommand\Ernpupt{42} % Errore nuova password è uguale alla password temporeanea
\newcommand\Ernpnv{43} % Errore nuova password non valida
\newcommand\Eruu{44} % Errore username già in uso
\newcommand\Ernru{45} % Errore nome reparto già in uso
\newcommand\Erns{46} % Errore nessun sensore selezionato
\newcommand\Ernlp{47} % Errore nessun livello di priorità selezionato
\newcommand\Ernss{48} % Errore nessun soglia di intervento selezionata

\begin{document}
\setcounter{tocdepth}{5}
\setcounter{secnumdepth}{5}

\begin{titlepage}
    \begin{center}
        \begin{center}
            \includegraphics[width=0.6\textwidth]{./img/logo.pdf}
        \end{center}
        \vspace{4cm}
        \huge\textbf{Analisi dei requisiti}\par
        \vspace{2cm}
        \large \textbf{SnakeByte} (Gruppo 1):\\
        \large Valeria Baleanu, Leonardo Pellizzon, Filippo Venzo, Giuseppe De Fina, \\
         Francesco Pasqual, Christian Libralato, Luca Granziero \\
        (2109911, 2111006, 2113705, 2113187, 2103119, 2101047, 2075512)
        \vfill
        \small
        \begin{center}
            \begin{tabular}{|c|c|c|c|}
                \hline
                \multicolumn{4}{|c|}{\textbf{Informazioni documento}} \\
                \hline
                \rowcolor{lightgray} \textbf{Versione} & \textbf{Data} & \textbf{Stato} & \textbf{Destinatari} \\
                \hline
                1.1.1 & 02/02/2026 & Verificato &       
                \begin{tabular}[c]{@{}l@{}}
                    \textbf{Interni:} SnakeByte \\
                    \textbf{Esterni:} prof. Vardanega Tullio, prof. Cardin Riccardo, Vimar
                \end{tabular} \\
                \hline
            \end{tabular}
        \end{center}
        \vfill
        \large Contatti: snakebyteteam@gmail.com
    \end{center}
\end{titlepage}

\begin{center}
    \begin{xltabular}{\textwidth}{|c|c|c|c|c|X|}
        \hline
        \multicolumn{6}{|c|}{\textbf{Registro delle modifiche}} \\
        \hline
        \rowcolor{lightgray} \textbf{Versione} & \textbf{Data} & \textbf{Autore} & \textbf{Verificatore} & \textbf{Approvatore} & \textbf{Descrizione} \\
        \hline
        1.1.1 & 26/01/2026 & L. Pellizzon & V. Baleanu & - & Modifica diagramma UML Use Cases UC31 \\
        \hline
        1.1.0 & 26/01/2026 & L. Pellizzon & V. Baleanu & - & Modifica diagrammi UML Use Cases da UC1 a UC7, aggiunta Use Cases da UC8 a UC48 e modifica Attori \\
        \hline
        1.0.7 & 15/12/2025 & F. Venzo & F. Pasqual & - & Aggiunti diagrammi UML Use Cases UC1, UC1.4, UC1.5, UC2, UC2.8, UC2.9  \\
        \hline
        1.0.6 & 14/12/2025 & G. De Fina & L. Pellizzon & - & Aggiunta Use Cases UC5.4 (Gestione utenti e permessi) e relativi sotto-casi  \\
        \hline
        1.0.5 & 14/12/2025 & G. De Fina & L. Pellizzon & - & Modifica UC5.1.4 per utilizzo gruppi/reparti invece di singoli operatori  \\
        \hline
        1.0.4 & 14/12/2025 & V. Baleanu & L. Pellizzon & - & Creazione diagrammi UML per UC4, UC4.1, UC4.2, UC4.3, UC4.4, UC4.5, UC4.6, UC4.7.\\
        \hline
        1.0.3 & 12/12/2025 & V. Baleanu & L. Pellizzon & - & Aggiunta UC10 e UC10.1. Creazione diagrammi UML per UC3, UC3.3.1, UC3.3.2, UC3.6.1, UC3.6.2, UC3.7.1, UC3.7.2, UC3.10.\\
        \hline
        1.0.2 & 10/12/2025 & V. Baleanu & L. Pellizzon & - & Sistemazione UC3. Aggiunta dispositivi in Scenario Principale di UC4. Rimozione UC8.\\
        \hline
        1.0.1 & 8/12/2025 & V. Baleanu & L. Pellizzon & - & Sistemazione UC3 e UC4. Aggiunta Use Cases UC7 e UC8. Aggiunta diagramma Attori. \\
        \hline
        1.0.0 & 5/12/2025 & F. Venzo & L. Pellizzon & F. Pasqual & Approvazione \\
        \hline
        0.5.0 & 5/12/2025 & F. Venzo & L. Pellizzon & - & Aggiunta Use Cases UC7 e correzioni varie  \\
        \hline
        0.4.0 & 3/12/2025 & F. Venzo & L. Pellizzon & - & Aggiunta Use Cases UC6, UC6.1, UC6.2  \\
        \hline
        0.3.0 & 1/12/2025 & G. De Fina & L. Pellizzon & - & Aggiunta Use Cases UC5, UC5.1, UC5.2, UC5.3  \\
        \hline
        0.2.0 & 1/12/2025 & V. Baleanu & L. Pellizzon & - & Aggiunta Use Cases UC4, UC4.1, UC4.2, UC4.3 UC4.4, UC4.5, UC4.6, UC4.7, UC4.8, UC4.9, UC4.10  \\
        \hline
        0.1.2 & 24/11/2025 & F. Venzo & V. Baleanu & - & Aggiunta Use Cases UC3, UC3.1, UC3.2, UC3.3, UC3.4, UC3.4, UC3.5, UC3.6, UC3.7 \\
        \hline
        0.1.1 & 20/11/2025 & F. Venzo & V. Baleanu & - & Aggiunta Use Cases UC1, UC1.1, UC1.2, UC1.3, UC1.4, UC1.5,
         UC2, UC2.1, UC2.2, UC2.3, UC2.4 \\
        \hline
        0.1.0 & 05/11/2025 & L. Granziero & L. Pellizon & - & Prima Stesura \\
        \hline
    \end{xltabular}
\end{center}

\newpage

\tableofcontents

\newpage

\listoffigures

\newpage

\section{Introduzione}
\subsection{Finalità del documento}{
Il documento di analisi dei requisiti, in un contesto di ingegneria del software, ha lo scopo fondamentale di tradurre l'esigenza dell'utenza e degli stakeholder, in questo caso la proponente Vimar S.p.A., in una specifica completa, coerente e verificabile di requisiti, destinata a guidare le fasi di progettazione e sviluppo, verifica e infine validazione del Sistema.

\noindent
\\
\textbf{Definire chiaramente “cosa” il Sistema deve fare e “in quali condizioni”:}
\begin{itemize}
    \item L’analisi dei requisiti serve a identificare le funzioni (requisiti funzionali) e le qualità (requisiti non funzionali: prestazioni, usabilità, affidabilità, portabilità...) del software.
    \item Definisce i limiti del Sistema e i vincoli (tecnici, di interfaccia...)
    \item Permette di evitare ambiguità o fraintendimenti sulle funzionalità richieste. 
\end{itemize}

\noindent
\textbf{Allineare tutti gli stakeholder su un linguaggio comune e condividere le aspettative:}
\begin{itemize}
    \item Questo documento funge da contratto tra cliente/committente e il team di sviluppo: specifica ciò che sarà consegnato.
    \item Aiuta a garantire che utenti, committenti, analisti, progettisti e tester abbiano la stessa comprensione del Sistema.
\end{itemize}

\noindent
\textbf{Fornire una base stabile per le fasi successive del ciclo di vita del software:}
\begin{itemize}
    \item Il documento di analisi dei requisiti serve come input per la progettazione del Sistema, per la pianificazione dello sviluppo e per la pianificazione dei test.
    \item Serve anche come base per la verifica e la convalida: si può usare come riferimento per capire se il prodotto finale soddisfa i requisiti richiesti. 
\end{itemize}
\noindent
\textbf{Gestire i rischi e controllare le modifiche:}
\begin{itemize}
    \item Durante l’analisi dei requisiti si identificano requisiti non realizzabili, conflitti tra requisiti, omissioni e incoerenze. Ciò consente di ridurre i rischi fin dalle prime fasi.
    \item Aiuta a limitare il fenomeno dello “scope creep” (ovvero l’aggiunta non controllata di funzionalità) e a mantenere il controllo sul cambiamento dei requisiti. 
\end{itemize}
}

\subsection{Sviluppo del documento}{
Il presente documento è stato sviluppato in modo graduale e incrementale, con lo scopo di facilitare modifiche future in base alle esigenze che verranno concordate tra il gruppo e l'azienda committente. Il documento è quindi soggetto a un processo di miglioramento continuo nel tempo.
}

\subsection{Riferimenti}

\subsubsection{Riferimenti Normativi}
\begin{itemize}
    \item \textbf{Norme di Progetto}: 
    
    \url{https://github.com/SnakeByteTeam/snakebyteteam.github.io/blob/develop/2-RTB/Norme_di_progetto/NdP_v0.2.0.pdf}  (link temporaneo, da cambiare una volta pubblicato nel sito)
    
    (consultato il 30/10/2025);
    \item \textbf{Vimar View4Life Capitolato di Ingegneria del Software Università di Padova 2025 - 2026:}
    
    \url{https://www.math.unipd.it/~tullio/IS-1/2025/Progetto/C9.pdf} 
    
    (consultato il 20/10/2025).
\end{itemize}

\subsubsection{Riferimenti Informativi}
\begin{itemize}
    \item \textbf{830-1998 - IEEE Recommended Practice for Software Requirements Specifications}
    
    \url{https://ieeexplore.ieee.org/document/720574} 
    
    (consultato il 20/10/2025).

    \item \textbf{Diagrammi Use Case - Riccardo Cardin}
    
    \url{https://www.math.unipd.it/~rcardin/swea/2022/Diagrammi%20Use%20Case.pdf} 
    
    (consultato il 18/10/2025).
\end{itemize}

\section{Descrizione del prodotto}
\subsection{Prospettiva del prodotto}
La prospettiva del prodotto è un Sistema domotico integrato per persone autosufficienti (ad esempio anziani) che si basa su dispositivi Vimar con connessione mesh Bluetooth.
Gli obiettivi principali sono la sicurezza e il comfort delle persone occupanti, l'aumento dell'efficienza energetica della struttura e la semplificazione della gestione operativa dell'impianto elettrico.
Tali risultati vengono raggiunti attraverso la centralizzazione del controllo di illuminazione, temperatura, televisione e dispositivi di sicurezza mediante l'app View e i relativi servizi cloud, con la possibilità di controllo da remoto da parte del personale medico.

\subsection{Obiettivi del prodotto}{
Il progetto consiste nella realizzazione di una piattaforma unica \textit{View4Life} per la gestione intelligente degli impianti \textit{Smart} nelle residenze protette, sfruttando i dispositivi domotici Vimar connessi in rete \textit{mesh Bluetooth} che vengono poi esposti tramite l'interfaccia \textit{API KNX IoT 3rd-party$_{G}$}. Questa soluzione mira a supportare il lavoro del \os fornendo uno strumento che integri un Sistema di gestione degli allarmi (come il rilevamento di cadute o presenze prolungate in determinate stanze) per garantire un intervento rapido e tempestivo. Inoltre, la piattaforma è progettata per permettere il monitoraggio del consumo energetico e la rilevazione di anomalie nell’impianto (ad esempio le disconnessioni).
}

\subsection{Funzionalità del prodotto}
Dal punto di vista degli utenti del \os l'applicativo svolge le seguenti funzioni:
\begin{itemize}
    \item Visualizzazione delle informazioni generali di allarmi, statistiche ed analitiche tramite cruscotto riassuntivo (Dashboard);
    \item Possibilità di essere notificati, visualizzare e gestire gli allarmi attivi;
    \item Possibilità di visualizzare e gestire i dispositivi dei vari impianti collocati in diverse residenze;
    \item Possibilità di visualizzare statistiche, tramite grafici, sui consumi dell'impianto, sulla variazione di temperatura e sugli allarmi passati;
    \item Possibilità di ricevere consigli, basati sulle statistiche, per ridurre i consumi energetici.
\end{itemize}

\subsection{Utenza di riferimento}{
Il prodotto si rivolge principalmente a quattro categorie principali di utenti, descritte di seguito:

\begin{itemize}
    \item \textbf{Personale medico e operatori sanitari} che utilizzano l'applicazione per monitorare lo stato di ambienti e dispositivi, oltre a ricevere notifiche di allarme e gestire da remoto funzioni come temperatura, illuminazione e sicurezza delle stanze degli ospiti presenti. 
    \item \textbf{Personale amministrativo} che si fa carico della configurazione e manutenzione del Sistema e del monitoraggio delle statistiche prodotte (ad esempio sui consumi elettrici o sulla variazione delle temperatura), e quindi a una conseguente ottimizzazione dell'efficienza dell'impianto elettrico.
\end{itemize}
Questa sezione evidenzia come il Sistema domotico proposto miri a centralizzare il controllo e la gestione degli impianti all’interno della residenza, semplificando le operazioni quotidiane del personale e migliorando la qualità della vita degli ospiti.
Attraverso l’app View e i servizi cloud Vimar, l’applicazione permette di gestire in modo integrato illuminazione, temperatura e sicurezza, contribuendo a maggior efficienza energetica, sicurezza e comfort abitativo per tutti gli utenti coinvolti.
}

\section{Casi d'uso}
Un \textit{caso d'uso$_{G}$} è la descrizione dettagliata, tramite \textit{diagramma UML$_G$} e descrizione testuale, di un insieme di scenari che hanno uno scopo comune, all'interno del Sistema, per un attore.
Permettono di comprendere al meglio le funzionalità che devono essere rese disponibili dal Sistema \textit{software}.

In particolare, le descrizioni dei casi d'uso contenute in questo documento conterranno le informazioni riportate nella seguente tabella:

\begin{center}
    \begin{tabularx}{\textwidth}{|c| >{\centering\arraybackslash}X|}
        \hline
        \rowcolor{lightgray} \textbf{Campo} & \textbf{Descrizione} \\
        \hline
        Attori & Coloro che partecipano attivamente al caso d'uso per raggiungere un preciso obiettivo.  \\
        \hline
        Pre-condizioni & Condizioni che devono essere soddisfatte prima dello scenario descritto dal caso d'uso.\\
        \hline
        Post-condizioni & Condizioni che risultano soddisfatte dopo il completamento dello scenario principale del caso d'uso. Se viene completato uno scenario alternativo, saranno soddisfatte le Post-condizioni di quest'ultimo. \\
        % \hline
        % Trigger & La motivazione che porta l'\user a svolgere i passi del caso d'uso. \\
        \hline
        Scenario principale & Sequenza di passi che l'\user deve seguire per completare il caso d'uso. \\
        \hline
        Scenari alternativi & Scenario divergente dal principale per il verificarsi di una particolare condizione. \\
        \hline
        Estensioni &  Casi d'uso ulteriori eseguiti al verificarsi di una particolare condizione nel caso d'uso primario. Modificano Scenario e Post-condizioni. \\
        \hline
        Inclusioni & Casi d'uso ulteriori eseguiti al fine di completare il caso d'uso principale. Vengono eseguiti tutti incondizionatamente. \\
        \hline
        Generalizzazioni & Casi d'uso più specifici che ereditano comportamento e caratteristiche da un caso d'uso più generale, potendo aggiungere o specializzare ulteriori funzionalità. \\  \\
        \hline
    \end{tabularx}
\end{center}

Non tutti gli attributi sono necessari per ogni caso d'uso. Nel caso in cui un campo sia assente in un caso d'uso, allora tale sarà assente anche nella sua descrizione e nel suo diagramma UML.

\subsection{Attori}
Di seguito vengono riportati gli attori individuati:
\begin{figure}[H]
    \centering
    \includegraphics[width=0.415\textwidth]{./img/Attori.pdf}
    \caption{Diagramma attori principali}
    \label{fig:attori}
\end{figure}

\begin{center}
    \begin{tabularx}{\textwidth}{|c| >{\centering\arraybackslash}X|}
        \hline
        \rowcolor{lightgray} \textbf{Attore} & \textbf{Descrizione} \\
        \hline
        \user & Rappresenta l'utente generico (sia \admin che \os) registrato nel Sistema. \\
        \hline
        \os & Rappresenta l'\user \os. \\
        \hline 
        \admin & Rappresenta un \user che ha particolari privilegi all'interno del Sistema. \\
        \hline  
        \db & Rappresenta il \db responsabile della persistenza dei dati. \\
        \hline 
        \cloud & Rappresenta il \cloud che mette a disposizione un'interfaccia per accedere ai dati riguardanti gli appartamenti, le stanze e i dispositivi. \\
        \hline          
    \end{tabularx}
\end{center}

\subsection{Lista dei casi d'uso}

% --------

\addAttore{\user}
\addAttoreSec{\db}
\addPrC{Il Sistema è attivo;}
\addPoC{L'\user è autenticato nel Sistema;}
\addPoC{L'\user è riconosciuto come \os.}
\addTrigger{L'\user vuole autenticarsi nel Sistema}
\addScenarioPrincipale{L'\user si autentica con le proprie credenziali:
\begin{itemize}
    \item inserimento username (\rif{\Au.1});
    \item inserimento password (\rif{\Au.2}).
\end{itemize}
}
\addInclusione{\rif{\Au.1}}
\addInclusione{\rif{\Au.2}}
\addScenarioAlternativo{L'\user non è registrato nel Sistema o la password è errata (\rif{\Erunr});}
\addEstensione{\rif{\Erunr}}
\makeUC[0.9]{\Au}{Autenticazione}

\addAttore{\user}
\addPrC{L'\user sta eseguendo l'autenticazione}
\addPoC{L'\user ha inserito l'username.}
\addScenarioPrincipale{L'\user inserisce l'username;}
\makeUC{\Au.1}{Inserimento username}

\addAttore{\user}
\addPrC{L'\user sta eseguendo l'autenticazione}
\addPoC{L'\user ha inserito la password.}
\addScenarioPrincipale{L'\user inserisce la password;}
\makeUC{\Au.2}{Inserimento password}

% --------

\addAttore{\user}
\addPrC{Il Sistema è attivo;}
\addPrC{L'\user si trova nella pagina registrazione;}
\addPoC{L'\user è riconosciuto come \os.}
\addScenarioPrincipale{L'\user si registra con le credenziali fornite dall'\admin:
\begin{itemize}
    \item inserimento username (\rif{\Aucp.1});
    \item inserimento password temporanea (\rif{\Aucp.2});
    \item inserimento nuova password (\rif{\Aucp.3});
\end{itemize}
}
\addScenarioAlternativo{L'\user non è registrato nel Sistema o la password temporanea errata (\rif{\Erpte});}
\addScenarioAlternativo{La nuova password è uguale alla password temporanea (\rif{\Ernpupt});}
\addScenarioAlternativo{La nuova password non è valida (\rif{\Ernpnv});}
\addInclusione{\rif{\Aucp.1}}
\addInclusione{\rif{\Aucp.2}}
\addInclusione{\rif{\Aucp.3}}
\addEstensione{\rif{\Erpte}}
\addEstensione{\rif{\Ernpupt}}
\addEstensione{\rif{\Ernpnv}}
\addTrigger{L'\os si è autenticato per la prima volta.}
\makeUC[1]{\Aucp}{Autenticazione con cambio password}

% \addAttore{\user}
% \addPrC{L'\user sta eseguendo la registrazione}
% \addPoC{Il Sistema riceve l'username dell'\os.}
% \addScenarioPrincipale{L'\user inserisce l'username;}
% \makeUC{\Aucp.1}{Inserimento username}

\addAttore{\user}
\addPrC{L'\user sta eseguendo l'autenticazione con cambio password}
\addPoC{Il Sistema riceve l'username dell'\os.}
\addScenarioPrincipale{L'\user inserisce l'username;}
\makeUC{\Aucp.1}{Inserimento username}

\addAttore{\user}
\addPrC{L'\user sta eseguendo l'autenticazione con cambio password}
\addPoC{Il Sistema riceve la password temporanea dell'\os.}
\addScenarioPrincipale{L'\user inserisce la password temporanea;}
\makeUC{\Aucp.2}{Inserimento password temporanea}

\addAttore{\user}
\addPrC{L'\user sta eseguendo l'autenticazione con cambio password}
\addPoC{Il Sistema riceve la nuova password dell'\os.}
\addScenarioPrincipale{L'\user inserisce la nuova password;}
\makeUC{\Aucp.3}{Inserimento nuova password}

% --------

\addAttore{\admin}
\addAttoreSec{\db}
\addPrC{Il Sistema è attivo;}
\addPrC{L'\admin è autenticato nel Sistema;}
\addPrC{L'\admin sta visualizzando la sezione di gestione account MyVimar.}
\addPoC{L'\admin visualizza se è presente un account MyVimar collegato.}
\addScenarioPrincipale{L'\admin visualizza se è presente un account MyVimar collegato.}
\makeUC[0.6]{\VaMV}{Visualizzazione account MyVimar collegato}

\addAttore{\admin}
\addPrC{Il Sistema è attivo;}
\addPrC{L'\admin è autenticato nel Sistema;}
\addPrC{L'\admin sta visualizzando la sezione di gestione account MyVimar.}
\addPoC{Il Sistema aggiunge le credenziali fornite dal portale di Vimar alla propria configurazione}
\addScenarioPrincipale{L'\admin viene reinderizzato al portale di Vimar per collegare l'account
segue la procedura.}
\makeUC[0.6]{\CaMV}{Collegamento account MyVimar}

\addAttore{\admin}
\addAttoreSec{\db}
\addAttoreSec{\cloud}
\addPrC{Il Sistema è attivo;}
\addPrC{L'\admin è autenticato nel Sistema;}
\addPrC{L'\admin sta visualizzando la sezione di gestione account MyVimar.}
\addPoC{Il Sistema rimuove le credenziali dell'account MyVimar salvate dalla propria configurazione}
\addScenarioPrincipale{L'\admin }
\makeUC[0.6]{\RaMV}{Rimozione account MyVimar}

% --------

\addAttore{\admin}
\addAttoreSec{\db}
\addPrC{Il Sistema è attivo;}
\addPrC{L'\admin è autenticato nel Sistema;}
\addPoC{Viene visualizzato l'elenco degli utenti.}
\addScenarioPrincipale{L'\admin seleziona dal menù l'opzione relativa alla gestione utenti}
\addScenarioPrincipale{L'\admin visualizza l'elenco degli utenti (\rif{\Vu.1}).}
\addInclusione{\rif{\Vu.1}}
\makeUC[0.6]{\Vu}{Visualizzazione elenco utenti}

\addPrC{L'\admin sta visualizzando l'elenco di tutti gli utenti.}
\addPoC{Viene visualizzato un singolo utente con le proprie informazioni}
\addScenarioPrincipale{L'\admin visualizza un singolo utente con le proprie informazioni:
\begin{itemize}
    \item nome dell'utente (\rif{\Vr.1.1}).
    \item cognome dell'utente (\rif{\Vr.1.2}).
    \item l'username dell'utente (\rif{\Vr.1.3}).
\end{itemize}
}
\addInclusione{\rif{\Vu.1.1}}
\addInclusione{\rif{\Vu.1.2}}
\addInclusione{\rif{\Vu.1.3}}
\makeUC{\Vu.1}{Visualizzazione elemento elenco utenti}

\addAttore{\admin}
\addPrC{L'\admin sta visualizzando un singolo utente.}
\addPoC{L'\admin visualizza il nome dell'utente.}
\addScenarioPrincipale{L'\admin visualizza il nome dell'utente.}
\makeUC{\Vu.1.1}{Visualizzazione nome}

\addAttore{\admin}
\addPrC{L'\admin sta visualizzando un singolo utente.}
\addPoC{L'\admin visualizza il cognome dell'utente.}
\addScenarioPrincipale{L'\admin visualizza il cognome dell'utente.}
\makeUC{\Vu.1.2}{Visualizzazione cognome}

\addAttore{\admin}
\addPrC{L'\admin sta visualizzando un singolo utente.}
\addPoC{L'\admin visualizza l'username dell'utente.}
\addScenarioPrincipale{L'\admin visualizza l'username dell'utente.}
\makeUC{\Vu.1.3}{Visualizzazione username}

\addAttore{\admin}
\addAttoreSec{\db}
\addPrC{Il Sistema è attivo;}
\addPrC{L'\admin è autenticato nel Sistema;}
\addPrC{L'\admin sta visualizzando la sezione gestione utenti.}
\addPoC{Un nuovo utente \os è stato creato all'interno del Sistema}
\addScenarioPrincipale{L'\admin inserisce i seguenti dati per creare l'\user:
\begin{itemize}
    \item inserimento nome (\rif{\Cu.1});
    \item inserimento cognome (\rif{\Cu.2});
    \item inserimento nuovo username (\rif{\Cu.3});
\end{itemize}
successivamente genera la password temporanea (\rif{\Cu.4}).
}
\addInclusione{\rif{\Cu.1}}
\addInclusione{\rif{\Cu.2}}
\addInclusione{\rif{\Cu.3}}
\addInclusione{\rif{\Cu.4}}
\addScenarioAlternativo{L'username inserito è già in uso (\rif{\Eruu})}
\addEstensione{\rif{\Eruu}}
%\addTrigger{L'\admin vuole registrare un nuovo \user del \os.}
\makeUC[1]{\Cu}{Creazione nuovo utente \os}

\addAttore{\admin}
\addPrC{L'\admin sta creando un nuovo utente \os;}
\addPoC{Il Sistema conosce il nome del'\os;}
\addScenarioPrincipale{L'\admin inserisce il nome per l'\os}
\makeUC{\Cu.1}{Inserimento nome}

\addAttore{\admin}
\addPrC{L'\admin sta creando un nuovo utente \os;}
\addPoC{Il Sistema conosce il cognome del'\os;}
\addScenarioPrincipale{L'\admin inserisce il cognome per l'\os}
\makeUC{\Cu.2}{Inserimento cognome}

\addAttore{\admin}
\addPrC{L'\admin sta creando un nuovo utente \os;}
\addPoC{Il Sistema conosce lo username del'\os;}
\addScenarioPrincipale{L'\admin inserisce un username per l'\os}
\makeUC{\Cu.3}{Inserimento username}

\addAttore{\admin}
\addPrC{L'\admin sta creando un nuovo utente \os;}
\addPoC{Il Sistema conosce la password temporanea per l'\os;}
\addScenarioPrincipale{L'\admin genera una password temporanea per l'\os}
\makeUC{\Cu.4}{Generazione password temporanea}

\addAttore{\admin}
\addPrC{Il Sistema è attivo;}
\addAttoreSec{\db}
\addPrC{L'\admin è autenticato nel Sistema;}
\addPrC{L'\admin sta visualizzando la sezione gestione utenti.}
\addPrC{Esiste almeno un \user del \os registrato.}
\addPoC{L'utente del \os è eliminato dal Sistema.}
%\addTrigger{L'\admin vuole eliminare un \user del \os}
\addScenarioPrincipale{L'\admin seleziona l'utente \os che vuole eliminare e conferma l'eliminazione}
\makeUC[0.6]{\Eu}{Eliminazione utente \os}

% --------

\addAttore{\admin}
\addAttoreSec{\db}
\addPrC{Il Sistema è attivo;}
\addPrC{L'\admin è autenticato nel Sistema;}
\addPoC{Viene visualizzato l'elenco dei reparti.}
\addScenarioPrincipale{L'\admin seleziona dal menù l'opzione relativa alla gestione reparti}
\addScenarioPrincipale{L'\admin visualizza l'elenco dei reparti (\rif{\Vr.1}).}
\addInclusione{\rif{\Vr.1}}
\makeUC[0.6]{\Vr}{Visualizzazione elenco reparti}

\addAttore{\admin}
\addPrC{L'\admin sta visualizzando l'elenco dei reparti.}
\addPoC{Viene visualizzato un singolo reparto con le proprie informazioni}
\addScenarioPrincipale{L'\admin visualizza un reparto con le proprie informazioni:
\begin{itemize}
    \item nome del reparto (\rif{\Vr.1.1})
    \item elenco degli appartamenti (\rif{\Vr.1.2})
\end{itemize}
}
\makeUC{\Vr.1}{Visualizzazione elemento elenco reparti}

\addAttore{\admin}
\addPrC{L'\admin sta visualizzando un reparto.}
\addPoC{L'\admin visualizza il nome della stanza.}
\addScenarioPrincipale{L'\admin visualizza il nome della stanza.}
\makeUC{\Vr.1.1}{Visualizzazione nome}

\addAttore{\admin}
\addPrC{L'\admin sta visualizzando un reparto.}
\addPoC{Viene visualizzato l'elenco degli appartamenti.}
\addScenarioPrincipale{L'\admin visualizza la lista degli appartamenti (\rif{\Vr.1.2.1})}
\addInclusione{\rif{\Vr.1.2.1}}
\makeUC{\Vr.1.2}{Visualizzazione elenco appartamenti}

\addAttore{\admin}
\addPrC{L'\admin sta visualizzando tutti gli appartamenti.}
\addPoC{Viene visualizzato un singolo appartamento con le proprie informazioni}
\addScenarioPrincipale{L'\admin visualizza un appartamento con le proprie informazioni:
\begin{itemize}
    \item nome dell'appartamento (\rif{\Vr.1.2.1.1})
\end{itemize}
}
\addInclusione{\rif{\Vr.1.2.1.1}}
\makeUC{\Vr.1.2.1}{Visualizzazione elemento elenco appartamenti}

\addAttore{\admin}
\addPrC{L'\admin sta visualizzando un appartamento.}
\addPoC{L'\admin visualizza il nome dell'appartamento.}
\addScenarioPrincipale{L'\admin visualizza il nome dell'appartamento.}
\makeUC{\Vr.1.2.1.1}{Visualizzazione nome appartamento}

% --------

\addAttore{\admin}
\addAttoreSec{\db}
\addPrC{Il Sistema è attivo;}
\addPrC{L'\admin è autenticato nel Sistema;}
\addPrC{L'\admin si trova nella sezione gestione reparti.}
\addPoC{Il reparto viene aggiunto nel Sistema.}
\addScenarioPrincipale{L'\admin inserisce i seguenti dati per creare un reparto:
\begin{itemize}
    \item inserimento nome (\rif{\Cr.1});
\end{itemize}
}
\addScenarioAlternativo{Il nome inserito è già in uso (\rif{\Ernru})}
\addEstensione{\rif{\Ernru}}
\addInclusione{\rif{\Cr.1}}
\makeUC[1]{\Cr}{Creazione reparto}

\addAttore{\admin}
\addPrC{L'\admin sta creando un nuovo reparto;}
\addPoC{Il Sistema conosce il nome del nuovo reparto;}
\addScenarioPrincipale{L'\admin inserisce il nome del nuovo reparto.}
\makeUC{\Cr.1}{Inserimento nome}

\addAttore{\admin}
\addAttoreSec{\db}
\addPrC{Il Sistema è attivo;}
\addPrC{L'\admin è autenticato nel Sistema;}
\addPrC{L'\admin si trova nella sezione gestione reparti.}
\addPoC{Il nome del reparto viene aggiornato dal Sistema.}
\addScenarioPrincipale{L'\admin modifica il nome del reparto il reparto}
\makeUC[0.6]{\Mr}{Modifica nome reparto}

\addAttore{\admin}
\addAttoreSec{\db}
\addPrC{Il Sistema è attivo;}
\addPrC{L'\admin è autenticato nel Sistema;}
\addPrC{L'\admin si trova nella sezione gestione reparti.}
\addPoC{Il reparto viene eliminato dal Sistema.}
\addScenarioPrincipale{L'\admin elimina il reparto}
\makeUC[0.6]{\Er}{Eliminazione reparto}

% --------

\addAttore{\admin}
\addAttoreSec{\db}
\addPrC{Il Sistema è attivo;}
\addPrC{L'\admin è autenticato nel Sistema;}
\addPrC{L'\admin si trova nella sezione gestione reparti.}
\addPoC{L'assegnazione \os - reparto viene registrata nel Sistema.}
\addScenarioPrincipale{L'\admin seleziona i seguenti elementi per assegnare un utente \os a un reparto:
\begin{itemize}
    \item selezione reparto (\rif{\Aaosr.1});
    \item selezione utente \os (\rif{\Aaosr.2});
\end{itemize}
}
\addInclusione{\rif{\Aaosr.1}}
\addInclusione{\rif{\Aaosr.2}}
\makeUC[0.6]{\Aaosr}{Aggiunta assegnazione utente \os - reparto}

\addAttore{\admin}
\addPrC{L'\admin sta aggiungendo un'assegnazione utente \os - reparto;}
\addPoC{Il Sistema conosce reparto che verrà assegnato all'utente \os;}
\addScenarioPrincipale{L'\admin seleziona il reparto.}
\makeUC{\Aaosr.1}{Selezione reparto}

\addAttore{\admin}
\addPrC{L'\admin sta aggiungendo un'assegnazione utente \os - reparto;}
\addPoC{Il Sistema conosce l'utente \os al quale verrà assegnato il reparto;}
\addScenarioPrincipale{L'\admin seleziona il l'utente \os.}
\makeUC{\Aaosr.2}{Selezione utente \os}

\addAttore{\admin}
\addAttoreSec{\db}
\addPrC{Il Sistema è attivo;}
\addPrC{L'\admin è autenticato nel Sistema;}
\addPrC{L'\admin si trova nella sezione gestione reparti.}
\addPoC{L'assegnazione utente \os - reparto viene rimossa dal Sistema.}
\addScenarioPrincipale{L'\admin rimuove l'assegnazione utente \os - reparto}
\makeUC[0.6]{\Raosr}{Rimozione assegnazione utente \os - reparto}

% --------

\addAttore{\admin}
\addAttoreSec{\db}
\addPrC{Il Sistema è attivo;}
\addPrC{L'\admin è autenticato nel Sistema;}
\addPrC{L'\admin si trova nella sezione gestione reparti.}
\addPoC{L'assegnazione appartamento - reparto viene registrata nel Sistema.}
\addScenarioPrincipale{L'\admin seleziona i seguenti elementi per assegnare un appartamento a un reparto:
\begin{itemize}
    \item selezione reparto (\rif{\Aaapr.1});
    \item selezione appartamento (\rif{\Aaapr.2});
\end{itemize}
}
\addInclusione{\rif{\Aaapr.1}}
\addInclusione{\rif{\Aaapr.2}}
\makeUC[0.6]{\Aaapr}{Aggiunta assegnazione appartamento - reparto}

\addAttore{\admin}
\addPrC{L'\admin sta aggiungendo un'assegnazione utente appartamento - reparto;}
\addPoC{Il Sistema conosce reparto che verrà assegnato l'appartamento;}
\addScenarioPrincipale{L'\admin seleziona il reparto.}
\makeUC{\Aaapr.1}{Selezione reparto}

\addAttore{\admin}
\addPrC{L'\admin sta aggiungendo un'assegnazione utente appartamento - reparto;}
\addPoC{Il Sistema conosce l'appartamento al quale verrà assegnato il reparto;}
\addScenarioPrincipale{L'\admin seleziona l'appartamento}
\makeUC{\Aaapr.2}{Selezione appartamento}

\addAttore{\admin}
\addAttoreSec{\db}
\addPrC{Il Sistema è attivo;}
\addPrC{L'\admin è autenticato nel Sistema;}
\addPrC{L'\admin si trova nella sezione gestione reparti.}
\addPoC{L'assegnazione appartamento - reparto viene rimossa dal Sistema.}
\addScenarioPrincipale{L'\admin rimuove l'assegnazione appartamento - reparto}
\makeUC[0.6]{\Raapr}{Rimozione assegnazione appartamento - reparto}

% --------

\addAttore{\user}
\addPrC{Il Sistema è attivo;}
\addPrC{L'\user è autenticato nel Sistema.}
\addScenarioPrincipale{L'\user seleziona dal menù l'opzione relativa alla visualizzazione dashboard}
\addScenarioPrincipale{L'\user visualizza il seguente modulo non opzionale: visualizzazione gestione allarmi (\rif{\Vd.1}).}
\addScenarioPrincipale{
L'\user visualizza i moduli opzionali presenti tra:
\begin{itemize}
    \item statistiche allarmi (\rif{\Vd.2});
    \item informazioni \user (\rif{\Vd.3});
    \item analisi clima (\rif{\Vd.4});
    \item analisi consumi (\rif{\Vd.5});
    \item analisi presenze (\rif{\Vd.6}).
\end{itemize}
}
%\addTrigger{L'\user vuole visualizzare la sezione Dashboard.}
\addInclusione{\rif{\Vd.1}}
\addInclusione{\rif{\Vd.2}}
\addInclusione{\rif{\Vd.3}}
\addInclusione{\rif{\Vd.4}}
\addInclusione{\rif{\Vd.5}}
\addInclusione{\rif{\Vd.6}}
\makeUC[0.6]{\Vd}{Visualizzazione dashboard}

\addAttore{\user}
\addPrC{L'\user sta visualizzando la dashboard.}
\addPoC{L'\user visualizza il modulo gestione allarmi}
\addScenarioPrincipale{L'\user visualizza il modulo gestione allarmi che consiste in un elenco di allarmi (\rif{\Vd.1.1})}
%\addTrigger{L'\user vuole visualizzare il modulo relativo alla gestione allarmi.}
\addInclusione{\rif{\Vd.1.1}}
\makeUC{\Vd.1}{Visualizzazione gestione allarmi}

\addAttore{\user}
\addPrC{L'\user visulizza il modulo gestione allarmi.}
\addPoC{L'\user visualizza le informazioni relative ad un allarme.}
\addScenarioPrincipale{Il Sistema mostra le informazioni relative ad un allarme:
\begin{itemize}
    \item il segnale di pericolo (\rif{\Vd.1.1.1});
    \item il nome dell'allarme (\rif{\Vd.1.1.2});
    \item il tempo trascorso dallo scatto dell'allarme (\rif{\Vd.1.1.3}).
\end{itemize}
}
\addInclusione{\rif{\Vd.1.1.1}}
\addInclusione{\rif{\Vd.1.1.2}}
\addInclusione{\rif{\Vd.1.1.3}}
\makeUC{\Vd.1.1}{Visualizzazione allarme}

\addAttore{\user}
\addPrC{L'\user visulizza un allarme.}
\addPoC{L'\user visualizza il segnale di pericolo.}
\addScenarioPrincipale{L'\user visualizza il segnale di pericolo.}
\makeUC{\Vd.1.1.1}{Visualizzazione segnale di pericolo}

\addAttore{\user}
\addPrC{L'\user visulizza un allarme.}
\addPoC{L'\user visualizza il nome di un allarme.}
\addScenarioPrincipale{L'\user visualizza il nome di un allarme.}
\makeUC{\Vd.1.1.2}{Visualizzazione nome allarme}

\addAttore{\user}
\addPrC{L'\user visulizza un allarme.}
\addPoC{L'\user visualizza il tempo trascorso dallo scatto dell'allarme.}
\addScenarioPrincipale{L'\user visualizza il tempo trascorso dallo scatto dell'allarme.}
\makeUC{\Vd.1.1.3}{Visualizzazione tempo trascorso dallo scatto dell'allarme}

\addAttore{\user}
\addPrC{L'\user sta visualizzando la dashboard.}
\addPoC{L'\user visualizza il modulo statistiche allarmi.}
\addScenarioPrincipale{L'\user visualizza il modulo statistiche allarmi, caratterizzate da:
\begin{itemize}
    \item numero di allarmi risolti (\rif{\Vd.2.1});
    \item numero di allarmi attivi (\rif{\Vd.2.2}).
\end{itemize}
}
\addInclusione{\rif{\Vd.2.1}}
\addInclusione{\rif{\Vd.2.2}}
%\addTrigger{L'\user vuole visualizzare il modulo opzionale relativo alle statistiche allarmi.}
\makeUC{\Vd.2}{Visualizzazione statistiche allarmi}

\addAttore{\user}
\addPrC{L'\user sta visualizzando statistiche allarmi.}
\addPoC{L'\user visualizza il numero di allarmi risolte.}
\addScenarioPrincipale{l'\user visualizza il numero di allarmi risolte.}
\makeUC{\Vd.2.1}{Visualizzazione numero di allarmi risolte}

\addAttore{\user}
\addPrC{L'\user sta visualizzando statistiche allarmi.}
\addPoC{L'\user visualizza il numero di allarmi attive.}
\addScenarioPrincipale{l'\user visualizza il numero di allarmi attive.}
\makeUC{\Vd.2.2}{Visualizzazione numero di allarmi attive}

\addAttore{\user}
\addPrC{L'\user sta visualizzando la dashboard.}
\addPoC{L'\user visualizza il modulo informazioni \user.}
\addScenarioPrincipale{L'\user visualizza il modulo informazioni \user, caratterizzate da:
\begin{itemize}
    \item nome dell'\user (\rif{\Vd.3.1});
    \item cognome dell'\user (\rif{\Vd.3.2}).
\end{itemize}
}
%\addTrigger{L'\user vuole visualizzare il modulo opzionale relativo alle informazioni \user}
\addInclusione{\rif{\Vd.3.1}}
\addInclusione{\rif{\Vd.3.2}}
\makeUC{\Vd.3}{Visualizzazione informazioni \user}

\addAttore{\user}
\addPrC{L'\user sta visualizzando le informazioni \user.}
\addPoC{L'\user visualizza il nome dell'\user.}
\addScenarioPrincipale{l'\user visualizza il nome dell'\user.}
\makeUC{\Vd.3.1}{Visualizzazione nome \user}

\addAttore{\user}
\addPrC{L'\user sta visualizzando le informazioni \user.}
\addPoC{L'\user visualizza il cognome dell'\user.}
\addScenarioPrincipale{l'\user visualizza il cognome dell'\user.}
\makeUC{\Vd.3.2}{Visualizzazione cognome \user}

\addAttore{\user}
\addPrC{L'\user sta visualizzando la dashboard.}
\addPoC{L'\user visualizza il grafico che riporta l'andamento della temperatura nelle ultime 12 ore.}
\addScenarioPrincipale{L'\user visualizza il grafico che riporta l'andamento della temperatura nelle ultime 12 ore.}
%\addTrigger{L'\user vuole visualizzare il modulo opzionale relativo all'analisi clima.}
\makeUC{\Vd.4}{Visualizzazione analisi clima}

\addAttore{\user}
\addPrC{L'\user sta visualizzando la dashboard.}
\addPoC{L'\user visualizza il grafico che riporta l'andamento dei consumi nelle ultime 12 ore.}
\addScenarioPrincipale{L'\user visualizza il grafico che riporta l'andamento dei consumi nelle ultime 12 ore.}
%\addTrigger{L'\user vuole visualizzare il modulo opzionale relativo all'analisi consumi.}
\makeUC{\Vd.5}{Visualizzazione analisi consumi}

\addAttore{\user}
\addPrC{L'\user sta visualizzando la dashboard.}
\addPoC{L'\user visualizza il grafico che riporta l'andamento della presenza nelle ultime 12 ore.}
\addScenarioPrincipale{L'\user visualizza il grafico che riporta l'andamento della presenza nelle ultime 12 ore.}
%\addTrigger{L'\user vuole visualizzare il modulo opzionale relativo all'andamento della presenza.}
\makeUC{\Vd.6}{Visualizzazione analisi presenze}

\addAttore{\admin}
\addAttoreSec{\db}
\addPrC{L'\admin si trova nella sezione Dashboard.}
\addPoC{Il Sistema aggiunge nella visualizzazione il modulo statistiche allarmi.}
\addScenarioPrincipale{L'\admin aggiunge alla visualizzazione il modulo statistiche allarmi.}
\makeUC[0.6]{18}{Aggiunta modulo statistiche allarmi}

\addAttore{\admin}
\addAttoreSec{\db}
\addPrC{L'\admin si trova nella sezione Dashboard.}
\addPoC{Il Sistema rimuove nella visualizzazione il modulo statistiche allarmi.}
\addScenarioPrincipale{L'\admin rimuove dalla visualizzazione il modulo statistiche allarmi.}
\makeUC[0.6]{19}{Rimozione modulo statistiche allarmi}

\addAttore{\admin}
\addAttoreSec{\db}
\addPrC{L'\admin si trova nella sezione Dashboard.}
\addPoC{Il Sistema aggiunge nella visualizzazione il modulo informazioni \user}
\addScenarioPrincipale{L'\admin aggiunge alla visualizzazione il modulo informazioni \user}
\makeUC[0.6]{20}{Aggiunta modulo informazioni \user}

\addAttore{\admin}
\addAttoreSec{\db}
\addPrC{L'\admin si trova nella sezione Dashboard.}
\addPoC{Il Sistema rimuove nella visualizzazione il modulo informazioni \user}
\addScenarioPrincipale{L'\admin rimuove dalla visualizzazione il modulo informazioni \user}
\makeUC[0.6]{21}{Rimozione modulo informazioni \user}

\addAttore{\admin}
\addAttoreSec{\db}
\addPrC{L'\admin si trova nella sezione Dashboard.}
\addPoC{Il Sistema aggiunge nella visualizzazione il modulo analisi clima.}
\addScenarioPrincipale{L'\admin rimuove dalla visualizzazione il modulo analisi clima.}
\makeUC[0.6]{22}{Aggiunta modulo analisi clima}

\addAttore{\admin}
\addAttoreSec{\db}
\addPrC{L'\admin si trova nella sezione Dashboard.}
\addPoC{Il Sistema rimuove nella visualizzazione il modulo analisi clima.}
\addScenarioPrincipale{L'\admin rimuove dalla visualizzazione il modulo analisi clima.}
\makeUC[0.6]{23}{Rimozione modulo analisi clima}

\addAttore{\admin}
\addAttoreSec{\db}
\addPrC{L'\admin si trova nella sezione Dashboard.}
\addPoC{Il Sistema aggiunge nella visualizzazione il modulo analisi consumi.}
\addScenarioPrincipale{L'\admin aggiunge alla visualizzazione il modulo analisi consumi.}
\makeUC[0.6]{24}{Aggiunta modulo analisi consumi}

\addAttore{\admin}
\addAttoreSec{\db}
\addPoC{Il Sistema rimuove nella visualizzazione il modulo analisi consumi.}
\addScenarioPrincipale{L'\admin rimuove dalla visualizzazione il modulo analisi consumi.}
\addPrC{L'\admin si trova nella sezione Dashboard.}
\makeUC[0.6]{25}{Rimozione modulo analisi consumi}

\addAttore{\user}
\addPrC{L'\user si trova nella sezione Dashboard.}
\addPoC{Il Sistema contrassegna l'allarme come risolta.}
\addScenarioPrincipale{L'\user contrassegna l'allarme come risolta.}
\makeUC[0.6]{26}{Risoluzione allarme}

% --------

\addAttore{\user}
\addPrC{Il Sistema è attivo;}
\addPrC{L'\user è autenticato nel Sistema.}
\addPoC{L'\user è all'interno della sezione analytics.}
\addTrigger{L’\user vuole visualizzare la sezione analytics.}
\addScenarioPrincipale{L'\user seleziona la sezione analytics dal menù principale;}
%\addScenarioPrincipale{L'\user seleziona il gruppo di impianti che si vuole analizzare (\rif{3.10});}
\addScenarioPrincipale{L'\user visualizza la sezione analytics, che permette di visualizzare i seguenti moduli relativi agli impianti 
monitorati negli ultimi 30 giorni:
\begin{itemize}
    \item suggerimenti per il risparmio energetico (\rif{\Va.1});
    \item grafico sul calcolo dell’energia consumata dall’illuminazione (\rif{\Va.2});
    \item grafico sulle anomalie dell'impianto (\rif{\Va.3});
    \item grafico sul rilevamento di presenza, assenza e caduta (\rif{\Va.4});
    \item grafico sul rilevamento di presenza prolungata nello stesso ambiente (\rif{\Va.5});
    \item grafico di variazione e cambio di temperatura (\rif{\Va.6});
    \item grafico relativo agli allarmi inviati e risolti per giorno dagli operatori sanitari (\rif{\Va.7});
    \item grafico relativo alla frequenza degli allarmi rilevati (\rif{\Va.8});
    \item grafico relativo alla frequenza delle cadute rilevate (\rif{\Va.9}).
\end{itemize}
}
\addInclusione{\rif{\Va.1}}
\addInclusione{\rif{\Va.2}}
\addInclusione{\rif{\Va.3}}
\addInclusione{\rif{\Va.4}}
\addInclusione{\rif{\Va.5}}
\addInclusione{\rif{\Va.6}}
\addInclusione{\rif{\Va.7}}
\addInclusione{\rif{\Va.8}}
\addInclusione{\rif{\Va.9}}
\makeUC[0.6][0.55]{\Va}{Visualizzazione analytics}

\addAttore{\user}
\addPrC{L'\user si trova nella sezione analytics.}
\addPoC{Il Sistema mostra l'elenco di tutti i suggerimenti risparmio energetico.}
%\addTrigger{L'\user vuole visualizzare il modulo relativo ai suggerimenti risparmio energetico.}
\addScenarioPrincipale{L'\user visualizza l'elenco di tutti i suggerimenti risparmio energetico (\rif{\Va.1.1})}
\addInclusione{\rif{\Va.1.1}}
\makeUC{\Va.1}{Visualizzazione elenco suggerimenti risparmio energetico}

\addAttore{\user}
\addPrC{L'\user sta visualizzando l'elenco dei suggerimenti risparmio energetico.}
\addPoC{Il Sistema mostra un suggerimenti risparmio energetico con le proprie informazioni.}
%\addTrigger{L'\user vuole visualizzare il modulo relativo ai suggerimenti risparmio energetico.}
\addScenarioPrincipale{L'\user visualizza un suggerimento risparmio energetico.}
\makeUC{\Va.1.1}{Visualizzazione elemento elenco suggerimenti risparmio energetico}

\addAttore{\user}
\addPrC{L'\user si trova nella sezione analytics.}
\addPoC{Il Sistema mostra il grafico dedicato al consumo energetico.}
%\addTrigger{L'\user vuole visualizzare il grafico dedicato al consumo energetico.}
\addScenarioPrincipale{L'\user visualizza il grafico dedicato al consumo energetico.}
\makeUC{\Va.2}{Visualizzazione grafico dedicato al consumo energetico}

\addAttore{\user}
\addPrC{L'\user si trova nella sezione analytics.}
\addPoC{Il Sistema mostra il grafico dedicato alle anomalie dell'impianto.}
%\addTrigger{L'\user vuole visualizzare il grafico dedicato alle anomalie dell'impianto.}
\addScenarioPrincipale{L'\user visualizza il grafico dedicato alle anomalie dell'impianto.}
\makeUC{\Va.3}{Visualizzazione grafico dedicato alle anomalie dell'impianto}

\addAttore{\user}
\addPrC{L'\user si trova nella sezione analytics.}
\addPoC{Il Sistema mostra il grafico relativo al rilevamento di presenza.}
%\addTrigger{L'\user vuole visualizzare il grafico relativo al rilevamento di presenza.}
\addScenarioPrincipale{L'\user visualizza il grafico relativo al rilevamento di presenza.}
\makeUC{\Va.4}{Visualizzazione grafico relativo al rilevamento di presenza}

\addAttore{\user}
\addPrC{L'\user si trova nella sezione analytics.}
\addPoC{Il Sistema mostra il grafico relativo alla presenza prolungata nello stesso ambiente.}
%\addTrigger{L'\user vuole visualizzare il grafico relativo alla presenza prolungata nello stesso ambiente.}
\addScenarioPrincipale{L'\user visualizza il grafico relativo alla presenza prolungata nello stesso ambiente.}
\makeUC{\Va.5}{Visualizzazione grafico relativo alla presenza prolungata nello stesso ambiente}

\addAttore{\user}
\addPrC{L'\user si trova nella sezione analytics.}
\addPoC{Il Sistema mostra il grafico relativo alle variazioni di temperatura.}
%\addTrigger{L'\user vuole visualizzare il grafico relativo alla presenza prolungata nello stesso ambiente.}
\addScenarioPrincipale{L'\user visualizza il grafico relativo alle variazioni di temperatura.}
\makeUC{\Va.6}{Visualizzazione grafico relativo alle variazioni di temperatura}

\addAttore{\user}
\addPrC{L'\user si trova nella sezione analytics.}
\addPoC{Il Sistema mostra il grafico relativo agli allarmi inviati e risolti.}
%\addTrigger{L'\user vuole visualizzare il grafico relativo agli allarmi inviati e risolti.}
\addScenarioPrincipale{L'\user visualizza il grafico relativo agli allarmi inviati e risolti.}
\makeUC{\Va.7}{Visualizzazione grafico relativo agli allarmi inviati e risolti}

\addAttore{\user}
\addPrC{L'\user si trova nella sezione analytics.}
\addPoC{Il Sistema mostra il grafico relativo alla frequenza degli allarmi.}
%\addTrigger{L'\user vuole visualizzare il grafico relativo alla frequenza degli allarmi.}
\addScenarioPrincipale{L'\user visualizza il grafico relativo alla frequenza degli allarmi.}
\makeUC{\Va.8}{Visualizzazione grafico relativo alla frequenza degli allarmi}

\addAttore{\user}
\addPrC{L'\user si trova nella sezione analytics.}
\addPoC{Il Sistema mostra il grafico relativo alla frequenza delle cadute.}
%\addTrigger{L'\user vuole visualizzare il grafico relativo alla frequenza delle cadute.}
\addScenarioPrincipale{L'\user visualizza il grafico relativo alla frequenza delle cadute.}
\makeUC{\Va.9}{Visualizzazione grafico relativo alla frequenza delle cadute}

\addAttore{\user}
\addPrC{Il Sistema è attivo;}
\addPrC{L'\user è autenticato nel Sistema.}
\addPrC{L'\user si trova nella sezione appartamento.}
\addScenarioPrincipale{L'\user visualizza:
\begin{itemize}
    \item il nome dell'appartamento (\rif{\Vap.1});
    \item la mappa degli allarmi (\rif{\Vap.2})
    \item le stanze dell'appartamento (\rif{\Vap.3})
\end{itemize}
}
\addInclusione{\rif{\Vap.1}}
\addInclusione{\rif{\Vap.2}}
\addInclusione{\rif{\Vap.3}}
\makeUC[0.6]{\Vap}{Visualizzazione appartamento}

\addAttore{\user}
\addPrC{l'\user sta visualizzando un appartamento}
\addPoC{l'\user visualizza il nome dell'appartamento}
\addScenarioPrincipale{l'\user visualizza il nome dell'appartamento}
\makeUC{\Vap.1}{Visualizzazione nome appartamento}

\addAttore{\admin}
\addPrC{L'\admin sta visualizzando un appartamento.}
\addPoC{L'\admin visualizza la mappa degli allarmi dell'appartamento.}
\addScenarioPrincipale{L'\admin visualizza la mappa degli allarmi dell'appartamento.}
\makeUC{\Vap.2}{Visualizzazione mappa allarmi}

\addAttore{\user}
\addPrC{L'\user sta visualizzando un appartamento}
\addPoC{Il Sistema mostra l'elenco di tutte le stanze}
\addScenarioPrincipale{l'\user visualizza l'elenco di tutte le stanze (\rif{\Vap.3.1})}
\addInclusione{\rif{\Vap.3.1}}
\makeUC{\Vap.3}{Visualizzazione elenco stanze}

\addAttore{\user}
\addPrC{L'\user sta visualizzando le stanze di un appartamento}
\addPoC{Il Sistema mostra una stanza con le proprie informazioni}
\addScenarioPrincipale{l'\user visualizza una stanza con le proprie informazioni:
\begin{itemize}
    \item nome della stanza (\rif{\Vap.3.1.1});
    \item tutti di dispositivi presenti nella stanza (\rif{\Vap.3.1.2}).
\end{itemize}}
\addInclusione{\rif{\Vap.3.1.1}}
\addInclusione{\rif{\Vap.3.1.2}}
\makeUC{\Vap.3.1}{Visualizzazione elemento elenco stanze}

\addAttore{\user}
\addPrC{l'\user sta visualizzando una stanza di un appartamento}
\addPoC{l'\user visualizza il nome della stanza}
\addScenarioPrincipale{l'\user visualizza il nome della stanza}
\makeUC{\Vap.3.1.1}{Visualizzazione nome}

\addAttore{\user}
\addPrC{L'\user sta visualizzando una stanza di un appartamento}
%\addTrigger{L'\os vuole visualizzare i dispositivi di una stanza.}
\addPoC{L'\user visualizza l'elenco dei dispositivi di una stanza.}
\addScenarioPrincipale{L'\user visualizza l'elenco dei dispositivi presenti in una stanza (\rif{\Vap.3.1.2.1})}
\addInclusione{\rif{\Vap.3.1.2.1}}
\makeUC{\Vap.3.1.2}{Visualizzazione elenco dispositivi}

% -- da rimuovere in caso --

\addAttore{\user}
\addPrC{L'\user visualizza la lista dei dispositivi di una stanza.}
\addPoC{Viene visualizzato un dispositivo con le proprie informazioni}
%\addTrigger{L'\os vuole visualizzare il termostato dell'impianto.}
\addScenarioPrincipale{L'\os visualizza le caratteristiche relative al singolo dispositivo:
\begin{itemize}
    \item il nome del dispositivo (\rif{\Vap.3.1.2.1.1});
    \item lo stato corrente del dispositivo (\rif{\Vap.3.1.2.1.2});
    \item le azioni eseguibili (\rif{\Vap.3.1.2.1.3}).
\end{itemize}
}
\addInclusione{\rif{\Vap.3.1.2.1.1}}
\addInclusione{\rif{\Vap.3.1.2.1.2}}
\addInclusione{\rif{\Vap.3.1.2.1.3}}
\makeUC[0.725]{\Vap.3.1.2.1}{Visualizzazione elemento elenco dispositivi}

\addAttore{\user}
\addPrC{l'\user sta visualizzando un dispositivo}
\addPoC{l'\user visualizza il nome del dispositivo}
\addScenarioPrincipale{l'\user visualizza il nome del dispositivo}
\makeUC{\Vap.3.1.2.1.1}{Visualizzazione nome}

\addAttore{\user}
\addPrC{l'\user sta visualizzando un dispositivo}
\addPoC{l'\user visualizza lo stato del dispositivo}
\addScenarioPrincipale{l'\user visualizza lo stato del dispositivo}
\makeUC{\Vap.3.1.2.1.2}{Visualizzazione stato}

\addAttore{\user}
\addPrC{l'\user sta visualizzando un dispositivo}
\addPoC{l'\user visualizza le azioni eseguibili per il dispositivo}
\addScenarioPrincipale{l'\user visualizza le azioni eseguibili per il dispositivo}
\makeUC{\Vap.3.1.2.1.3}{Visualizzazione azioni eseguibili}

\addAttore{\user}
\addPrC{L'\user visualizza la lista dei dispositivi di una stanza.}
\addPoC{Viene visualizzato un termostato con le proprie informazioni}
%\addTrigger{L'\os vuole visualizzare il termostato dell'impianto.}
\addScenarioPrincipale{L'\user visualizza le caratteristiche relative al dispositivo termostato:
\begin{itemize}
    \item il nome del dispositivo;
    \item lo stato corrente del dispositivo;
    \item le azioni eseguibili.
\end{itemize}
}
\makeUC{\Vap.3.1.2.2}{Visualizzazione termostato}

\addAttore{\user}
\addPrC{L'\user visualizza la lista dei dispositivi di una stanza.}
\addPoC{Viene visualizzato un sensore di caduta con le proprie informazioni}
\addScenarioPrincipale{L'\user visualizza le caratteristiche relative al dispositivo sensore caduta:
\begin{itemize}
    \item il nome del dispositivo;
    \item lo stato corrente del dispositivo;
    \item le azioni eseguibili.
\end{itemize}}
\makeUC{\Vap.3.1.2.3}{Visualizzazione sensore di caduta}

\addAttore{\user}
\addPrC{L'\user visualizza la lista dei dispositivi di una stanza.}
\addPoC{Il Sistema mostra per un sensore di presenza le sue informazioni.}
\addScenarioPrincipale{L'\user visualizza le caratteristiche relative al dispositivo sensore di presenza:
\begin{itemize}
    \item il nome del dispositivo;
    \item lo stato corrente del dispositivo;
    \item le azioni eseguibili.
\end{itemize}}
\makeUC{\Vap.3.1.2.4}{Visualizzazione sensore di presenza}

\addAttore{\user}
\addPrC{L'\user visualizza la lista dei dispositivi di una stanza.}
\addPoC{L'\user visualizza il dispositivo luce di un impianto specifico.}
\addScenarioPrincipale{L'\user visualizza le caratteristiche relative al dispositivo luce:
\begin{itemize}
    \item il nome del dispositivo;
    \item lo stato corrente del dispositivo;
    \item le azioni eseguibili.
\end{itemize}}
\makeUC{\Vap.3.1.2.5}{Visualizzazione punto luce}

\addAttore{\user}
\addPrC{L'\user visualizza la lista dei dispositivi di una stanza.}
\addPoC{L'\user visualizza il dispositivo pulsante di allarme di un impianto specifico.}
\addScenarioPrincipale{L'\user visualizza le caratteristiche relative al dispositivo pulsante di allarme:
\begin{itemize}
    \item il nome del dispositivo;
    \item lo stato corrente del dispositivo;
    \item le azioni eseguibili.
\end{itemize}}
\makeUC{\Vap.3.1.2.6}{Visualizzazione pulsante di allarme}

\addAttore{\user}
\addPrC{L'\user visualizza la lista dei dispositivi di una stanza.}
\addPoC{L'\user visualizza il dispositivo porta di ingresso di un impianto specifico.}
\addScenarioPrincipale{L'\user visualizza le caratteristiche relative al dispositivo porta di ingresso:
\begin{itemize}
    \item il nome del dispositivo;
    \item lo stato corrente del dispositivo;
    \item le azioni eseguibili.
\end{itemize}}
\makeUC{\Vap.3.1.2.7}{Visualizzazione porta di ingresso}

\addAttore{\user}
\addPrC{L'\user visualizza la lista dei dispositivi di una stanza.}
\addPoC{L'\user visualizza il dispositivo tapparella di un impianto specifico.}
\addScenarioPrincipale{L'\user visualizza le caratteristiche relative al dispositivo tapparella:
\begin{itemize}
    \item il nome del dispositivo;
    \item lo stato corrente del dispositivo;
    \item le azioni eseguibili.
\end{itemize}}
\makeUC{\Vap.3.1.2.8}{Visualizzazione tapparella}

% \addAttore{\user}
% \addPrC{l'\user sta visualizzando una stanza di un appartamento}
% %\addTrigger{L'\os vuole visualizzare i dispositivi di una stanza.}
% \addPoC{L'\user visualizza l'elenco dei dispositivi di una stanza.}
% \addScenarioPrincipale{L'\user visualizza l'elenco dei dispositivi presenti in una stanza tra:
% \begin{itemize}
%     \item termostato (\rif{\Vap.3.1.2.1});
%     \item sensori di caduta (\rif{\Vap.3.1.2.2});
%     \item sensori di presenza (\rif{\Vap.3.1.2.3});
%     \item punti luci (\rif{\Vap.3.1.2.4});
%     \item pulsante di allarme (\rif{\Vap.3.1.2.5});
%     \item porta di ingresso (\rif{\Vap.3.1.2.6});
%     \item tapparella (\rif{\Vap.3.1.2.7}).
% \end{itemize}}
% \addInclusione{\rif{\Vap.3.1.2.1}}
% \addInclusione{\rif{\Vap.3.1.2.2}}
% \addInclusione{\rif{\Vap.3.1.2.3}}
% \addInclusione{\rif{\Vap.3.1.2.4}}
% \addInclusione{\rif{\Vap.3.1.2.5}}
% \addInclusione{\rif{\Vap.3.1.2.6}}
% \addInclusione{\rif{\Vap.3.1.2.7}}
% \makeUC{\Vap.3.1.2}{Visualizzazione elenco dispositivi}

% \addAttore{\user}
% \addPrC{L'\user visualizza la lista dei dispositivi di una stanza.}
% %\addTrigger{L'\os vuole visualizzare il termostato dell'impianto.}
% \addScenarioPrincipale{L'\os visualizza le caratteristiche relative al dispositivo termostato:
% \begin{itemize}
%     \item il nome del dispositivo (\rif{\Vap.3.1.2.1.1});
%     \item lo stato corrente del dispositivo (\rif{\Vap.3.1.2.1.2});
%     \item le azioni eseguibili (\rif{\Vap.3.1.2.1.3}).
% \end{itemize}
% }
% \addInclusione{\rif{\Vap.3.1.2.1.1}}
% \addInclusione{\rif{\Vap.3.1.2.1.2}}
% \addInclusione{\rif{\Vap.3.1.2.1.3}}
% \makeUC{\Vap.3.1.2.1}{Visualizzazione termostato}

% \addAttore{\user}
% \addPrC{l'\user sta visualizzando un termostato}
% \addPoC{l'\user visualizza il nome del termostato}
% \addScenarioPrincipale{l'\user visualizza il nome del termostato}
% \makeUC{\Vap.3.1.2.1.1}{Visualizzazione nome}

% \addAttore{\user}
% \addPrC{l'\user sta visualizzando un termostato}
% \addPoC{l'\user visualizza lo stato del termostato}
% \addScenarioPrincipale{l'\user visualizza lo stato del termostato}
% \makeUC{\Vap.3.1.2.1.2}{Visualizzazione stato}

% \addAttore{\user}
% \addPrC{l'\user sta visualizzando un termostato}
% \addPoC{l'\user visualizza le azioni eseguibili per il termostato}
% \addScenarioPrincipale{l'\user visualizza le azioni eseguibili per il termostato}
% \makeUC{\Vap.3.1.2.1.3}{Visualizzazione azioni eseguibili}

% \addAttore{\user}
% \addPrC{L'\user visualizza la lista dei dispositivi di una stanza.}
% \addScenarioPrincipale{L'\user visualizza le caratteristiche relative al dispositivo sensore caduta:
% \begin{itemize}
%     \item il nome del dispositivo (\rif{\Vap.3.1.2.2.1});
%     \item lo stato corrente del dispositivo (\rif{\Vap.3.1.2.2.2});
%     \item le azioni eseguibili (\rif{\Vap.3.1.2.2.3}).
% \end{itemize}}
% \addInclusione{\rif{\Vap.3.1.2.2.1}}
% \addInclusione{\rif{\Vap.3.1.2.2.2}}
% \addInclusione{\rif{\Vap.3.1.2.2.3}}
% \makeUC{\Vap.3.1.2.2}{Visualizzazione sensore di caduta}

% \addAttore{\user}
% \addPrC{l'\user sta visualizzando un sensore di caduta}
% \addPoC{l'\user visualizza il nome del sensore di caduta}
% \addScenarioPrincipale{l'\user visualizza il nome del sensore di caduta}
% \makeUC{\Vap.3.1.2.2.1}{Visualizzazione nome}

% \addAttore{\user}
% \addPrC{l'\user sta visualizzando un sensore di caduta}
% \addPoC{l'\user visualizza lo stato del sensore di caduta}
% \addScenarioPrincipale{l'\user visualizza lo stato del sensore di caduta}
% \makeUC{\Vap.3.1.2.2.2}{Visualizzazione stato}

% \addAttore{\user}
% \addPrC{l'\user sta visualizzando un sensore di caduta}
% \addPoC{l'\user visualizza le azioni eseguibili per il sensore di caduta}
% \addScenarioPrincipale{l'\user visualizza le azioni eseguibili per il sensore di caduta}
% \makeUC{\Vap.3.1.2.2.3}{Visualizzazione azioni eseguibili}

% \addAttore{\user}
% \addPrC{L'\user visualizza la lista dei dispositivi di una stanza.}
% \addPoC{Il Sistema mostra per un sensore di presenza le sue informazioni.}
% %\addTrigger{L'\os vuole visualizzare il sensore \textit{UWB} presenza dell'impianto.}
% \addScenarioPrincipale{L'\os visualizza le caratteristiche relative al dispositivo sensore di presenza:
% \begin{itemize}
%     \item il nome del dispositivo (\rif{\Vap.3.1.2.3.1});
%     \item lo stato corrente del dispositivo (\rif{\Vap.3.1.2.3.2});
%     \item le azioni eseguibili (\rif{\Vap.3.1.2.3.3}).
% \end{itemize}}
% \addInclusione{\rif{\Vap.3.1.2.3.1}}
% \addInclusione{\rif{\Vap.3.1.2.3.2}}
% \addInclusione{\rif{\Vap.3.1.2.3.3}}
% \makeUC{\Vap.3.1.2.3}{Visualizzazione sensore di presenza}

% \addAttore{\user}
% \addPrC{l'\user sta visualizzando un sensore di presenza}
% \addPoC{l'\user visualizza il nome del sensore di presenza}
% \addScenarioPrincipale{l'\user visualizza il nome del sensore di presenza}
% \makeUC{\Vap.3.1.2.3.1}{Visualizzazione nome}

% \addAttore{\user}
% \addPrC{l'\user sta visualizzando un sensore di presenza}
% \addPoC{l'\user visualizza lo stato del sensore di presenza}
% \addScenarioPrincipale{l'\user visualizza lo stato del sensore di presenza}
% \makeUC{\Vap.3.1.2.3.2}{Visualizzazione stato}

% \addAttore{\user}
% \addPrC{l'\user sta visualizzando un sensore di presenza}
% \addPoC{l'\user visualizza le azioni eseguibili per il sensore di presenza}
% \addScenarioPrincipale{l'\user visualizza le azioni eseguibili per il sensore di presenza}
% \makeUC{\Vap.3.1.2.3.3}{Visualizzazione azioni eseguibili}

% \addAttore{\user}
% \addPrC{L'\user visualizza la lista dei dispositivi di una stanza.}
% \addPoC{L'\user visualizza il dispositivo luce di un impianto specifico.}
% %\addTrigger{L'\os vuole visualizzare la luce dell'impianto.}
% \addScenarioPrincipale{L'\user visualizza le caratteristiche relative al dispositivo luce:
% \begin{itemize}
%     \item il nome del dispositivo (\rif{\Vap.3.1.2.4.1});
%     \item lo stato corrente del dispositivo (\rif{\Vap.3.1.2.4.2});
%     \item le azioni eseguibili (\rif{\Vap.3.1.2.4.3}).
% \end{itemize}}
% \addInclusione{\rif{\Vap.3.1.2.4.1}}
% \addInclusione{\rif{\Vap.3.1.2.4.2}}
% \addInclusione{\rif{\Vap.3.1.2.4.3}}
% \makeUC{\Vap.3.1.2.4}{Visualizzazione punto luce}

% \addAttore{\user}
% \addPrC{l'\user sta visualizzando un punto luce}
% \addPoC{l'\user visualizza il nome del punto luce}
% \addScenarioPrincipale{l'\user visualizza il nome del punto luce}
% \makeUC{\Vap.3.1.2.4.1}{Visualizzazione nome}

% \addAttore{\user}
% \addPrC{l'\user sta visualizzando un punto luce}
% \addPoC{l'\user visualizza lo stato del punto luce}
% \addScenarioPrincipale{l'\user visualizza lo stato del punto luce}
% \makeUC{\Vap.3.1.2.4.2}{Visualizzazione stato}

% \addAttore{\user}
% \addPrC{l'\user sta visualizzando un punto luce}
% \addPoC{l'\user visualizza le azioni eseguibili per il punto luce}
% \addScenarioPrincipale{l'\user visualizza le azioni eseguibili per il punto luce}
% \makeUC{\Vap.3.1.2.4.3}{Visualizzazione azioni eseguibili}

% \addAttore{\user}
% \addPrC{L'\user visualizza la lista dei dispositivi di una stanza.}
% \addPoC{L'\user visualizza il dispositivo pulsante di allarme di un impianto specifico.}
% %\addTrigger{L'\os vuole visualizzare il pulsante di allarme dell'impianto.}
% \addScenarioPrincipale{L'\os visualizza le caratteristiche relative al dispositivo pulsante di allarme:
% \begin{itemize}
%     \item il nome del dispositivo (\rif{\Vap.3.1.2.5.1});
%     \item lo stato corrente del dispositivo (\rif{\Vap.3.1.2.5.2});
%     \item le azioni eseguibili (\rif{\Vap.3.1.2.5.3}).
% \end{itemize}}
% \addInclusione{\rif{\Vap.3.1.2.5.1}}
% \addInclusione{\rif{\Vap.3.1.2.5.2}}
% \addInclusione{\rif{\Vap.3.1.2.5.3}}
% \makeUC{\Vap.3.1.2.5}{Visualizzazione pulsante di allarme}

% \addAttore{\user}
% \addPrC{l'\user sta visualizzando un pulsante di allarme}
% \addPoC{l'\user visualizza il nome del pulsante di allarme}
% \addScenarioPrincipale{l'\user visualizza il nome del pulsante di allarme}
% \makeUC{\Vap.3.1.2.5.1}{Visualizzazione nome}

% \addAttore{\user}
% \addPrC{l'\user sta visualizzando un pulsante di allarme}
% \addPoC{l'\user visualizza lo stato del pulsante di allarme}
% \addScenarioPrincipale{l'\user visualizza lo stato del pulsante di allarme}
% \makeUC{\Vap.3.1.2.5.2}{Visualizzazione stato}

% \addAttore{\user}
% \addPrC{l'\user sta visualizzando un pulsante di allarme}
% \addPoC{l'\user visualizza le azioni eseguibili per il pulsante di allarme}
% \addScenarioPrincipale{l'\user visualizza le azioni eseguibili per il pulsante di allarme}
% \makeUC{\Vap.3.1.2.5.3}{Visualizzazione azioni eseguibili}

% \addAttore{\user}
% \addPrC{L'\user visualizza la lista dei dispositivi di una stanza.}
% \addPoC{L'\user visualizza il dispositivo porta di ingresso di un impianto specifico.}
% %\addTrigger{L'\os vuole visualizzare la porta di ingresso dell'impianto.}
% \addScenarioPrincipale{L'\os visualizza le caratteristiche relative al dispositivo porta di ingresso:
% \begin{itemize}
%     \item il nome del dispositivo (\rif{\Vap.3.1.2.6.1});
%     \item lo stato corrente del dispositivo (\rif{\Vap.3.1.2.6.2});
%     \item le azioni eseguibili (\rif{\Vap.3.1.2.6.3}).
% \end{itemize}}
% \addInclusione{\rif{\Vap.3.1.2.6.1}}
% \addInclusione{\rif{\Vap.3.1.2.6.2}}
% \addInclusione{\rif{\Vap.3.1.2.6.3}}
% \makeUC{\Vap.3.1.2.6}{Visualizzazione porta di ingresso}

% \addAttore{\user}
% \addPrC{l'\user sta visualizzando una porta di ingresso}
% \addPoC{l'\user visualizza il nome della porta di ingresso}
% \addScenarioPrincipale{l'\user visualizza il nome della porta di ingresso}
% \makeUC{\Vap.3.1.2.6.1}{Visualizzazione nome}

% \addAttore{\user}
% \addPrC{l'\user sta visualizzando una porta di ingresso}
% \addPoC{l'\user visualizza lo stato della porta di ingresso}
% \addScenarioPrincipale{l'\user visualizza lo stato della porta di ingresso}
% \makeUC{\Vap.3.1.2.6.2}{Visualizzazione stato}

% \addAttore{\user}
% \addPrC{l'\user sta visualizzando una porta di ingresso}
% \addPoC{l'\user visualizza le azioni eseguibili per la porta di ingresso}
% \addScenarioPrincipale{l'\user visualizza le azioni eseguibili per la porta di ingresso}
% \makeUC{\Vap.3.1.2.6.3}{Visualizzazione azioni eseguibili}

% \addAttore{\user}
% \addPrC{L'\user visualizza la lista dei dispositivi di una stanza.}
% \addPoC{L'\user visualizza il dispositivo tapparella di un impianto specifico.}
% %\addTrigger{L'\os vuole visualizzare la tapparella dell'impianto.}
% \addScenarioPrincipale{L'\os visualizza le caratteristiche relative al dispositivo tapparella:
% \begin{itemize}
%     \item il nome del dispositivo (\rif{\Vap.3.1.2.7.1});
%     \item lo stato corrente del dispositivo (\rif{\Vap.3.1.2.7.2});
%     \item le azioni eseguibili (\rif{\Vap.3.1.2.7.3}).
% \end{itemize}}
% \addInclusione{\rif{\Vap.3.1.2.7.1}}
% \addInclusione{\rif{\Vap.3.1.2.7.2}}
% \addInclusione{\rif{\Vap.3.1.2.7.3}}
% \makeUC{\Vap.3.1.2.7}{Visualizzazione tapparella}

% \addAttore{\user}
% \addPrC{l'\user sta visualizzando una tapparella}
% \addPoC{l'\user visualizza il nome della tapparella}
% \addScenarioPrincipale{l'\user visualizza il nome della tapparella}
% \makeUC{\Vap.3.1.2.7.1}{Visualizzazione nome}

% \addAttore{\user}
% \addPrC{l'\user sta visualizzando una tapparella}
% \addPoC{l'\user visualizza lo stato della tapparella}
% \addScenarioPrincipale{l'\user visualizza lo stato della tapparella}
% \makeUC{\Vap.3.1.2.7.2}{Visualizzazione stato}

% \addAttore{\user}
% \addPrC{l'\user sta visualizzando una tapparella}
% \addPoC{l'\user visualizza le azioni eseguibili per la tapparella}
% \addScenarioPrincipale{l'\user visualizza le azioni eseguibili per la tapparella}
% \makeUC{\Vap.3.1.2.7.3}{Visualizzazione azioni eseguibili}

\addAttore{\admin}
\addAttoreSec{\db}
\addPrC{Il Sistema è attivo;}
\addPrC{L'\admin è autenticato nel Sistema;}
\addPrC{L'\admin si trova nella sezione dedicata alla visualizzazione di un appartamento.}
\addPoC{Il Sistema abilita l'appartamento}
\addScenarioPrincipale{L'\admin abilita l'appartamento}
\makeUC[0.6]{\Aapp}{Abilita appartamento}

\addAttore{\admin}
\addAttoreSec{\db}
\addPrC{Il Sistema è attivo;}
\addPrC{L'\admin è autenticato nel Sistema;}
\addPrC{L'\admin si trova nella sezione dedicata alla visualizzazione di un appartamento.}
\addPoC{Il Sistema disabilita l'appartamento}
\addScenarioPrincipale{L'\admin disabilita l'appartamento}
\makeUC[0.6]{\Dapp}{Disabilita appartamento}

% --------

\addAttore{\admin}
\addPrC{Il Sistema è attivo;}
\addPrC{L'\admin è autenticato nel Sistema;}
\addPrC{L'\admin sta visualizzando la sezione di configurazione allarmi.}
\addPoC{Un nuovo allarme è aggiunto alla configurazione del Sistema.}
%\addTrigger{L'\admin vuole aggiungere un nuovo allarme.}
\addScenarioPrincipale{L'\admin seleziona le seguenti opzioni per creare l'allarme:
\begin{itemize}
    \item seleziona l'appartamento (\rif{\Aa.1});
    \item seleziona il sensore (\rif{\Aa.2});
    \item seleziona il livello di priorità (\rif{\Aa.3});
    \item seleziona la soglia di intervento (\rif{\Aa.8});
    \item seleziona l'orario di attivazione (\rif{\Aa.9});
    \item seleziona l'orario di disattivazione (\rif{\Aa.10});
\end{itemize}
}
\addScenarioAlternativo{Nessun sensore selezionato (\rif{\Erns})}
\addScenarioAlternativo{Nessun livello di priorità selezionato (\rif{\Ernlp})}
\addScenarioAlternativo{Nessuna soglia di intervento selezionata (\rif{\Ernss})}
\addEstensione{\rif{\Erns}}
\addEstensione{\rif{\Ernlp}}
\addEstensione{\rif{\Ernss}}
\addInclusione{\rif{\Aa.1}}
\addInclusione{\rif{\Aa.2}}
\addInclusione{\rif{\Aa.3}}
\addInclusione{\rif{\Aa.8}}
\addInclusione{\rif{\Aa.9}}
\addInclusione{\rif{\Aa.10}}
\makeUC[1]{\Aa}{Creazione allarme}

\addAttore{\admin}
\addPrC{L'\admin sta aggiungendo un nuovo allarme;}
\addPoC{L'appartamento è selezionato per l'allarme che si vuole creare.}
%\addTrigger{L'\admin deve selezionare un sensore per l'allarme.}
\addScenarioPrincipale{L'\admin seleziona l'appartamento per l'allarme che vuole}
\makeUC{\Aa.1}{Selezione appartamento}

\addAttore{\admin}
\addPrC{L'\admin sta aggiungendo un nuovo allarme;}
\addPoC{Il sensore è selezionato per l'allarme che si vuole creare.}
%\addTrigger{L'\admin deve selezionare un sensore per l'allarme.}
\addScenarioPrincipale{L'\admin seleziona il sensore per l'allarme che vuole}
\makeUC{\Aa.2}{Selezione sensore}

\addAttore{\admin}
\addPrC{L'\admin sta aggiungendo un nuovo allarme;}
\addPoC{La priorità è selezionata per l'allarme che si vuole creare.}
%\addTrigger{L'\admin deve assegnare una priorità all'allarme.}
\addScenarioPrincipale{L'\admin seleziona la priorità dell'allarme che si vuole creare.
}
\makeUC[0.575]{\Aa.3}{Selezione livello priorità allarme}

\addGeneralizzazione{\rif{\Aa.2}}
\addPrC{L'\admin sta creando un nuovo allarme;}
\addPoC{La priorità è livello 1 per l'allarme che si vuole creare.}
\addScenarioPrincipale{L'\admin seleziona la priorità livello 1 per l'allarme che vuole creare}
\makeUC{\Aa.4}{Selezione livello priorità allarme 1 (bianco)}

\addGeneralizzazione{\rif{\Aa.2}}
\addPrC{L'\admin sta creando un nuovo allarme;}
\addPoC{La priorità è livello 2 per l'allarme che si vuole creare.}
\addScenarioPrincipale{L'\admin seleziona la priorità livello 2 per l'allarme che vuole creare}
\makeUC{\Aa.5}{Selezione livello priorità allarme 2 (verde)}

\addGeneralizzazione{\rif{\Aa.2}}
\addPrC{L'\admin sta creando un nuovo allarme;}
\addPoC{La priorità è livello 3 per l'allarme che si vuole creare.}
\addScenarioPrincipale{L'\admin seleziona la priorità livello 3 per l'allarme che vuole creare}
\makeUC{\Aa.6}{Selezione livello priorità allarme 3 (arancione)}

\addGeneralizzazione{\rif{\Aa.2}}
\addPrC{L'\admin sta creando un nuovo allarme;}
\addPoC{La priorità è livello 4 per l'allarme che si vuole creare.}
\addScenarioPrincipale{L'\admin seleziona la priorità livello 4 per l'allarme che vuole creare}
\makeUC{\Aa.7}{Selezione livello priorità allarme 4 (rosso)}

\addAttore{\admin}
\addPrC{L'\admin sta creando un nuovo allarme;}
\addPoC{Le l'orario di attivazione è configurato per l'allarme che si vuole creare.}
%\addTrigger{L'\admin deve configurare le soglie di attivazione dell'allarme.}
\addScenarioPrincipale{L'\admin seleziona una soglia di intervento dell'allarme che vuole creare}
\makeUC{\Aa.8}{Selezione soglia intervento allarme}

\addAttore{\admin}
\addPrC{L'\admin sta creando un nuovo allarme;}
\addPoC{Le l'orario di attivazione è configurato per l'allarme che si vuole creare.}
%\addTrigger{L'\admin deve configurare le soglie di attivazione dell'allarme.}
\addScenarioPrincipale{L'\admin seleziona l'orario di attivazione dell'allarme che vuole creare}
\makeUC{\Aa.9}{Selezione orario attivazione allarme}

\addAttore{\admin}
\addPrC{L'\admin sta creando un nuovo allarme;}
\addPoC{Le l'orario di disattivazione è configurato per l'allarme che si vuole creare.}
%\addTrigger{L'\admin deve configurare le soglie di attivazione dell'allarme.}
\addScenarioPrincipale{L'\admin seleziona l'orario di disattivazione dell'allarme che vuole creare}
\makeUC{\Aa.10}{Selezione orario disattivazione allarme}

% --------

\addAttore{\admin}
\addAttoreSec{\db}
\addPrC{Il Sistema è attivo;}
\addPrC{L'\admin è autenticato nel Sistema;}
\addPrC{L'\admin sta modificando un allarme esistente.}
\addPoC{La priorità dell'allarme è aggiornata.}
%\addTrigger{L'\admin vuole modificare la priorità di un allarme.}
\addScenarioPrincipale{L'\admin seleziona la nuova priorità;}
\makeUC[0.6]{\Mpa}{Modifica priorità allarme}

\addAttore{\admin}
\addAttoreSec{\db}
\addPrC{Il Sistema è attivo;}
\addPrC{L'\admin è autenticato nel Sistema;}
\addPrC{L'\admin sta modificando un allarme esistente.}
\addPoC{La soglia d'intervento dell'allarme è aggiornata.}
%\addTrigger{L'\admin vuole modificare la priorità di un allarme.}
\addScenarioPrincipale{L'\admin seleziona la nuova soglia d'intervento;}
\makeUC[0.6]{\Msia}{Modifica soglia intervento allarme}

\addAttore{\admin}
\addAttoreSec{\db}
\addPrC{Il Sistema è attivo;}
\addPrC{L'\admin è autenticato nel Sistema;}
\addPrC{L'\admin sta modificando un allarme esistente.}
\addPoC{La soglia d'intervento dell'allarme è aggiornata.}
%\addTrigger{L'\admin vuole modificare la priorità di un allarme.}
\addScenarioPrincipale{L'\admin seleziona la nuova soglia d'intervento;}
\makeUC[0.6]{\Moaa}{Modifica orario attivazione allarme}

\addAttore{\admin}
\addAttoreSec{\db}
\addPrC{Il Sistema è attivo;}
\addPrC{L'\admin è autenticato nel Sistema;}
\addPrC{L'\admin sta modificando un allarme esistente.}
\addPoC{La soglia d'intervento dell'allarme è aggiornata.}
%\addTrigger{L'\admin vuole modificare la priorità di un allarme.}
\addScenarioPrincipale{L'\admin seleziona la nuova soglia d'intervento;}
\makeUC[0.6]{\Moda}{Modifica orario disattivazione allarme}

\addAttore{\admin}
\addAttoreSec{\db}
\addPrC{Il Sistema è attivo;}
\addPrC{L'\admin è autenticato nel Sistema;}
\addPrC{L'\admin sta visualizzando la sezione di configurazione allarmi.}
\addPoC{L'allarme viene abilitato dal Sistema.}
\addScenarioPrincipale{L'\admin abilita un allarme dal Sistema.}
\makeUC[0.6]{\Aal}{Abilita allarme}

\addAttore{\admin}
\addAttoreSec{\db}
\addPrC{Il Sistema è attivo;}
\addPrC{L'\admin è autenticato nel Sistema;}
\addPrC{L'\admin sta visualizzando la sezione di configurazione allarmi.}
\addPoC{L'allarme viene disabilitato dal Sistema.}
\addScenarioPrincipale{L'\admin disabilita un allarme dal Sistema.}
\makeUC[0.6]{\Dal}{Disabilita allarme}

% --------

\addAttore{\admin}
\addPrC{Il Sistema è attivo;}
\addPrC{L'\admin è autenticato nel Sistema;}
\addPrC{L'\admin sta visualizzando la sezione di configurazione allarmi.}
\addPoC{L'allarme viene eliminato dal Sistema.}
%\addTrigger{L'\admin vuole eliminare un allarme esistente.}
\addScenarioPrincipale{L'\admin elimina un allarme dal Sistema.}
\makeUC[0.6]{\Ea}{Elimina allarme}

% --------

\addAttore{\user}
\addPrC{Il Sistema è attivo;}
\addPrC{L'\user è autenticato nel Sistema;}
\addPoC{Viene visualizzato l'elenco delle notifiche}
\addScenarioPrincipale{L'\user seleziona dal menù l'opzione relativa alla visualizzazione delle notifiche}
\addScenarioPrincipale{L'\user visualizza l'elenco delle notifiche (\rif{\Vn.1})}
\addInclusione{\rif{\Vn.1}}
\makeUC[0.6]{\Vn}{Visualizzazione elenco notifiche}

\addAttore{\user}
\addPrC{L'\user sta visualizzando l'elenco delle notifiche.}
\addPoC{Viene visualizzato una singola notifica con le proprie informazioni}
\addScenarioPrincipale{L'\user visualizza una notifica con le proprie informazioni:
\begin{itemize}
    \item titolo della notifica (\rif{\Vn.1.1});
    \item tempo trascorso dall'invio della notifica (\rif{\Vn.1.2});
\end{itemize}
}
\addInclusione{\rif{\Vn.1.1}}
\addInclusione{\rif{\Vn.1.2}}
\makeUC{\Vn.1}{Visualizzazione elemento elenco notifiche}

\addAttore{\user}
\addPrC{L'\user sta visualizzando una notifica.}
\addPoC{L'\user visualizza il titolo della notifica}
\addScenarioPrincipale{L'\user visualizza il titolo della notifica}
\makeUC{\Vn.1.1}{Visualizzazione titolo}

\addAttore{\user}
\addPrC{L'\user sta visualizzando una notifica.}
\addPoC{L'\user visualizza il tempo trascorso dall'invio della notifica}
\addScenarioPrincipale{L'\user visualizza il tempo trascorso dall'invio della notifica}
\makeUC{\Vn.1.2}{Visualizzazione tempo trascorso dall'invio}

% ---- Autenticazione ---- 

\addAttore{\user}
\addPrC{L'\user ha inserito un username che non corrisponde ad un \user nel Sistema o una password
errata per l'username inserito;}
\addPoC{Il Sistema non registra il nuovo \user}
\addScenarioPrincipale{Il Sistema segnala l'errore di username non registrato o password errata.}
%\addTrigger{L'\admin sceglie uno username non registrato nel Sistema.}
\makeUC{\Erunr}{Errore username non registrato o password errata}

\addAttore{\user}
\addPrC{L'\user ha inserito un username che non corrisponde ad un \user nel Sistema o una password 
temporanea errata per l'username inserito;}
\addPoC{Il Sistema non registra il nuovo \user}
\addScenarioPrincipale{Il Sistema segnala l'errore di username non registrato o password temporanea errata.}
%\addTrigger{L'\os inserisce una nuova password non valida.}
\makeUC{\Erpte}{Errore username non registrato o password temporanea errata}

\addAttore{\user}
\addPrC{L'\user ha inserito una nuova password che è uguale alla password temporanea;}
\addPoC{Il Sistema non registra il nuovo \user}
\addScenarioPrincipale{Il Sistema segnala l'errore di password attuale errata.}
%\addTrigger{L'\os inserisce una password attuale errata.}
\makeUC{\Ernpupt}{Errore nuova password uguale alla password temporanea}

\addAttore{\user}
\addPrC{L'\user ha inserito una nuova password che non rispetta i criteri richiesti;}
\addPoC{Il Sistema non registra il nuovo \user}
\addScenarioPrincipale{Il Sistema segnala l'errore di nuova password non valida.}
%\addTrigger{L'\os inserisce una nuova password non valida.}
\makeUC{\Ernpnv}{Errore nuova password non valida}

% --- Creazione \user ---

\addAttore{\admin}
\addPrC{L'\admin ha inserito un username \os che è già registrato nel Sistema;}
\addPoC{Il Sistema non crea un nuovo \user dall'username inserito.}
%\addTrigger{L'\admin sceglie uno username già in uso nel Sistema.}
\addScenarioPrincipale{Il Sistema segnala l'errore di username già in uso.}
\makeUC{\Eruu}{Errore username già in uso}

% --- Creazione reparto ---

\addAttore{\admin}
\addPrC{L'\admin ha inserito un nome reparto che è già registrato nel Sistema;}
\addPoC{Il Sistema non crea un nuovo reparto dal nome inserito.}
%\addTrigger{L'\admin sceglie uno username già in uso nel Sistema.}
\addScenarioPrincipale{Il Sistema segnala l'errore di nome reparto già in uso.}
\makeUC{\Ernru}{Errore nome reparto già in uso}

% --- Aggiunta allarme ---

\addAttore{\admin}
\addPrC{L'\admin non ha selezionato alcun sensore a cui affidare l'allarme;}
\addPoC{Il Sistema non aggiunge l'allarme}
%\addTrigger{Nessun sensore è stato assegnato all'allarme.}
\addScenarioPrincipale{Il Sistema segnala l'errore nessun sensore selezionato.}
\makeUC{\Erns}{Errore nessun sensore selezionato}

\addAttore{\admin}
\addPrC{L'\admin non ha selezionato alcun livello di priorità;}
\addPoC{Il Sistema non aggiunge l'allarme}
%\addTrigger{Nessun livello di priorità è stato assegnato all'allarme.}
\addScenarioPrincipale{Il Sistema segnala l'errore nessun livello di priorità.}
\makeUC{\Ernlp}{Errore nessun livello di priorità selezionato}

\addAttore{\admin}
\addPrC{L'\admin non ha selezionato alcuna soglia di intervento;}
\addPoC{Il Sistema non aggiunge l'allarme}
%\addTrigger{Nessun livello di priorità è stato assegnato all'allarme.}
\addScenarioPrincipale{Il Sistema segnala l'errore nessuna soglia di intervento selezionata.}
\makeUC{\Ernss}{Errore nessuna soglia di intervento selezionata}

\newpage

\section{Requisiti}{
    In questa sezione viene presentata la specifica dei requisiti. In particolare i requisiti di dividono in:
    \begin{itemize}
        \item Funzionali;
        \item Di qualità;
        \item Di vincolo.
    \end{itemize}

    % --- Autenticazione ---
    \addReqF{2}{L'\user deve potersi autenticare presso il Sistema}{\rif{\Au}}
    \addReqF{2}{L'\user deve inserire il proprio username per autenticarsi presso il Sistema}{\rif{\Au}\\\rif{\Au.1}}
    \addReqF{2}{L'\user deve inserire la propria password per autenticarsi presso il Sistema}{\rif{\Au}\\\rif{\Au.2}}
    \addReqF{2}{L'\user deve ricevere un errore se l'username non è registrato nel Sistema o 
    la password non è corretta per l'\user inserito}{\rif{\Erunr}}

    % --- Autenticazione con cambio psw ---
    \addReqF{2}{L'\user deve potersi autenticare presso il Sistema e impostare una nuova password}{\rif{\Aucp}}
    % \addReqF{2}{L'\user deve inserire il proprio \user per registrarsi presso il Sistema}{}
    \addReqF{2}{L'\user deve inserire la password temporanea che gli è stata fornita dall'amministratore per
    autenticarsi presso il Sistema e impostare una nuova password}{\rif{\Aucp}\\\rif{\Aucp.1}}
    \addReqF{2}{L'\user deve inserire la nuova password per registrarsi presso il Sistema}{\rif{\Aucp}\\\rif{\Aucp.2}}
    \addReqF{2}{L'\user deve ricevere un errore se l'username non è registrato nel Sistema}{\rif{\Erunr}}
    \addReqF{2}{L'\user deve ricevere un errore se l'username non è registrato nel Sistema o 
    la password temporanea non è corretta per l'\user inserito}{\rif{\Erpte}}
    \addReqF{2}{L'\user deve ricevere un errore se la nuova password è uguale alla password temporanea}{\rif{\Ernpupt}}
    \addReqF{2}{L'\user deve ricevere un errore se la nuova password non rispetta i criteri richiesti}{\rif{\Ernpnv}}

    % --- MYVimar e ab. dis. appartamenti ---
    \addReqF{2}{L'\admin deve poter visualizzare se un account MyVimar è configurato nel Sistema}{\rif{\VaMV}}
    \addReqF{2}{L'\admin deve poter collegare un account MyVimar nel Sistema}{\rif{\CaMV}}
    \addReqF{2}{L'\admin deve poter rimuovere un account MyVimar dal Sistema}{\rif{\RaMV}}

    % --- CRUD utente ---
    \addReqF{1}{L'\admin deve poter visualizzare l'elenco degli utenti del Sistema}{\rif{\Vu}}
    \addReqF{2}{L'\admin visualizzando l'elenco degli utenti del sistema deve visualizzare un utente del Sistema nel dettaglio}{\rif{\Vu}\\\rif{\Vu.1}}
    \addReqF{2}{L'\admin visualizzando un utente del sistema nel dettaglio deve visualizzare il nome}{\rif{\Vu.1}\\\rif{\Vu.1.1}}
    \addReqF{2}{L'\admin visualizzando un utente del sistema nel dettaglio deve visualizzare il cognome}{\rif{\Vu.1}\\\rif{\Vu.1.2}}
    \addReqF{2}{L'\admin visualizzando un utente del sistema nel dettaglio deve visualizzare l'username}{\rif{\Vu.1}\\\rif{\Vu.1.3}}
    \addReqF{1}{L'\admin deve poter creare un utente \os}{\rif{\Cu}}
    \addReqF{1}{L'\admin deve inserire il nome dell'\os per creare un utente \os}{\rif{\Cu}\\\rif{\Cu.1}}
    \addReqF{1}{L'\admin deve inserire il cognome dell'\os per creare un utente \os}{\rif{\Cu}\\\rif{\Cu.2}}
    \addReqF{1}{L'\admin deve inserire un username per creare un utente \os}{\rif{\Cu}\\\rif{\Cu.3}}
    \addReqF{1}{L'\admin deve generare la password temporanea per creare un utente \os}{\rif{\Cu}\\\rif{\Cu.4}}
    \addReqF{1}{L'\admin deve ricevere un errore se l'username è già in uso}{\rif{\Eruu}}
    \addReqF{1}{L'\admin deve poter eliminare un utente \os}{\rif{4}}

    % --- CRUD reparti ---
    \addReqF{2}{L'\admin deve poter visualizzare tutti i reparti del Sistema}{\rif{\Vr}}
    \addReqF{2}{L'\admin deve poter visualizzare un reparto del Sistema nel dettaglio}{\rif{\Vr}\\\rif{\Vr.1}}
    \addReqF{2}{L'\admin visualizzando un reparto del Sistema deve poter visualizzare il nome del reparto}{\rif{\Vr.1}\\\rif{\Vr.1.1}}
    \addReqF{2}{L'\admin deve poter creare un nuovo reparto nel Sistema}{\rif{\Cr}}
    \addReqF{2}{L'\admin deve inserire il nome del nuovo reparto per crearlo nel Sistema}{\rif{\Cr}\\\rif{\Cr.1}}
    \addReqF{2}{L'\admin deve ricevere un errore se il nome del nuovo reparto è già in uso}{\rif{\Ernru}}
    \addReqF{2}{L'\admin deve poter modificare il nome di un reparto del Sistema}{\rif{\Mr}}
    \addReqF{2}{L'\admin deve poter eliminare un reparto del Sistema}{\rif{\Er}}

    % --- Assegnazioni ---
    \addReqF{2}{L'\admin deve poter assegnare un reparto a un utente \os}{\rif{\Aaosr}}
    \addReqF{2}{L'\admin deve selezionare un utente \os per assegnare a un utente \os un reparto}{\rif{\Aaosr}\\\rif{\Aaosr.1}}
    \addReqF{2}{L'\admin deve selezionare un reparto per assegnare un reparto a un utente \os}{\rif{\Aaosr}\\\rif{\Aaosr.2}}
    \addReqF{2}{L'\admin deve poter eliminare l'assegnazione utente \os - reparto}{\rif{\Raosr}}

    \addReqF{2}{L'\admin deve poter assegnare un reparto a un utente \os}{\rif{\Aaapr}}
    \addReqF{2}{L'\admin deve selezionare un utente \os per assegnare a un appartamento un reparto}{\rif{\Aaapr}\\\rif{\Aaapr.1}}
    \addReqF{2}{L'\admin deve selezionare un reparto per assegnare un reparto a un appartamento}{\rif{\Aaapr}\\\rif{\Aaapr.2}}
    \addReqF{2}{L'\admin deve poter eliminare l'assegnazione appartamento - reparto}{\rif{\Raapr}}

    % --- Dashboard UC10 ---
    \addReqF{2}{L'\user deve poter visualizzare una dashboard che rappresenta lo stato del Sistema}{\rif{\Vd}}
    \addReqF{2}{L'\user visualizzando la dashboard deve poter visualizzare un modulo gestione allarmi}{\rif{\Vd}\\\rif{\Vd.1}}
    \addReqF{2}{L'\user visualizzando il modulo gestione allarmi deve poter visualizzare 
    ogni singola allarme}{\rif{\Vd.1}\\\rif{\Vd.1.1}}
    \addReqF{2}{L'\user visualizzando un sigolo allarme deve poter visualizzare il segnale di pericolo}{\rif{\Vd.1.1}\\\rif{\Vd.1.1.1}}
    \addReqF{2}{L'\user visualizzando un sigolo allarme deve poter visualizzare il nome}{\rif{\Vd.1.1}\\\rif{\Vd.1.1.2}}
    \addReqF{2}{L'\user visualizzando un sigolo allarme deve poter visualizzare il tempo trascorso dallo scatto
    dell'allarme}{\rif{\Vd.1.1}\\\rif{\Vd.1.1.3}}
    \addReqF{2}{L'\user visualizzando la dashboard deve poter visualizzare un modulo statistiche allarmi}{\rif{\Vd}\\\rif{\Vd.2}}
    \addReqF{2}{L'\user visualizzando il modulo statistiche allarmi deve poter visualizzare il numero
    di allarmi risolte}{\rif{\Vd.2}\\\rif{\Vd.2.1}}
    \addReqF{2}{L'\user visualizzando il modulo statistiche allarmi deve poter visualizzare il numero
    di allarmi attive}{\rif{\Vd.2}\\\rif{\Vd.2.2}}
    \addReqF{2}{L'\user visualizzando la dashboard deve poter visualizzare un modulo informazioni \user}{\rif{\Vd}\\\rif{\Vd.3}}
    \addReqF{2}{L'\user visualizzando il modulo modulo informazioni \user deve poter visualizzare il 
    nome \user}{\rif{\Vd.3}\\\rif{\Vd.3.1}}
    \addReqF{2}{L'\user visualizzando il modulo modulo informazioni \user deve poter visualizzare il 
    cognome \user}{\rif{\Vd.3}\\\rif{\Vd.3.2}}
    \addReqF{2}{L'\user visualizzando la dashboard deve poter visualizzare un modulo analisi clima}{\rif{\Vd}\\\rif{\Vd.4}}
    \addReqF{2}{L'\user visualizzando la dashboard deve poter visualizzare un modulo analisi consumi}{\rif{\Vd}\\\rif{\Vd.5}}
    \addReqF{2}{L'\user visualizzando la dashboard deve poter visualizzare un modulo analisi presenze}{\rif{\Vd}\\\rif{\Vd.6}}

    % --- Modifiche Dashboard UC11-18 ---
    \addReqF{1}{L'\admin deve poter aggiungere alla visualizzazione della dashboard il modulo statistiche allarmi}{\rif{11}}
    \addReqF{1}{L'\admin deve poter rimuovere dalla visualizzazione della dashboard il modulo statistiche allarmi}{\rif{12}}
    \addReqF{1}{L'\admin deve poter aggiungere alla visualizzazione della dashboard il modulo informazioni \user}{\rif{13}}
    \addReqF{1}{L'\admin deve poter rimuovere dalla visualizzazione della dashboard il modulo informazioni \user}{\rif{14}}
    \addReqF{1}{L'\admin deve poter aggiungere alla visualizzazione della dashboard il modulo analisi clima}{\rif{15}}
    \addReqF{1}{L'\admin deve poter rimuovere dalla visualizzazione della dashboard il modulo analisi clima}{\rif{16}}
    \addReqF{1}{L'\admin deve poter aggiungere alla visualizzazione della dashboard il modulo analisi consumi}{\rif{17}}
    \addReqF{1}{L'\admin deve poter rimuovere dalla visualizzazione della dashboard modulo analisi consumi}{\rif{18}}

    \addReqF{2}{L'\user deve poter risolvere un allarme}{\rif{19}}

    % --- Analytics UC20 --- 
    \addReqF{2}{L'\user deve poter visualizzare le analytics}{\rif{\Va}}
    \addReqF{2}{L'\user visualizzando le analytics deve visualizzare l'elenco dei suggerimenti risparmio energetico}{\rif{\Va.1}}
    \addReqF{2}{L'\user visualizzando l'elenco dei suggerimenti risparmio energetico deve visualizzare un suggerimento
    risparmio energetico}{\rif{\Va.1.1}}
    \addReqF{2}{L'\user visualizzando le analytics deve visualizzare il grafico dedicato al consumo energetico}{\rif{\Va.2}}
    \addReqF{2}{L'\user visualizzando le analytics deve visualizzare il grafico dedicato alle anomalie dell'impianto}{\rif{\Va.3}}
    \addReqF{2}{L'\user visualizzando le analytics deve visualizzare il grafico relativo al rilevamento di presenza}{\rif{\Va.4}}
    \addReqF{2}{L'\user visualizzando le analytics deve visualizzare il grafico relativo alla presenza prolungata nello stesso 
    ambiente}{\rif{\Va.5}}
    \addReqF{2}{L'\user visualizzando le analytics deve visualizzare il grafico relativo alle variazioni di temperatura}{\rif{\Va.6}}
    \addReqF{2}{L'\user visualizzando le analytics deve visualizzare il grafico relativo agli allarmi inviati e risolti}{\rif{\Va.7}}
    \addReqF{0}{L'\user visualizzando le analytics deve visualizzare il grafico relativo alla frequenza degli allarmi}{\rif{\Va.8}}
    \addReqF{0}{L'\user visualizzando le analytics deve visualizzare il grafico relativo alla frequenza delle cadute}{\rif{\Va.9}}

    % --- Appartamento --- 
    \addReqF{2}{L'\user deve poter visualizzare un appartamento del Sistema}{\rif{\Vap}}
    \addReqF{2}{L'\user visualizzando un appartamento del Sistema deve visualizzare il nome dell'appartamento}{\rif{\Vap}\\\rif{\Vap.1}}
    \addReqF{0}{L'\user visualizzando un appartamento del Sistema deve visualizzare la mappa degli allarmi dell'appartamento}{\rif{\Vap}\\\rif{\Vap.2}}
    \addReqF{2}{L'\user visualizzando un appartamento del Sistema deve visualizzare le stanze dell'appartamento}{\rif{\Vap}\\\rif{\Vap.3}}
    \addReqF{2}{L'\user deve visualizzare una stanza nel dettaglio}{\rif{\Vap.3}\\\rif{\Vap.3.1}}
    \addReqF{2}{L'\admin visualizzando una stanza deve visualizzare il nome della stanza}{\rif{\Vap.3.1}\\\rif{\Vap.3.1.1}}
    \addReqF{2}{L'\admin visualizzando una stanza deve visualizzare l'elenco dei dispositivi della stanza}{\rif{\Vap.3.1}\rif{\Vap.3.1.2}}
    \addReqF{2}{L'\admin visualizzando l'elenco dei dispositivi deve visualizzare un singolo dispositivo di tipo:
    \begin{itemize}
        \item termostato;
        \item sensore di caduta;
        \item sensore di presenza;
        \item punto luce;
        \item pulsante di allarme;
        \item porta di ingresso;
        \item tapparella.
    \end{itemize}
    }{\rif{\Vap.3.1.2}\\\rif{\Vap.3.1.2.1}\\\rif{\Vap.3.1.2.2}\\\rif{\Vap.3.1.2.3}\\\rif{\Vap.3.1.2.4}\\\rif{\Vap.3.1.2.6}\\\rif{\Vap.3.1.2.7}\\\rif{\Vap.3.1.2.8}}
    \addReqF{2}{L'\admin visualizzando un dispositivo deve vedere il nome}{\rif{\Vap.3.1.2.1}\\\rif{\Vap.3.1.2.1.1}}
    \addReqF{2}{L'\admin visualizzando un dispositivo deve vedere lo stato}{\rif{\Vap.3.1.2.1}\\\rif{\Vap.3.1.2.1.2}}
    \addReqF{0}{L'\admin visualizzando un dispositivo deve vedere le azioni eseguibili}{\rif{\Vap.3.1.2.1}\\\rif{\Vap.3.1.2.1.3}}
    % \addReqF{2}{L'\admin visualizzando i dispositivi associati a una stanza deve vedere i dispositivi di tipo termostato}{\rif{\Vap.3.1.2}\\\rif{\Vap.3.1.2.1}}
    % \addReqF{2}{L'\admin visualizzando un dispositivo di tipo termostato deve vedere il nome}{\rif{\Vap.3.1.2.1}\\\rif{\Vap.3.1.2.1.1}}
    % \addReqF{2}{L'\admin visualizzando un dispositivo di tipo termostato deve vedere lo stato}{\rif{\Vap.3.1.2.1}\\\rif{\Vap.3.1.2.1.2}}
    % \addReqF{0}{L'\admin visualizzando un dispositivo di tipo termostato deve vedere le azioni eseguibili}{\rif{\Vap.3.1.2.1}\\\rif{\Vap.3.1.2.1.3}}
    % \addReqF{2}{L'\admin visualizzando i dispositivi associati a una stanza deve vedere i dispositivi di tipo sensore di caduta}{\rif{\Vap.3.1.2}\\\rif{\Vap.3.1.2.2}}
    % \addReqF{2}{L'\admin visualizzando un dispositivo di tipo sensore di caduta deve vedere il nome}{\rif{\Vap.3.1.2.2}\\\rif{\Vap.3.1.2.2.1}}
    % \addReqF{2}{L'\admin visualizzando un dispositivo di tipo sensore di caduta deve vedere lo stato}{\rif{\Vap.3.1.2.2}\\\rif{\Vap.3.1.2.2.2}}
    % \addReqF{0}{L'\admin visualizzando un dispositivo di tipo sensore di caduta deve vedere le azioni eseguibili}{\rif{\Vap.3.1.2.2}\\\rif{\Vap.3.1.2.2.3}}
    % \addReqF{2}{L'\admin visualizzando i dispositivi associati a una stanza deve vedere i dispositivi di tipo sensore di presenza}{\rif{\Vap.3.1.2}\\\rif{\Vap.3.1.2.3}}
    % \addReqF{2}{L'\admin visualizzando un dispositivo di tipo sensore di presenza deve vedere il nome}{\rif{\Vap.3.1.2.3}\\\rif{\Vap.3.1.2.3.1}}
    % \addReqF{2}{L'\admin visualizzando un dispositivo di tipo sensore di presenza deve vedere lo stato}{\rif{\Vap.3.1.2.3}\\\rif{\Vap.3.1.2.3.2}}
    % \addReqF{0}{L'\admin visualizzando un dispositivo di tipo sensore di presenza deve vedere le azioni eseguibili}{\rif{\Vap.3.1.2.3}\\\rif{\Vap.3.1.2.3.3}}
    % \addReqF{2}{L'\admin visualizzando i dispositivi associati a una stanza deve vedere i dispositivi di tipo punto luce}{\rif{\Vap.3.1.2}\\\rif{\Vap.3.1.2.4}}
    % \addReqF{2}{L'\admin visualizzando un dispositivo di tipo punto luce deve vedere il nome}{\rif{\Vap.3.1.2.4}\\\rif{\Vap.3.1.2.4.1}}
    % \addReqF{2}{L'\admin visualizzando un dispositivo di tipo punto luce deve vedere lo stato}{\rif{\Vap.3.1.2.4}\\\rif{\Vap.3.1.2.4.2}}
    % \addReqF{0}{L'\admin visualizzando un dispositivo di tipo punto luce deve vedere le azioni eseguibili}{\rif{\Vap.3.1.2.4}\\\rif{\Vap.3.1.2.4.3}}
    % \addReqF{2}{L'\admin visualizzando i dispositivi associati a una stanza deve vedere i dispositivi di tipo pulsante di allarme}{\rif{\Vap.3.1.2}\\\rif{\Vap.3.1.2.5}}
    % \addReqF{2}{L'\admin visualizzando un dispositivo di tipo pulsante di allarme deve vedere il nome}{\rif{\Vap.3.1.2.5}\\\rif{\Vap.3.1.2.5.1}}
    % \addReqF{2}{L'\admin visualizzando un dispositivo di tipo pulsante di allarme deve vedere lo stato}{\rif{\Vap.3.1.2.5}\\\rif{\Vap.3.1.2.5.2}}
    % \addReqF{0}{L'\admin visualizzando un dispositivo di tipo pulsante di allarme deve vedere le azioni eseguibili}{\rif{\Vap.3.1.2.5}\\\rif{\Vap.3.1.2.5.3}}
    % \addReqF{2}{L'\admin visualizzando i dispositivi associati a una stanza deve vedere i dispositivi di tipo porta di ingresso}{\rif{\Vap.3.1.2}\\\rif{\Vap.3.1.2.6}}
    % \addReqF{2}{L'\admin visualizzando un dispositivo di tipo porta di ingresso deve vedere il nome}{\rif{\Vap.3.1.2.6}\\\rif{\Vap.3.1.2.6.1}}
    % \addReqF{2}{L'\admin visualizzando un dispositivo di tipo porta di ingresso deve vedere lo stato}{\rif{\Vap.3.1.2.6}\\\rif{\Vap.3.1.2.6.2}}
    % \addReqF{0}{L'\admin visualizzando un dispositivo di tipo porta di ingresso deve vedere le azioni eseguibili}{\rif{\Vap.3.1.2.6}\\\rif{\Vap.3.1.2.6.3}}
    % \addReqF{2}{L'\admin visualizzando i dispositivi associati a una stanza deve vedere i dispositivi di tipo tapparella}{\rif{\Vap.3.1.2}\\\rif{\Vap.3.1.2.7}}
    % \addReqF{2}{L'\admin visualizzando un dispositivo di tipo tapparella deve vedere il nome}{\rif{\Vap.3.1.2.7}\\\rif{\Vap.3.1.2.7.1}}
    % \addReqF{2}{L'\admin visualizzando un dispositivo di tipo tapparella deve vedere lo stato}{\rif{\Vap.3.1.2.7}\\\rif{\Vap.3.1.2.7.2}}
    % \addReqF{0}{L'\admin visualizzando un dispositivo di tipo tapparella deve vedere le azioni eseguibili}{\rif{\Vap.3.1.2.7}\\\rif{\Vap.3.1.2.7.3}}
    \addReqF{2}{L'\admin deve poter abilitare un appartamento nel Sistema}{\rif{\Aapp}}
    \addReqF{2}{L'\admin deve poter disabilitare un appartamento nel Sistema}{\rif{\Dapp}}
    % aggiungere tutti i sotto dispositivi

    % --- Modifiche allarme --- 
    \addReqF{2}{L'\admin deve poter creare un allarme nel Sistema}{\rif{\Aa}}
    \addReqF{2}{L'\admin deve selezionare l'appartamento per creare un allarme nel Sistema}{\rif{\Aa}\\\rif{\Aa.1}}
    \addReqF{2}{L'\admin deve selezionare il sensore per creare un allarme nel Sistema}{\rif{\Aa}\\\rif{\Aa.2}}
    \addReqF{2}{L'\admin deve selezionare il livello di priorità tra:
    \begin{itemize}
        \item priorità 1 (bianco);
        \item priorità 2 (verde);
        \item priorità 3 (arancione);
        \item priorità 4 (rosso).
    \end{itemize}
    per creare un allarme nel Sistema}{\parbox[t]{\linewidth}{\rif{\Aa}\\\rif{\Aa.3}\\\rif{\Aa.4}\\\rif{\Aa.5}\\\rif{\Aa.6}\\\rif{\Aa.7}}}
    \addReqF{2}{L'\admin deve selezionare una soglia di intervetto per creare un allarme nel Sistema}{\rif{\Aa}\\\rif{\Aa.8}}
    \addReqF{2}{L'\admin deve selezionare un orario di attivazione per creare un allarme nel Sistema}{\rif{\Aa}\\\rif{\Aa.9}}
    \addReqF{2}{L'\admin deve selezionare un orario di disattivazione per creare un allarme nel Sistema}{\rif{\Aa}\\\rif{\Aa.10}}
    \addReqF{2}{L'\admin deve ricevere un errore se non ha selezionato alcun sensore}{\rif{\Erns}}
    \addReqF{2}{L'\admin deve ricevere un errore se non ha selezionato alcun livello di priorità}{\rif{\Ernlp}}
    \addReqF{2}{L'\admin deve ricevere un errore se non ha selezionato alcuna soglia di intervento}{\rif{\Ernss}}
    \addReqF{2}{L'\admin deve poter modificare la priorità di un allarme del Sistema}{\rif{\Mpa}}
    \addReqF{2}{L'\admin deve poter modificare la soglia d'intervento di un allarme del Sistema}{\rif{\Msia}}
    \addReqF{2}{L'\admin deve poter modificare l'orario di attivazione di un allarme del Sistema}{\rif{\Moaa}}
    \addReqF{2}{L'\admin deve poter modificare l'orario di disattivazione di un allarme del Sistema}{\rif{\Moda}}
    \addReqF{2}{L'\admin deve poter abilitare un allarme del Sistema}{\rif{\Aal}}
    \addReqF{2}{L'\admin deve poter disabilitare un allarme del Sistema}{\rif{\Dal}}
    \addReqF{2}{L'\admin deve poter eliminare un allarme nel Sistema}{\rif{\Ea}}

    % --- Notifiche --- 
    \addReqF{0}{L'\user deve poter visualizzare le notifiche inviate dal Sistema}{\rif{\Vn}}
    \addReqF{0}{L'\user visualizzando le notifiche deve poter visualizzare una notifica nel dettaglio}{\rif{\Vn}\\\rif{\Vn.1}}
    \addReqF{0}{L'\user visualizzando una notifica deve poter visualizzare il titolo della notifica}{\rif{\Vn.1}\\\rif{\Vn.1.1}}
    \addReqF{0}{L'\user visualizzando una notifica deve poter visualizzare il tempo trascorso dall'invio della 
    notifica}{\rif{\Vn.1}\\\rif{\Vn.1.2}}
    \makeReqFtable

    \addReqQ{2}{Il Sistema deve essere corredato da test di unità con una copertura di almeno 75\%}{\capitolato\S5.3\\\capitolato\S6.2}
    \addReqQ{0}{Il Sistema deve essere corredato da test di unità con una copertura di almeno 90\%}{\capitolato\S5.3\\\capitolato\S6.2}
    \addReqQ{2}{Il Sistema deve essere corredato da test di integrazione con una copertura di almeno 75\%}{\capitolato\S5.3\\\capitolato\S6.2}
    \addReqQ{0}{Il Sistema deve essere corredato da test di integrazione con una copertura di almeno 90\%}{\capitolato\S5.3\\\capitolato\S6.2}
    \addReqQ{2}{Il Sistema deve essere corredato da test di Sistema con una copertura di almeno 75\%}{\capitolato\S5.3\\\capitolato\S6.2}
    \addReqQ{2}{Il Sistema deve essere corredato da test end-to-end con una copertura di almeno 80\%}{\capitolato\S5.3\\\capitolato\S6.2}
    \addReqQ{2}{Consegnare una documentazione sull’architettura realizzata tramite il modello C4 (diagramma di contesto e di
    container) e/o altri tipi di schemi}{\capitolato\S6.1}
    \addReqQ{2}{Consegnare bozzetti grafici relativi alla progettazione di design dell’applicativo web responsive}{\capitolato\S6.1}
    \addReqQ{2}{Consegnare documentazione sintetica della libreria backend (se implementata)}{\capitolato\S6.1}
    \addReqQ{2}{Consegnare spese stimate della propria soluzione in Cloud AWS in base ai servizi previsti}{\capitolato\S6.1}
    \addReqQ{2}{Consegnare la documentazione dei casi d'uso per l’utilizzo dell’applicativo web.}{\capitolato\S6.1}
    \addReqQ{0}{Consegnare la documentazione delle API backend}{\capitolato\S6.1}
    \addReqQ{0}{Consegnare la documentazione di tutti i test realizzati e report, con relativo coverage e risultato.}{\capitolato\S6.1}
    \addReqQ{0}{Consegnare codice sorgente (accessibile al pubblico dominio) con licenza open source e con un file README
    contenente le istruzioni di installazione e primo utilizzo.}{\capitolato\S6.1}
    \addReqQ{0}{Consegnare screenshot e video dimostrativo della soluzione in funzione.}{\capitolato\S6.1}
    \addReqQ{0}{Consegnare manuale di utilizzo dell’applicativo web per gli utenti.}{\capitolato\S6.1}
    \makeReqQtable

    \addReqV{2}{L’infrastruttura Cloud deve usare Docker con docker-compose al fine di far valere il principio
    di infrastructure as code.}{\capitolato \S 5.1}
    \addReqV{0}{Per l’infrastruttura Cloud è concesso utilizzare Terraform, AWS CDK V2 o Ansible}{\capitolato \S 5.1}
    \addReqV{2}{L’applicativo deve prevedere l’uso della tecnologia KNX IoT 3rd party API.}{\capitolato \S 5.1}
    \addReqV{2}{L’applicativo deve prevedere l’utilizzo del meccanismo di ricezione delle notifiche di
    impianto (push) messo a disposizione dallo standard KNX IoT (i.e. Subscription).}{\capitolato \S 5.1}
    \addReqV{2}{Il repository di lavoro deve essere versionato tramite GIT e deve essere pubblicamente
    accessibile (es. GitHub, BitBucket, GitLab) e per i sorgenti la licenza dovrà essere open
    source (es. MIT, Apache 2)}{\capitolato \S 5.1}
    \addReqV{2}{Le librerie o i framework che verranno utilizzati dovranno avere licenza open source}{\capitolato \S 5.1}
    \addReqV{0}{Distribuire con AWS LightSail oppure AWS EC2 con Docker installato l’infrastruttura Cloud}{\capitolato\S 5.2.1}
    \addReqV{1}{La soluzione viene implementata con un componente libreria backend realizzato ad hoc per l’interfacciamento con le 
    API di KNX IoT, potenzialmente indipendente e riutilizzabile separatamente come “SDK” (Software Development Kit).}{\capitolato\S 6.2}
    \addReqV{0}{Utilizzare servizi serverless come AWS SimpleNotificationService (SNS) e AWS Lambda per gli allarmi tramite 
    Web Push Notification}{\capitolato\S 5.2.1}
    \addReqV{0}{Utilizzare per la raccolta e la rielaborazione dati per le analytics è possibile valutare servizi serverless 
    di AWS Kinesis, AWS Athena oppure AWS CloudWatch, AWS S3}{\capitolato\S 5.2.1}
    \makeReqVtable

    % NICE TO HAVE: La gestione impianto dell’applicativo web permette di eseguire azioni con i dispositivi di impianto (es. accendere una luce).
}

\end{document}
