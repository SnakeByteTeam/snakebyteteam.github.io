\documentclass[10pt, letterpaper]{article}
\usepackage[nomarginpar, margin=2.75cm, tmargin=3cm, bmargin=1.75cm]{geometry}
\usepackage[
    colorlinks=true,      
    linkcolor=black,      
    urlcolor=blue,       
    citecolor=black       
]{hyperref}
\usepackage{template}
\usepackage{float}
\usepackage{graphicx}
\usepackage{caption}
\usepackage[table, x11names]{xcolor}
\usepackage{lastpage} 
\renewcommand{\arraystretch}{1.05} % migliora la leggibilità
\renewcommand{\contentsname}{Indice}
\usepackage{fancyhdr}
%Comandi per livello di sottosezioni = 3
\setcounter{tocdepth}{4}
\setcounter{secnumdepth}{4}
\newcommand{\trisubsection}[1]{\paragraph{#1}\mbox{}\\}

\pagestyle{fancy}
\fancyhf{}
\fancyhead[L]{SnakeByte} 
\fancyhead[R]{Analisi dei Requisiti}
\fancyfoot[C]{Pagina \thepage\ di \pageref{LastPage}}

\begin{document}

\begin{titlepage}
    \begin{center}
        \begin{center}
            \includegraphics[width=0.6\textwidth]{./img/logo.pdf}
        \end{center}
        \vspace{4cm}
        \huge\textbf{Analisi dei requisiti}\par
        \vspace{2cm}
        \large \textbf{SnakeByte} (Gruppo 1):\\
        \large Valeria Baleanu, Leonardo Pellizzon, Filippo Venzo, Giuseppe De Fina, \\
         Francesco Pasqual, Christian Libralato, Luca Granziero \\
        (2109911, 2111006, 2113705, 2113187, 2103119, 2101047, 2075512)
        \vfill
        \small
        \begin{center}
            \begin{tabular}{|c|c|c|c|}
                \hline
                \multicolumn{4}{|c|}{\textbf{Informazioni documento}} \\
                \hline
                \rowcolor{lightgray} \textbf{Versione} & \textbf{Data} & \textbf{Stato} & \textbf{Destinatari} \\
                \hline
                1.0.1 & 8/12/2025 & Da verificare &       
                \begin{tabular}[c]{@{}l@{}}
                    \textbf{Interni:} SnakeByte \\
                    \textbf{Esterni:} prof. Vardanega Tullio, prof. Cardin Riccardo, Vimar
                \end{tabular} \\
                \hline
            \end{tabular}
        \end{center}
        \vfill
        \large Contatti: snakebyteteam@gmail.com
    \end{center}
\end{titlepage}

\begin{center}
    \begin{tabularx}{\textwidth}{|c|c|c|c|c|X|}
        \hline
        \multicolumn{6}{|c|}{\textbf{Registro delle modifiche}} \\
        \hline
        \rowcolor{lightgray} \textbf{Versione} & \textbf{Data} & \textbf{Autore} & \textbf{Verificatore} & \textbf{Approvatore} & \textbf{Descrizione} \\
        \hline
        1.0.1 & 8/12/2025 & V. Baleanu & L. Pellizzon & - & Sistemazione UC3 e UC4. Aggiunta Use Cases UC7 e UC8. Aggiunta diagramma Attori. \\
        \hline
        1.0.0 & 5/12/2025 & F. Venzo & L. Pellizzon & F. Pasqual & Approvazione \\
        \hline
        0.5.0 & 5/12/2025 & F. Venzo & L. Pellizzon & - & Aggiunta Use Cases UC7 e correzioni varie  \\
        \hline
        0.4.0 & 3/12/2025 & F. Venzo & L. Pellizzon & - & Aggiunta Use Cases UC6, UC6.1, UC6.2  \\
        \hline
        0.3.0 & 1/12/2025 & G. De Fina & L. Pellizzon & - & Aggiunta Use Cases UC5, UC5.1, UC5.2, UC5.3  \\
        \hline
        0.2.0 & 1/12/2025 & V. Baleanu & L. Pellizzon & - & Aggiunta Use Cases UC4, UC4.1, UC4.2, UC4.3 UC4.4, UC4.5, UC4.6, UC4.7, UC4.8, UC4.9, UC4.10  \\
        \hline
        0.1.2 & 24/11/2025 & F. Venzo & V. Baleanu & - & Aggiunta Use Cases UC3, UC3.1, UC3.2, UC3.3, UC3.4, UC3.4, UC3.5, UC3.6, UC3.7 \\
        \hline
        0.1.1 & 20/11/2025 & F. Venzo & V. Baleanu & - & Aggiunta Use Cases UC1, UC1.1, UC1.2, UC1.3, UC1.4, UC1.5,
         UC2, UC2.1, UC2.2, UC2.3, UC2.4 \\
        \hline
        0.1.0 & 05/11/2025 & L. Granziero & L. Pellizon & - & Prima Stesura \\
        \hline
    \end{tabularx}
\end{center}

\newpage

\tableofcontents

\newpage

\listoffigures

\newpage

\section{Introduzione}
\subsection{Finalità del documento}{
Il documento di analisi dei requisiti, in un contesto di ingegneria del software, ha lo scopo fondamentale di tradurre l'esigenza dell'utenza e degli stakeholder, in questo caso la proponente Vimar S.p.A., in una specifica completa, coerente e verificabile di requisiti, destinata a guidare le fasi di progettazione e sviluppo, verifica e infine validazione del sistema.

\noindent
\\
\textbf{Definire chiaramente “cosa” il sistema deve fare e “in quali condizioni”:}
\begin{itemize}
    \item L’analisi dei requisiti serve a identificare le funzioni (requisiti funzionali) e le qualità (requisiti non funzionali: prestazioni, usabilità, affidabilità, portabilità...) del software.
    \item Definisce i limiti del sistema e i vincoli (tecnici, di interfaccia...)
    \item Permette di evitare ambiguità o fraintendimenti sulle funzionalità richieste. 
\end{itemize}

\noindent
\textbf{Allineare tutti gli stakeholder su un linguaggio comune e condividere le aspettative:}
\begin{itemize}
    \item Questo documento funge da contratto tra cliente/committente e il team di sviluppo: specifica ciò che sarà consegnato.
    \item Aiuta a garantire che utenti, committenti, analisti, progettisti e tester abbiano la stessa comprensione del sistema.
\end{itemize}

\noindent
\textbf{Fornire una base stabile per le fasi successive del ciclo di vita del software:}
\begin{itemize}
    \item Il documento di analisi dei requisiti serve come input per la progettazione del sistema, per la pianificazione dello sviluppo e per la pianificazione dei test.
    \item Serve anche come base per la verifica e la convalida: si può usare come riferimento per capire se il prodotto finale soddisfa i requisiti richiesti. 
\end{itemize}
\noindent
\textbf{Gestire i rischi e controllare le modifiche:}
\begin{itemize}
    \item Durante l’analisi dei requisiti si identificano requisiti non realizzabili, conflitti tra requisiti, omissioni e incoerenze. Ciò consente di ridurre i rischi fin dalle prime fasi.
    \item Aiuta a limitare il fenomeno dello “scope creep” (ovvero l’aggiunta non controllata di funzionalità) e a mantenere il controllo sul cambiamento dei requisiti. 
\end{itemize}
}

\subsection{Sviluppo del documento}{
Il presente documento è stato sviluppato in modo graduale e incrementale, con lo scopo di facilitare modifiche future in base alle esigenze che verranno concordate tra il gruppo e l'azienda committente. Il documento è quindi soggetto a un processo di miglioramento continuo nel tempo.
}

\subsection{Riferimenti}

\subsubsection{Riferimenti Normativi}
\begin{itemize}
    \item \textbf{Norme di Progetto}: 
    
    \url{link-alle-norme-di-progetto}  (non ci sono sul sito)
    
    (consultato il 30/10/2025);
    \item \textbf{Vimar View4Life Capitolato di Ingegneria del Software Università di Padova 2025 - 2026:}
    
    \url{https://www.math.unipd.it/~tullio/IS-1/2025/Progetto/C9.pdf} 
    
    (consultato il 20/10/2025).
\end{itemize}

\subsubsection{Riferimenti Informativi}
\begin{itemize}
    \item \textbf{830-1998 - IEEE Recommended Practice for Software Requirements Specifications}
    
    \url{https://ieeexplore.ieee.org/document/720574} 
    
    (consultato il 20/10/2025).

    \item \textbf{Diagrammi Use Case - Riccardo Cardin}
    
    \url{https://www.math.unipd.it/~rcardin/swea/2022/Diagrammi%20Use%20Case.pdf} 
    
    (consultato il 18/10/2025).
\end{itemize}

\section{Descrizione del prodotto}
\subsection{Prospettiva del prodotto}
La prospettiva del prodotto è un sistema domotico integrato per anziani autosufficienti che si basa su dispositivi Vimar con connessione mesh Bluetooth.
Gli obiettivi principali sono la sicurezza e il comfort delle persone occupanti, l'aumento dell'efficienza energetica della struttura e la semplificazione della gestione operativa dell'impianto elettrico.
Tali risultati vengono raggiunti attraverso la centralizzazione del controllo di illuminazione, temperatura, televisione e dispositivi di sicurezza mediante l'app View e i relativi servizi cloud, con la possibilità di controllo da remoto da parte del personale medico.

\subsection{Obiettivi del prodotto}{
Il progetto consiste nella realizzazione di una piattaforma unica \textit{View4Life} per la gestione intelligente degli impianti \textit{Smart} nelle residenze protette per anziani, sfruttando i dispositivi domotici Vimar connessi in rete \textit{mesh Bluetooth} tramite l’\textit{API KNX IoT 3rd-party$_{G}$}. Questa soluzione mira a supportare il lavoro del \os fornendo uno strumento che integri un sistema di gestione degli allarmi (come il rilevamento di cadute o presenze prolungate in determinate stanze) per garantire un intervento rapido e tempestivo. Inoltre, la piattaforma è progettata per permettere il monitoraggio del consumo energetico e la rilevazione di anomalie nell’impianto.
}

\subsection{Funzionalità del prodotto}
Dal punto di vista degli utenti del \os l'applicativo svolge le seguenti funzioni:
\begin{itemize}
    \item Visualizzazione delle informazioni generali di allarmi, statistiche ed analitiche tramite cruscotto riassuntivo (Dashboard);
    \item Possibilità di essere notificati, visualizzare e gestire gli allarmi attivi;
    \item Possibilità di visualizzare e gestire i dispositivi dei vari impianti collocati in diverse residenze;
    \item Possibilità di visualizzare statistiche, tramite grafici, sui consumi dell'impianto, sulla variazione di temperatura e sugli allarmi passati;
    \item Possibilità di ricevere consigli, basati sulle statistiche, per ridurre i consumi energetici.
\end{itemize}

\subsection{Utenza di riferimento}{
Il prodotto si rivolge principalmente a quattro categorie principali di utenti, descritte di seguito:

\begin{itemize}
    \item \textbf{Personale medico e operatori sanitari} che utilizzano l'applicazione per monitorare lo stato di ambienti e dispositivi, oltre a ricevere notifiche di allarme e gestire da remoto funzioni come temperatura, illuminazione e sicurezza delle stanze degli ospiti presenti. 
    \item \textbf{Personale amministrativo} che si fa carico della configurazione e manutenzione del sistema e del monitoraggio dei consumi elettrici, e quindi a una conseguente ottimizzazione dell'efficienza dell'impianto elettrico.
\end{itemize}
Questa sezione evidenzia come il sistema domotico proposto miri a centralizzare il controllo e la gestione degli impianti all’interno della residenza, semplificando le operazioni quotidiane del personale e migliorando la qualità della vita degli ospiti.
Attraverso l’app View e i servizi cloud Vimar, l’applicazione permette di gestire in modo integrato illuminazione, temperatura e sicurezza, contribuendo a maggior efficienza energetica, sicurezza e comfort abitativo per tutti gli utenti coinvolti.
}





\section{Casi d'uso}
Un \textit{caso d'uso$_{G}$} è la descrizione dettagliata, tramite \textit{diagramma UML$_G$} e descrizione testuale, di un insieme di scenari che hanno uno scopo comune, all'interno del Sistema, per un attore.
Permettono di comprendere al meglio le funzionalità che devono essere rese disponibili dal Sistema \textit{software}.

In particolare, le descrizioni dei casi d'uso contenute in questo documento conterranno le informazioni riportate nella seguente tabella:

\begin{center}
    \begin{tabularx}{\textwidth}{|c| >{\centering\arraybackslash}X|}
        \hline
        \rowcolor{lightgray} \textbf{Campo} & \textbf{Descrizione} \\
        \hline
        Attori & Coloro che partecipano attivamente al caso d'uso per raggiungere un preciso obiettivo.  \\
        \hline
        Pre-condizioni & Condizioni che devono essere soddisfatte prima dello scenario descritto dal caso d'uso.\\
        \hline
        Post-condizioni & Condizioni che risultano soddisfatte dopo il completamento dello scenario principale del caso d'uso. Se viene completato uno scenario alternativo, saranno soddisfatte le Post-condizioni di quest'ultimo. \\
        \hline
        Trigger & La motivazione che porta l'utente a svolgere i passi del caso d'uso. \\
        \hline
        Scenario principale & Sequenza di passi che l'utente deve seguire per completare il caso d'uso. \\
        \hline
        Scenari alternativi & Scenario divergente dal principale per il verificarsi di una particolare condizione. \\
        \hline
        Estensioni &  Casi d'uso ulteriori eseguiti al verificarsi di una particolare condizione nel caso d'uso primario. Modificano Scenario e Post-condizioni. \\
        \hline
        Inclusioni & Casi d'uso ulteriori eseguiti al fine di completare il caso d'uso principale. Vengono eseguiti tutti incondizionatamente. \\
        \hline
        
    \end{tabularx}
\end{center}

Non tutti gli attributi sono necessari per ogni caso d'uso. Nel caso in cui un campo sia assente in un caso d'uso, allora tale sarà assente anche nella sua descrizione e nel suo diagramma UML.

\subsection{Attori}
Di seguito vengono riportati gli attori individuati:
\begin{figure}[H]
    \centering
    \includegraphics[width=0.6\textwidth]{./img/Attori.pdf}
    \caption{Diagramma attori principali}
    \label{fig:attori}
\end{figure}

\begin{center}
    \begin{tabularx}{\textwidth}{|c| >{\centering\arraybackslash}X|}
        \hline
        \rowcolor{lightgray} \textbf{Attore} & \textbf{Descrizione} \\
        \hline
        \userna & Rappresenta l'utente che vuole accedere nel Sistema. \\
        \hline
        Utente & Rappresenta l'utente generico (sia \admin che \os) autenticato nel Sistema. \\
        \hline
        \os & Rappresenta l'utente \os. \\
        \hline 
        \admin & Rappresenta l'amministratore della struttura il quale gestice la struttura e l'\os. \\
        \hline          
    \end{tabularx}
\end{center}



\subsection{Lista dei casi d'uso}

\addAttore{\userna.}
\addPreCondizione{Il Sistema è attivo;}
\addPreCondizione{L'utente non è autenticato nel Sistema.}
\addPostCondizione{L'utente è autenticato nel Sistema}
\addTrigger{L'utente vuole autenticarsi nel Sistema}
\addScenarioPrincipale{L'utente inserisce il proprio username;}
\addScenarioPrincipale{L'utente inserisce la propria Password.}
\addInclusione{Inserimento username \rif{1.1}}
\addInclusione{Inserimento Password \rif{1.2}}
\addScenarioAlternativo{L'utente inserisce username o Password errate.}
\addEstensione{Autenticazione fallita \rif{1.3}}
\makeUC{1}{Autenticazione}

\addAttore{\userna.}
\addPreCondizione{Il Sistema è attivo;}
\addPreCondizione{L'utente non è autenticato nel Sistema;}
\addPreCondizione{Il Sistema non conosce l'username dell'utente.}
\addPostCondizione{Il Sistema conosce l'username dell'utente.}
\addScenarioPrincipale{L'utente inserisce il proprio username.}
\addTrigger{L'utente ha selezionato l'opzione di inserimento dell'username.}
\makeUC{1.1}{Inserimento username}

\addAttore{\userna.}
\addPreCondizione{Il Sistema è attivo;}
\addPreCondizione{L'utente non è autenticato nel Sistema;}
\addPreCondizione{Il Sistema non conosce la Password dell'utente.}
\addPostCondizione{Il Sistema conosce la Password dell'utente.}
\addScenarioPrincipale{L'utente inserisce la propria Password.}
\addTrigger{L'utente ha selezionato l'opzione di inserimento della Password;}
\makeUC{1.2}{Inserimento Password}

\addAttore{\userna.}
\addPreCondizione{Il Sistema è attivo;}
\addPreCondizione{L'utente non è autenticato nel Sistema;}
\addPostCondizione{L'utente non è autenticato.}
\addScenarioPrincipale{Il Sistema segnala l'errore di autenticazione.}
\addTrigger{L'utente ha immesso username o Password errati.}
\makeUC{1.3}{Autenticazione fallita}

\addAttore{\admin.}
\addPreCondizione{Il Sistema è attivo;}
\addPreCondizione{L'\admin è autenticato nel Sistema.}
\addPostCondizione{Un nuovo utente del \os è registrato presso il Sistema}
\addScenarioPrincipale{L'\admin crea uno username per l'utente del \os;}
\addScenarioPrincipale{L'\admin crea una password temporanea per l'utente del \os;}
\addInclusione{Creazione username utente \os \rif{1.4.1}}
\addInclusione{Generazione password temporanea \rif{1.4.2}}
\addTrigger{L'\admin vuole registrare un nuovo utente del \os.}
\makeUC{1.4}{Creazione nuovo utente \os}

\addAttore{\admin.}
\addPreCondizione{Il Sistema è attivo;}
\addPreCondizione{L'\admin è autenticato nel Sistema.}
\addPostCondizione{Il Sistema conosce lo username dell'utente del \os;}
\addScenarioPrincipale{L'\admin sceglie uno username per l'utente del \os.}
\addScenarioAlternativo{L'username scelto è già in uso nel Sistema;}
\addEstensione{Errore username già in uso \rif{1.4.3}}
\makeUC{1.4.1}{Creazione username utente \os}

\addAttore{\admin.}
\addPreCondizione{Il Sistema è attivo;}
\addPreCondizione{L'\admin è autenticato nel Sistema.}
\addPostCondizione{Il Sistema conosce la password temporanea per l'utente del \os;}
\addScenarioPrincipale{L'\admin genera una Password temporanea per l'utente del \os.}
\makeUC{1.4.2}{Generazione password temporanea}

\addAttore{\admin.}
\addPreCondizione{Il Sistema è attivo;}
\addPreCondizione{L'\admin è autenticato nel Sistema.}
\addPostCondizione{Il Sistema non registra il nuovo utente del \os.}
\addScenarioPrincipale{Il Sistema segnala l'errore di username già in uso.}
\addScenarioPrincipale{Il Sistema ripropone all'\admin di scegliere uno username.}
\addTrigger{L'\admin sceglie uno username già in uso nel Sistema.}
\addInclusione{Creazione username utente \os \rif{1.4.1}}
\makeUC{1.4.3}{Errore username già in uso}

\addAttore{\os.}
\addPreCondizione{Il Sistema è attivo;}
\addPreCondizione{L'utente del \os non è autenticato nel Sistema.}
\addPostCondizione{L'utente del \os è autenticato nel Sistema}
\addScenarioPrincipale{L'utente del personale si autentica con le credenziali fornite dall'\admin;}
\addScenarioPrincipale{Il Sistema obbliga l'utente del \os ad aggiornare la Password temporanea;}
\addInclusione{Autenticazione \rif{UC1}}
\addInclusione{Aggiornamento Password \rif{1.5.1}}
\addTrigger{L'utente del \os si è autenticato per la prima volta.}
\makeUC{1.5}{Registrazione utente \os}

\addAttore{\os.}
\addPreCondizione{Il Sistema è attivo;}
\addPreCondizione{L'utente del \os è autenticato nel Sistema con una Password temporanea.}
\addPostCondizione{L'utente del \os ha aggiornato la propria Password.}
\addScenarioPrincipale{L'utente del \os inserisce una nuova Password.}
\makeUC{1.5.1}{Aggiornamento Password}

\addAttore{\user.}
\addPreCondizione{Il Sistema è attivo;}
\addPreCondizione{L'utente è autenticato nel Sistema.}
\addPostCondizione{L'utente visualizza la sezione Dashboard.}
\addScenarioPrincipale{L'utente seleziona dal menù l'opzione relativa alla sezione Dashboard;}
\addScenarioPrincipale{L'utente visualizza i seguenti Moduli opzionali presenti nella Dashboard:
\begin{itemize}
    \item Informazioni utente;
    \item statistiche allarmi;
    \item analisi clima;
    \item analisi consumi;
    \item dispositivi accesi;
    \item temperatura impostata.
\end{itemize}
L'utente inoltre visualizza il seguente Modulo non modificabile:
\begin{itemize}
    \item Gestione allarmi.
\end{itemize}
}
\addTrigger{L'utente vuole visualizzare la sezione Dashboard.}
\addInclusione{Visualizzazione informazioni utente \rif{2.1}}
\addInclusione{Visualizzazione statistiche allarmi \rif{2.2}}
\addInclusione{Visualizzazione analisi clima \rif{2.3}}
\addInclusione{Visualizzazione analisi consumi \rif{2.4}}
\addInclusione{Visualizzazione dispositivi accesi \rif{2.5}}
\addInclusione{Visualizzazione temperatura impostata \rif{2.6}}
\addInclusione{Visualizzazione gestione allarmi \rif{2.7}}
\makeUC{2}{Visualizzazione Dashboard}

\addAttore{\user.}
\addPreCondizione{L'utente si trova nella sezione Dashboard.}
\addPostCondizione{Il Sistema mostra il Modulo relativo alle informazioni utente.}
\addScenarioPrincipale{L'utente visualizza le informazioni utente, caratterizzate da:
\begin{itemize}
    \item Nome e cognome dell'utente;
    \item ruolo dell'utente.
\end{itemize}
}
\addTrigger{L'utente vuole visualizzare il Modulo opzionale relativo alle informazioni utente.}
\addInclusione{Visualizzazione Nome e cognome utente \rif{2.1.1}}
\addInclusione{Visualizzazione ruolo utente \rif{2.1.2}}
\makeUC{2.1}{Visualizzazione informazioni utente}


\addAttore{\user.}
\addPreCondizione{L'utente si trova nella sezione Dashboard.}
\addPostCondizione{Il Sistema mostra il Modulo relativo alle statistiche allarmi.}
\addScenarioPrincipale{L'utente visualizza le statistiche allarmi, caratterizzate da:
\begin{itemize}
    \item ???
\end{itemize}
}
\addTrigger{L'utente vuole visualizzare il Modulo opzionale relativo alle statistiche allarmi.}
\makeUC{2.2}{Visualizzazione statistiche allarmi}


\addAttore{\user.}
\addPreCondizione{L'utente si trova nella sezione Dashboard.}
\addPostCondizione{Il Sistema mostra il Modulo relativo all'analisi del clima.}
\addScenarioPrincipale{L'utente visualizza le analisi del clima, caratterizzate da:
\begin{itemize}
    \item ???
\end{itemize}
}
\addTrigger{L'utente vuole visualizzare il Modulo opzionale relativo all'analisi clima.}
\makeUC{2.3}{Visualizzazione analisi clima}


\addAttore{\user.}
\addPreCondizione{L'utente si trova nella sezione Dashboard.}
\addPostCondizione{Il Sistema mostra il Modulo relativo all'analisi dei consumi.}
\addScenarioPrincipale{L'utente visualizza le analisi dei consumi, caratterizzate da:
\begin{itemize}
    \item ???
\end{itemize}
}
\addTrigger{L'utente vuole visualizzare il Modulo opzionale relativo all'analisi consumi.}
\makeUC{2.4}{Visualizzazione analisi consumi}


\addAttore{\user.}
\addPreCondizione{L'utente si trova nella sezione Dashboard.}
\addPostCondizione{Il Sistema mostra il Modulo relativo ai dispositivi accesi.}
\addScenarioPrincipale{L'utente visualizza i dispositivi accesi, caratterizzati da:
\begin{itemize}
    \item Nome dispositivo;
    \item stato dispositivo.
\end{itemize}
}
\addTrigger{L'utente vuole visualizzare il Modulo opzionale relativo ai dispositivi accesi.}
\makeUC{2.5}{Visualizzazione dispositivi accesi}


\addAttore{\user.}
\addPreCondizione{L'utente si trova nella sezione Dashboard.}
\addPostCondizione{Il Sistema mostra il Modulo relativo alla temperatura impostata.}
\addScenarioPrincipale{L'utente visualizza la temperatura impostata, caratterizzata da:
\begin{itemize}
    \item Nome impianto;
    \item Temperatura impostata.
\end{itemize}
}
\addTrigger{L'utente vuole visualizzare il Modulo opzionale relativo alla temperatura impostata.}
\makeUC{2.6}{Visualizzazione temperatura impostata}

\addAttore{\user.}
\addPreCondizione{L'utente si trova nella sezione Dashboard.}
\addPostCondizione{Il Sistema mostra il Modulo relativo alla gestione allarmi.}
\addScenarioPrincipale{L'utente visualizza la gestione allarmi, caratterizzata da:
\begin{itemize}
    \item Allarmi in corso;
    \item azioni da poter eseguire sugli allarmi.
\end{itemize}
}
\addTrigger{L'utente vuole visualizzare il Modulo relativo alla gestione allarmi.}
\makeUC{2.7}{Visualizzazione gestione allarmi}

\addAttore{\user.}
\addPreCondizione{L'utente si trova nella sezione Dashboard.}
\addPostCondizione{L'utente visualizza nella sezione Dashboard il Modulo selezionato.}
\addScenarioPrincipale{L'utente seleziona l'opzione di modifica della Dashboard;}
\addScenarioPrincipale{L'utente seleziona un Modulo non presente nella Dashboard.}
\addScenarioAlternativo{Il Modulo selezionato è già presente nella Dashboard;}
\addScenarioAlternativo{È stato raggiunto il massimo numero di Moduli visualizzabili nella Dashboard.}
\addEstensione{Selezione Modulo già presente \rif{2.8.1}}
\addEstensione{Limite Moduli raggiunti \rif{2.8.2}}
\addTrigger{L'utente vuole aggiungere un Modulo nella sezione Dashboard.}
\makeUC{2.8}{Aggiunta Modulo opzionale Dashboard}

\addAttore{Utente.}
\addPreCondizione{L'utente si trova nella sezione Dashboard;}
\addPreCondizione{L'utente sta modificando quali Moduli sono visualizzati nella Dashboard.}
\addPostCondizione{Il Modulo selezionato è rimosso dalla Dashboard.}
\addScenarioPrincipale{Il Sistema rimuove il Modulo selezionato dalla Dashboard.}
\addTrigger{L'utente vuole aggiungere un Modulo già presente nella Dashboard.}
\makeUC{2.8.1}{Selezione Modulo già presente nella Dashboard}

\addAttore{Utente.}
\addPreCondizione{L'utente si trova nella sezione Dashboard;}
\addPreCondizione{L'utente sta modificando quali Moduli sono visualizzati nella Dashboard.}
\addPreCondizione{Il limite massimo di Moduli visualizzabili nella Dashboard è stato raggiunto.}
\addPostCondizione{Il Modulo selezionato non è inserito nella Dashboard.}
\addScenarioPrincipale{Il Sistema non inserisce il Modulo selezionato nella Dashboard;}
\addScenarioPrincipale{Il Sistema segnala la condizione di massima capienza raggiunta.}
\addTrigger{L'utente vuole aggiungere un Modulo nella Dashboard.}
\makeUC{2.8.2}{Errore limite Moduli visualizzabili nella Dashboard}

\addAttore{Utente.}
\addPreCondizione{L'utente si trova nella sezione Dashboard;}
\addPostCondizione{Il Sistema rimuove dalla Dashboard il Modulo selezionato dall'utente.}
\addScenarioPrincipale{L'utente seleziona l'opzione di modifica della Dashboard;}
\addScenarioPrincipale{L'utente seleziona un Modulo opzionale attualmente presente nella Dashboard.}
\addScenarioAlternativo{Il Modulo selezionato non è attualmente presente nella Dashboard.}
\addInclusione{(???) Selezione Modulo già presente \rif{2.8.1}}
\addEstensione{Selezione Modulo non presente \rif{2.9.1}}
\addTrigger{L'utente vuole rimuovere un Modulo opzionale dalla Dashboard.}
\makeUC{2.9}{Rimozione Modulo opzionale Dashboard}

\addAttore{Utente.}
\addPreCondizione{Il Sistema è attivo;}
\addPreCondizione{L'utente è autenticato nel Sistema;}
\addPreCondizione{L'utente sta modificando quali Moduli sono visualizzati nella Dashboard.}
\addPostCondizione{Il Modulo selezionato viene aggiunto alla Dashboard.}
\addScenarioPrincipale{L'utente seleziona un Modulo, non presente nella Dashboard, tra:
                        \begin{itemize}
                            \item informazioni utente;
                            \item statistiche allarmi;
                            \item analisi clima;
                            \item analisi consumi;
                            \item dispositivi accessi;
                            \item temperatura impostata.
                        \end{itemize} }
\addScenarioPrincipale{Il Sistema aggiunge alla Dashboard il Modulo selezionato.}
\addTrigger{L'utente seleziona un Modulo non presente nella Dashboard}
\makeUC{2.9.1}{Selezione Modulo non presente in Dashboard}


% ======= UC Analytics =======
\addAttore{\os.}
\addPreCondizione{Il Sistema è attivo;}
\addPreCondizione{L'\os è autenticato nel Sistema.}
\addPostCondizione{L'\os è all'interno della sezione Analytics.}
\addTrigger{L’\os vuole visualizzare la sezione Analytics.}
\addScenarioPrincipale{L'\os seleziona la sezione Analytics dal menù principale;}
\addScenarioPrincipale{L'\os visualizza la sezione Analytics, che permette di visualizzare i seguenti moduli relativi agli impianti monitorati negli ultimi 30 giorni (o in un periodo superiore):
\begin{itemize}
    \item Suggerimenti per il risparmio energetico (Vedi \rif{3.1});
    \item grafico sulle anomalie di impianto (Vedi \rif{3.2});
    \item grafico sul calcolo dell’energia consumata dall’illuminazione, a partire dal tempo (Vedi \rif{3.3});
    \item grafico sul rilevamento di presenza, assenza e caduta (Vedi \rif{3.4});
    \item grafico sul rilevamento di presenza prolungata nello stesso ambiente (Vedi \rif{3.5});
    \item grafico di variazione e cambio di temperatura (Vedi \rif{3.6});
    \item grafico sugli allarmi inviati e risolti per giorno dagli operatori sanitari (Vedi \rif{3.7});
    \item grafico sulla frequenza degli allarmi rilevati nell’arco di un periodo a scelta, suddivisi per ore (Vedi \rif{3.8});
    \item grafico sulla frequenza delle cadute rilevate nell’arco di un periodo a scelta, suddivise per ore (Vedi \rif{3.9}).
\end{itemize}
Inoltre, è possibile modificare la visualizzazione dei seguenti moduli:
\begin{itemize}
    \item Grafico sul calcolo dell’energia consumata dall’illuminazione (Vedi inserimento \rif{3.3.1} e rimozione \rif{3.3.2});
    \item grafico di variazione e cambio di temperatura (Vedi inserimento \rif{3.6.1} e rimozione \rif{C3.6.2});
    \item grafico sugli allarmi inviati e risolti (Vedi inserimento \rif{3.7.1} e rimozione \rif{3.7.2}).
\end{itemize}}
\addInclusione{\rif{3.1}}
\addInclusione{\rif{3.2}}
\addInclusione{\rif{3.3}}
\addInclusione{\rif{3.4}}
\addInclusione{\rif{3.5}}
\addInclusione{\rif{3.6}}
\addInclusione{\rif{3.7}}
\addInclusione{\rif{3.8}}
\addInclusione{\rif{3.9}}
\addEstensione{\rif{3.3.1}}
\addEstensione{\rif{3.3.2}}
\addEstensione{\rif{3.6.1}}
\addEstensione{\rif{3.6.2}}
\addEstensione{\rif{3.7.1}}
\addEstensione{\rif{3.7.2}}
\makeUC{3}{Visualizzazione Analytics}

% ======= UC Analytics: Suggerimenti risparmio energetico =======
\addAttore{\os.}
\addPreCondizione{L'\os è all'interno della sezione Analytics.}
\addPostCondizione{L'\os visualizza i suggerimenti relativi al risparmio energetico.}
\addTrigger{L'\os vuole visualizzare i suggerimenti per ridurre i consumi energetici.}
\addScenarioPrincipale{L'\os seleziona i suggerimenti per il risparmio energetico;}
\addScenarioPrincipale{L'\os seleziona il periodo temporale da analizzare;}
\addScenarioPrincipale{L'\os visualizza i suggerimenti per il risparmio energetico.}
\addScenarioAlternativo{Il Sistema non ha sufficienti dati per generare suggerimenti (Vedi \rif{7})}
\addEstensione{\rif{7}}
\makeUC{3.1}{Visualizzazione suggerimenti risparmio energetico}

% ======= UC Analytics: Grafico anomalie di impianto =======
\addAttore{\os.}
\addPreCondizione{L'\os è all'interno della sezione Analytics.}
\addPostCondizione{L'\os visualizza grafico dedicato alle anomalie di impianto.}
\addTrigger{L’\os vuole visualizzare grafico dedicato alle anomalie di impianto.}
\addScenarioPrincipale{L'\os seleziona le anomalie di impianto;}
\addScenarioPrincipale{L'\os seleziona il periodo temporale da analizzare.}
\addScenarioAlternativo{Il Sistema non dispone di sufficenti dati per generare il grafico (Vedi \rif{7})}
\addEstensione{\rif{7}}
\makeUC{3.2}{Visualizzazione grafico anomalie di impianto}

% ======= UC Analytics: Grafico energia consumata =======
\addAttore{\os.}
\addPreCondizione{L'\os è all'interno della sezione Analytics.}
\addPostCondizione{L'\os visualizza il grafico dedicato all'energia consumata dall’illuminazione in base al tempo.}
\addTrigger{L’\os vuole visualizzare grafico dedicato all'energia consumata dall’illuminazione.}
\addScenarioPrincipale{L'\os seleziona l'energia consumata;}
\addScenarioPrincipale{L'\os seleziona il periodo temporale da analizzare.}
\addScenarioAlternativo{Il Sistema non dispone di sufficenti dati per generare il grafico (Vedi \rif{7})}
\addEstensione{\rif{7}}
\makeUC{3.3}{Visualizzazione grafico energia consumata}

% ======= UC Analytics: Grafico rilevamento presenza, assenza e caduta =======
\addAttore{\os.}
\addPreCondizione{L'\os si trova nella sezione Analytics.}
\addPostCondizione{L'\os visualizza il grafico relativo al rilevamento di presenza, assenza e caduta.}
\addTrigger{L’\os vuole visualizzare il grafico relativo a presenza, assenza e caduta.}
\addScenarioPrincipale{L'\os seleziona il rilevamento di presenza, assenza e caduta;}
\addScenarioPrincipale{L'\os seleziona il periodo temporale da analizzare.}
\addScenarioAlternativo{Il Sistema non dispone di dati sufficienti per generare il grafico (Vedi \rif{7})}
\addEstensione{\rif{7}}
\makeUC{3.4}{Visualizzazione grafico rilevamento presenza, assenza e caduta}

% ======= UC Analytics: Grafico rilevamento di presenza prolungata nello stesso ambiente =======
\addAttore{\os.}
\addPreCondizione{L'\os si trova nella sezione Analytics.}
\addPostCondizione{L'\os visualizza il grafico relativo alla presenza prolungata nello stesso ambiente.}
\addTrigger{L’\os vuole visualizzare il grafico sulla presenza prolungata nello stesso ambiente.}
\addScenarioPrincipale{L'\os seleziona la presenza prolungata;}
\addScenarioPrincipale{L'\os seleziona il periodo temporale da analizzare.}
\addScenarioAlternativo{Il Sistema non dispone di dati sufficienti per generare il grafico (Vedi \rif{7}).}
\addEstensione{\rif{7}}
\makeUC{3.5}{Visualizzazione grafico presenza prolungata nello stesso ambiente}

% ======= UC Analytics: Grafico variazione e cambio temperatura =======
\addAttore{\os.}
\addPreCondizione{L'\os si trova nella sezione Analytics.}
\addPostCondizione{L'\os visualizza il grafico delle variazioni di temperatura nel tempo.}
\addTrigger{L’\os vuole visualizzare il grafico di variazione e cambio temperatura.}
\addScenarioPrincipale{L'\os seleziona la temperatura;}
\addScenarioPrincipale{L'\os seleziona il periodo temporale da analizzare.}
\addScenarioAlternativo{Il Sistema non dispone di dati sufficienti per generare il grafico (Vedi \rif{7})}
\addEstensione{\rif{7}}
\makeUC{3.6}{Visualizzazione grafico variazione e cambio temperatura}

% ======= UC Analytics: Grafico allarmi inviati e risolti =======
\addAttore{\os.}
\addPreCondizione{L'\os si trova nella sezione Analytics.}
\addPostCondizione{L'\os visualizza il grafico sugli allarmi inviati e risolti giornalmente.}
\addTrigger{L’\os vuole visualizzare il grafico sugli allarmi.}
\addScenarioPrincipale{L'\os seleziona  gli allarmi;}
\addScenarioPrincipale{L'\os seleziona il periodo temporale da analizzare.}
\addScenarioAlternativo{Il Sistema non dispone di dati sufficienti (Vedi \rif{7})}
\addEstensione{\rif{7}}
\makeUC{3.7}{Visualizzazione grafico allarmi inviati e risolti}

% ======= UC Analytics: Grafico frequenza degli allarmi rilevati =======
\addAttore{\os.}
\addPreCondizione{L'\os si trova nella sezione Analytics.}
\addPostCondizione{L'\os visualizza la frequenza oraria degli allarmi in un periodo selezionato.}
\addTrigger{L’\os vuole visualizzare il grafico sulla frequenza degli allarmi.}
\addScenarioPrincipale{L'\os seleziona la frequenza degli allarmi rilevati;}
\addScenarioPrincipale{L'\os seleziona il periodo temporale da analizzare.}
\addScenarioAlternativo{Il Sistema non dispone di dati sufficienti (Vedi \rif{7})}
\addEstensione{\rif{7}}
\makeUC{3.8}{Visualizzazione frequenza degli allarmi rilevati}

% ======= UC Analytics: Grafico frequenza degli allarmi rilevati =======
\addAttore{\os.}
\addPreCondizione{L'\os si trova nella sezione Analytics.}
\addPostCondizione{L'\os visualizza la frequenza oraria delle cadute in un periodo selezionato.}
\addTrigger{L’\os vuole visualizzare il grafico sulla frequenza delle cadute.}
\addScenarioPrincipale{L'\os seleziona il grafico sulla frequenza delle cadute rilevate;}
\addScenarioPrincipale{L'\os seleziona il periodo temporale da analizzare.}
\addScenarioAlternativo{Il Sistema non dispone di dati sufficienti (Vedi \rif{7}).}
\addEstensione{\rif{7}}
\makeUC{3.9}{Visualizzazione frequenza delle cadute rilevate}

% ======= UC Analytics: Inserimento Modulo grafico energia consumata =======
\addAttore{\os.}
\addPreCondizione{L'\os si trova nella sezione Analytics;}
\addPreCondizione{L'\os ha selezionato la funzionalità di modifica della sezione Analytics.}
\addPostCondizione{L'\os aggiunge la visualizzazione del grafico.}
\addTrigger{Il \os vuole aggiungere la visualizzazione del grafico.}
\addScenarioPrincipale{L'\os seleziona il grafico sul calcolo dell'energia consumata dall'illuminazione;}
\addScenarioPrincipale{L'\os seleziona la funzionalità di aggiunta del grafico.}
\makeUC{3.3.1}{Inserimento grafico sul calcolo dell'energia consumata dall'illuminazione}

% ======= UC Analytics: Rimozione Modulo grafico energia consumata =======
\addAttore{\os.}
\addPreCondizione{L'\os si trova nella sezione Analytics;}
\addPreCondizione{L'\os ha selezionato la funzionalità di modifica della sezione Analytics.}
\addPostCondizione{L'\os rimuove la visualizzazione del grafico.}
\addTrigger{Il \os vuole rimuovere la visualizzazione del grafico.}
\addScenarioPrincipale{L'\os seleziona il grafico sul calcolo dell'energia consumata dall'illuminazione;}
\addScenarioPrincipale{L'\os seleziona la funzionalità di rimozione del grafico.}
\makeUC{3.3.2}{Rimozione grafico sul calcolo dell'energia consumata dall'illuminazione}

% ======= UC Analytics: Inserimento Modulo grafico variazione cambio temp =======
\addAttore{\os.}
\addPreCondizione{L'\os si trova nella sezione Analytics;}
\addPreCondizione{L'\os ha selezionato la funzionalità di modifica della sezione Analytics.}
\addPostCondizione{L'\os aggiunge la visualizzazione del grafico.}
\addTrigger{Il \os vuole aggiungere la visualizzazione del grafico.}
\addScenarioPrincipale{L'\os seleziona il grafico sul calcolo di variazione e cambio temperatura;}
\addScenarioPrincipale{L'\os seleziona la funzionalità di aggiunta del grafico.}
\makeUC{3.6.1}{Inserimento grafico di variazione e cambio temperatura}

% ======= UC Analytics: Rimozione Modulo grafico variazione cambio temp. =======
\addAttore{\os.}
\addPreCondizione{L'\os si trova nella sezione Analytics;}
\addPreCondizione{L'\os ha selezionato la funzionalità di modifica della sezione Analytics.}
\addPostCondizione{L'\os rimuove la visualizzazione del grafico.}
\addTrigger{Il \os vuole rimuovere la visualizzazione del grafico.}
\addScenarioPrincipale{L'\os seleziona il grafico di variazione e cambio temperatura;}
\addScenarioPrincipale{L'\os seleziona la funzionalità di rimozione del grafico.}
\makeUC{3.6.2}{Rimozione grafico di variazione e cambio temperatura}

% ======= UC Analytics: Inserimento Modulo grafico sugli allarmi inviati e risolti =======
\addAttore{\os.}
\addPreCondizione{L'\os si trova nella sezione Analytics;}
\addPreCondizione{L'\os ha selezionato la funzionalità di modifica della sezione Analytics.}
\addPostCondizione{L'\os aggiunge la visualizzazione del grafico.}
\addTrigger{Il \os vuole aggiungere la visualizzazione del grafico.}
\addScenarioPrincipale{L'\os seleziona il grafico sugli allarmi inviati e risolti;}
\addScenarioPrincipale{L'\os seleziona la funzionalità di aggiunta del grafico.}
\makeUC{3.7.1}{Inserimento grafico sugli allarmi inviati e risolti}

% ======= UC Analytics: Rimozione Modulo grafico sugli allarmi inviati e risolti =======
\addAttore{\os.}
\addPreCondizione{L'\os si trova nella sezione Analytics;}
\addPreCondizione{L'\os ha selezionato la funzionalità di modifica della sezione Analytics.}
\addPostCondizione{L'\os rimuove la visualizzazione del grafico.}
\addTrigger{Il \os vuole rimuovere la visualizzazione del grafico.}
\addScenarioPrincipale{L'\os seleziona il grafico sugli allarmi inviati e risolti;}
\addScenarioPrincipale{L'\os seleziona la funzionalità di rimozione del grafico.}
\makeUC{3.7.2}{Rimozione grafico sugli allarmi inviati e risolti}

% ======= UC Dispositivi =======
\addAttore{\os.}
\addPreCondizione{Il Sistema è attivo;}
\addPreCondizione{L'\os è autenticato nel Sistema.}
\addTrigger{L'\os vuole visualizzare i dispositivi di un impianto specifico.}
\addPostCondizione{L'\os visualizza i dispositivi di un impianto specifico.}
\addScenarioPrincipale{L'\os seleziona la sezione Dispositivi nel menù principale;}
\addScenarioPrincipale{L'\os seleziona l'impianto in cui risiedono i dispositivi da visualizzare;}
\addScenarioPrincipale{L'\os visualizza l'elenco di tutti i dispositivi associati all'impianto selezionato. I dispositivi visualizzabili sono:
\begin{itemize}
    \item Termostato (Vedi \rif{4.1});
    \item Tapparelle (Vedi \rif{4.2});
    \item Sensori \textit{Ultra Wide Band} di caduta (Vedi \rif{4.3});
    \item Sensori \textit{Ultra Wide Band} di presenza (Vedi \rif{4.4});
    \item Luci (Vedi\rif{4.5}).
\end{itemize}}
\makeUC{4}{Visualizzazione Dispositivi}

% ======= UC Termostato =======
\addAttore{\os.}
\addPreCondizione{L'\os si trova nella sezione Dispositivi;}
\addPreCondizione{L'\os ha selezionato un impianto;}
\addPreCondizione{L'\os ha selezionato il dispositivo termostato.}
\addPostCondizione{L'\os visualizza il dispositivo termostato di un impianto specifico.}
\addTrigger{L'\os vuole visualizzare il termostato dell'impianto.}
\addScenarioPrincipale{L'\os visualizza le caratteristiche relative al dispositivo termostato:
\begin{itemize}
    \item Lo stato corrente del dispositivo (Vedi \rif{4.6});
    \item il nome del dispositivo (Vedi \rif{4.7});
    \item l'ambiente in cui risiede (Vedi \rif{4.8});
    \item le azioni eseguibili sul dispositivo (Vedi \rif{4.9}).
\end{itemize}}
\addScenarioAlternativo{Il Sistema non riesce a visualizzare le caratteristiche del dispositivo (Vedi \rif{8}).}
\addEstensione{\rif{8}}
\addInclusione{\rif{4.6}}
\addInclusione{\rif{4.7}}
\addInclusione{\rif{4.8}}
\addInclusione{\rif{4.9}}
\makeUC{4.1}{Visualizzazione Termostato}

% ======= UC Tapparelle =======
\addAttore{\os.}
\addPreCondizione{L'\os si trova nella sezione Dispositivi;}
\addPreCondizione{L'\os ha selezionato un impianto;}
\addPreCondizione{L'\os ha selezionato il dispositivo tapparella.}
\addPostCondizione{L'\os visualizza il dispositivo tapparella di un impianto specifico.}
\addTrigger{L'\os vuole visualizzare la tapparella dell'impianto.}
\addScenarioPrincipale{L'\os visualizza le caratteristiche relative al dispositivo tapparella:
\begin{itemize}
    \item Lo stato corrente del dispositivo (Vedi \rif{4.6});
    \item il nome del dispositivo (Vedi \rif{4.7});
    \item l'ambiente in cui risiede (Vedi \rif{4.8});
    \item le azioni eseguibili sul dispositivo (Vedi \rif{4.9}).
\end{itemize}}
\addScenarioAlternativo{Il Sistema non riesce a visualizzare le caratteristiche del dispositivo (Vedi \rif{8}).}
\addEstensione{\rif{8}}
\addInclusione{\rif{4.6}}
\addInclusione{\rif{4.7}}
\addInclusione{\rif{4.8}}
\addInclusione{\rif{4.9}}
\makeUC{4.2}{Visualizzazione Tapparella}

% ======= UC UWB CADUTA =======
\addAttore{\os.}
\addPreCondizione{L'\os si trova nella sezione Dispositivi;}
\addPreCondizione{L'\os ha selezionato un impianto;}
\addPreCondizione{L'\os ha selezionato il dispositivo sensore \textit{UWB} caduta.}
\addPostCondizione{L'\os visualizza il dispositivo sensore \textit{UWB} caduta di un impianto specifico.}
\addTrigger{L'\os vuole visualizzare il sensore \textit{UWB} caduta dell'impianto.}
\addScenarioPrincipale{L'\os visualizza le caratteristiche relative al dispositivo sensore \textit{UWB} caduta:
\begin{itemize}
    \item Lo stato corrente del dispositivo (Vedi \rif{4.6});
    \item il nome del dispositivo (Vedi \rif{4.7});
    \item l'ambiente in cui risiede (Vedi \rif{4.8});
    \item le azioni eseguibili sul dispositivo (Vedi \rif{4.9}).
\end{itemize}}
\addScenarioAlternativo{Il Sistema non riesce a visualizzare le caratteristiche del dispositivo (Vedi \rif{8}).}
\addEstensione{\rif{8}}
\addInclusione{\rif{4.6}}
\addInclusione{\rif{4.7}}
\addInclusione{\rif{4.8}}
\addInclusione{\rif{4.9}}
\makeUC{4.3}{Visualizzazione sensore \textit{UWB} di caduta}

% ======= UC UWB PRESENZA =======
\addPreCondizione{L'\os si trova nella sezione Dispositivi;}
\addPreCondizione{L'\os ha selezionato un impianto;}
\addPreCondizione{L'\os ha selezionato il dispositivo sensore \textit{UWB} presenza.}
\addPostCondizione{L'\os visualizza il dispositivo sensore \textit{UWB} presenza di un impianto specifico.}
\addTrigger{L'\os vuole visualizzare il sensore \textit{UWB} presenza dell'impianto.}
\addScenarioPrincipale{L'\os visualizza le caratteristiche relative al dispositivo sensore \textit{UWB} presenza:
\begin{itemize}
    \item Lo stato corrente del dispositivo (Vedi \rif{4.6});
    \item il nome del dispositivo (Vedi \rif{4.7});
    \item l'ambiente in cui risiede (Vedi \rif{4.8});
    \item le azioni eseguibili sul dispositivo (Vedi \rif{4.9}).
\end{itemize}}
\addScenarioAlternativo{Il Sistema non riesce a visualizzare le caratteristiche del dispositivo (Vedi \rif{8}).}
\addEstensione{\rif{8}}
\addInclusione{\rif{4.6}}
\addInclusione{\rif{4.7}}
\addInclusione{\rif{4.8}}
\addInclusione{\rif{4.9}}
\makeUC{4.4}{Visualizzazione sensore \textit{UWB} di presenza}

% ======= UC Luci =======
\addAttore{\os.}
\addPreCondizione{L'\os si trova nella sezione Dispositivi;}
\addPreCondizione{L'\os ha selezionato un impianto;}
\addPreCondizione{L'\os ha selezionato il dispositivo luce.}
\addPostCondizione{L'\os visualizza il dispositivo luce di un impianto specifico.}
\addTrigger{L'\os vuole visualizzare la luce dell'impianto.}
\addScenarioPrincipale{L'\os visualizza le caratteristiche relative al dispositivo luce:
\begin{itemize}
    \item Lo stato corrente del dispositivo (Vedi \rif{4.6});
    \item il nome del dispositivo (Vedi \rif{4.7});
    \item l'ambiente in cui risiede (Vedi \rif{4.8});
    \item le azioni eseguibili sul dispositivo (Vedi \rif{4.9}).
\end{itemize}}
\addScenarioAlternativo{Il Sistema non riesce a visualizzare le caratteristiche del dispositivo (Vedi \rif{8}).}
\addEstensione{\rif{8}}
\addInclusione{\rif{4.6}}
\addInclusione{\rif{4.7}}
\addInclusione{\rif{4.8}}
\addInclusione{\rif{4.9}}
\makeUC{4.5}{Visualizzazione luce}

% ======= UC Accesso =======
% VALERIA: in che senso 'controllo dell'accesso'???
% l'utente operatore può modificare o solo leggere questi dispositivi?

% ======= UC Stato dispositivo =======
\addAttore{\os.}
\addPreCondizione{L'\os si trova nella sezione Dispositivi;}
\addPreCondizione{L'\os ha selezionato un impianto;}
\addPreCondizione{L'\os ha selezionato un dispositivo.}
\addPostCondizione{L'\os visualizza lo stato del dispositivo.}
\addTrigger{L'\os vuole visualizzare lo stato di un dispositivo.}
\addScenarioPrincipale{L'\os visualizza lo stato del dispositivo, caratterizzato dallo stato corrente e dalla data di ultima modifica dello stato.}
\makeUC{4.6}{Visualizzazione stato del dispositivo}

% ======= UC Nome dispositivo =======
\addAttore{\os.}
\addPreCondizione{L'\os si trova nella sezione Dispositivi;}
\addPreCondizione{L'\os ha selezionato un impianto;}
\addPreCondizione{L'\os ha selezionato un dispositivo.}
\addPostCondizione{L'\os visualizza il nome del dispositivo.}
\addTrigger{L'\os vuole visualizzare il nome di un dispositivo.}
\addScenarioPrincipale{L'\os visualizza il nome del dispositivo.}
\makeUC{4.7}{Visualizzazione nome del dispositivo}

% ======= UC Ambiente dispositivo =======
\addAttore{\os.}
\addPreCondizione{L'\os si trova nella sezione Dispositivi;}
\addPreCondizione{L'\os ha selezionato un impianto;}
\addPreCondizione{L'\os ha selezionato un dispositivo.}
\addPostCondizione{L'\os visualizza l'ambiente in cui risiede il dispositivo.}
\addTrigger{L'\os vuole visualizzare l'ambiente in cui risiede il dispositivo.}
\addScenarioPrincipale{L'\os visualizza l'ambiente in cui risiede il dispositivo.}
\makeUC{4.8}{Visualizzazione ambiente del dispositivo}

% ======= UC Azioni dispositivo =======
\addAttore{\os.}
\addPreCondizione{L'\os si trova nella sezione Dispositivi;}
\addPreCondizione{L'\os ha selezionato un impianto;}
\addPreCondizione{L'\os ha selezionato un dispositivo.}
\addPostCondizione{L'\os visualizza le azioni disponibili del dispositivo.}
\addTrigger{L'\os vuole visualizzare le azioni disponibili di un dispositivo.}
\addScenarioPrincipale{L'\os visualizza le azioni disponibili del dispositivo.}
\makeUC{4.9}{Visualizzazione azioni del dispositivo}

%
%
%
%
% configurazione allarmi e utenti
%
%
%
%

\addAttore{\admin.}
\addPreCondizione{Il Sistema è attivo;}
\addPreCondizione{L'\admin è autenticato nel Sistema;}
\addPreCondizione{L'\admin è nella sezione allarmi.}
\addPostCondizione{L'\admin ha modificato la configurazione degli allarmi.}
\addTrigger{L'\admin vuole modificare la configurazione degli allarmi.}
\addScenarioPrincipale{L'\admin accede alla sezione di configurazione allarmi;}
\addScenarioPrincipale{L'\admin visualizza gli allarmi configurati;}
\addScenarioPrincipale{L'\admin modifica la configurazione.}
\addScenarioAlternativo{Non ci sono allarmi configurati nel Sistema.}
\addEstensione{Aggiungi allarme \rif{5.1}}
\addEstensione{Modifica allarme \rif{5.2}}
\addEstensione{Elimina allarme \rif{5.3}}
\makeUC{5}{Modifica configurazione allarmi}

\addAttore{\admin.}
\addPreCondizione{Il Sistema è attivo;}
\addPreCondizione{L'\admin è autenticato nel Sistema;}
\addPreCondizione{L'\admin sta visualizzando la sezione di configurazione allarmi.}
\addPostCondizione{Un nuovo allarme è aggiunto alla configurazione del Sistema.}
\addTrigger{L'\admin vuole aggiungere un nuovo allarme.}
\addScenarioPrincipale{L'\admin seleziona il sensore per l'allarme;}
\addScenarioPrincipale{L'\admin seleziona la priorità dell'allarme;}
\addScenarioPrincipale{L'\admin configura le soglie parametrizzabili;}
\addScenarioPrincipale{L'\admin configura il multicast per la notifica;}
\addScenarioPrincipale{Il Sistema salva il nuovo allarme.}
\addScenarioAlternativo{Il sensore selezionato è già associato a un allarme;}
\addScenarioAlternativo{I parametri inseriti non sono validi.}
\addInclusione{Selezionare il sensore \rif{5.1.1}}
\addInclusione{Selezionare priorità allarme \rif{5.1.2}}
\addInclusione{Configurare le soglie parametrizzabili \rif{5.1.3}}
\addInclusione{Configurare multicast \rif{5.1.4}}
\addEstensione{Errore aggiunta allarme \rif{5.1.5}}
\makeUC{5.1}{Aggiungi allarme}

\addAttore{\admin.}
\addPreCondizione{Il Sistema è attivo;}
\addPreCondizione{L'\admin è autenticato nel Sistema;}
\addPreCondizione{L'\admin sta aggiungendo un nuovo allarme;}
\addPreCondizione{Esistono sensori disponibili nel Sistema.}
\addPostCondizione{Il sensore è selezionato per l'allarme.}
\addTrigger{L'\admin deve selezionare un sensore per l'allarme.}
\addScenarioPrincipale{L'\admin visualizza la lista dei sensori disponibili (caduta, temperatura, presenza, pulsante emergenza, gateway offline);}
\addScenarioPrincipale{L'\admin seleziona il tipo di sensore;}
\addScenarioPrincipale{Il Sistema associa il sensore all'allarme.}
\addScenarioAlternativo{Non ci sono sensori disponibili.}
\makeUC{5.1.1}{Selezionare il sensore}

\addAttore{\admin.}
\addPreCondizione{Il Sistema è attivo;}
\addPreCondizione{L'\admin è autenticato nel Sistema;}
\addPreCondizione{L'\admin sta aggiungendo un nuovo allarme;}
\addPreCondizione{Esistono livelli di priorità configurati nel Sistema (3-4 livelli).}
\addPostCondizione{La priorità dell'allarme è selezionata.}
\addTrigger{L'\admin deve assegnare una priorità all'allarme.}
\addScenarioPrincipale{L'\admin visualizza i livelli di priorità disponibili;}
\addScenarioPrincipale{L'\admin seleziona il livello di priorità appropriato;}
\addScenarioPrincipale{Il Sistema associa la priorità all'allarme.}
\makeUC{5.1.2}{Selezionare priorità allarme}

\addAttore{\admin.}
\addPreCondizione{Il Sistema è attivo;}
\addPreCondizione{L'\admin è autenticato nel Sistema;}
\addPreCondizione{L'\admin sta aggiungendo un nuovo allarme;}
\addPreCondizione{Il sensore selezionato supporta soglie parametrizzabili.}
\addPostCondizione{Le soglie parametrizzabili sono configurate per l'allarme.}
\addTrigger{L'\admin deve configurare le soglie di attivazione dell'allarme.}
\addScenarioPrincipale{L'\admin definisce le soglie di attivazione in base al tipo di sensore (temperatura in gradi, presenza in minuti);}
\addScenarioPrincipale{Il Sistema valida i parametri inseriti;}
\addScenarioPrincipale{Il Sistema associa le soglie all'allarme.}
\addScenarioAlternativo{Le soglie inserite non sono nel range valido.}
\addEstensione{Errore configurazione soglie \rif{6}}
\makeUC{5.1.3}{Configurare le soglie parametrizzabili}

\addAttore{\admin.}
\addPreCondizione{Il Sistema è attivo;}
\addPreCondizione{L'\admin è autenticato nel Sistema;}
\addPreCondizione{L'\admin sta aggiungendo un nuovo allarme;}
\addPreCondizione{Esistono utenti OSS registrati nel Sistema.}
\addPostCondizione{Il multicast per la notifica dell'allarme è configurato.}
\addTrigger{L'\admin deve configurare a quali operatori inviare l'allarme.}
\addScenarioPrincipale{L'\admin visualizza la lista degli operatori OSS disponibili;}
\addScenarioPrincipale{L'\admin seleziona gli operatori che riceveranno la notifica;}
\addScenarioPrincipale{Il Sistema associa il multicast all'allarme.}
\addScenarioAlternativo{Non ci sono operatori OSS registrati.}
\addEstensione{Errore configurazione multicast \rif{C5.1.4.1}}
\makeUC{5.1.4}{Configurare multicast}

\addAttore{\admin.}
\addPreCondizione{Il Sistema è attivo;}
\addPreCondizione{L'\admin è autenticato nel Sistema;}
\addPreCondizione{La configurazione del multicast non è valida.}
\addPostCondizione{Il Sistema notifica l'errore.}
\addTrigger{Nessun operatore è stato assegnato all'allarme.}
\addScenarioPrincipale{Il Sistema mostra un messaggio di errore.}
\makeUC{5.1.4.1}{Errore configurazione multicast}

\addAttore{\admin.}
\addPreCondizione{Il Sistema è attivo;}
\addPreCondizione{L'\admin è autenticato nel Sistema;}
\addPreCondizione{L'\admin sta tentando di aggiungere un allarme;}
\addPreCondizione{Si verifica un errore (sensore già utilizzato, parametri non validi).}
\addPostCondizione{Il Sistema notifica l'errore e non aggiunge l'allarme.}
\addTrigger{Si verifica un errore durante l'aggiunta dell'allarme.}
\addScenarioPrincipale{Il Sistema rileva l'errore;}
\addScenarioPrincipale{Il Sistema mostra un messaggio di errore con i dettagli.}
\makeUC{5.1.5}{Errore aggiunta allarme}

\addAttore{\admin.}
\addPreCondizione{Il Sistema è attivo;}
\addPreCondizione{L'\admin è autenticato nel Sistema;}
\addPreCondizione{L'\admin sta visualizzando la sezione di configurazione allarmi;}
\addPreCondizione{Esiste almeno un allarme configurato nel Sistema.}
\addPostCondizione{L'allarme selezionato è modificato.}
\addTrigger{L'\admin vuole modificare un allarme esistente.}
\addScenarioPrincipale{L'\admin seleziona un allarme dalla lista;}
\addScenarioPrincipale{L'\admin visualizza la configurazione attuale;}
\addScenarioPrincipale{L'\admin modifica i parametri desiderati;}
\addScenarioPrincipale{Il Sistema salva le modifiche.}
\addScenarioAlternativo{I nuovi parametri inseriti non sono validi;}
\addScenarioAlternativo{Si verifica un errore durante il salvataggio.}
\addInclusione{Modifica priorità allarme \rif{5.2.1}}
\addInclusione{Modifica soglie parametrizzabili \rif{5.2.2}}
\addInclusione{Modifica assegnatari allarme \rif{5.2.3}}
\addEstensione{Errore configurazione soglie \rif{6}}
\makeUC{5.2}{Modifica allarme}

\addAttore{\admin.}
\addPreCondizione{Il Sistema è attivo;}
\addPreCondizione{L'\admin è autenticato nel Sistema;}
\addPreCondizione{L'\admin sta modificando un allarme esistente.}
\addPostCondizione{La priorità dell'allarme è aggiornata.}
\addTrigger{L'\admin vuole modificare la priorità di un allarme.}
\addScenarioPrincipale{L'\admin visualizza la priorità attuale dell'allarme;}
\addScenarioPrincipale{L'\admin seleziona la nuova priorità;}
\addScenarioPrincipale{Il Sistema aggiorna la priorità dell'allarme.}
\makeUC{5.2.1}{Modifica priorità allarme}

\addAttore{\admin.}
\addPreCondizione{Il Sistema è attivo;}
\addPreCondizione{L'\admin è autenticato nel Sistema;}
\addPreCondizione{L'\admin sta modificando un allarme esistente;}
\addPreCondizione{Il sensore dell'allarme supporta soglie parametrizzabili.}
\addPostCondizione{Le soglie parametrizzabili dell'allarme sono aggiornate.}
\addTrigger{L'\admin vuole modificare le soglie di attivazione dell'allarme.}
\addScenarioPrincipale{L'\admin visualizza le soglie attuali;}
\addScenarioPrincipale{L'\admin inserisce i nuovi valori delle soglie;}
\addScenarioPrincipale{Il Sistema valida i parametri inseriti;}
\addScenarioPrincipale{Il Sistema aggiorna le soglie.}
\addScenarioAlternativo{Le soglie inserite non sono nel range valido.}
\addEstensione{Errore configurazione soglie \rif{6}}
\makeUC{5.2.2}{Modifica soglie parametrizzabili}

\addAttore{\admin.}
\addPreCondizione{Il Sistema è attivo;}
\addPreCondizione{L'\admin è autenticato nel Sistema;}
\addPreCondizione{L'\admin sta modificando un allarme esistente;}
\addPreCondizione{Esistono utenti OSS registrati nel Sistema.}
\addPostCondizione{Gli assegnatari dell'allarme sono aggiornati.}
\addTrigger{L'\admin vuole modificare gli operatori che ricevono l'allarme.}
\addScenarioPrincipale{L'\admin visualizza gli assegnatari attuali;}
\addScenarioPrincipale{L'\admin modifica la lista degli operatori OSS;}
\addScenarioPrincipale{Il Sistema aggiorna gli assegnatari.}
\addScenarioAlternativo{Nessun operatore viene assegnato ad un allarme critico.}
\makeUC{5.2.3}{Modifica assegnatari allarme}

\addAttore{\admin.}
\addPreCondizione{Il Sistema è attivo;}
\addPreCondizione{L'\admin è autenticato nel Sistema;}
\addPreCondizione{L'\admin sta visualizzando la sezione di configurazione allarmi;}
\addPreCondizione{Esiste almeno un allarme configurato nel Sistema.}
\addPostCondizione{L'allarme selezionato è eliminato dal Sistema.}
\addTrigger{L'\admin vuole eliminare un allarme esistente.}
\addScenarioPrincipale{L'\admin seleziona un allarme dalla lista;}
\addScenarioPrincipale{L'\admin conferma l'eliminazione;}
\addScenarioPrincipale{Il Sistema elimina l'allarme.}
\addScenarioAlternativo{L'allarme è critico e non può essere eliminato;}
\addScenarioAlternativo{L'utente annulla l'operazione.}
\addEstensione{Errore eliminazione allarme \rif{5.3.1}}
\makeUC{5.3}{Elimina allarme}

\addAttore{\admin.}
\addPreCondizione{Il Sistema è attivo;}
\addPreCondizione{L'\admin è autenticato nel Sistema;}
\addPreCondizione{L'allarme selezionato è critico o si verifica un errore.}
\addPostCondizione{Il Sistema notifica l'errore e non elimina l'allarme.}
\addTrigger{L'allarme non può essere eliminato.}
\addScenarioPrincipale{Il Sistema mostra un messaggio che spiega perché l'allarme non può essere eliminato.}
\makeUC{5.3.1}{Errore eliminazione allarme}

\addAttore{\admin.}
\addPreCondizione{Il Sistema è attivo;}
\addPreCondizione{L'\admin è autenticato nel Sistema;}
\addPreCondizione{L'\admin sta configurando o modificando le soglie parametrizzabili;}
\addPreCondizione{I valori delle soglie non sono nel range valido.}
\addPostCondizione{Il Sistema notifica l'errore specifico e richiede correzione.}
\addTrigger{I valori delle soglie non rispettano i vincoli del Sistema.}
\addScenarioPrincipale{Il Sistema valida i valori inseriti;}
\addScenarioPrincipale{Il Sistema rileva l'errore;}
\addScenarioPrincipale{Il Sistema mostra un messaggio di errore specifico.}
\addEstensione{La soglia (valore) è troppo alta \rif{6.1}}
\addEstensione{La soglia (valore) è troppo bassa \rif{6.2}}
\makeUC{6}{Errore configurazione soglie}

\addAttore{\admin.}
\addPreCondizione{Il Sistema è attivo;}
\addPreCondizione{L'\admin è autenticato nel Sistema;}
\addPreCondizione{Il valore della soglia inserito supera il limite massimo consentito.}
\addPostCondizione{Il Sistema mostra il valore massimo consentito.}
\addTrigger{Il valore della soglia è troppo alto.}
\addScenarioPrincipale{Il Sistema mostra un messaggio indicando il valore massimo consentito per quella soglia.}
\makeUC{6.1}{La soglia (valore) è troppo alta}

\addAttore{\admin.}
\addPreCondizione{Il Sistema è attivo;}
\addPreCondizione{L'\admin è autenticato nel Sistema;}
\addPreCondizione{Il valore della soglia inserito è inferiore al limite minimo consentito.}
\addPostCondizione{Il Sistema mostra il valore minimo consentito.}
\addTrigger{Il valore della soglia è troppo basso.}
\addScenarioPrincipale{Il Sistema mostra un messaggio indicando il valore minimo consentito per quella soglia.}
\makeUC{6.2}{La soglia (valore) è troppo bassa}


%\subsection*{UC7: Gestione utenti e permessi (\admin)}
%\texttt{TBD, sincronizzare con pippovenzo}


% ======= UC Analytics: Errore dati insufficienti =======
\addAttore{\os.}
\addPreCondizione{L'\os si trova nella sezione Analytics;}
\addPreCondizione{L'\os ha selezionato un grafico oppure un suggerimento da visualizzare.}
\addPostCondizione{L'\os visualizza un messaggio di errore.}
\addTrigger{Il Sistema non dispone di un numero sufficiente di dati per generare il grafico o il suggerimento richiesto dall'\os.}
\addScenarioPrincipale{Il Sistema mostra all'\os un messaggio di errore.}
\makeUC{7}{Errore dati insufficienti}


% ======= UC Errore dispositivo non disponibile  =======
\addAttore{\os.}
\addPreCondizione{L'\os si trova nella sezione Dispositivi;}
\addPreCondizione{L'\os ha selezionato un impianto;}
\addPreCondizione{L'\os ha selezionato un dispositivo.}
\addTrigger{Il Sistema non trova informazioni relative al dispositivo.}
\addPostCondizione{L'\os visualizza un messaggio di errore.}
\addScenarioPrincipale{L'\os visualizza la lista di dispositivi associati all'impianto;}
\addScenarioPrincipale{Il Sistema mostra all'\os un messaggio di errore.}
\makeUC{8}{Errore dispositivo non risponde (\textit{offline})}

\end{document}
