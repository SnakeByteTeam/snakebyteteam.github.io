\documentclass[10pt, letterpaper]{article}
\usepackage[nomarginpar, margin=2.75cm, tmargin=3cm, bmargin=1.75cm]{geometry}
\usepackage{graphicx}
\usepackage[
    colorlinks=true,      
    linkcolor=black,      
    urlcolor=blue,       
    citecolor=black       
]{hyperref}
\usepackage[table, x11names]{xcolor}
\renewcommand{\contentsname}{Indice}
\usepackage{fancyhdr}
\usepackage{tabularx}

%Comandi per livello di sottosezioni = 3
\setcounter{tocdepth}{4}
\setcounter{secnumdepth}{4}
\newcommand{\trisubsection}[1]{\paragraph{#1}\mbox{}\\}
\pagestyle{fancy}
\fancyhf{}
\fancyhead[L]{SnakeByte} 
\fancyhead[R]{Norme di Progetto}
\fancyfoot[C]{\thepage}


\begin{document}

\begin{titlepage}
    \begin{center}
        \begin{center}
            \includegraphics[width=0.6\textwidth]{./img/logo.pdf}
        \end{center}
        \vspace{4cm}
        \huge\textbf{Norme di Progetto}\par
        \vspace{2cm}
        \large \textbf{SnakeByte} (Gruppo 1):\\
        \large Valeria Baleanu, Leonardo Pellizzon, Filippo Venzo, Giuseppe De Fina, \\
         Francesco Pasqual, Christian Libralato, Luca Granziero \\
        (2109911, 2111006, 2113705, 2113187, 2103119, 2101047, 2075512)
        \vfill
        \small
        \begin{center}
            \begin{tabular}{|c|c|c|c|}
                \hline
                \multicolumn{4}{|c|}{\textbf{Informazioni documento}} \\
                \hline
                \rowcolor{lightgray} \textbf{Versione} & \textbf{Data} & \textbf{Stato} & \textbf{Destinatari} \\
                \hline
                0.2.0 & 08/12/2025 & Verificato & SnakeByte, prof. Vardanega Tullio, prof. Cardin Riccardo \\
                \hline
            \end{tabular}
        \end{center}
        \vfill
        \large Contatti: snakebyteteam@gmail.com
    \end{center}
\end{titlepage}
\begin{center}
\begin{tabularx}{\textwidth}{|c|c|c|c|c|X|}
    \hline
    \multicolumn{6}{|c|}{\textbf{Registro delle modifiche}} \\
    \hline
    \rowcolor{lightgray} \textbf{Versione} & \textbf{Data} & \textbf{Autore} & \textbf{Verifica} & \textbf{Approvazione} & \textbf{Descrizione} \\
    \hline
    0.2.0 & 08/12/2025 & C. Libralato & L. Pellizzon & - & Aggiunte a processo di verifica e approvazione (sez. 3.1.4, 3.1.5). Aggiunta Sviluppo con Analisi Requisiti ai processi primari (sez. 2.2 )\\ 
    \hline
    0.1.6 & 29/11/2025 & F. Pasqual & V. Baleanu & - & Aggiunto processo di verifica tramite \textit{pull request} \\ 
    \hline
    0.1.5 & 8/11/2025 & Filippo Venzo & - & - & Aggiunta tabella attività completate nella struttura dei verbali\\
    \hline
    0.1.4 & 30/10/2025 & Filippo Venzo & Luca Granziero & - & Modifica struttura tabella attività, modifiche tipografiche\\
    \hline
    0.1.3 & 26/10/2025 & Filippo Venzo & Luca Granziero & - & Aggiunta sezioni e termini glossario, modifica convenzione date nei nomi dei file \\
    \hline
    0.1.2 & 23/10/2025 & Filippo Venzo & Luca Granziero & - & Correzione errori ortografici e aggiunta link \\
    \hline
    0.1.1 & 22/10/2025 & Filippo Venzo & Luca Granziero & - & Aggiunta sezioni e sotto sezioni \\
    \hline
    0.1.0 & 17/10/2025 & Filippo Venzo & Luca Granziero & - & Prima stesura \\
    \hline
\end{tabularx}    
\end{center}

\newpage

\tableofcontents

\newpage

\section{Introduzione}
\subsection{Finalità del documento}
Il presente documento intende fissare le linee guida che il gruppo \textit{SnakeByte} si impegna a rispettare ed attuare per perseguire la migliore efficienza ed efficacia nel processo di realizzazione del progetto didattico.

Il documento è strutturato secondo le norme dello Standard ISO/IEC 12207:1995 e segue quanto descritto nel \textit{Regolamento del progetto didattico (A.a. 2025/2026)}. Presenta una descrizione dei \textit{processi} del ciclo di vita del \textit{software} e delle \textit{attività} di cui sono composti.
A sua volta, ogni attività, è composta da una serie di procedure metodiche dotate di obiettivi e strumenti ben definiti.

\'E importante notare che il documento in questione è in continua evoluzione fino al suo ritiro, poiché le norme contenute al suo interno vengono costantemente revisionate, ottimizzate ed aggiornate, seguendo un approccio incrementale.

Ogni attività svolta nell'interesse del progetto didattico e nei suoi materiali è regolamentata precedentemente all'esecuzione della stessa.

\subsection{Glossario}
Il documento cita alcuni termini la cui definizione può risultare ambigua. Per questo, è possibile consultare il \textit{glossario$_{G}$} il quale contiene le definizioni di tali espressioni, che saranno marcate da una lettera \textit{G} a pedice.

\subsection{Riferimenti Normativi}
\begin{itemize}
    \item \textbf{Standard ISO/IEC 12207:1995}: 
    
    \url{https://www.math.unipd.it/~tullio/IS-1/2009/Approfondimenti/ISO_12207-1995.pdf} 
    
    (consultato il 30/10/2025);
    \item \textbf{Regolamento del progetto didattico}: 
    
    \url{https://www.math.unipd.it/~tullio/IS-1/2025/Dispense/PD1.pdf} 
    
    (consultato il 23/10/2025).
\end{itemize}

\subsection{Riferimenti Informativi}
\begin{itemize}
    \item \textbf{Sito dedicato alla documentazione}
    
    \url{https://snakebyteteam.github.io}

    (consultato il 23/10/2025)

    \item \textbf{Glossario}:

    \url{https://snakebyteteam.github.io/glossary.html}

    (consultato il 30/10/2025)

\end{itemize}

\section{Processi primari}
Le attività  che compongono i processi primari considerate in questa sede sono un sottoinsieme proprio delle attività previste dallo Standard ISO/IEC 12207:1995. 
\subsection{Fornitura}
Il processo di fornitura, come specificato nello Standard ISO/IEC 12207:1995, definisce le attività dell'organizzazione che fornisce il prodotto software all'acquirente, dalla concezione fino alla consegna del prodotto. Viene istanziato conseguentemente alla redazione della \textit{Valutazione dei \textit{capitolati$_{G}$}}.

\subsubsection{Attività}
Il processo di fornitura si compone delle seguenti attività
\begin{itemize}
    \item \textbf{Avviamento}: revisione delle proposte dei richiedenti. Per i capitolati di maggiore interesse vengono mandate delle comunicazioni via mail per eventuali approfondimenti;
    
    \item \textbf{Preparazione della risposta}: viene scelto il capitolato per cui ci vuole candidare sulla base delle considerazioni fatte nella fase precedente e viene preparato il documento di \textbf{candidatura}.
\end{itemize}

\subsubsection{Documentazione risultante}
La documentazione prodotta durante le attività di fornitura, la quale verrà consegnata ai committenti$_{G}$ quali prof. Tullio Vardanega, prof. Riccardo Cardin e all'azienda proponente è la seguente 

\begin{itemize}
    \item \textbf{Valutazioni dei capitolati} contenente 
    \begin{itemize}
        \item Titolo del capitolato e nome dell'azienda proponente;
        \item Una breve descrizione del capitolato e dei suoi obiettivi;
        \item Considerazioni del gruppo;
    \end{itemize}
    \item \textbf{Dichiarazione degli impegni} contenente
    \begin{itemize}
        \item Impegni orari e suddivisione dei ruoli;
        \item Preventivo dei costi totali del progetto (calcolato secondo il \textit{Regolamento del progetto didattico});
        \item Data prevista di consegna.
    \end{itemize}
    \item \textbf{Lettera di presentazione} contenente
    \begin{itemize}
        \item Scelta del capitolato;
        \item Motivazione della scelta;
        \item Riassunto costi complessivi e data prevista di consegna.
    \end{itemize}
\end{itemize}

\subsection{Sviluppo}
Il processo di Sviluppo consiste in un'insieme di attività necessarie per la realizzazione del prodotto software, queste sono particolarmente
orientate verso analisi dei requisiti, design, codifica, integrazione e testing; si tratta infatti del processo primario principale che
guida la realizzazione del prodotto software per la quasi totalità del ciclo di vita.

\subsubsection{Attività}
Il processo di sviluppo si compone delle seguenti attività:
\begin{itemize}
    \item \textbf{System requirements analysis};
    \item \textbf{Software requirements analysis};
    \item \textbf{Software architectural design};
    \item \textbf{Software detailed design};
    \item \textbf{Software coding and testing};
    \item \textbf{Software qualification testing};
    \item \textbf{Software installation};
\end{itemize}

\subsubsection{Documentazione risultante}
La documentazione prodotta durante il processo di Sviluppo consiste in:
\begin{itemize}
    \item \textbf{\textit{Diagrammi UML$_{G}$}}: diagrammi realizzati secondo lo standard \textit{UML2.5$_{G}$} utili alla definizione dei \textit{casi d'uso$_{G}$} all'interno dell'Analisi dei Requisiti.
    \item \textbf{Analisi dei Requisiti}: documento che racchiude i risultati della \textit{System requirements analysis} (requisiti funzionali) e \textit{Software requirements analysis} (requisiti non funzionali).

\end{itemize}

\subsubsection{Analisi dei Requisiti}
Il suddetto documento, la cui redazione è affidata al ruolo dell'Analista, è necessario al fine di determinare e tracciare l'insieme di
requisiti che l'applicativo deve soddisfare. 

L'analisi è strutturata in maniera tale da ricavare i requisiti a partire dai \textit{casi d'uso}, che 
consistono in una serie di scenari che condividono uno scopo per un utente che interagisce con il sistema. La ricerca dei casi d'uso avviene tramite \textit{brainstorming$_{G}$} interno al gruppo e si
avvale di \textit{feedback$_{G}$} da parte della Proponente.\\
Il documento è strutturato nel seguente modo:

\trisubsection{Introduzione}
Una breve introduzione descrive le finalità del documento, i Riferimenti Informativi e Normativi.

\trisubsection{Descrizione del prodotto}
In questa sezione viene descritto lo scopo e l'obiettivo principale del prodotto, oltre che  le principali funzionalità
che deve possedere e l'insieme di utenti a cui è destinato.

\trisubsection{Elenco \textit{casi d'uso}}
I \textit{casi d'uso} vengono elencati seguendo la numerazione associata, sono presentati attraverso un diagramma \textit{UML} e una descrizione testuale in forma tabellare contenente sia le informazioni
all'interno del diagramma sia quelle non rappresentabili da esso, tra cui Pre-condizioni e le Post-condizioni.\\
La struttura è la seguente:

%==============================
%Da cambiare con template che però non è ancora confermato
\begin{center}
    \begin{tabularx}{\textwidth}{|c| >{\centering\arraybackslash}X|}
        \hline
        \rowcolor{lightgray} \textbf{Campo} & \textbf{Descrizione} \\
        \hline
        Attori & Coloro che partecipano attivamente al caso d'uso per raggiungere un preciso obiettivo  \\
        \hline
        Pre-condizioni & Condizioni che devono essere soddisfatte prima dello scenario descritto dal caso d'uso\\
        \hline
        Post-condizioni & Condizioni che risultano soddisfatte dopo il completamento dello scenario principale del caso d'uso. Se viene completato uno scenario alternativo, saranno soddisfatte le Post-condizioni di quest'ultimo \\
        \hline
        Trigger & La motivazione che porta l'utente a svolgere i passi del caso d'uso \\
        \hline
        Scenario principale & Sequenza di passi che l'utente deve seguire per completare il caso d'uso \\
        \hline
        Scenari alternativi & Scenario divergente dal principale per il verificarsi di una particolare condizione \\
        \hline
        Estensioni &  Casi d'uso ulteriori eseguiti al verificarsi di una particolare condizione nel caso d'uso primario. Modificano Scenario e Post-condizioni \\
        \hline
        Inclusioni & Casi d'uso ulteriori eseguiti al fine di completare il caso d'uso principale. Vengono eseguiti tutti incondizionatamente. \\
        \hline
        
    \end{tabularx}
\end{center}
%==========================
La crezione della tabella è facilitata dall'utilizzo del template \texttt{templateAdR.sty}.\\
Non tutti i \textit{casi d'uso} necessitano della tabella nella sua interezza, la presenza dei campi: Trigger, Scenari alternativi, Estensioni e Inclusioni
dipende dalla situazione.


 
\trisubsection{\textbf{Elenco requisiti}}
I requisiti sono identificati da un codice univoco e sono associati agli casi d'uso da cui sono stati generati.

\section{Processi di supporto}
\subsection{Documentazione}
\subsubsection{Struttura generale dei documenti} 
\label{sec:Struttura}
Le seguenti sezioni illustrano le componenti che ogni documento creato deve avere. Ogni documento deve essere redatto utilizzato il linguaggio \LaTeX$_{G}$, in particolare il file \texttt{template.sty} e template forniti nel repository interno.

\trisubsection{Prima pagina}
La prima pagina di ogni documento deve riportare, in ordine di posizionamento dall'alto verso il basso

\begin{itemize}
    \item Logo del gruppo \textit{SnakeByte};
    \item Titolo del documento;
    \item Il nome e il numero del gruppo \textit{SnakeByte};
    \item Nome e cognome di ogni componente e relativo numero di matricola UniPD;
    \item Informazioni generali del documento quali
    \begin{itemize}
        \item Versione attuale;
        \item Data di creazione della versione;
        \item Lo stato attuale;
        \item I destinatari del documento.
    \end{itemize}
    \item Contatto email del gruppo \textit{SnakeByte}.
\end{itemize}

\trisubsection{Intestazione}
Ogni pagina di qualsiasi documento deve riportare come intestazione
\begin{itemize}
    \item Nome del gruppo \textit{SnakeByte};
    \item Titolo del relativo documento.
\end{itemize}

\trisubsection{Registro delle modifiche}
Tutti i documenti, interni ed esterni, devono riportare a partire dalla seconda pagina il \textit{registro delle modifiche} sottoforma di tabella, la quale deve riassumere

\begin{itemize}
    \item Versione del documento;
    \item Data di creazione della versione;
    \item Autore della versione;
    \item Verificatore della versione;
    \item Approvatore della versione;
    \item Descrizione riassuntiva delle modifiche alla versione precedente.
\end{itemize}


Ogni modifica ad un documento scatena la creazione di una nuova versione di esso e quindi la compilazione di una nuova riga, verso il basso, della tabella.

\trisubsection{Indice}
Ogni documento deve riportare l'indice dove saranno elencati i titoli di tutte le sezioni e sottosezioni. \\
Ogni titolo deve essere provvisto di link che porta alla sezione associata all'interno dello stesso documento.\\

\textbf{Metodologie}
\begin{itemize}
    \item Questo deve essere fatto tramite il pacchetto \texttt{hyperref} fornito dal linguaggio \LaTeX$_{G}$
\end{itemize}

\subsubsection{Struttura dei verbali}
I verbali sia interni che esterni, oltre alla struttura descritta nel capitolo \hyperref[sec:Struttura]{\textit{§3.1.1}}, devono essere composti dalle seguenti sezioni

\begin{itemize}
    \item \textbf{Informazioni}, contenente
    \begin{itemize}
        \item Data di svolgimento;
        \item Ora inizio;
        \item Ora fine;
        \item Modalità di svolgimento (Presenza, Online o tramite canali asincroni).
    \end{itemize}
    \item \textbf{Presenze}, contenente, in forma tabellare, le seguenti informazioni
    \begin{itemize}
        \item Nome e cognome di tutti i membri;
        \item Ruolo (ND se non definito);
        \item Presenza alla riunione.
    \end{itemize}
    \item \textbf{Ordine del giorno}, con all'interno una lista degli argomenti che vengono trattati all'interno dell'incontro;
    \item \textbf{Approfondimento} degli argomenti ordine del giorno;
    \item \textbf{Decisioni} (\hyperref[sec:TabellaDecisioni]{Sezione \textit{§3.1.2.1}});
    \item \textbf{Attività da svolgere} (\hyperref[sec:TabellaToDo]{Sezione \textit{§3.1.2.2}});
    \item \textbf{Attività svolte} (\hyperref[sec:TabellaDone]{Sezione \textit{§3.1.2.3}})
\end{itemize}

\label{sec:TabellaDecisioni}
\trisubsection{Tabella delle decisioni}
Per il tracciamento e l'organizzazione di ogni decisione presa collettivamente dal gruppo \textit{SnakeByte}, al termine di ogni verbale, deve essere presente una tabella che riporta le decisioni prese in seguito alla riunione in questione. Per ogni decisione deve essere riportato
\begin{itemize}
    \item Identificativo alfanumerico della decisioni, così composto\\ \texttt{v\{i, e\}}\_AAAA\_MM\_GG.d\texttt{<numero\_decisione>};
    \item Descrizione testuale della decisione presa.
\end{itemize}

\label{sec:TabellaToDo}
\trisubsection{Tabella delle attività da svolgere}
Per il tracciamento delle attività da eseguire, emerse durante l'incontro trattato dal verbale, deve essere presente una tabella che riporta una lista di queste, riassumendo le seguenti informazioni
\begin{itemize}
    \item Identificativo alfanumerico dell'attività, così composto\\ \texttt{v\{i, e\}}\_AAAA\_MM\_GG.a\texttt{<numero\_attività>};
    \item Descrizione testuale dell'attività;
    \item Id GitHub Issue associata all'attività (carattere \textit{"-"} se non presente);
    \item Assegnatario dell'attività;
    \item Scadenza di completamento.
\end{itemize}

\label{sec:TabellaDone}
\trisubsection{Tabella delle attività svolte}
Per il tracciamento delle attività svolte, deve essere presente una tabella che riporta una lista di queste, riassumendo le seguenti informazioni
\begin{itemize}
    \item Identificativo alfanumerico dell'attività, come definito nella \hyperref[sec:TabellaToDo]{Sezione \textit{§3.1.2.2}};
    \item Id GitHub Issue associata all'attività (carattere \textit{"-"} se non presente);
    \item Data di completamento.
\end{itemize}



\subsubsection{Redazione dei documenti}
La stesura e l'aggiornamento dei documenti è affidata alla persona che ricopre, in quel periodo, il ruolo di \textit{amministratore$_{G}$}. Le informazioni che verranno usate per redarre i documenti saranno ricavate dalle riunioni e comunicazioni con gli altri componenti del gruppo.

\trisubsection{Redazione dei verbali}
Differentemente dagli altri documenti, ogni verbale viene redatto da una persona decisa durante la rispettiva riunione. 

\trisubsection{Redazione dell'analisi dei requisiti}
Differentemente dagli altri documenti, l'analisi dei requisiti viene redatta dalla persona incaricata in quel momento del ruolo di 
\textit{analista} (\hyperref[sec:Analista]{Sezione \textit{§4.1.1.5}}). 

\subsubsection{Verifica dei documenti}
Ogni documento dopo essere stato redatto, deve essere verificato, ovvero la correttezza delle informazioni in esso contenute deve essere confermata.

L'azione di verifica dei documenti viene effettuata dalla persona che ricopre, in quel periodo, il ruolo di Verificatore.

\label{sec:Verifica}
\trisubsection{Processo di verifica tramite \textit{pull request$_G$}}
Il processo di verifica dei documenti adotta il meccanismo delle \textit{pull request} offerto dalla piattaforma \textit{GitHub}.
Tale strumento garantisce che nessuna modifica venga integrata nel \textit{branch} principale (\texttt{main}) o nel branch di sviluppo principale (\texttt{develop}) senza un controllo esplicito del verificatore,
il quale può approvare il documento o richiedere ulteriori modifiche direttamente all'interno della sezione \textit{pull request} di \textit{GitHub} tramite i commenti. \\
L'intero ciclo di verifica, incluse eventuali correzioni successive, avviene all'interno di un'unica \textit{pull request}, 
garantendo la tracciabilità delle discussioni e dei \textit{commit} di modifica. \\

Di seguito viene definito il \textit{workflow} operativo:

\begin{enumerate}
    \item \textbf{Esecuzione modifiche} \\
    Il redattore crea un nuovo branch denominato \texttt{modifica-<nome\_documento>} nel suo repository locale (la versione del documento non va inclusa all'interno del nome del branch di modifica).\\
    Per definizione, le modifiche su documenti diversi dovranno essere presentate attraverso branch di modifica e pull request diverse.

    \item \textbf{Apertura della Richiesta} \\
    Il redattore, completato il lavoro in locale sul \textit{branch} dedicato, effettua il \textit{push} in remoto del branch di modifica e apre una nuova \textit{pull request} 
    verso il branch \texttt{develop}. In questa fase deve:
    \begin{itemize}
        \item definire come assegnatario e reviewer il verificatore e impostare i \textit{tag};
        \item collegare la richiesta alla relativa \textit{Issue} di progetto.
    \end{itemize}

    \item \textbf{Verifica} \\
    Il verificatore riceve la notifica ed esamina le modifiche. A questo punto si presentano due scenari:
    \begin{enumerate}
        \item \textbf{Esito Positivo:} Se il lavoro è corretto e completo, il verificatore approva la \textit{pull request} 
        e procede al \textit{merge} automatico del branch di modifica nel branch di \texttt{develop}. L'operazione porta all'eliminazione 
        automatica del branch di modifica previa attivazione dell'impostazione dedicata di GitHub (Settings$>$General$>$Automatically delete head branches)\\
        Il processo di verifica è concluso.\\
        Spetta all'autore eliminare manualmente il branch di modifica nel suo repository locale.
        \item \textbf{Esito Negativo:} Se sono necessari cambiamenti, il verificatore inserisce commenti \textit{inline} 
        sui punti critici e imposta lo stato della revisione su \textit{Request changes}.\\
    \end{enumerate}

    \item \textbf{Risoluzione e Nuova Verifica}\\
    Nel caso siano state richieste modifiche, il redattore:
    \begin{enumerate}
        \item viene notificato;
        \item applica le correzioni in locale ed effettua un nuovo \textit{push} sullo stesso branch 
        (aggiornando automaticamente la \textit{pull request} esistente);
        \item risponde ai commenti o li risolve e richiede una nuova revisione (\textit{Re-request review});
        \item in caso di esito di verifica negativo itera il punto 4.
    \end{enumerate}
\end{enumerate}

\subsubsection{Approvazione dei documenti}
Ogni documento, dopo essere stato verificato, deve essere approvato per poter essere rilasciato come documentazione ufficiale del progetto 
nel branch \texttt{main}. L'approvazione avviene in maniera simile alla verifica da chi, in quel periodo, ricopre la figura di \textit{responsabile$_{G}$}.

I documenti vengono approvati in maniera progressiva al fine di evitare l'accumulo di un carico di lavoro insostenibile per il responsabile
quando si raggiunge una baseline. Una volta che tutti i documenti contenuti nel branch \texttt{develop} sono stati approvati si può procedere
con il \textit{merge} del branch \texttt{develop} verso il branch \texttt{main}.

Il momento opportuno per sottoporre un documento ad approvazione non è a seguito della verifica ma è definito in base alle tempistiche imposte dalle baselines.\\
L'approvazione di un singolo documento consiste nei seguenti passi:
\begin{enumerate}
    \item \textbf{Aggiornamento versione}\\
    Il redattore crea un branch locale denominato \texttt{approvazione-<nome\_documento>} (la versione del documento non va inclusa nel nome) 
    in cui modifica \textbf{esclusivamente} la versione aumentando l'indice \textit{Major} e azzerando gli altri (X.0.0).
    \item \textbf{Apertura Richiesta}\\
    Viene effettuata la \textit{push} e aperta una \textit{pull request} verso il branch \texttt{develop}. Devono essere impostati i \textit{tag}, l'assegnatario e il \textit{reviewer} devono coincidere con il responsabile.
    \item \textbf{Approvazione}\\
    Il responsabile viene notificato, controlla il documento e fornisce un verdetto:
    \begin{enumerate}
        \item \textbf{Esito Positivo:} il documento è approvato con successo, il responsabile approva la \textit{pull request} e il 
        \textit{merge} sul branch \texttt{develop}.\\
        Spetta al redattore eliminare manualmente il branch di approvazione in locale.
        \item \textbf{Esito Negativo:} il documento non è adatto e richiede variazioni che vengono comunicate come commenti, lo stato
        revisione viene impostato a \textit{Request Changes}.
    \end{enumerate}
    \item \textbf{Risoluzione e Nuova Approvazione}\\
    In caso di esito di approvazione negativo, il redattore:
    \begin{enumerate}
        \item viene notificato;
        \item applica le modifiche e ripete il processo di verifica (\hyperref[sec:Verifica]{Sezione \textit{§3.1.4.1}});
        \item rinnova la \textit{pull request} esistente per l'approvazione;
        \item in caso di esito negativo itera il punto 4.
    \end{enumerate}

\end{enumerate}

\subsubsection{Nomenclatura dei documenti}
\trisubsection{Acronimi}
Nella nomenclatura e all'interno dei documenti sono utilizzati i seguenti acronimi

\begin{center}
    \begin{tabular}{|c|c|}
        \hline
        \rowcolor{lightgray} \textbf{Abbreviazione} & \textbf{Significato} \\
        \hline
            VI & Verbale interno \\
        \hline
            VE & Verbale esterno \\
        \hline
            NdP & Norme di Progetto \\
        \hline
            PB & Product Baseline \\
        \hline
            RTB & Requirements and Technology Baseline\\
        \hline
    \end{tabular}
\end{center}

\trisubsection{Convenzione}
La convenzione in uso per la nomenclatura dei file è la seguente (dove X, Y e Z sono le versione descritte nella sezione \hyperref[sec:Versionamento]{\textit{§3.2.1}})
\begin{itemize}
    \item \textbf{Verbali}: vi\_AAAA\_MM\_GG\_vX.Y.Z    
    \item \textbf{Generale}: \texttt{<ACR.DOC.>}\_vX.Y.Z (dove \texttt{<ACR.DOC.>} è l'acronimo del documento)
\end{itemize}
\subsection{Gestione della configurazione}
La gestione della configurazione è l'insieme delle attività volte a identificare, tracciare e controllare le modifiche apportate a qualsiasi elemento nel progetto.

Secondo lo Standard ISO/IEC 12207:1995, le azioni di gestione della configurazione devono controllare le modifiche e i rilasci degli elementi, registrare e segnalare lo stato degli elementi e
le richieste di modifica, garantire la completezza, la coerenza e la correttezza degli elementi, e controllare
l'immagazzinamento, la movimentazione e la consegna degli elementi.

\label{sec:Versionamento}
\subsubsection{Numeri di versione documenti}
La convenzione per il numero di versione dei documenti è il formato X.Y.Z (\textit{MAJOR}.\textit{MINOR}.\textit{PATCH}) ogni documento parte dalla prima versione 0.1.0
\begin{itemize}
    \item Ogni modifica minore (\textit{patch version}), come correzione di errori grammaticali o aggiunta d'informazioni meno significative, fa avanzare Z di una unità;
    \item Ogni modifica maggiore (\textit{minor version}), come aggiunta di sezioni o modifiche sostanziali fa avanzare Y di una unità;
    \item La cifra X avanza solamente quando l'approvazione da parte del responsabile termina con successo (\textit{major version}).
\end{itemize}
Ogni avanzamento delle cifre X e Y riportano le cifre alla loro destra a 0.\\ \\
%Al termine del processo di verifica, se questo termina con esito positivo, le modifiche apportate vengono confermate. In caso negativo la versione non viene confermata e si deve procedere con ulteriori modifiche per soddisfare il processo di verifica
%draft del verificatore
\subsubsection{Repository}
I repository creati per la gestione della configurazione sono
\begin{itemize}
    %\item \textbf{Documents}: repository che contiene tutti la documentazione creata, in formato PDF;
    %\item \textbf{Sorgenti}: repository che contiene tutta la documentazione create, in formato \LaTeX;
    \item \textbf{MVP}: repository contenente il Minimum Viable Product del progetto;
    \item \textbf{snakebyte.github.io}: repository contenente il codice sorgente del sito web dedicato alla\\ documentazione del progetto e la documentazione stessa.
\end{itemize}

\trisubsection{Strumenti}
Gli strumenti adottati per la gestione della configurazione e del versionamento dei file di progetto sono:
\begin{itemize}
    \item \textbf{Git}$_{G}$: Version Control System distribuito e OpenSource
    \item \textbf{Github}$_{G}$: servizio di hosting per progetti software e implementazione di Git.
\end{itemize}


\section{Processi organizzativi}
\subsection{Processo di gestione}
Il processo di gestione contiene le attività e i compiti generici che possono essere impiegati da qualsiasi
ruolo che debba gestire i rispettivi processi. 

\subsubsection{Pianificazione}

\trisubsection{Assegnazione delle responsabilità}
Per tutta la durata del progetto, per ragioni formative, i membri del gruppo ricopriranno a rotazione 6 ruoli fondamentali nel campo dell'ingegneria del software. Ogni componente del gruppo \textit{SnakeByte} dovrà ricoprire almeno una volta tutti i seguenti ruoli:
\begin{itemize}
    \item Responsabile (\hyperref[sec:Responsabile]{Sezione \textit{§4.1.1.2}})
    \item Amministratore (\hyperref[sec:Amministratore]{Sezione \textit{§4.1.1.3}})
    \item Progettista (\hyperref[sec:Progettista]{Sezione \textit{§4.1.1.4}})
    \item Analista (\hyperref[sec:Analista]{Sezione \textit{§4.1.1.5}})
    \item Verificatore (\hyperref[sec:Verificatore]{Sezione \textit{§4.1.1.6}})
    \item Programmatore (\hyperref[sec:Programmatore]{Sezione \textit{§4.1.1.7}})
\end{itemize}
La rotazione dei ruoli avviene tra la fine di un periodo (\textit{sprint$_{G}$}) e l'inizio di quello successivo.

\label{sec:Responsabile}
\trisubsection{Responsabile}
Il responsabile (\textit{Project Manager}) governa il \textit{team} e rappresenta il progetto verso l'esterno. Deve avere conoscenze e capacità tecniche per valutare rischi, scelte e alternative.

I compiti principali del responsabile sono
\begin{itemize}
    \item Prendere scelte nell'interesse del gruppo;
    \item Approvare il lavoro realizzato dagli altri ruoli;
    \item Pianificare le attività e gestire le risorse necessarie;
    \item Coordinare e relazionarsi con l'esterno 
\end{itemize}
Le persone incaricate come responsabile devono essere una per periodo.

\label{sec:Amministratore}
\trisubsection{Amministratore}
L'amministratore di sistema (\textit{Sysadmin}) definisce, controlla e manutiene l'ambiente informatico di lavoro. 
\begin{itemize}
    \item Definisce, seleziona e mette in opera le risorse informatiche a supporto delle \textit{Norme di Progetto};
    \item Gestisce le segnalazioni (\textit{ticket}) sul non funzionamento dell'infrastruttura.
\end{itemize}
Le persone incaricate come amministratore devono essere una per periodo.

\label{sec:Progettista}
\trisubsection{Progettista}
Il progettista ha competenze tecniche e tecnologiche aggiornate e per questo ha il ruolo di determinare le scelte realizzative del prodotto. Segue il progetto durante la fase di sviluppo software ma non durante la sua manutenzione. \\

I principali compiti del progettista sono
\begin{itemize}
    \item Ideare l'architettura del prodotto in modo da ottenere la migliore efficienza ed efficacia possibile;
    \item Supervisionare la fase di codifica supportando i programmatori.
\end{itemize}
La realizzazione del progetto a regola d'arte richiede la presenza di più progettisti, i quali devono aver la possibilità di confrontarsi tra di loro in modo da prendere le scelte attuative in maniera critica.

\label{sec:Analista}
\trisubsection{Analista}
L'analista conosce il dominio del problema ed ha competenze avanzate sull'analisi dei requisiti. Ha molta influenza sul successo del prodotto. Non segue il progetto fino alla consegna, ma il suo intervento è richiesto solo nei momenti di bisogno.

I suoi compiti principali sono
\begin{itemize}
    \item Analizzare i requisiti e il dominio applicativo;
    \item Definire fattibilità e obiettivi delle attività;
    \item Redigere il documento \textit{Analisi dei requisiti}.
\end{itemize}
Più membri possono essere occupati in questo ruolo contemporaneamente.

\label{sec:Verificatore}
\trisubsection{Verificatore}
Il compito principale del verificatore è analizzare la conformità agli obiettivi del lavoro svolto dagli altri ruoli presenti nel progetto. Ha competenze tecniche e profonda conoscenza delle \textit{Norme di Progetto}.

I principali compiti del verificatore sono
\begin{itemize}
    \item Verificare che gli artefatti software e di documentazione siano aderenti alle \textit{Norme di Progetto} e agli obiettivi;
    \item Segnalare eventuali errori da correggere;
    \item Redigere test di verifica e di aderenza ai requisiti per i prodotti software.
\end{itemize}
Più membri possono essere occupati in questo ruolo contemporaneamente.

\label{sec:Programmatore}
\trisubsection{Programmatore}
Il programmatore è la figura responsabile della realizzazione e manutenzione del prodotto software. Ha competenze tecniche ampie ma deleghe limitate.

I principali compiti del programmatore sono
\begin{itemize}
    \item Scrivere codice secondo le specifiche del \textit{progettista} e che persegue gli obiettivi e requisiti definiti dall'\textit{analista};
    \item Redigere la documentazione tecnica (\textit{Manuale}) del prodotto. 
\end{itemize}

\'E il ruolo in cui vengono occupati più membri contemporaneamente.


\subsubsection{Coordinamento}

\trisubsection{Riunione interna fissata}
Il gruppo \textit{SnakeByte} ha scelto di fissare un giorno all'interno della settimana lavorativa in cui si svolge una riunione in presenza a cui prendono parte i componenti del gruppo. La partecipazione dei membri è richiesta ma non tassativa. Inoltre nel caso di impossibilità della maggior parte del gruppo la riunione viene annullata.

Il giorno attuale in cui si tiene la riunione è il \textit{Lunedì} alle ore \textit{12.15} circa.

Altre eventuali riunioni possono essere fissate tramite confronto e accordo tra i componenti del gruppo su giorno e ora.

\trisubsection{Comunicazioni interne}
Per le comunicazioni interne, il gruppo \textit{SnakeByte} ha individuato come mezzo per la trasmissione di informazioni in formato sincrono e asincrono la piattaforma \textit{\textit{Discord$_{G}$}}.

\trisubsection{Comunicazioni esterne}
Per le comunicazioni esterne, il gruppo \textit{SnakeByte} ha individuato come mezzo per la trasmissione di informazioni in formato asincrono i messaggi email, esclusivamente tramite l'indirizzo ufficiale del gruppo (\textit{snakebyteteam@gmail.com}). Mentre per le comunicazioni sincrone l'applicazione di videoconferenze \textit{Google Meet}. 

%Scrivere di , google sheets, validazione dei documenti.

\end{document}
