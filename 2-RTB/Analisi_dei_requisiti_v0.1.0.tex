\documentclass[10pt, letterpaper]{article}
\usepackage[nomarginpar, margin=2.75cm, tmargin=3cm, bmargin=1.75cm]{geometry}
\usepackage[
    colorlinks=true,      
    linkcolor=black,      
    urlcolor=blue,       
    citecolor=black       
]{hyperref}
\usepackage{graphicx}
\usepackage[table, x11names]{xcolor}
\usepackage{tabularx}
\usepackage{lastpage} 
\renewcommand{\arraystretch}{1.2} % migliora la leggibilità
\renewcommand{\contentsname}{Indice}
\usepackage{fancyhdr}
\pagestyle{fancy}
\fancyhf{}
\fancyhead[L]{SnakeByte} 
\fancyhead[R]{Analisi dei Requisiti}
\fancyfoot[C]{Pagina \thepage\ di \pageref{LastPage}}


\begin{document}

\begin{titlepage}
    \begin{center}
        \begin{center}
            \includegraphics[width=0.6\textwidth]{./img/logo.pdf}
        \end{center}
        \vspace{4cm}
        \huge\textbf{Analisi dei requisiti}\par
        \vspace{2cm}
        \large \textbf{SnakeByte} (Gruppo 1):\\
        \large Valeria Baleanu, Leonardo Pellizzon, Filippo Venzo, Giuseppe De Fina, \\
         Francesco Pasqual, Christian Libralato, Luca Granziero \\
        (2109911, 2111006, 2113705, 2113187, 2103119, 2101047, 2075512)
        \vfill
        \small
        \begin{center}
            \begin{tabular}{|c|c|c|c|}
                \hline
                \multicolumn{4}{|c|}{\textbf{Informazioni documento}} \\
                \hline
                \rowcolor{lightgray} \textbf{Versione} & \textbf{Data} & \textbf{Stato} & \textbf{Destinatari} \\
                \hline
                0.1.0 & 05/10/2025 & stato & prof. Vardanega Tullio, prof. Cardin Riccardo, Vimar S.p.A. \\
                \hline
            \end{tabular}
        \end{center}
        \vfill
        \large Contatti: snakebyteteam@gmail.com
    \end{center}
\end{titlepage}

\begin{center}
    \begin{tabularx}{\textwidth}{|c|c|c|c|c|X|}
        \hline
        \multicolumn{6}{|c|}{\textbf{Registro delle modifiche}} \\
        \hline
        \rowcolor{lightgray} \textbf{Versione} & \textbf{Data} & \textbf{Autore} & \textbf{Verificatore} & \textbf{Approvatore} & \textbf{Descrizione} \\
        \hline
        0.1.0 & 05/10/2025 & L. Granziero & - & - & Prima Stesura \\
        \hline
    \end{tabularx}
\end{center}

\newpage

\tableofcontents

\newpage
\section{Introduzione}
\subsection{Scopo del documento}{
Il documento di analisi dei requisiti, in un contesto di ingegneria del software, ha lo scopo fondamentale di tradurre l'esigenza dell'utenza e degli stakeholder, in questo caso la proponente Vimar S.p.A., in una specifica completa, coerente e verificabile di requisiti, destinata a guidare le fasi di progettazione e sviluppo, verifica e infine validazione del sistema.
Andiamo a vedere quindi più nello specifico gli obbiettivi principali.
\\
\noindent
\\
\textbf{Definire chiaramente “cosa” il sistema deve fare e “in quali condizioni”:}
\begin{itemize}
    \item L’analisi dei requisiti serve a identificare le funzioni (requisiti funzionali) e le qualità (requisiti non funzionali: prestazioni, usabilità, affidabilità, portabilità...) del software.
    \item Definisce i limiti del sistema e i vincoli (tecnici, di interfaccia...)
    \item Permette di evitare ambiguità o fraintendimenti sulle funzionalità richieste. 
\end{itemize}

\noindent
\textbf{Allineare tutti gli stakeholder su un linguaggio comune e condividere le aspettative:}
\begin{itemize}
    \item Questo documento funge da contratto tra cliente/committente e il team di sviluppo: specifica ciò che sarà consegnato.
    \item Aiuta a garantire che utenti, committenti, analisti, progettisti e tester abbiano la stessa comprensione del sistema.
\end{itemize}

\noindent
\textbf{Fornire una base stabile per le fasi successive del ciclo di vita del software:}
\begin{itemize}
    \item Il documento di analisi dei requisiti serve come input per la progettazione del sistema, per la pianificazione dello sviluppo e per la pianificazione dei test.
    \item Serve anche come base per la verifica e la convalida: si può usare come riferimento per capire se il prodotto finale soddisfa i requisiti richiesti. 
\end{itemize}
\noindent
\textbf{Gestire i rischi e controllare le modifiche:}
\begin{itemize}
    \item Durante l’analisi dei requisiti si identificano requisiti non realizzabili, conflitti tra requisiti, omissioni e incoerenze. Ciò consente di ridurre i rischi fin dalle prime fasi.
    \item Aiuta a limitare il fenomeno dello “scope creep” (ovvero l’aggiunta non controllata di funzionalità) e a mantenere il controllo sul cambiamento dei requisiti. 
\end{itemize}
}
\subsection{Scopo del progetto}{
Lo scopo del progetto è l'implementazione di un sistema domotico integrato per anziani autosufficienti che si basa su dispositivi Vimar con connessione mesh Bluetooth.
L'obbiettivo è la miglioria del confort e della sicurezza delle persone occupanti, l'aumento dell'efficienza energetica della struttura e la semplificazione della gestione operativa dell'impianto elettrico.
Tali risultati vengono raggiunti attraverso la centralizzazione del controllo di illuminazione, temperatura, televisione e dispositivi di sicurezza mendiante l'app View e i relativi servizi cloud, con la possibilità di controllo da remoto da parte del personale medico.
}

\subsection{Sviluppo del documento}{
Il presente documento è stato sviluppato in modo graduale e incrementale, con lo scopo di facilitare modifiche future in base alle esigenze che verranno concordate tra il gruppo e l'azienda committente. Il documento è quindi soggetto a un processo di miglioramento continuo nel tempo.
}

\section{Relazione del prodotto}
\subsection{Obbiettivo del prodotto}{
Il progetto consiste nella realizzazione di una piattaforma unica \textit{View4Life} per la gestione intelligente degli impianti \textit{Smart} nelle residenze protette per anziani, sfruttando i dispositivi domotici Vimar connessi in rete \textit{mesh Bluetooth} tramite l’\textit{API KNX IoT 3rd-party$_{G}$}. Questa soluzione mira a supportare il lavoro del personale sanitario fornendo uno strumento che integri un sistema di gestione degli allarmi (come il rilevamento di cadute o presenze prolungate in determinate stanze) per garantire un intervento rapido e tempestivo. Inoltre, la piattaforma è progettata per permettere il monitoraggio del consumo energetico e la rilevazione di anomalie nell’impianto.
}
\subsection{Funzionalità del prodotto}{
La funzionalità del prodotto è quella di fornire un’interfaccia unica e intuitiva per il controllo, il monitoraggio e l’automazione dei dispositivi domotici della struttura, migliorando la sicurezza, il comfort e l’efficienza energetica.
}
\subsection{Utenza di riferimento}{
Il prodotto si rivolge principalmente a quattro categorie principali di utenti, descritte di seguito:
\noindent
\begin{itemize}
    \item \textbf{Personale medico e operatori sanitari} che utilizzano l'applicazione per monitorare lo stato di ambienti e dispositivi, oltre a ricevere notifiche di allarme e gestire da remoto funzioni come temperatura, illuminazione e sicurezza delle stanze degli ospiti presenti. 
    \item \textbf{Ospiti della struttura (anziani autosufficienti)} che permette loro di interagire in modo semplice e instintivo con l'impiando domotico per la regolazione di luci, tapparelle e temperatura, migliorando il confort.
    \item \textbf{Personale tecnico e amministrativo} che si fa a carico della configurazione e manutenzione del sistema e del monitoraggio dei consumi elettrici, e quindi a una conseguente ottimizzazione dell'efficienza dell'impianto elettrico.
    \item \textbf{Team di progetto} il quale è responsabile di progettazione, sviluppo e integrazione della soluzione domotica attraverso i dispositivi Vimar e relativi servizi cloud, impegnandosi a garantire sicurezza e affidibilità del sistema.
\end{itemize}
Questa sezione evidenzia come il sistema domotico proposto miri a centralizzare il controllo e la gestione degli impianti all’interno della residenza, semplificando le operazioni quotidiane del personale e migliorando la qualità della vita degli ospiti.
Attraverso l’app View e i servizi cloud Vimar, l’applicazione permette di gestire in modo integrato illuminazione, temperatura e sicurezza, contribuendo a maggior efficienza energetica, sicurezza e comfort abitativo per tutti gli utenti coinvolti.
}
\section{Tecnologie utilizzate}
Il progetto nel suo insieme delle parti di front-end e back-end richiede l'utilizzo di molteplici tecnologie che il team ha dovuto in un primo luogo imparare ad usare. Queste sono state scelte tenendo conto delle esigenze del proponente e del progetto stesso, al fine di garantire un prodotto di qualità soddisfacente.\\
\noindent
\\
Come linguaggio di programmazione di è deciso quindi di usare:\\
\noindent
\\
Per la parte del back-end si è deciso di utilizzare: \\
\noindent
\\
Per la parte del front-end si è deciso di utilizzare: 


\subsection{Riferimenti}

\end{document}