\documentclass[10pt, letterpaper]{article}
\usepackage[nomarginpar, margin=2.75cm, tmargin=3cm, bmargin=1.75cm]{geometry}
\usepackage[
    colorlinks=true,      
    linkcolor=black,      
    urlcolor=blue,       
    citecolor=black       
]{hyperref}
\usepackage{template}
\usepackage{float}
\usepackage{graphicx}
\usepackage{caption}
\usepackage{lastpage} 
\renewcommand{\arraystretch}{1.05} % migliora la leggibilità
\renewcommand{\contentsname}{Indice}
\renewcommand{\listfigurename}{Elenco dei grafici}
\usepackage{fancyhdr}
%Comandi per livello di sottosezioni = 3
\setcounter{tocdepth}{4}
\setcounter{secnumdepth}{4}
\newcommand{\trisubsection}[1]{\paragraph{#1}\mbox{}\\}

\pagestyle{fancy}
\fancyhf{}
\fancyhead[L]{SnakeByte}
\fancyhead[R]{Piano di qualifica}
\fancyfoot[C]{Pagina \thepage\ di \pageref{LastPage}}

\begin{document}
\setcounter{tocdepth}{5}
\setcounter{secnumdepth}{5}

\begin{titlepage}
    \begin{center}
        \begin{center}
            \includegraphics[width=0.6\textwidth]{../../template/img/logo.pdf}
        \end{center}
        \vspace{4cm}
        \huge\textbf{Piano di qualifica}\par
        \vspace{2cm}
        \large \textbf{SnakeByte} (Gruppo 1):\\
        \large Valeria Baleanu, Leonardo Pellizzon, Filippo Venzo, Giuseppe De Fina, \\
         Francesco Pasqual, Christian Libralato, Luca Granziero \\
        (2109911, 2111006, 2113705, 2113187, 2103119, 2101047, 2075512)
        \vfill
        \small
        \begin{center}
            \begin{tabular}{|c|c|c|c|}
                \hline
                \multicolumn{4}{|c|}{\textbf{Informazioni documento}} \\
                \hline
                \rowcolor{lightgray} \textbf{Versione} & \textbf{Data} & \textbf{Stato} & \textbf{Destinatari} \\
                \hline
                0.2.1 & 4/02/2026 & Da verificare &       
                \begin{tabular}[c]{@{}l@{}}
                    \textbf{Interni:} SnakeByte \\
                    \textbf{Esterni:} prof. Vardanega Tullio, prof. Cardin Riccardo, Vimar
                \end{tabular} \\
                \hline
            \end{tabular}
        \end{center}
        \vfill
        \large Contatti: snakebyteteam@gmail.com
    \end{center}
\end{titlepage}

\begin{center}
    \begin{tabularx}{\textwidth}{|c|c|c|c|c|X|}
        \hline
        \multicolumn{6}{|c|}{\textbf{Registro delle modifiche}} \\
        \hline
        \rowcolor{lightgray} \textbf{Versione} & \textbf{Data} & \textbf{Autore} & \textbf{Verificatore} & \textbf{Approvatore} & \textbf{Descrizione} \\
        0.2.1 & 4/02/2026 & F. Venzo & - & - & Aggiunta descrizione ai grafici, campagna di test di unità e campagna di test di integrazione\\
        \hline
        0.2.0 & 28/12/2025 & V. Baleanu & F. Pasqual & - & Aggiunta tabelle delle soglie per le metriche di processo e di prodotto \\
        \hline
        0.1.1 & 27/12/2025 & V. Baleanu & F. Pasqual & - & Aggiunta Introduzione \\
        \hline
        0.1.0 & 22/12/2025 & V. Baleanu & F. Pasqual & - & Prima Stesura \\
        \hline
    \end{tabularx}
\end{center}

\newpage

\tableofcontents

\newpage
\listoffigures

\newpage

\section{Introduzione}
\subsection{Finalità del documento}
Il presente documento ha lo scopo di definire e documentare gli obiettivi di qualità 
che il gruppo \textit{SnakeByte} si impegna a perseguire durante l'intero ciclo di vita del 
progetto. In particolare, il Piano di qualifica${_G}$ comprende i seguenti elementi:
\begin{itemize}
    \item \textbf{Piano della Qualità}: tutte le attività volte a definire gli obiettivi 
    qualitativi e a stabilire i processi e le risorse necessarie per il loro conseguimento;
    \item \textbf{Controllo di Qualità}: le attività finalizzate ad accertare la conformità dei processi 
    e dei prodotti agli standard, ai requisiti e alle metriche definite;
    \item \textbf{Miglioramento continuo}: le azioni intraprese per analizzare i risultati 
    del controllo della qualità e per introdurre interventi correttivi e migliorativi 
    sui processi e sul prodotto.
\end{itemize}
Il documento si rivolge sia ai membri del gruppo, come guida per il mantenimento 
degli standard qualitativi, sia ai committenti e al proponente, come dimostrazione 
dell'impegno del gruppo verso la qualità del prodotto finale.
\subsection{Glossario}
Il documento cita alcuni termini la cui definizione può risultare ambigua. Per questo è possibile consultare
il glossario${_G}$ il quale contiene le definizioni di tali espressioni, che saranno marcate da una lettera G a
pedice.
\subsection{Riferimenti}
\subsubsection{Riferimenti Normativi}
\begin{itemize}
    \item \textbf{Norme di Progetto}: 
    \\ \url{https://github.com/SnakeByteTeam/snakebyteteam.github.io/blob/develop/2-RTB/Norme_di_progetto/NdP_v0.2.0.pdf}
    \\ (consultato il 22/12/2025).
    \item \textbf{Vimar View4Life Capitolato di Ingegneria del Software Università di Padova 2025-2026}:
    \\ \url{https://www.math.unipd.it/~tullio/IS-1/2025/Progetto/C9.pdf}
    \\ (consultato il 22/12/2025).
\end{itemize}
\subsubsection{Riferimenti Informativi}
\begin{itemize}
    \item \textbf{Glossario}: 
    \\ \url{https://snakebyteteam.github.io/glossary.html}
    \\ (consultato il 22/12/2025).
    \item \textbf{Standard ISO 9000:2015 - Quality management systems. Fundamentals and vocabulary}:
    \\ \url{https://www.iso.org/standard/45481.html}
    \\ (consultato il 23/12/2025).
    \item \textbf{Standard ISO/IEC 12207:1995}: 
    \\ \url{https://www.math.unipd.it/~tullio/IS-1/2009/Approfondimenti/ISO_12207-1995.pdf}
    \\ (consultato il 27/12/2025).
    \item \textbf{Standard ISO/IEC 9126:2001}:
    \\ \url{https://it.wikipedia.org/wiki/ISO/IEC_9126}
    \\ (consultato il 28/12/2025).
\end{itemize}

\section{Qualità di processo}
La qualità di processo${_G}$ ha lo scopo di garantire che le attività di sviluppo
del software siano pianificate, eseguite, monitorate e migliorate in modo
sistematico, al fine di ridurre il rischio di errori o incongruenze 
durante l'intero ciclo di vita del progetto.

Per la definizione e la classificazione dei processi di qualità, il progetto
fa riferimento allo standard ISO/IEC 12207:1995, che descrive i processi 
del ciclo di vita del software e li suddivide nelle seguenti categorie:
\begin{itemize}
    \item \textbf{Processi primari}: sono direttamente coinvolti nella
    realizzazione, fornitura e mantenimento del prodotto \textit{software};
    \item \textbf{Processi di supporto}: affiancano i processi primari fornendo 
    attività di supporto come la documentazione e la verifica;
    \item \textbf{Processi organizzativi}: forniscono le risorse necessarie 
    all'esecuzione dei processi primari e di supporto, favorendo il 
    miglioramento continuo dell'organizzazione.
\end{itemize}
\subsection{Processi primari}
\subsubsection{Fornitura}
Nella tabella che segue il termine $BAC$ (\textit{Budget At Completion}) 
rappresenta il budget totale previsto inizialmente per il completamento 
del progetto.
\sogliatabella{
    \sogliariga{MPC1}{Planned Value}{$\geq 0$}{$\leq BAC$}
    \sogliariga{MPC2}{Earned Value}{$\geq 0$}{$\leq EAC$}
    \sogliariga{MPC3}{Actual Cost}{$\geq 0$}{$\leq EAC$}
    \sogliariga{MPC4}{Cost Performance Index}{$\geq 0.80$}{$1.00$}
    \sogliariga{MPC5}{Schedule Performance Index}{$\geq 0.80$}{$1.00$}
    \sogliariga{MPC6}{Estimate At Completion (EAC)}{$\leq 120\% \cdot BAC$}{$\leq BAC$}
    \sogliariga{MPC7}{Estimate To Complete}{$\geq 0$}{$\leq BAC$}
}

\subsubsection{Sviluppo}
\sogliatabella{
    \sogliariga{MPC8}{Requirements Stability Index}{$\geq 80\%$}{$100\%$}
}

\subsection{Processi di supporto}
\subsubsection{Documentazione}
\sogliatabella{
    \sogliariga{MPC9}{Indice di Gulpease}{$\geq 60$}{$\geq 80$}
    \sogliariga{MPC10}{Indice di Frammentazione (IF)}{$5\% \leq IF \leq 20\%$}{$10\% \leq IF \leq 15\%$}
}

\subsubsection{Verifica}
\sogliatabella{
    \sogliariga{MPC11}{Test Success Rate}{$\geq 80\%$}{$100\%$}
}

\subsubsection{Gestione della qualità}
\sogliatabella{
    \sogliariga{MPC12}{Percentuale Metriche Soddisfatte}{$\geq 80\%$}{$100\%$}
}

\subsection{Processi organizzativi}
\subsubsection{Gestione dei processi}
\sogliatabella{
    \sogliariga{MPC13}{Percentuale Rischi Inattesi}{$\leq 20\%$}{$0\%$}
    \sogliariga{MPC14}{Labor Efficiency}{$\geq 70\%$}{$\geq 100\%$}
}

\section{Qualità di prodotto}
La qualità di prodotto$_G$ è definita secondo un modello basato sullo standard 
ISO/IEC 9126, che struttura la qualità in caratteristiche e sottocaratteristiche 
misurabili attraverso metriche specifiche.
\\Per il prodotto \textit{software} sviluppato, il gruppo \textit{SnakeByte} 
ha individuato le seguenti caratteristiche qualitative da monitorare e garantire:
\begin{itemize}
    \item \textbf{Funzionalità}: capacità di fornire funzioni che soddisfino 
    i requisiti definiti;
    \item \textbf{Affidabilità}: capacità di mantenere il livello di prestazione 
    richiesto quando utilizzato in condizioni specifiche;
    \item \textbf{Usabilità}: facilità con cui gli utenti possono apprendere, 
    comprendere e utilizzare il prodotto;
    \item \textbf{Efficienza}: capacità di fornire prestazioni adeguate in 
    relazione alle risorse utilizzate;
    \item \textbf{Manutenibilità}: facilità con cui il prodotto può essere 
    modificato, corretto o esteso, favorendo interventi di manutenzione;
    \item \textbf{Portabilità}: capacità di essere trasferito e adattato a 
    diversi ambienti operativi, piattaforme o configurazioni hardware e software.
\end{itemize}
\subsection{Funzionalità}
\sogliatabella{
    \sogliariga{MPD1}{Percentuale Requisiti Obbligatori Soddisfatti}{$100\%$}{$100\%$}
    \sogliariga{MPD2}{Percentuale Requisiti Opzionali Soddisfatti}{$0\%$}{$100\%$}
    \sogliariga{MPD3}{Percentuale Requisiti Desiderabili Soddisfatti}{$0\%$}{$100\%$}
}

\subsection{Affidabilità}
\sogliatabella{
    \sogliariga{MPD4}{Line Coverage}{$\geq 75\%$}{$100\%$}
    \sogliariga{MPD5}{Branch Coverage}{$\geq 75\%$}{$100\%$}
    \sogliariga{MPD6}{Statement Coverage}{$\geq 75\%$}{$100\%$}
}

\subsection{Usabilità}
\sogliatabella{
    \sogliariga{MPD7}{Profondità Massima di Navigazione }{$\leq 7$ click}{$\leq 3$ click}
}

\subsection{Efficienza}
\sogliatabella{
    \sogliariga{MPD8}{Tempo Medio di Risposta}{$\leq 5$ secondi}{$\leq 1$ secondo}
}

\subsection{Manutenibilità}
\sogliatabella{
    \sogliariga{MPD9}{Complessità Ciclomatica}{$\leq 20$}{$\leq 10$}
}

\subsection{Portabilità}
\sogliatabella{
    \sogliariga{MPD10}{Compatibilità Cross-Browser}{$\geq 80\%$}{100\%}
}

\section{Strategie di Testing}
\subsection{Test di sistema}

\addTestS{Verificare che un Utente non autenticato possa effettuare l'autenticazione con il Sistema}{RF1-OB}{NI}
\addTestS{Verificare che l'Utente non autenticato, durante il processo di autenticazione, possa inserire il proprio username}{RF2-OB}{NI}
\addTestS{Verificare che l'Utente non autenticato, durante il processo di autenticazione, possa inserire la propria password}{RF3-OB}{NI}
\addTestS{Verificare che l'Utente non autenticato riceva un messaggio errore a seguito di un tentativo di autenticazione fallito}{RF4-OB}{NI}
\addTestS{Verificare che l'Utente autenticato possa impostare una nuova password}{RF5-OB}{NI}
\addTestS{Verificare che l'Utente non registrato possa essere autenticato con la password temporanea fornita dall'amministratore e impostare una nuova password}{RF6-OB}{NI}
\addTestS{Verificare che l'Utente autenticato con la password temporanea possa impostare una nuova password}{RF7-OB}{NI}
\addTestS{Verificare che l'Utente non registrato riceva un errore a seguito di un tentativo di autenticazione con password temporanea fallito}{RF8-OB RF9-OB}{NI}
\addTestS{Verificare che l'Utente autenticato con la password temporanea riceva un errore nel caso la nuova password sia uguale alla password temporanea}{RF10-OB}{NI}
\addTestS{Verificare che l'Utente autenticato con la password temporanea riceva un errore nel caso la nuova password non rispetta i criteri richiesti}{RF11-OB}{NI}
\addTestS{Verificare che l'amministratore possa visualizzare se un account MyVimar è configuarato nel Sistema}{RF12-OB}{NI}
\addTestS{Verificare che l'amministratore possa collegare un account MyVimar al Sistema}{RF13-OB}{NI}
\addTestS{Verificare che l'amministratore possa rimuovere un account MyVimar dal Sistema}{RF14-OB}{NI}
\addTestS{Verificare che l'amministratore possa visualizzare l'elenco degli utenti del Sistema}{RF15-OB}{NI}
\addTestS{Verificare che l'amministratore, visualizzando l'elenco degli utenti del Sistema, possa visualizzare un utente del Sistema nel dettaglio}{RF16-OB}{NI}
\addTestS{Verificare che l'amministratore, visualizzando un utente del Sistema nel dettaglio, ne visualizzi il nome}{RF17-OB}{NI}
\addTestS{Verificare che l'amministratore, visualizzando un utente del Sistema nel dettaglio, ne visualizzi il cognome}{RF18-OB}{NI}
\addTestS{Verificare che l'amministratore, visualizzando un utente del Sistema nel dettaglio, ne visualizzi l'username}{RF19-OB}{NI}
\addTestS{Verificare che l'amministratore possa creare un utente Operatore Sanitario}{RF20-OB}{NI}
\addTestS{Verificare che l'amministratore, creando un utente Operatore Sanitario, possa inserirne il nome}{RF21-OB}{NI}
\addTestS{Verificare che l'amministratore, creando un utente Operatore Sanitario, possa inserirne il cognome}{RF22-OB}{NI}
\addTestS{Verificare che l'amministratore, creando un utente Operatore Sanitario, possa inserire un username}{RF23-OB}{NI}
\addTestS{Verificare che l'amministratore, creando un utente Operatore Sanitario. possa generare una password temporanea}{RF24-OB}{NI}
\addTestS{Verificare che l'amministratore riceva un errore se l'username utilizzato per la creazione di un utente Operatore Sanitario è già in uso}{RF25-OB}{NI}
\addTestS{Verificare che l'amministratore possa eliminare un utente Operatore Sanitario}{RF26-OB}{NI}
\addTestS{Verificare che l'amministratore possa visualizzare tutti i reparti del Sistema}{RF27-OB}{NI}
\addTestS{Verificare che l'amministratore possa visualizzare un reparto del Sistema nel dettaglio}{RF28-OB}{NI}
\addTestS{Verificare che l'amministratore, visualizzando un reparto del Sistema del dettaglio, possa visualizzarne il nome}{RF29-OB}{NI}
\addTestS{Verificare che l'amministratore possa creare un nuovo reparto}{RF30-OB}{NI}
\addTestS{Verificare che l'amministratore, creando un nuovo reparto, possa inserirne il nome}{RF31-OB}{NI}
\addTestS{Verificare che l'amministratore riceva un errore se il nome inserito per creare il nuovo reparto è già in uso}{RF32-OB}{NI}
\addTestS{Verificare che l'amministratore possa modificare il nome di un reparto del Sistema}{RF33-OB}{NI}
\addTestS{Verificare che l'amministratore possa eliminare un reparto del Sistema}{RF34-OB}{NI}
\addTestS{Verificare che l'amministratore possa assegnare un reparto ad un utente Operatore Sanitario}{RF35-OB}{NI}
\addTestS{Verificare che l'amministratore, assegnando un reparto ad un utente Operatore Sanitario, selezioni un utente Operatore Sanitario}{RF36-OB}{NI}
\addTestS{Verificare che l'amministratore, assegnando un reparto ad un utente Operatore Sanitario, selezioni un reparto del Sistema}{RF37-OB}{NI}
\addTestS{Verificare che l'amministratore possa rimuovere un utente Operatore Sanitario dall'assegnazione ad un reparto}{RF38-OB}{NI}
\addTestS{Verificare che l'amministratore possa assegnare un appartamento ad un reparto del Sistema}{RF39-OB}{NI}
\addTestS{Verificare che l'amministratore, assegnando un appartamento ad un reparto del Sistema, selezioni un appartamento}{RF40-OB}{NI}
\addTestS{Verificare che l'amministratore, assegnando un appartamento ad un reparto del Sistema, selezioni un reparto del Sistema}{RF41-OB}{NI}
\addTestS{Verificare che l'amministratore possa rimuovere un appartamento  dall'assegnazione ad un reparto}{RF42-OB}{NI}


\makeTSTable

\section{Cruscotto di qualità}

\subsection{MPC1 e MPC2 - Planned Value e Earned Value}
\begin{figure}[H]
    \centering
    \includegraphics[width=1\textwidth]{./grafici/MPC1_2.pdf}
    \caption{Andamento PV e EV in base al periodo}
    \label{fig:mpc1-mpc2}
\end{figure}

\subsection{MPC3 e MPC7 - Actual Cost e Estimate To Complete}
\begin{figure}[H]
    \centering
    \includegraphics[width=1\textwidth]{./grafici/MPC3_7.pdf}
    \caption{Andamento AC e ETC in base al periodo}
    \label{fig:mpc1-mpc2}
\end{figure}

\subsection{MPC4 e MPC5 - Cost Performance Index e Schedule Performance Index}
\begin{figure}[H]
    \centering
    \includegraphics[width=1\textwidth]{./grafici/MPC4_5.pdf}
    \caption{Andamento CPI e SPI in base al periodo}
    \label{fig:mpc1-mpc2}
\end{figure}

\subsection{MPC6 - Estimate At Completion}
\begin{figure}[H]
    \centering
    \includegraphics[width=1\textwidth]{./grafici/MPC6_7.pdf}
    \caption{Andamento EAC in base al periodo}
    \label{fig:mpc1-mpc2}
\end{figure}

\subsection{MPC8 - Requirements Stability Index}
\begin{figure}[H]
    \centering
    \includegraphics[width=1\textwidth]{./grafici/MPC8.pdf}
    \caption{Andamento RSI in base al periodo}
    \label{fig:mpc1-mpc2}
\end{figure}

\subsection{MPC9 - Indice di Gulpease}
\begin{figure}[H]
    \centering
    \includegraphics[width=1\textwidth]{./grafici/MPC9.pdf}
    \caption{Andamento IG dei documenti principali, in base al periodo}
    \label{fig:mpc1-mpc2}
\end{figure}

\subsection{MPC10 - Indice di Frammentazione}
\begin{figure}[H]
    \centering
    \includegraphics[width=1\textwidth]{./grafici/MPC10.pdf}
    \caption{Andamento IF dei documenti principali, in base al periodo}
    \label{fig:mpc1-mpc2}
\end{figure}

Spiegare che il Glossario non è incluso perché per come è strutturato è ovviamente frammentato (un termine equivale ad una section).


\subsection{MPC12 - Percentuale Metriche Soddisfatte}
\begin{figure}[H]
    \centering
    \includegraphics[width=1\textwidth]{./grafici/MPC12.pdf}
    \caption{Andamento PMS in base al periodo}
    \label{fig:mpc1-mpc2}
\end{figure}

\subsection{MPC13 - Percentuale Rischi Inattesi}
\begin{figure}[H]
    \centering
    \includegraphics[width=1\textwidth]{./grafici/MPC13.pdf}
    \caption{Andamento PRI in base al periodo}
    \label{fig:mpc1-mpc2}
\end{figure}

\subsection{MPC14 - Labor Efficiency}
\begin{figure}[H]
    \centering
    \includegraphics[width=1\textwidth]{./grafici/MPC14.pdf}
    \caption{Andamento LE in base al periodo}
    \label{fig:mpc1-mpc2}
\end{figure}


\end{document}