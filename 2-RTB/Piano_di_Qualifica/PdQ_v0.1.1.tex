\documentclass[10pt, letterpaper]{article}
\usepackage[nomarginpar, margin=2.75cm, tmargin=3cm, bmargin=1.75cm]{geometry}
\usepackage[
    colorlinks=true,      
    linkcolor=black,      
    urlcolor=blue,       
    citecolor=black       
]{hyperref}
\usepackage{template}
\usepackage{float}
\usepackage{graphicx}
\usepackage{caption}
\usepackage[table, x11names]{xcolor}
\usepackage{lastpage} 
\renewcommand{\arraystretch}{1.05} % migliora la leggibilità
\renewcommand{\contentsname}{Indice}
\usepackage{fancyhdr}
%Comandi per livello di sottosezioni = 3
\setcounter{tocdepth}{4}
\setcounter{secnumdepth}{4}
\newcommand{\trisubsection}[1]{\paragraph{#1}\mbox{}\\}

\pagestyle{fancy}
\fancyhf{}
\fancyhead[L]{SnakeByte}
\fancyhead[R]{Piano di qualifica}
\fancyfoot[C]{Pagina \thepage\ di \pageref{LastPage}}

\begin{document}
\setcounter{tocdepth}{5}
\setcounter{secnumdepth}{5}

\begin{titlepage}
    \begin{center}
        \begin{center}
            \includegraphics[width=0.6\textwidth]{../../template/img/logo.pdf}
        \end{center}
        \vspace{4cm}
        \huge\textbf{Piano di qualifica}\par
        \vspace{2cm}
        \large \textbf{SnakeByte} (Gruppo 1):\\
        \large Valeria Baleanu, Leonardo Pellizzon, Filippo Venzo, Giuseppe De Fina, \\
         Francesco Pasqual, Christian Libralato, Luca Granziero \\
        (2109911, 2111006, 2113705, 2113187, 2103119, 2101047, 2075512)
        \vfill
        \small
        \begin{center}
            \begin{tabular}{|c|c|c|c|}
                \hline
                \multicolumn{4}{|c|}{\textbf{Informazioni documento}} \\
                \hline
                \rowcolor{lightgray} \textbf{Versione} & \textbf{Data} & \textbf{Stato} & \textbf{Destinatari} \\
                \hline
                0.1.0 & 22/12/2025 & Verificato &       
                \begin{tabular}[c]{@{}l@{}}
                    \textbf{Interni:} SnakeByte \\
                    \textbf{Esterni:} prof. Vardanega Tullio, prof. Cardin Riccardo, Vimar
                \end{tabular} \\
                \hline
            \end{tabular}
        \end{center}
        \vfill
        \large Contatti: snakebyteteam@gmail.com
    \end{center}
\end{titlepage}

\begin{center}
    \begin{tabularx}{\textwidth}{|c|c|c|c|c|X|}
        \hline
        \multicolumn{6}{|c|}{\textbf{Registro delle modifiche}} \\
        \hline
        \rowcolor{lightgray} \textbf{Versione} & \textbf{Data} & \textbf{Autore} & \textbf{Verificatore} & \textbf{Approvatore} & \textbf{Descrizione} \\
        0.1.1 & 27/12/2025 & V. Baleanu & F. Pasqual & - & Aggiunta Introduzione \\
        \hline
        0.1.0 & 22/12/2025 & V. Baleanu & F. Pasqual & - & Prima Stesura \\
        \hline
    \end{tabularx}
\end{center}

\newpage

\tableofcontents

\newpage

\section{Introduzione}
\subsection{Finalità del documento}
Il presente documento ha lo scopo di definire e documentare gli obiettivi di qualità 
che il gruppo \textit{SnakeByte} si impegna a perseguire durante l'intero ciclo di vita del 
progetto. In particolare, il Piano di Qualifica${_G}$ comprende i seguenti elementi:
\begin{itemize}
    \item \textbf{Piano della Qualità}: tutte le attività volte a definire gli obiettivi 
    qualitativi e a stabilire i processi e le risorse necessarie per il loro conseguimento;
    \item \textbf{Controllo di Qualità}: le attività finalizzate ad accertare la conformità dei processi 
    e dei prodotti agli standard, ai requisiti e alle metriche definite;
    \item \textbf{Miglioramento continuo}: le azioni intraprese per analizzare i risultati 
    del controllo della qualità e per introdurre interventi correttivi e migliorativi 
    sui processi e sul prodotto.
\end{itemize}
Il documento si rivolge sia ai membri del gruppo, come guida per il mantenimento 
degli standard qualitativi, sia ai committenti e al proponente, come dimostrazione 
dell'impegno del gruppo verso la qualità del prodotto finale.
\subsection{Glossario}
Il documento cita alcuni termini la cui definizione può risultare ambigua. Per questo è possibile consultare
il glossario${_G}$ il quale contiene le definizioni di tali espressioni, che saranno marcate da una lettera G a
pedice.
\subsection{Riferimenti}
\subsubsection{Riferimenti Normativi}
\begin{itemize}
    \item \textbf{Norme di Progetto}: 
    \url{https://github.com/SnakeByteTeam/snakebyteteam.github.io/blob/develop/2-RTB/Norme_di_progetto/NdP_v0.2.0.pdf}
    \\ (consultato il 22/12/2025).
    \item \textbf{Vimar View4Life Capitolato di Ingegneria del Software Università di Padova 2025-2026}:
    \url{https://www.math.unipd.it/~tullio/IS-1/2025/Progetto/C9.pdf}
    \\ (consultato il 22/12/2025).
\end{itemize}
\subsubsection{Riferimenti Informativi}
\begin{itemize}
    \item \textbf{Glossario}: \url{https://snakebyteteam.github.io/glossary.html}
    \\ (consultato il 22/12/2025).
    \item \textbf{Standard ISO 9000:2015 - Quality management systems. Fundamentals and vocabulary}:
    \url{https://www.iso.org/standard/45481.html}
    \\ (consultato il 23/12/2025).
    \item \textbf{Standard ISO/IEC 12207:1995}: 
    \url{https://www.math.unipd.it/~tullio/IS-1/2009/Approfondimenti/ISO_12207-1995.pdf}
    \\ (consultato il 27/12/2025).
\end{itemize}

\section{Qualità di processo}
La qualità di processo${_G}$ ha lo scopo di garantire che le attività di sviluppo
del software siano pianificate, eseguite, monitorate e migliorate in modo
sistematico, al fine di ridurre il rischio di errori o incongruenze 
durante l'intero ciclo di vita del progetto.

Per la definizione e la classificazione dei processi di qualità, il progetto
fa riferimento allo standard ISO/IEC 12207:1995, che descrive i processi 
del ciclo di vita del software e li suddivide nelle seguenti categorie:
\begin{itemize}
    \item \textbf{Processi primari}: sono direttamente coinvolti nella
    realizzazione, fornitura e mantenimento del prodotto software;
    \item \textbf{Processi di supporto}: affiancano i processi primari fornendo 
    attività di supporto come la documentazione e la verifica;
    \item \textbf{Processi organizzativi}: forniscono le risorse necessarie 
    all'esecuzione dei processi primari e di supporto, favorendo il 
    miglioramento continuo dell'organizzazione.
\end{itemize}
\subsection{Processi primari}
\subsubsection{Fornitura}
\subsubsection{Sviluppo}

\subsection{Processi di supporto}
\subsubsection{Documentazione}
\subsubsection{Verifica}

\subsection{Processi organizzativi}
\subsubsection{Gestione dei processi}


\section{Qualità di prodotto}

\section{Strategie di Testing}


\end{document}