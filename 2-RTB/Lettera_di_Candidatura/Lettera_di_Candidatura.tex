\documentclass[10pt, letterpaper]{article}
\usepackage{multicol}
\usepackage[nomarginpar, margin=2.75cm, tmargin=3cm, bmargin=1.75cm]{geometry}
\usepackage[
    colorlinks=false,                 % disabilita colore nel testo (usa bordi/underline)
    linkbordercolor={0 0 1},          % colore bordo/underline per link interni (RGB)
    urlbordercolor={0 0 1},           % colore bordo/underline per URL
    citebordercolor={0 0 0},          % colore bordo/underline per citazioni
    pdfborderstyle={/S/U/W 1}          % stile bordo: U = underline, W = spessore
]{hyperref}
\usepackage{graphicx}
\usepackage[table, x11names]{xcolor}
\usepackage{tabularx}
\usepackage[gen]{eurosym}
\renewcommand{\arraystretch}{1.2} 
\renewcommand{\contentsname}{Indice}
\usepackage{fancyhdr}
\pagestyle{fancy}
\fancyhf{}
\fancyhead[L]{SnakeByte} 
\fancyhead[R]{Lettera di Candidatura RTB}
\fancyfoot[C]{\thepage}


\begin{document}

\begin{titlepage}
    \begin{center}
        \begin{center}
            \includegraphics[width=0.6\textwidth]{./img/logo.pdf}
        \end{center}
        \vspace{4cm}
        \huge\textbf{Lettera di Candidatura RTB}\par
        \vspace{2cm}
        \large \textbf{SnakeByte} (Gruppo 1):\\
        \large Valeria Baleanu, Leonardo Pellizzon, Filippo Venzo, Giuseppe De Fina, \\
         Francesco Pasqual, Christian Libralato, Luca Granziero \\
        (2109911, 2111006, 2113705, 2113187, 2103119, 2101047, 2075512)
        \vfill
        \small
        \begin{center}
            \begin{tabular}{|c|c|}
                \hline
                \multicolumn{2}{|c|}{\textbf{Informazioni documento}} \\
                \hline
                \rowcolor{lightgray} \textbf{Data} & \textbf{Destinatari} \\
                \hline
                13/02/2026 & SnakeByte, prof. Vardanega Tullio e prof. Cardin Riccardo \\
                \hline
            \end{tabular}
        \end{center}
        \vfill
        \large Contatti: snakebyteteam@gmail.com
    \end{center}
\end{titlepage}

\newpage

\noindent 
Egregio Prof. Vardanega, \\
Egregio Prof. Cardin, \\
\\
con il presente documento il gruppo \textit{SnakeByte} intende comunicare la propria volontà di sottoporsi alla revisione della \textit{Requirements and Technology Baseline} in data 13/02/2026, riguardo al lavoro svolto nella realizzazione del capitolato proposto da \textbf{Vimar S.p.A.} con titolo: 
\begin{center}
    \textbf{Vimar View4Life}
\end{center}
La documentazione a sostegno di tale candidatura, costituita dai verbali interni ed esterni, dall'analisi dei requisiti, dalle norme di progetto, dal piano di progetto e piano di qualità è disponibile \href{https://snakebyteteam.github.io/}{al sito web ufficiale del gruppo}\footnote{\href{https://snakebyteteam.github.io/}{https://snakebyteteam.github.io/}}.\\
\\
All'interno di esso è possibile trovare i seguenti documenti: 
\begin{multicols}{2}
    \begin{itemize}
        \item Lettera di candidatura RTB (attuale documento);
        \item Piano di Qualifica;
        \item Piano di Progetto;
        \item Analisi dei Requisiti; 
        \item Norme di Progetto;
        \item Glossario;
        \item Verbale Interno 07/11/2025
        \item Verbale Interno 17/11/2025
        \item Verbale Interno 25/11/2025
        \item Verbale Interno 01/12/2025
        \item Verbale Interno 15/12/2025
        \item Verbale Interno 29/12/2025
        \item Verbale Interno 13/01/2026
        \item Verbale Interno 02/02/2026
        \item Verbale Interno 09/02/2026
        \item Verbale Esterno 12/11/2025
        \item Verbale Esterno 27/11/2025
        \item Verbale Esterno 10/12/2025
        \item Verbale Esterno 23/12/2025
        \item Verbale Esterno 07/01/2026
        \item Verbale Esterno 21/01/2026
        \item Verbale Esterno 11/02/2026
    \end{itemize}
\end{multicols}


Il gruppo \textit{SnakeByte}, a supporto dell'analisi dei requisiti e come prova di fattibilità tecnologica di questi, ha sviluppato un \textit{Proof Of Concept}, reperibile al \href{https://github.com/SnakeByteTeam/PoC}{repository}\footnote{\href{https://github.com/SnakeByteTeam/PoC}{https://github.com/SnakeByteTeam/PoC}} dedicato\\
\\
Infine intendiamo comunicare che la data di ultima consegna del progetto è stata spostata al \textbf{5 aprile 2026}, con costo stimato massimo di \textbf{\euro 13.045}
\begin{center}
    \begin{tabular}{|c|c|}
        \hline
        \rowcolor{lightgray} \textbf{Membro} & \textbf{Matricola} \\
        \hline
        Valeria Baleanu & 2109911 \\
        \hline
        Leonardo Pellizzon & 2111006 \\
        \hline
        Filippo Venzo & 2113705 \\
        \hline
        Giuseppe De Fina & 2113187 \\
        \hline
        Francesco Pasqual & 2103119 \\
        \hline
        Christian Libralato & 2101047 \\
        \hline
        Luca Granziero & 2075512 \\
        \hline
    \end{tabular}
\end{center}
Cordiali Saluti,
\\
\textit{SnakeByte}


\end{document}