\documentclass[10pt, letterpaper]{article}
\usepackage[nomarginpar, margin=2.75cm, tmargin=3cm, bmargin=1.75cm]{geometry}
\usepackage[
    colorlinks=true,      
    linkcolor=black,      
    urlcolor=blue,       
    citecolor=black       
]{hyperref}
\usepackage{graphicx}
\usepackage[table, x11names]{xcolor}
\usepackage{tabularx}
\usepackage{lastpage} 
\renewcommand{\arraystretch}{1.2} % migliora la leggibilità
\renewcommand{\contentsname}{Indice}
\usepackage{fancyhdr}
\usepackage{template}

%Comandi per livello di sottosezioni = 3
\setcounter{tocdepth}{4}
\setcounter{secnumdepth}{4}
\newcommand{\trisubsection}[1]{\paragraph{#1}\mbox{}\\}

\pagestyle{fancy}
\fancyhf{}
\fancyhead[L]{SnakeByte} 
\fancyhead[R]{Analisi dei Requisiti}
\fancyfoot[C]{Pagina \thepage\ di \pageref{LastPage}}


\begin{document}

\begin{titlepage}
    \begin{center}
        \begin{center}
            \includegraphics[width=0.6\textwidth]{./img/logo.pdf}
        \end{center}
        \vspace{4cm}
        \huge\textbf{Analisi dei requisiti}\par
        \vspace{2cm}
        \large \textbf{SnakeByte} (Gruppo 1):\\
        \large Valeria Baleanu, Leonardo Pellizzon, Filippo Venzo, Giuseppe De Fina, \\
         Francesco Pasqual, Christian Libralato, Luca Granziero \\
        (2109911, 2111006, 2113705, 2113187, 2103119, 2101047, 2075512)
        \vfill
        \small
        \begin{center}
            \begin{tabular}{|c|c|c|c|}
                \hline
                \multicolumn{4}{|c|}{\textbf{Informazioni documento}} \\
                \hline
                \rowcolor{lightgray} \textbf{Versione} & \textbf{Data} & \textbf{Stato} & \textbf{Destinatari} \\
                \hline
                0.1.2 & 24/11/2025 & Modificato & prof. Vardanega Tullio, prof. Cardin Riccardo, Vimar S.p.A. \\
                \hline
            \end{tabular}
        \end{center}
        \vfill
        \large Contatti: snakebyteteam@gmail.com
    \end{center}
\end{titlepage}

\begin{center}
    \begin{tabularx}{\textwidth}{|c|c|c|c|c|X|}
        \hline
        \multicolumn{6}{|c|}{\textbf{Registro delle modifiche}} \\
        \hline
        \rowcolor{lightgray} \textbf{Versione} & \textbf{Data} & \textbf{Autore} & \textbf{Verificatore} & \textbf{Approvatore} & \textbf{Descrizione} \\
        \hline
        0.1.2 & 24/10/2025 & F. Venzo & - & - & Modifica UC1, UC2 \\
        \hline
        0.1.1 & 20/10/2025 & F. Venzo & V. Baleanu & - & Aggiunta Use Cases UC1, UC1.1, UC1.2, UC1.3, UC2, UC2.1, UC2.1, UC2.3, UC2.4, UC2.5, UC2.6, UC2.7, UC3 \\
        \hline
        0.1.0 & 05/10/2025 & L. Granziero & L. Pellizon & - & Prima Stesura \\
        \hline
    \end{tabularx}
\end{center}

\newpage

\tableofcontents

\newpage
\section{Introduzione}
\subsection{Finalità del documento}{
Il documento di analisi dei requisiti, in un contesto di ingegneria del software, ha lo scopo fondamentale di tradurre l'esigenza dell'utenza e degli stakeholder, in questo caso la proponente Vimar S.p.A., in una specifica completa, coerente e verificabile di requisiti, destinata a guidare le fasi di progettazione e sviluppo, verifica e infine validazione del sistema.

\noindent
\\
\textbf{Definire chiaramente “cosa” il sistema deve fare e “in quali condizioni”:}
\begin{itemize}
    \item L’analisi dei requisiti serve a identificare le funzioni (requisiti funzionali) e le qualità (requisiti non funzionali: prestazioni, usabilità, affidabilità, portabilità...) del software.
    \item Definisce i limiti del sistema e i vincoli (tecnici, di interfaccia...)
    \item Permette di evitare ambiguità o fraintendimenti sulle funzionalità richieste. 
\end{itemize}

\noindent
\textbf{Allineare tutti gli stakeholder su un linguaggio comune e condividere le aspettative:}
\begin{itemize}
    \item Questo documento funge da contratto tra cliente/committente e il team di sviluppo: specifica ciò che sarà consegnato.
    \item Aiuta a garantire che utenti, committenti, analisti, progettisti e tester abbiano la stessa comprensione del sistema.
\end{itemize}

\noindent
\textbf{Fornire una base stabile per le fasi successive del ciclo di vita del software:}
\begin{itemize}
    \item Il documento di analisi dei requisiti serve come input per la progettazione del sistema, per la pianificazione dello sviluppo e per la pianificazione dei test.
    \item Serve anche come base per la verifica e la convalida: si può usare come riferimento per capire se il prodotto finale soddisfa i requisiti richiesti. 
\end{itemize}
\noindent
\textbf{Gestire i rischi e controllare le modifiche:}
\begin{itemize}
    \item Durante l’analisi dei requisiti si identificano requisiti non realizzabili, conflitti tra requisiti, omissioni e incoerenze. Ciò consente di ridurre i rischi fin dalle prime fasi.
    \item Aiuta a limitare il fenomeno dello “scope creep” (ovvero l’aggiunta non controllata di funzionalità) e a mantenere il controllo sul cambiamento dei requisiti. 
\end{itemize}
}

\subsection{Sviluppo del documento}{
Il presente documento è stato sviluppato in modo graduale e incrementale, con lo scopo di facilitare modifiche future in base alle esigenze che verranno concordate tra il gruppo e l'azienda committente. Il documento è quindi soggetto a un processo di miglioramento continuo nel tempo.
}

\subsection{Riferimenti}

\subsubsection{Riferimenti Normativi}
\begin{itemize}
    \item \textbf{Norme di Progetto}: 
    
    \url{link-alle-norme-di-progetto}  (non ci sono sul sito)
    
    (consultato il 30/10/2025);
    \item \textbf{Vimar View4Life Capitolato di Ingegneria del Software Università di Padova 2025 - 2026:}
    
    \url{https://www.math.unipd.it/~tullio/IS-1/2025/Progetto/C9.pdf} 
    
    (consultato il 20/10/2025).
\end{itemize}

\subsubsection{Riferimenti Informativi}
\begin{itemize}
    \item \textbf{830-1998 - IEEE Recommended Practice for Software Requirements Specifications}
    
    \url{https://ieeexplore.ieee.org/document/720574} 
    
    (consultato il 20/10/2025).

    \item \textbf{Diagrammi Use Case - Riccardo Cardin}
    
    \url{https://www.math.unipd.it/~rcardin/swea/2022/Diagrammi%20Use%20Case.pdf} 
    
    (consultato il 18/10/2025).
\end{itemize}

\section{Descrizione del prodotto}
\subsection{Prospettiva del prodotto}
La prospettiva del prodotto è un sistema domotico integrato per anziani autosufficienti che si basa su dispositivi Vimar con connessione mesh Bluetooth.
Gli obiettivi principali sono la sicurezza e il comfort delle persone occupanti, l'aumento dell'efficienza energetica della struttura e la semplificazione della gestione operativa dell'impianto elettrico.
Tali risultati vengono raggiunti attraverso la centralizzazione del controllo di illuminazione, temperatura, televisione e dispositivi di sicurezza mediante l'app View e i relativi servizi cloud, con la possibilità di controllo da remoto da parte del personale medico.

\subsection{Obiettivi del prodotto}{
Il progetto consiste nella realizzazione di una piattaforma unica \textit{View4Life} per la gestione intelligente degli impianti \textit{Smart} nelle residenze protette per anziani, sfruttando i dispositivi domotici Vimar connessi in rete \textit{mesh Bluetooth} tramite l’\textit{API KNX IoT 3rd-party$_{G}$}. Questa soluzione mira a supportare il lavoro del personale sanitario fornendo uno strumento che integri un sistema di gestione degli allarmi (come il rilevamento di cadute o presenze prolungate in determinate stanze) per garantire un intervento rapido e tempestivo. Inoltre, la piattaforma è progettata per permettere il monitoraggio del consumo energetico e la rilevazione di anomalie nell’impianto.
}

\subsection{Funzionalità del prodotto}
Dal punto di vista degli utenti del personale sanitario l'applicativo svolge le seguenti funzioni:
\begin{itemize}
    \item Visualizzazione delle informazioni generali di allarmi, statistiche ed analitiche tramite cruscotto riassuntivo (Dashboard);
    \item Possibilità di essere notificati, visualizzare e gestire gli allarmi attivi;
    \item Possibilità di visualizzare e gestire i dispositivi dei vari impianti collocati in diverse residenze;
    \item Possibilità di visualizzare statistiche, tramite grafici, sui consumi dell'impianto, sulla variazione di temperatura e sugli allarmi passati;
    \item Possibilità di ricevere consigli, basati sulle statistiche, per ridurre i consumi energetici.
\end{itemize}

\subsection{Utenza di riferimento}{
Il prodotto si rivolge principalmente a quattro categorie principali di utenti, descritte di seguito:

\begin{itemize}
    \item \textbf{Personale medico e operatori sanitari} che utilizzano l'applicazione per monitorare lo stato di ambienti e dispositivi, oltre a ricevere notifiche di allarme e gestire da remoto funzioni come temperatura, illuminazione e sicurezza delle stanze degli ospiti presenti. 
    \item \textbf{Personale amministrativo} che si fa carico della configurazione e manutenzione del sistema e del monitoraggio dei consumi elettrici, e quindi a una conseguente ottimizzazione dell'efficienza dell'impianto elettrico.
\end{itemize}
Questa sezione evidenzia come il sistema domotico proposto miri a centralizzare il controllo e la gestione degli impianti all’interno della residenza, semplificando le operazioni quotidiane del personale e migliorando la qualità della vita degli ospiti.
Attraverso l’app View e i servizi cloud Vimar, l’applicazione permette di gestire in modo integrato illuminazione, temperatura e sicurezza, contribuendo a maggior efficienza energetica, sicurezza e comfort abitativo per tutti gli utenti coinvolti.
}





\section{Casi d'uso}
Un \textit{caso d'uso$_{G}$} è la descrizione dettagliata, tramite \textit{diagramma UML$_G$} e descrizione testuale, di un insieme di scenari che hanno uno scopo comune, all'interno del Sistema, per un attore.
Permettono di comprendere al meglio le funzionalità che devono essere rese disponibili dal Sistema \textit{software}.

In particolare, le descrizioni dei casi d'uso contenute in questo documento conterranno le informazioni riportate nella seguente tabella:

\begin{center}
    \begin{tabularx}{\textwidth}{|c| >{\centering\arraybackslash}X|}
        \hline
        \rowcolor{lightgray} \textbf{Campo} & \textbf{Descrizione} \\
        \hline
        Attori & Coloro che partecipano attivamente al caso d'uso per raggiungere un preciso obiettivo  \\
        \hline
        Pre-condizioni & Condizioni che devono essere soddisfatte prima dello scenario descritto dal caso d'uso\\
        \hline
        Post-condizioni & Condizioni che risultano soddisfatte dopo il completamento dello scenario principale del caso d'uso. Se viene completato uno scenario alternativo, saranno soddisfatte le Post-condizioni di quest'ultimo \\
        \hline
        Trigger & La motivazione che porta l'utente a svolgere i passi del caso d'uso \\
        \hline
        Scenario principale & Sequenza di passi che l'utente deve seguire per completare il caso d'uso \\
        \hline
        Scenari alternativi & Scenario divergente dal principale per il verificarsi di una particolare condizione \\
        \hline
        Estensioni &  Casi d'uso ulteriori eseguiti al verificarsi di una particolare condizione nel caso d'uso primario. Modificano Scenario e Post-condizioni \\
        \hline
        Inclusioni & Casi d'uso ulteriori eseguiti al fine di completare il caso d'uso principale. Vengono eseguiti tutti incondizionatamente. \\
        \hline
        
    \end{tabularx}
\end{center}

Non tutti gli attributi sono necessari per ogni caso d'uso. Nel caso in cui un campo sia assente in un caso d'uso, allora tale sarà assente anche nella sua descrizione e nel suo diagramma UML.

\subsection{Attori}
Di seguito vengono riportati gli attori individuati 

\begin{center}
    \begin{tabularx}{\textwidth}{|c| >{\centering\arraybackslash}X|}
        \hline
        \rowcolor{lightgray} \textbf{Attore} & \textbf{Descrizione} \\
        \hline
        Utente & Rappresenta l'utente generico (sia Admin che del personale sanitario) \\
        \hline
        Utente personale sanitario & Rappresenta il personale sanitario. \\
        \hline 
        Utente admin & Rappresenta l'amministratore della struttura il quale gestice la struttura e il personale sanitario \\
        \hline          
    \end{tabularx}
\end{center}



\subsection{Lista dei casi d'uso}

\subsubsection{UC1: Autenticazione} \label{UC1}

\begin{center}
    \includegraphics[width=0.6\textwidth]{./img/UC1.pdf}
\end{center}

\addAttore{Utente}
\addPreCondizione{Il Sistema è attivo;}
\addPreCondizione{L'utente non è autenticato nel Sistema.}
\addPostCondizione{L'utente è autenticato nel Sistema}
\addTrigger{L'utente vuole autenticarsi nel Sistema}
\addScenarioPrincipale{L'utente inserisce il proprio username;}
\addScenarioPrincipale{L'utente inserisce la propria Password.}
\addInclusione{Inserimento username \hyperref[UC1.1]{\textit{§UC1.1}}}
\addInclusione{Inserimento Password \hyperref[UC1.2]{\textit{§UC1.2}}}
\addScenarioAlternativo{L'utente inserisce username o Password errate.}
\addEstensione{Autenticazione fallita \hyperref[UC1.3]{\textit{§UC1.3}}}

\makeTable


\trisubsection{UC1.1 Inserimento username} \label{UC1.1}

\addAttore{Utente}
\addPreCondizione{Il Sistema è attivo;}
\addPreCondizione{L'utente non è autenticato nel Sistema;}
\addPreCondizione{Il Sistema non conosce l'username dell'utente.}

\addPostCondizione{Il Sistema conosce l'username dell'utente.}
\addScenarioPrincipale{L'utente inserisce il proprio username.}

\addTrigger{L'utente ha selezionato l'opzione di inserimento dell'username.}

\makeTable

\trisubsection{UC1.2 Inserimento Password} \label{UC1.2}

\addAttore{Utente}
\addPreCondizione{Il Sistema è attivo;}
\addPreCondizione{L'utente non è autenticato nel Sistema;}
\addPreCondizione{Il Sistema non conosce la Password dell'utente.}

\addPostCondizione{Il Sistema conosce la Password dell'utente.}
\addScenarioPrincipale{L'utente inserisce la propria Password.}

\addTrigger{L'utente ha selezionato l'opzione di inserimento della Password;}
\makeTable

\trisubsection{UC1.3 Autenticazione fallita} \label{UC1.3}

\addAttore{Utente}
\addPreCondizione{Il Sistema è attivo;}
\addPreCondizione{L'utente non è autenticato nel Sistema;}

\addPostCondizione{L'utente non è autenticato.}

\addScenarioPrincipale{Il Sistema segnala l'errore di autenticazione.}

\addTrigger{L'utente ha immesso username o Password errati.}

\makeTable

\trisubsection{UC1.4 Creazione nuovo utente personale sanitario} \label{UC1.4}
\begin{center}
    \includegraphics[width=0.6\textwidth]{./img/UC1.4.pdf}
\end{center}

\addAttore{Utente admin;}
\addPreCondizione{Il Sistema è attivo;}
\addPreCondizione{L'utente admin è autenticato nel Sistema.}

\addPostCondizione{Un nuovo utente del personale sanitario è registrato presso il Sistema}

\addScenarioPrincipale{L'utente admin crea uno username per l'utente del personale sanitario;}
\addScenarioPrincipale{L'utente admin crea una password temporanea per l'utente del personale sanitario;}

\addInclusione{Creazione username utente personale sanitario \hyperref[UC1.4.1]{\textit{§UC1.4.1}}}
\addInclusione{Generazione password temporanea \hyperref[UC1.4.2]{\textit{§UC1.4.2}}}

\addTrigger{L'utente admin vuole registrare un nuovo utente del personale sanitario.}

\makeTable


\trisubsection{UC1.4.1 Creazione username utente personale sanitario} \label{UC1.4.1}

\addAttore{Utente admin}
\addPreCondizione{Il Sistema è attivo;}
\addPreCondizione{L'utente admin è autenticato nel Sistema.}

\addPostCondizione{Il Sistema conosce lo username dell'utente del personale sanitario;}

\addScenarioPrincipale{L'utente admin sceglie uno username per l'utente del personale sanitario.}

\addScenarioAlternativo{L'username scelto è già in uso nel Sistema;}
\addEstensione{Errore username già in uso \hyperref[UC1.4.3]{\textit{§UC1.4.3}}}

\makeTable

\trisubsection{UC1.4.2 Generazione password temporanea} \label{UC1.4.2}

\addAttore{Utente admin}
\addPreCondizione{Il Sistema è attivo;}
\addPreCondizione{L'utente admin è autenticato nel Sistema.}

\addPostCondizione{Il Sistema conosce la password temporanea per l'utente del personale sanitario;}

\addScenarioPrincipale{L'utente admin genera una Password temporanea per l'utente del personale sanitario.}

\makeTable

\trisubsection{UC1.4.3 Errore username già in uso} \label{UC1.4.3}
\addAttore{Utente admin}
\addPreCondizione{Il Sistema è attivo;}
\addPreCondizione{L'utente admin è autenticato nel Sistema.}

\addPostCondizione{Il Sistema non registra il nuovo utente del personale sanitario.}

\addScenarioPrincipale{Il Sistema segnala l'errore di username già in uso.}
\addScenarioPrincipale{Il Sistema ripropone all'utente admin di scegliere uno username.}

\addTrigger{L'utente admin sceglie uno username già in uso nel Sistema.}

\addInclusione{Creazione username utente personale sanitario \hyperref[UC1.4.1]{\textit{§UC1.4.1}}}

\makeTable


\trisubsection{UC1.5 Registrazione utente personale sanitario} \label{UC1.5}

\addAttore{Utente personale sanitario}

\addPreCondizione{Il Sistema è attivo;}
\addPreCondizione{L'utente del personale sanitario non è autenticato nel Sistema.}

\addPostCondizione{L'utente del personale sanitario è autenticato nel Sistema}

\addScenarioPrincipale{L'utente del personale si autentica con le credenziali fornite dall'utente admin;}
\addScenarioPrincipale{Il Sistema obbliga l'utente del personale sanitario ad aggiornare la Password temporanea;}

\addInclusione{Autenticazione \hyperref[UC1]{\textit{§UC1}}}
\addInclusione{Aggiornamento Password \hyperref[UC1.5.1]{\textit{§UC1.5.1}}}        

\addTrigger{L'utente del personale sanitario si è autenticato per la prima volta.}

\makeTable


\trisubsection{UC1.5.1 Aggiornamento Password} \label{UC1.5.1}
\addAttore{Utente personale sanitario}
\addPreCondizione{Il Sistema è attivo;}
\addPreCondizione{L'utente del personale sanitario è autenticato nel Sistema con una Password temporanea.}

\addPostCondizione{L'utente del personale sanitario ha aggiornato la propria Password.}

\addScenarioPrincipale{L'utente del personale sanitario inserisce una nuova Password.}

\makeTable



\subsubsection{UC2 Visualizzazione Dashboard} \label{UC2}
\begin{center}
    \includegraphics{./img/UC2.pdf}
\end{center}

\addAttore{Utente}
\addPreCondizione{Il Sistema è attivo;}
\addPreCondizione{L'utente è autenticato nel Sistema.}


\addPostCondizione{Il Sistema mostra a schermo la Dashboard, contenente i Moduli.}

\addScenarioPrincipale{L'utente seleziona dal menù l'opzione relativa alla Dashboard;}
\addScenarioPrincipale{L'utente visualizza i Moduli opzionali presenti nella Dashboard.}
\addScenarioPrincipale{L'utente visualizza il Modulo di gestione allarmi presenti nella Dashboard.}

\addTrigger{L'utente vuole visualizzare la Dashboard.}

\addInclusione{Visualizzazione Modulo opzionale Dashboard \hyperref[UC2.1]{\textit{§UC2.1}}}
\addInclusione{Visualizzazione Modulo gestione allarmi \hyperref[UC2.2]{\textit{§UC2.2}}}

\makeTable

\trisubsection{UC2.1 Visualizzazione Modulo opzionale Dashboard} \label{UC2.1}

\addAttore{Utente}
\addPreCondizione{Il Sistema è attivo;}
\addPreCondizione{L'utente è autenticato nel Sistema.}
\addPreCondizione{L'utente sta visualizzando la Dashboard.}

\addPostCondizione{Il Sistema mostra i moduli opzionali presenti nella Dashboard.}

\addScenarioPrincipale{L'utente visualizza, nella Dashboard, i Moduli da lui selezionati. (Vedi \hyperref[UC2.3]{\textit{§UC2.3}})}

\addTrigger{L'utente vuole visualizzare un Modulo opzionale nella Dashboard.}
\makeTable

\trisubsection{UC2.2 Visualizzazione Modulo gestione allarmi} \label{UC2.2}

\addAttore{Utente}
\addPreCondizione{Il Sistema è attivo;}
\addPreCondizione{L'utente è autenticato nel Sistema;}
\addPreCondizione{L'utente sta visualizzando la Dashboard.}

\addPostCondizione{Il Sistema mostra il Modulo di gestione allarmi nella Dashboard.}

\addScenarioPrincipale{L'utente visualizza, nella Dashboard, il Modulo di gestione allarmi il quale non può essere rimosso.}

\addTrigger{L'utente vuole visualizzare il Modulo di gestione allarmi nella Dashboard.}

\makeTable


\trisubsection{UC2.3 Aggiunta Modulo opzionale Dashboard} \label{UC2.3}

\begin{center}
    \includegraphics[width=0.6\textwidth]{./img/UC2.3.pdf}
\end{center}

\addAttore{Utente}
\addPreCondizione{Il Sistema è attivo;}
\addPreCondizione{L'utente è autenticato nel Sistema.}
\addPreCondizione{L'utente sta visualizzando la Dashboard.}

\addPostCondizione{Il Sistema aggiunge alla Dashboard il Modulo selezionato dall'utente.}

\addScenarioPrincipale{L'utente seleziona l'opzione di modifica della Dashboard;}
\addScenarioPrincipale{L'utente sceglie un Modulo non presente nella Dashboard.}

\addScenarioAlternativo{Il Modulo selezionato è già presente nella Dashboard;}
\addScenarioAlternativo{È stato raggiunto il massimo numero di Moduli visualizzabili nella Dashboard.}

\addInclusione{Selezione Modulo non presente \hyperref[UC2.4.1]{\textit{§UC2.4.1}}}

\addEstensione{Selezione Modulo già presente \hyperref[UC2.3.1]{\textit{§UC2.3.1}}}
\addEstensione{Limite Moduli raggiunti \hyperref[UC2.3.2]{\textit{§UC2.3.2}}}

\addTrigger{L'utente vuole modificare quali Moduli sono visualizzati nella Dashboard}
\makeTable


\trisubsection{UC2.3.1 Selezione Modulo già presente nella Dashboard} \label{UC2.3.1}

\addAttore{Utente}
\addPreCondizione{Il Sistema è attivo;}
\addPreCondizione{L'utente è autenticato nel Sistema;}
\addPreCondizione{L'utente sta modificando quali Moduli sono visualizzati nella Dashboard.}

\addPostCondizione{Il Modulo selezionato è rimosso dalla Dashboard}

\addScenarioPrincipale{Il Sistema rimuove il Modulo selezionato dalla Dashboard}

\addTrigger{L'utente seleziona un Modulo già presente nella Dashboard}
\makeTable

\trisubsection{UC2.3.2 Errore limite Moduli visualizzabili nella Dashboard} \label{UC2.3.2}

\addAttore{Utente}
\addPreCondizione{Il Sistema è attivo;}
\addPreCondizione{L'utente è autenticato nel Sistema;}
\addPreCondizione{L'utente sta modificando quali Moduli sono visualizzati nella Dashboard; }
\addPreCondizione{Il limite massimo di Moduli visualizzabili nella Dashboard è stato raggiunto.}

\addPostCondizione{Il Modulo selezionato non è inserito nella Dashboard}

\addScenarioPrincipale{Il Sistema non inserisce il Modulo selezionato nella Dashboard;}
\addScenarioPrincipale{Il Sistema segnala la condizione di massima capienza raggiunta.}

\addTrigger{L'utente seleziona un Modulo.}

\makeTable


\trisubsection{UC2.4 Rimozione Modulo opzionale Dashboard}

\begin{center}
    \includegraphics[width=0.6\textwidth]{./img/UC2.4.pdf}
\end{center}

\addAttore{Utente}
\addPreCondizione{Il Sistema è attivo;}
\addPreCondizione{L'utente è autenticato nel Sistema.}
\addPreCondizione{L'utente sta visualizzando la Dashboard.}

\addPostCondizione{Il Sistema rimuove dalla Dashboard il Modulo selezionato dall'utente.}

\addScenarioPrincipale{L'utente seleziona l'opzione di modifica della Dashboard;}
\addScenarioPrincipale{L'utente seleziona un Modulo opzionale attualmente presente nella Dashboard}

\addScenarioAlternativo{Il Modulo selezionato non è presente nella Dashboard.}

\addInclusione{Selezione Modulo già presente \hyperref[UC2.3.1]{\textit{§UC2.3.1}}}

\addEstensione{Selezione Modulo non presente \hyperref[UC2.4.1]{\textit{§UC2.4.1}}}


\addTrigger{L'utente vuole rimuovere un Modulo opzionale dalla Dashboard}
\makeTable


\trisubsection{UC2.4.1 Selezione modulo non presente in Dashboard} \label{UC2.4.1}
\addAttore{Utente}
\addPreCondizione{Il Sistema è attivo;}
\addPreCondizione{L'utente è autenticato nel Sistema;}
\addPreCondizione{L'utente sta modificando quali Moduli sono visualizzati nella Dashboard.}

\addPostCondizione{Il Modulo selezionato viene aggiunto alla Dashboard.}

\addScenarioPrincipale{L'utente seleziona un Modulo, non presente nella Dashboard, tra:
                        \begin{itemize}
                            \item informazioni utente;
                            \item statistiche allarmi;
                            \item analisi clima;
                            \item analisi consumi;
                            \item dispositivi accessi;
                            \item temperatura impostata.
                        \end{itemize} }

\addScenarioPrincipale{Il Sistema aggiunge alla Dashboard il Modulo selezionato.}

\addTrigger{L'utente seleziona un Modulo non presente nella Dashboard}
\makeTable

\end{document}