
%============================
% Verbale riunione – Pianificazione RTB
%============================
\documentclass[10pt, letterpaper]{article}
\usepackage{../../../../template/template}

% ------------------------
% Impostazioni documento
% ------------------------

\setDocTitle{Verbale esterno 11/02/2026}
\setDocVersion{1.0.0}
\setDocData{13/02/2026}
\setDocIndex{ve\_2026\_01\_11}
\setDocState{2} % 0=Da verificare
\setDocRecipient{2} % solo interni
\begin{document}

\makeTitlePage

\newpage
\addChangelog{1.0.0}{13/02/2026}{ - }{ - }{F. Venzo}{\approval}
\addChangelog{0.1.0}{12/02/2026}{F. Venzo}{F. Pasqual}{ - }{Prima stesura}
\makeChangelog


\newpage

\tableofcontents

\newpage

\makeMeetingInfoTable{11/02/2026}{16:00}{16:40}{via \textit{Microsoft Teams$_G$}}
% ------------------------
% Partecipanti
% ------------------------
\addParticipant{Valeria}{Baleanu}{Programmatore}{P}
\addParticipant{Francesco}{Pasqual}{Verificatore}{P}
\addParticipant{Christian}{Libralato}{Programmatore}{P}
\addParticipant{Luca}{Granziero}{Verificatore}{P}
\addParticipant{Leonardo}{Pellizzon}{Programmatore}{P}
\addParticipant{Filippo}{Venzo}{Responsabile}{P}
\addParticipant{Giuseppe}{De Fina}{Amministratore}{P}
\makeMeetingParticipantsTable


\addPrecedentTodo{ve\_2026\_01\_21.a3}{\#81}{C. Libralato, F. Pasqual, V. Baleanu}{2/02/2026}

\makePrecedentTodoTable


\section{Ordine del giorno}
\begin{itemize}
  \item Cadenza dei SAL$_{G}$;
  \item stato del progetto e pianificazione;
  \item nuovi strumenti di gestione;
  \item limiti riscontrati dal gruppo.
\end{itemize}

\section{Approfondimento}

\subsection*{Organizzazione dei SAL}
È stata decisa, in accordo con la Proponente, una cadenza settimanale, piuttosto che bisettimanale, per i prossimi SAL (Stato Avanzamento Lavori). Tali incontri avranno durata di 30-40 minuti con la possibilità di estenderli su richiesta preventiva nel caso in cui sia necessario approfondire temi specifici.

\subsection*{Stato del progetto e pianificazione}
È stato discusso lo stato del progetto, in particolare il \textit{PoC$_{G}$}, che con la presentazione del 28/01/2025, è stato ritenuto concluso e ha ricevuto \textit{feedback} positivi da parte dei referenti di \textit{Vimar S.p.A.}. È stata comunicata alla Proponente la decisione del gruppo di posticipare la data di consegna del progetto al 5/04/2026.

Inoltre, il Responsabile ha presentato le attività previste per lo \textit{sprint} corrente, l'ottavo, comunicando l'intenzione del gruppo di candidarsi alla revisione RTB entro il giorno 13/02/2026. La Proponente ha espresso l'interesse a conoscere l'esito di quest'ultima.

\subsection*{Nuovi strumenti organizzativi}
La Proponente ha reso disponibile, al fine di migliorare l'organizzazione dei prossimi SAL e le comunicazioni tra il gruppo e l'azienda, una \textit{Github Project Board$_{G}$}. Tramite quest'ultima è possibile monitorare lo stato di avanzamento delle macro-attività e i criteri espressi nel capitolato raggruppati per categoria (obbligatori, \textit{nice-to-have} e opzionali).  

È stato inoltre deciso che la \textit{board} verrà compilata dalle figure gestionali del gruppo, quali Responsabile e Amministratore. 

\subsection*{Limiti riscontrati dal gruppo}
Sono stati comunicati alla Proponente i limiti che \textit{SnakeByte} ha riscontrato durante gli ultimi periodi come i rallentamenti dovuti alla sessione di esami appena terminata.

% ------------------------
% Decisioni
% ------------------------
\addDecision{Cadenza settimanale per i SAL}
\addDecision{Compilazione, da parte di Responsabile e Amministratore, della \textit{Github Project Board} per organizzare i temi dei prossimi SAL}
\makeDecisionTable

\addTodo{\noOne}{Comunicare l'esito della revisione RTB alla Proponente}{Gruppo \teamName}{23/02/2026}
\makeTodoTable


\end{document}

