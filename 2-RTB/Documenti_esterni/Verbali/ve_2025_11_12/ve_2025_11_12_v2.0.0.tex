\documentclass[10pt, letterpaper]{article}
\usepackage{../../../../template/template}

\setDocIndex{vi\_2025\_11\_12}
\setDocTitle{Verbale esterno 12/11/2025}
\setDocData{26/01/2026}
\setDocVersion{2.0.0}
\setDocState{2}
\setDocRecipient{2}

\begin{document}

\makeTitlePage

\addChangelog{2.0.0}{26/01/2026}{\noOne}{\noOne}{L. Pellizzon}{\approval}
\addChangelog{1.0.1}{26/01/2026}{L. Pellizzon}{F. Pasqual}{\noOne}{Modifica e correzione dello stato del documento}
\addChangelog{1.0.0}{10/12/2025}{-}{-}{F. Pasqual}{\approval}
\addChangelog{0.1.0}{12/11/2025}{C. Libralato}{L. Pellizzon}{\noOne}{\firstDraft}
\makeChangelog

\newpage
\tableofcontents

\newpage

\makeMeetingInfoTable{\docData}{15:30}{17:30}{in Presenza}

\addParticipant{Filippo}{Venzo}{Responsabile}{P}
\addParticipant{Christian}{Libralato}{Amministratore}{P}
\addParticipant{Leonardo}{Pellizzon}{Verificatore}{P}
\addParticipant{Valeria}{Baleanu}{Analista}{P}
\addParticipant{Giuseppe}{De Fina}{Analista}{P}
\addParticipant{Luca}{Granziero}{Analista}{P}
\addParticipant{Francesco}{Pasqual}{Analista}{P}

\makeMeetingParticipantsTable

\section{Ordine del giorno}
\begin{itemize}
    \item Definizione riunioni periodiche;
    \item proposte e supporto aziendale;
    \item consegna materiale;
    \item Q\&A;
    \item To Do.
\end{itemize}


\section{Approfondimento}
Segue una sintesi dei temi trattati:
\subsection*{Definizione riunioni periodiche}
L'azienda ha proposto dei \textit{SAL$_G$} con cadenza ogni 14 giorni, con ora e giorno da definire, che avranno luogo sulla piattaforma \textit{Microsoft
    Teams$_G$}. Le riunioni, aventi una durata di 60 minuti, potranno essere svolte anche in presenza, qualora richiesto dal gruppo con sufficiente anticipo.\\
A seguito del Proof of Concept la frequenza degli incontri verrà ridotta.\\
Per i \textit{SAL} è stato proposto il seguente ordine del giorno:
\begin{enumerate}
    \item esposizione dei ruoli;
    \item panoramica degli avanzamenti settimanali;
    \item attività terminate;
    \item attività programmate;
    \item limiti o blocchi incontrati.
\end{enumerate}
Al fine di rendere più efficiente la scrittura dei verbali esterni verranno utilizzate, come materiale di supporto, le trascrizioni
delle riunioni generate da \textit{Microsoft Copilot$_G$}.

\subsection*{Proposte e supporto aziendale}
Per ogni \textit{SAL} è atteso un verbale esterno che dovrà essere firmato dalla Proponente, al tal fine
è proposto l'impiego di \textit{Round Review$_G$}, che prevede un processo iterativo di caricamento documenti e inserimento di commenti da parte dell'azienda
fino a raggiungere l'approvazione. \textit{Round View} potrà essere eventualmente applicata anche
per documenti diversi dai verbali.\\ \\
Al fine di facilitare il processo di supporto da parte dell'azienda è stata richiesta la possibilità di condividere il codice sorgente
con un utente aziendale avente accesso in sola lettura al repository \textit{GitHub$_G$}. (Username: \texttt{sciaccom-vimar})\\ \\
Sono stati successivamente individuati come mezzi di comunicazione asincrona le chat di \textit{Microsoft Teams} e le Email per questioni di maggiore entità,
si raccomanda di rispettare gli orari aziendali. \\
Riferimenti:\\
A: \href{mailto:mariano.sciacco@vimar.com}{mariano.sciacco@vimar.com}\\
CC: \href{mailto:francesca.stival@vimar.com}{francesca.stival@vimar.com}\\ \\
Per facilitare il primo approccio con le tecnologie che verranno utlizzate, l'azienda si è resa disponibile a tenere degli incontri di
approfondimento. In prima istanza sono state proposte le seguenti tematiche:
\begin{itemize}
    \item \textit{KNX IoT$_G$};
    \item \textit{Container$_G$}, \textit{Docker$_G$};
    \item altre tecnologie o prodotti Vimar.
\end{itemize}
Gli incontri avranno durata massima di un'ora e richiederanno una partecipazione attiva dei componenti
del gruppo tramite domande e criticità emerse da un precedente studio autonomo degli argomenti.

\subsection*{Consegna materiale}
\textit{Leonardo Pellizzon} è il responsabile del materiale fornito dall'azienda, che consiste in:
\begin{itemize}
    \item 2x Kit di impianto portatile Smart (fisico, nero);
    \item credenziali per accesso all'API \textit{REST$_G$} di \textit{KNX IoT} (cartacea);
    \item documentazione per integrazioni di terze parti per \textit{KNX IoT} (cartacea).
\end{itemize}
Si raccomanda attenzione a non diffondere materiali e credenziali private.

\subsection*{Q\&A}
Nella sezione \textit{questions-and-answers} sono stati trattati i seguenti temi:
\begin{itemize}
    \item riguardo agli strumenti e i frameworks proposti all'interno del capitolato sono state confermate come preferibili, dalla Proponente, le seguenti: \textit{Angular$_G$} per la parte front-end, \textit{Express$_G$} per la parte back-end, \textit{PostgreSQL$_G$} e \textit{TimescaleDB$_G$} rispettivamente a livello di database e analytics;
    \item per quanto concerne il livello di specificità del glossario è emerso che esso è riservato ad un utilizzo interno al gruppo, eventuali definizioni utili agli
          utenti finali dovrebbero essere inserite nel Manuale Utente;
    \item dalla ricerca dei temi iniziali sui quali è utile rafforzare le conoscenze sono emersi i \textit{Container}, \textit{Git$_G$}, \textit{GitHub} e relative funzionalità, API e tecnologie per storicizzazione dei dati.
\end{itemize}

\subsection*{To Do}
L'azienda, oltre a richiedere l'invio da parte del gruppo dei nomi e delle email istituzionali dei componenti, ha comunicato che a breve invierà:
\begin{itemize}
    \item l'invito ai \textit{SAL};
    \item i documenti di integrazione e di informativa sulla privacy, che dovrà ritornare firmata;
    \item link per un sondaggio iniziale.
\end{itemize}



\addDecision{Conferma dell'ordine del giorno proposto per i \textit{SAL}}
\addDecision{Accolta la proposta di utilizzo di \textit{Round Review}}
\addDecision{Approvata la condivisione del codice sorgente}
\addDecision{Confermato incontro di approfondimento sulle tecnologie}
\makeDecisionTable

\addTodo{\noOne}{Comunicare alla Proponente ora e giorno di preferenza per la cadenza dei \textit{SAL}}{F. Venzo}{17/11/2025}
\addTodo{\noOne}{Inviare all'azienda i nominativi e le email istituzionali dei membri del gruppo}{F. Venzo}{17/11/2025}
\addTodo{\noOne}{Fornire all'utente aziendale accesso al repository dei sorgenti}{C. Libralato}{17/11/2025}

\makeTodoTable

\end{document}