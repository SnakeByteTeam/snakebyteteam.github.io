
\documentclass[10pt, letterpaper]{article}
\usepackage{../../../../template/template}

% ------------------------
% Impostazioni documento
% ------------------------

\setDocTitle{Verbale esterno 07/01/2026}
\setDocVersion{0.1.0}
\setDocData{11/01/2026}
\setDocIndex{ve\_2026\_01\_07}
\setDocState{0} % 0=Da verificare
\setDocRecipient{2} % solo interni
\begin{document}

\makeTitlePage

\newpage 

\addChangelog{\docVersion}{\docData}{L. Granziero}{F. Venzo}{\noOne}{\firstDraft}
\makeChangelog


\newpage

\tableofcontents

\newpage

\makeMeetingInfoTable{07/01/2026}{16:10}{17:30}{via \textit{Microsoft Teams$_G$}}
% ------------------------
% Partecipanti
% ------------------------
\addParticipant{Luca}{Granziero}{Responsabile}{P}
\addParticipant{Leonardo}{Pellizzon}{Programmatore}{P}
\addParticipant{Filippo}{Venzo}{Verificatore}{P}
\addParticipant{Christian}{Libralato}{Programmatore}{P}
\addParticipant{Giuseppe}{De Fina}{Verificatore}{P}
\addParticipant{Valeria}{Baleanu}{Amministratore}{P}
\addParticipant{Francesco}{Pasqual}{Programmatore}{P}
\makeMeetingParticipantsTable


\addPrecedentTodo{ve\_2025\_12\_23.a1}{-}{\teamName}{08/01/2026}
\addPrecedentTodo{vi\_2025\_12\_23.a2}{-}{\teamName}{29/12/2025}

\makePrecedentTodoTable


\section{Ordine del giorno}
\begin{itemize}
  \item Pianificazione revisioni RTB;
  \item Stato avanzamento Sprint$_G$ 5;
  \item Implementazione e requisiti minimi del PoC$_G$;
  \item Q\&A tecnico e architetturale.
\end{itemize}

\section{Approfondimento}
\subsection{Pianificazione revisioni RTB}
Nel corso della riunione è stata discussa la pianificazione delle revisioni tecniche e organizzative per la RTB, tenendo conto degli impegni accademici del team e delle disponibilità dei docenti Cardin e Vardanega. È stato concordato di separare la revisione tecnica dalla revisione di marketing, rimandando quest’ultima a una fase successiva.

\subsection{Stato avanzamento Sprint 5}
Il responsabile ha presentato lo stato di avanzamento dello Sprint 5, illustrando la suddivisione dei ruoli, le attività completate e quelle in corso. È stato evidenziato l’avvio dell’implementazione del PoC, con particolare attenzione alla necessità di adattare soluzioni esistenti e di concentrarsi sulle funzionalità chiave.

\subsection{Implementazione e requisiti minimi del PoC}
Viene illustrata l’architettura del server del PoC, basata su un approccio ispirato all’architettura esagonale, ricevendo suggerimenti per migliorare la separazione tra porte e adapter e per mantenere distinta la logica di business.

\subsection{Q\&A tecnico e architetturale}Sono stati infine chiariti diversi dubbi tecnici riguardanti metriche di qualità, utilizzo di Prisma ORM$_G$, requisiti minimi del PoC e uso del logo Vimar.



% ------------------------
% Decisioni
% ------------------------
\addDecision{La revisione tecnica interna per la RTB verrà pianificata indicativamente nella settimana 19--23 gennaio}

\addDecision{Il PoC dovrà includere almeno la gestione dell’impianto, degli allarmi, grafici analitici, un’utenza base e l’integrazione delle Time Series.}
\addDecision{È consentito l’utilizzo dell’ORM Prisma, a condizione che la scelta sia motivata e coerente con i requisiti accademici e infrastrutturali.}
\addDecision{È consentito esclusivamente l’utilizzo del logo Vimar fornito nel capitolato, con esclusione del logo ufficiale aziendale.}
\makeDecisionTable

% ------------------------
% Attività da completare
% ------------------------
\addTodo{\noOne}{Definire la data dell’incontro di revisione tecnica interna (settimana 19--23 gennaio).}{\teamName}{12/01/2026}
\addTodo{\noOne}{Raffinare l’architettura esagonale, con particolare attenzione alla separazione tra porte e adapter.}{\teamName}{22/01/2026}
\addTodo{\noOne}{Revisionare e sistemare i file Dockerfile e docker-compose.}{\teamName}{22/01/2026}

\makeTodoTable


\end{document}

