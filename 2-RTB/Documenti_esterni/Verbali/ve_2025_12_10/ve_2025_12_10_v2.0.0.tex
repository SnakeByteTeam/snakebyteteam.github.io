
\documentclass[10pt, letterpaper]{article}

\usepackage{../../../../template/template}

\setDocTitle{Verbale esterno 10/12/2025}
\setDocVersion{2.0.0}
\setDocData{26/01/2026}
\setDocIndex{ve\_2025\_12\_10}
\setDocState{2}
\setDocRecipient{2}

% comandi extra (rimuovere questi commenti una volta approvato il documento)
%
% -- costanti --
% \noOne [-]
% \noRole [ND]
% \firstDraft [Prima stesura]
% \approval [Approvazione]
% \teamName [SnakeByte]
% \teamEmail [snakebyteteam@gmail.com]
% \teamRepo [https://github.com/SnakeByteTeam/snakebyteteam.github.io]
% \teamSite [https://snakebyteteam.github.io/]
% \recipientTeachers [prof. Vardanega Tullio, prof. Cardin Riccardo]
% \vimar [Vimar]
% \vimarspa [Vimar S.p.A.]
%
% -- variabili --
% \docIndex | \setDocIndex{nome documento senza versione}
% \docTitle | \setDocTitle{titolo documento} 
% \docData | \setDocData{data documento} [GG/MM/AAAA]
% \docVersion | \setDocVersion{versione documento}
% \docState | \setDocState{0|1|2} [Da verificare][Verificato][Approvato]
% \docRecipient [table] | \setDocRecipient{0|1|2} [SnakeByte][SnakeByte, prof][SnakeByte, prof, vimar]
% 
% -- tabelle -- [gli id sono automatici dove serve]
% \makeMeetingInfoTable{data}{ora inizio}{ora fine}{modalità}
% \addParticipant{nome}{cognome}{ruolo}{presenza} | \makeMeetingParticipantsTable
% \addTodo{id github issue}{descrizione}{assegnatario}{scadenza} | \makeTodoTable
% \addPrecedentTodo{id todo}{id github issue todo}{assegnatario todo}{data} | \makePrecedentTodoTable
% \addDecision{descrizione decisione} | \makeDecisionTable
%
% -- pagine intere --
% \makeTitlePage
% \addChangelog{versione}{data}{autore}{verificatore}{approvatore}{descrizione} | \makeTodoTable

\begin{document}

\makeTitlePage

\newpage

\addChangelog{2.0.0}{26/01/2026}{\noOne}{\noOne}{L. Pellizzon, M. Sciacco}{\approval}
\addChangelog{1.0.1}{26/01/2026}{L. Pellizzon}{F. Pasqual}{\noOne}{Modifica e correzione dello stato del documento}
\addChangelog{1.0.0}{16/01/2026}{\noOne}{\noOne}{V. Baleanu, M. Sciacco}{Approvazione interna ed esterna del documento}
\addChangelog{0.1.0}{19/12/2025}{C. Libralato}{F. Pasqual}{\noOne}{\firstDraft}
\makeChangelog

\newpage

\tableofcontents

\newpage

\makeMeetingInfoTable{10/12/2025}{16:00}{17:00}{via \textit{Microsoft Teams$_G$}}

\addParticipant{Francesco}{Pasqual}{Responsabile}{P}
\addParticipant{Christian}{Libralato}{Amministratore}{P}
\addParticipant{Luca}{Granziero}{Progettista}{P}
\addParticipant{Leonardo}{Pellizzon}{Verificatore}{P}
\addParticipant{Valeria}{Baleanu}{Analista}{P}
\addParticipant{Filippo}{Venzo}{Analista}{P}
\addParticipant{Giuseppe}{De Fina}{Analista}{P}
\makeMeetingParticipantsTable

\addPrecedentTodo{ve\_2025\_11\_27.a1}{\#18}{C. Libralato, V. Baleanu, G. De Fina, F. Pasqual}{10/12/2025}
\addPrecedentTodo{ve\_2025\_11\_27.a2}{\noOne}{\teamName}{10/12/2025}
\addPrecedentTodo{ve\_2025\_11\_27.a3}{\noOne}{L. Pellizzon}{10/12/2025}
\addPrecedentTodo{ve\_2025\_11\_27.a4}{\noOne}{F. Pasqual}{09/12/2025}

\makePrecedentTodoTable


\section{Ordine del giorno}
\begin{itemize}
    \item Stato avanzamento lavori;
    \item Q\&A riguardo l'AdR;
    \item consigli;
    \item chiarimenti sul \textit{PoC$_G$}.
\end{itemize}

\section{Approfondimento}
\subsection*{Stato avanzamento lavori}
Il team ha elencato le attività portate a termine nell'ultimo periodo e quelle in corso d'opera, in riferimento alla
documentazione, in particolare all'Analisi dei Requisiti e al Piano di Progetto, e a questioni più tecniche come
la configurazione del kit fornito e l'autenticazione con \textit{KNX 3rd Party API$_G$}.

\subsection*{Q\&A riguardo l'AdR}
Dalla risoluzione dei dubbi riguardo l'Analisi dei Requisiti è emerso che:
\begin{itemize}
    \item il database interno, atto alla permanenza dei dati, non può essere considerato attore secondario in quanto interno al sistema, è utile invece identificare
          come attori esterni al sistema le entità che inviano notifiche o segnali, che in base all'interpretazione di alto livello possono
          coincidere con un paziente, con il \textit{Vimar Cloud}, o con l'\textit{API KNK IoT}.
    \item la scelta tra il mantenimento di \textit{casi d'uso$_G$} aventi solo estensioni o la loro disgregazione nelle singole parti
          può dipendere dal contesto, a tal fine è importante porre l'attenzione sull'obiettivo dell'attore e le pre e post condizioni del \textit{caso d'uso} aggregato. L'approccio è utile
          per raggruppare concettualmente più operazioni, tuttavia in merito a ciò è consigliato di attendere anche il feedback del prof. Cardin.
    \item la distinzione tra i \textit{casi d'uso} legati alla visualizzazione delle Analytics e dei Dispositivi è corretta in quanto le due sezioni soddisfano obiettivi diversi.
\end{itemize}

\subsection*{Consigli}
Per quanto concerne i diagrammi dei \textit{casi d'uso} è stato consigliata la realizzazione di diagrammi
unici e completi, senza avere diagrammi separati per ogni sotto-caso che non provochi flussi alternativi rilevanti.\\\\
Riguardo la visualizzazione delle Analytics è stato caldamente suggerito di non sviluppare le grafiche dal principio ma di usare
template e temi gratuiti pre-esistenti al fine di ottenere soluzioni robuste e verificate senza aumentare notevolmente il carico di lavoro.\\\\
È infine emerso il concetto di reparto (o gruppo logico), a cui associare utenti e impianti. Questo metodo di gestione offre maggior flessibilità
e facilita la gestione di notifiche, allarmi e Analytics.

\subsection*{Chiarimenti sul \textit{PoC}}
Il team ha espresso alcuni dubbi riguardanti l'essenza, dal punto di vista pratico, del Proof of Concept che sono stati risolti con una breve digressione.

\addDecision{Utilizzo di template e temi esistenti per la grafica delle Analytics.}
\makeDecisionTable

\addTodo{\noOne}{Sostenere il colloquio con il prof. Cardin}{\teamName}{17/10/2025}
\addTodo{\noOne}{Ottenere approvazione del verbale esterno del 27/11/2025 su RoundReview}{F. Pasqual}{20/10/2025}
\makeTodoTable

\end{document}
