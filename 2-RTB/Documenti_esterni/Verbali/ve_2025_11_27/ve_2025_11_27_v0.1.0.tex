\documentclass[10pt, letterpaper]{article}

\usepackage{../../../../template/template}

\setDocTitle{Verbale esterno 27/11/2025}
\setDocVersion{0.1.0}
\setDocData{27/11/2025}
\setDocIndex{ve\_2025\_11\_27}
\setDocState{1}
\setDocRecipient{2}

% comandi extra (rimuovere questi commenti una volta approvato il documento)
%
% -- costanti --
% \noOne [-]
% \noRole [ND]
% \firstDraft [Prima stesura]
% \approval [Approvazione]
% \teamName [SnakeByte]
% \teamEmail [snakebyteteam@gmail.com]
% \teamRepo [https://github.com/SnakeByteTeam/snakebyteteam.github.io]
% \teamSite [https://snakebyteteam.github.io/]
% \recipientTeachers [prof. Vardanega Tullio, prof. Cardin Riccardo]
% \vimar [Vimar]
% \vimarspa [Vimar S.p.A.]
%
% -- variabili --
% \docIndex | \setDocIndex{nome documento senza versione}
% \docTitle | \setDocTitle{titolo documento} 
% \docData | \setDocData{data documento} [GG/MM/AAAA]
% \docVersion | \setDocVersion{versione documento}
% \docState | \setDocState{0|1|2} [Da verificare][Verificato][Approvato]
% \docRecipient [table] | \setDocRecipient{0|1|2} [SnakeByte][SnakeByte, prof][SnakeByte, prof, vimar]
% 
% -- tabelle -- [gli id sono automatici dove serve]
% \makeMeetingInfoTable{data}{ora inizio}{ora fine}{modalità}
% \addParticipant{nome}{cognome}{ruolo}{presenza} | \makeMeetingParticipantsTable
% \addTodo{id github issue}{descrizione}{assegnatario}{scadenza} | \makeTodoTable
% \addPrecedentTodo{id todo}{id github issue todo}{assegnatario todo}{data} | \makePrecedentTodoTable
% \addDecision{descrizione decisione} | \makeDecisionTable
%
% -- pagine intere --
% \makeTitlePage
% \addChangelog{versione}{data}{autore}{verificatore}{approvatore}{descrizione} | \makeTodoTable

\begin{document}

\makeTitlePage

\newpage

\addChangelog{\docVersion}{\docData}{F. Pasqual}{L. Pellizzon}{\noOne}{\firstDraft}
\makeChangelog

\newpage

\tableofcontents

\newpage

\makeMeetingInfoTable{27/11/2025}{16:00}{17:30}{via \textit{Microsoft Teams$_G$}}

\addParticipant{Giuseppe}{De Fina}{Responsabile}{P}
\addParticipant{Francesco}{Pasqual}{Amministratore}{P}
\addParticipant{Leonardo}{Pellizzon}{Progettista}{P}
\addParticipant{Valeria}{Baleanu}{Verificatore}{P}
\addParticipant{Filippo}{Venzo}{Analista}{P}
\addParticipant{Christian}{Libralato}{Analista}{P}
\addParticipant{Luca}{Granziero}{Analista}{P}
\makeMeetingParticipantsTable

\addPrecedentTodo{ve\_2025\_11\_12.a1}{\noOne}{F. Venzo}{17/11/2025}
\addPrecedentTodo{ve\_2025\_ 11\_12.a2}{\noOne}{F. Venzo}{17/11/2025}
\addPrecedentTodo{ve\_2025\_11\_12.a3}{\noOne}{C. Libralato}{17/11/2025}
\makePrecedentTodoTable

\section{Ordine del giorno}
\begin{itemize}
    \item Stato avanzamento lavori e discussione del \textit{Gitflow};
    \item valutazione strumenti di tracciamento delle attività;
    \item Analisi dei Requisiti e proposta di revisione anticipata;
    \item Q\&A tecnico-funzionale: autenticazione, permessi, allarmi, gestione impianti, dashboard e interfaccia.
    \item valutazioni su \textit{analytics} e intelligenza artificiale;
    \item debugging autenticazione \textit{KNX IoT 3rd party API$_G$};
    \item logistica hardware: consegna del secondo \textit{kit di impianto portatile Smart$_G$}. 
\end{itemize}

\section{Approfondimento}

\subsection*{Stato avanzamento lavori e discussione del \textit{Gitflow}}
Il team ha illustrato i progressi svolti, che includono l'aggiornamento dell'Analisi dei Requisiti (ancora in corso), 
la definizione delle responsabilità operative e l'aggiornamento delle Norme di Progetto. È stato inoltre creato un template \textit{LaTeX$_G$} per uniformare 
e velocizzare la redazione dei documenti. \\
Riguardo al \textit{workflow} di versionamento:
\begin{itemize}
    \item il gruppo sta attualmente utilizzando per i documenti un flusso ispirato a \textit{Gitflow} che prevede branch di modifica, develop, release 
    (per l'accumulo dei documenti approvati) e main (per la pubblicazione finale);
    \item per la gestione del codice sorgente futuro, la proponente ha consigliato di valutare alternative più semplici rispetto al classico \textit{Gitflow}, 
    che può risultare complesso e incline a conflitti di merge se non gestito con esperienza. 
    È stato suggerito di considerare il \textit{Feature Branch Flow} (solo main e feature branches, gestendo i rilasci tramite tag sul main), 
    mantenendo il processo il più lineare possibile.
\end{itemize}


\subsection*{Valutazione strumenti \textit{project management}}
Il team è in fase di valutazione tra l'utilizzo di \textit{Jira$_G$} e \textit{GitHub Projects$_G$}, con il dubbio principale legato al rapporto tra i benefici offerti 
e il tempo necessario per il setup e la creazione di automazioni.
E' stato suggerito di assegnare a due membri del team un'analisi comparativa dei due strumenti,
al fine di evidenziare pro e contro di ciascuna soluzione e convergere a una decisione finale che eviti troppi oneri nella gestione delle attività.\\
Per il processo di revisione, si è discusso l'uso dei commenti nelle \textit{Pull Request$_G$} come strumento di tracciamento 
delle correzioni; l'approccio è stato validato positivamente, a patto di trovare un metodo efficace per riferire e storicizzare tali commenti.

\subsection*{Analisi dei Requisiti e proposta di revisione anticipata}
Il documento di Analisi dei Requisiti risulta aggiornato ma ancora in fase di lavorazione. Per garantire la qualità dell'elaborato e prevenire criticità riscontrate in progetti degli anni passati, 
è stata concordata una revisione anticipata. \\
Il team si impegnerà per inviare una bozza del documento entro il 05/12/2025.


\subsection*{Q\_\&A tecnico-funzionale: autenticazione, permessi, allarmi, gestione impianti}
Per quanto riguarda l'autenticazione e i permessi:
\begin{itemize}
    \item è stata stabilita la necessità di disaccoppiare i permessi. Gli utenti finali (OSS) non devono dipendere
    dall'account \textit{MyVimar} dell'amministratore di impianto;
    \item la piattaforma dovrà gestire un proprio layer di permessi interno, 
    dove l'amministratore può creare, gestire e assegnare reparti/strutture agli Operatori Socio-Sanitari (OSS).
\end{itemize}
Per quanto riguarda gli allarmi e le loro priorità:
\begin{itemize}
    \item il team ha ricevuto conferma che solo dispositivi specifici (pulsanti, sensori di caduta, termostati) generano allarmi, mentre l'attuazione (luci, tapparelle) no;
    \item si è definito che l'evento "Gateway Offline" deve essere considerato un allarme prioritario, in quanto nega la possibilità di ricevere notifiche in cloud;
    \item la proponente ha suggerito di rendere le soglie di allarme e le logiche di priorità (es. in base alla frequenza di attivazione) parametrizzabili da parte dell'amministratore.
\end{itemize}
Per quanto riguarda la gestione degli impianti:
\begin{itemize}
    \item è stato chiarito che la piattaforma non deve fornire la possibilità di modificare la struttura fisica degli impianti (aggiungere o togliere stanze/dispositivi);
    \item l'amministratore dovrà, al primo accesso, selezionare quali impianti gestire tra quelli disponibili tramite l'account \textit{KNX IoT}.
\end{itemize}
Per quanto riguarda la dashboard e l'interfaccia:
\begin{itemize}
    \item è auspicabile che la dashboard permetta anche di prendere in carico gli allarmi (azione \textit{nice to have});
    \item si è concordato di definire chiaramente i requisiti della dashboard prima di valutare la mole di lavoro necessaria per una eventuale personalizzazione dei widget;
    \item per la visualizzazione dei dispositivi, l'obiettivo è una mappa topologica (a box annidati) per una migliore esperienza utente, 
    anche se una semplice tabella con filtri è accettata come alternativa se i tempi non lo permetteranno.
\end{itemize}

\subsection*{Valutazioni su \textit{analytics} e intelligenza artificiale}
E' stato fortemente sconsigliato l'uso di \textit{Large Language Models} (LLM) in locale.
La parte di suggerimenti e \textit{analytics} dovrà essere gestita tramite logiche statiche, soglie predefinite e algoritmi di \textit{AI} semplici.

\subsection*{Logistica hardware: consegna del secondo \textit{kit di impianto portatile Smart}}
La proponente è stata informata del fatto che F. Pasqual sarà il responsabile del secondo kit di sensori. 
Privatamente, seguiranno contatti per l'organizzazione della consegna.


\addDecision{Mantenere il flusso Git il più semplice possibile; il team valuterà l'adozione del \textit{Feature Branch Flow} al posto del più complesso \textit{Gitflow} per il codice sorgente}
\addDecision{Adottare un sistema di autenticazione e permessi disaccoppiato da \textit{MyVimar}, con l'amministratore della piattaforma responsabile della gestione degli utenti (OSS) interni}
\addDecision{La piattaforma non deve consentire la modifica della struttura fisica dell'impianto. L'amministratore dovrà selezionare gli impianti da monitorare}
\addDecision{Il "Gateway Offline" è da considerarsi un allarme ad alta priorità. Le logiche di priorità devono essere parametrizzabili}
\addDecision{Definire chiaramente i requisiti della dashboard}
\addDecision{Impostare logiche statiche e soglie predefinite per la creazione di \textit{analytics} e suggerimenti}
\addDecision{L'obiettivo di design è una visualizzazione topologica degli impianti e dei loro dispositivi}
\makeDecisionTable

\addTodo{\#18}{Inviare bozza dell'Analisi dei requisiti}{C. Libralato, V. Baleanu, G. De Fina, F. Pasqual}{05/12/2025}
\addTodo{\noOne}{Decisione definitiva sul sistema di tracciamente dello attività di progetto}{\teamName}{01/12/2025}
\addTodo{\noOne}{Correggere l'implementazione dell'autenticazione API Vimar}{L. Pellizzon}{07/12/2025}
\addTodo{\noOne}{Contattare il referente di \vimarspa per la consegna del secondo kit hardware}{F. Pasqual}{28/11/2025}
\makeTodoTable




\end{document}