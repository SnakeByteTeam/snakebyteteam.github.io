
\documentclass[10pt, letterpaper]{article}

\usepackage{../../../../template/template}

\setDocTitle{Verbale esterno 23/12/2025}
\setDocVersion{0.1.0}
\setDocData{28/12/2025}
\setDocIndex{ve\_2025\_12\_23}
\setDocState{1}
\setDocRecipient{2}

% comandi extra (rimuovere questi commenti una volta approvato il documento)
%
% -- costanti --
% \noOne [-]
% \noRole [ND]
% \firstDraft [Prima stesura]
% \approval [Approvazione]
% \teamName [SnakeByte]
% \teamEmail [snakebyteteam@gmail.com]
% \teamRepo [https://github.com/SnakeByteTeam/snakebyteteam.github.io]
% \teamSite [https://snakebyteteam.github.io/]
% \recipientTeachers [prof. Vardanega Tullio, prof. Cardin Riccardo]
% \vimar [Vimar]
% \vimarspa [Vimar S.p.A.]
%
% -- variabili --
% \docIndex | \setDocIndex{nome documento senza versione}
% \docTitle | \setDocTitle{titolo documento} 
% \docData | \setDocData{data documento} [GG/MM/AAAA]
% \docVersion | \setDocVersion{versione documento}
% \docState | \setDocState{0|1|2} [Da verificare][Verificato][Approvato]
% \docRecipient [table] | \setDocRecipient{0|1|2} [SnakeByte][SnakeByte, prof][SnakeByte, prof, vimar]
% 
% -- tabelle -- [gli id sono automatici dove serve]
% \makeMeetingInfoTable{data}{ora inizio}{ora fine}{modalità}
% \addParticipant{nome}{cognome}{ruolo}{presenza} | \makeMeetingParticipantsTable
% \addTodo{id github issue}{descrizione}{assegnatario}{scadenza} | \makeTodoTable
% \addPrecedentTodo{id todo}{id github issue todo}{assegnatario todo}{data} | \makePrecedentTodoTable
% \addDecision{descrizione decisione} | \makeDecisionTable
%
% -- pagine intere --
% \makeTitlePage
% \addChangelog{versione}{data}{autore}{verificatore}{approvatore}{descrizione} | \makeTodoTable

\begin{document}

\makeTitlePage

\newpage

\addChangelog{\docVersion}{\docData}{C. Libralato}{F. Pasqual}{\noOne}{\firstDraft}
\makeChangelog

\newpage

\tableofcontents

\newpage

\makeMeetingInfoTable{23/12/2025}{9:30}{10:30}{via \textit{Microsoft Teams$_G$}}

\addParticipant{Christian}{Libralato}{Responsabile}{P}
\addParticipant{Leonardo}{Pellizzon}{Analista}{P}
\addParticipant{Giuseppe}{De Fina}{Amministratore}{P}
\addParticipant{Valeria}{Baleanu}{Progettista}{P}
\addParticipant{Luca}{Granziero}{Programmatore}{P}
\addParticipant{Filippo}{Venzo}{Programmatore}{P}
\addParticipant{Francesco}{Pasqual}{Verificatore}{P}
\makeMeetingParticipantsTable

\addPrecedentTodo{ve\_2025\_12\_10.a1}{\noOne}{\teamName}{17/12/2025}

\makePrecedentTodoTable

\section{Ordine del giorno}
\begin{itemize}
    \item Stato avanzamento lavori;
    \item Q\&A;
    \item stato del \textit{PoC$_G$};
    \item stato dell'Analisi dei Requisiti.
\end{itemize}

\section{Approfondimento}
\subsection*{Stato avanzamento lavori}
Sono state presentate le attività terminate nell'ultimo periodo di avanzamento e le attività da terminare, la quasi totalità
di esse concerneva la documentazione.

\subsection*{Q\&A}
I dubbi e le domande si sono concentrate sui seguenti argomenti:
\begin{itemize}
    \item \textbf{Metriche di avanzamento}: nel contesto del Piano di Qualifica il team ha esposto alcuni dubbi legati alla specificità
    delle metriche per misurare l'avanzamento ed ha espresso la preferenza di ricercare un numero contenuto di metriche, circa 5, significative e di alto livello
    che si dimostrino utili e comprensibili.
    \item \textbf{Gestione ore}: dal consuntivo dei primi \textit{sprint$_G$} è emersa una differenza tra le ore programmate e quelle reali svolte da ogni ruolo. È dunque sorto il dubbio
    relativo alla gestione delle ore non consumate, il gruppo ha proposto di mantenere un contatore d'accumulo, da cui attingere in caso
    di sottostime orarie future. In aggiunta a ciò, il team ha comunicato che spesso è stato necessario effettuare variazioni dei ruoli previsti nello sprint successivo.
    La Proponente ha espresso un'opinione positiva riguardo al metodo proposto per la gestione delle ore in eccesso, e, in relazione al secondo problema, qualora le previsioni siano molto diverse dalla realtà,
    ha suggerito di effettuare una revisione completa della distribuzione dei ruoli considerando eventualmente anche le ore in eccesso. In ogni caso è stato suggerito un confronto
    con il prof. Vardanega per trattare queste tematiche.
    \item \textbf{Duplicazione Amministratore}: dato che nel periodo corrente l'Amministratore ha il compito di redigere i Piani di Progetto e di Qualifica, il team ha proposto una duplicazione del ruolo Amministratore
    per il periodo rimanente dello \textit{sprint} ed eventualmente anche per il periodo successivo. Questa variazione porterebbe alla temporanea assenza della figura del
    Progettista. Anche in questo caso è stato suggerito un confronto con il prof. Vardanega in quanto, sebbene la presenza di più amministratori non sia comune, potrebbe rivelarsi utile per il caso in esame.
\end{itemize}

\subsection*{Stato del \textit{PoC}}
Riguardo al \textit{PoC} sono emerse alcune difficoltà riguardanti la traduzione dalla teoria alla pratica nel contesto delle architetture e alcuni dubbi sul grado di riutilizzabilità dello stesso.\\\\
In riferimento all'implementazione è stato consigliato di sperimentare con le API reali senza ricorrere a \textit{Mocking$_G$} per esse, se non in casi bloccanti,  per i quali la Proponente si è resa 
disponibile ad offrire supporto in caso di necessità.\\\\
È inoltre stata ricordata l'importanza dello sviluppo di una micro-libreria, come requisito \textit{nice-to-have}, per disaccoppiare la logica applicativa dalle API e fornire una semantica di alto livello, al fine di 
favorire la riutilizzabilità, una maggior separazione e un potenziale rilascio open-source.\\\\
A livello di prospettive future è stata definita una possibile presentazione del \textit{PoC} a fine gennaio, che coinvolgerebbe varie componenti aziendali tra cui Responsabile sviluppo software, Tutor, Team cloud, ed eventualmente il reparto marketing, 
al fine di avere un feedback tecnico, architetturale e di valore di business.

\subsection*{Stato dell'Analisi dei Requisiti}
Il ciclo di vita degli allarmi è stato definito come segue:
\begin{itemize}
    \item Creazione: effettuata dall'amministratore, prevede salvataggio della configurazione e la creazione della subscription;
    \item Attivazione: l'evento è inviato dal sensore all'endpoint registrato, il contesto viene ricostruito e vengono identificati i destinatari;
    \item Notifica: è distinta dall'allarme e viene inviata ai destinatari definiti dall'area;
    \item Risoluzione: l'operatore prende in carico l'allarme e lo stato viene aggiornato per tutti gli altri;
    \item Riattivazione: l'allarme si riattiva in automatico e può nuovamente scattare.
\end{itemize}
È poi stata discussa la gestione temporale degli allarmi, tramite semplice controllo manuale o con la possibilità di definire fasce orarie di attivazione per aumentare l'automazione, 
quest'ultima opzione è preferibile.\\\\
Da una successiva analisi dei permessi tra ruoli per le sezioni Analytics e Dashboard sono emerse le seguenti considerazioni:
\begin{itemize}
    \item le Analytics sono visibili a tutti ma la configurazione è riservata all'amministratore;
    \item la gestione del layout della Dashboard è analoga, l'amministratore determina widget e disposizione mentre gli operatori attingono passivamente alle informazioni.
\end{itemize}
Infine è stata trattata la gestione dei requisiti non funzionali, la cui inclusione è stata consigliata per completezza documentale.

\addDecision{Struttura del ciclo di vita degli allarmi confermat.}
\addDecision{Permessi dei ruoli per le sezioni Analytics e Dashboard.}
\addDecision{Inclusione requisiti non funzionali nell'Analisi dei Requisiti.}

\makeDecisionTable

\addTodo{\noOne}{Chidere conferma al prof. Vardanega riguardo la duplicazione Amministratore.}{\teamName}{11/01/2026}
\addTodo{\noOne}{Effettuare revisione della distribuzione ruoli e ore.}{\teamName}{11/01/2026}

\makeTodoTable

\end{document}
