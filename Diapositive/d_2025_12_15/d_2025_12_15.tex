\documentclass{beamer}

\usepackage{../tema}

\title{Diario di bordo}
\author[F. Pasqual, L. Pellizzon]{Francesco Pasqual, Leonardo Pellizzon}
\setDate{15/12/2025}

\begin{document}

{ 
  \setbeamertemplate{footline}{}
  \begin{frame}
    \titlepage
  \end{frame}
}

\begin{frame}{Finora}
  In ordine di tempo:
  \begin{enumerate}
    \item Riunione interna per retrospettiva sprint concluso e avanzamento
    \item Riunione interna per allineamento su \textit{use case} tra analisti 
    \item Continuazione Analisi dei Requisiti e consegna bozza del documento alla proponente
    \item Presentazione del SAL con la proponente il 10/12/2025
    \item Stesura e prime compilazioni del documento di Piano di Progetto
  \end{enumerate}
\end{frame}

\begin{frame}{Cosa faremo}
  In ordine di tempo:
  \begin{enumerate}
    \item Continuare l'Analisi dei Requisiti
    \item Ricevimento con il prof. Cardin
    \item Riunione interna di retrospettiva e avanzamento
    \item Analisi approfondita delle tecnologie per sviluppare il PoC
  \end{enumerate}
\end{frame}


\begin{frame}{Difficoltà e dubbi incontrati}
  \begin{center}
  \begin{itemize}
    \item Redazione della bozza del documento di Analisi dei Requisiti: modalità asincrona e terminologia differente
    \item Creazione \textit{use case} per alcuni dei requisiti obbligatori: dubbi su possibili ridondanze
    \item Piano di Progetto: valutazione dei rischi e di come esporli nel documento
  \end{itemize}
  \end{center}
\end{frame}

\end{document}
