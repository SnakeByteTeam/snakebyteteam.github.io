\documentclass{beamer}
\usepackage[italian]{babel}
\usepackage{multicol}

\usepackage{tikz}
\usepackage{graphicx}
\usepackage[table, x11names]{xcolor}
\usepackage{xstring}
\usepackage{etoolbox}
\usepackage{tabularx}
\usepackage[export]{adjustbox}
\usepackage{eurosym}

\newcommand{\Date}{}

\newcommand{\setDate}[1]{\renewcommand{\Date}{#1}}

\useinnertheme{default}
\useoutertheme{default}
\usecolortheme{default}

\definecolor{aziendaPrimary}{RGB}{84, 150, 104}
\setbeamercolor{structure}{fg=aziendaPrimary}
\setbeamercolor{frametitle}{bg=aziendaPrimary!10,fg=aziendaPrimary}

% rimuovere i simboli di navigazione
\setbeamertemplate{navigation symbols}{}


% pagina iniziale
\setbeamertemplate{title page}{
  \vspace{4em}
  \begin{center}
    \hspace{1.5cm}
    \begin{minipage}{0.3\textwidth}
      \centering
      \includegraphics[height=2.5cm, right]{../logo-crop.pdf}
    \end{minipage}
    \begin{minipage}{0.3\textwidth}
      \centering
      \includegraphics[height=2.5cm, left]{img/logounipd.png}
    \end{minipage}
  \end{center}
  \begin{center}
    \vspace{0.5em}
      {\usebeamerfont{title}\color{black}\textbf{\inserttitle}\par}
    \vspace{0.7em}
    {\usebeamerfont{date}\insertauthor\par}
    \vspace{0.5cm}
    {\usebeamerfont{date}\Date\par}
  \end{center}
}

% footer
\setbeamertemplate{footline}{
  \vspace{2ex}
  \hbox{
    \begin{minipage}[t]{0.40\paperwidth}
      \hspace*{0.5em}
      \raggedright\usebeamerfont{footline}{Presentazione RTB XX/02/2026}
    \end{minipage}%
    \begin{minipage}[t]{0.20\paperwidth}
      \centering\usebeamerfont{footline}Slide \insertframenumber{} di \inserttotalframenumber
    \end{minipage}%
    \begin{minipage}[t]{0.40\paperwidth}
      \hfill
      \insertshortauthor
      \hspace*{0.8em}
      \raisebox{-0.25\height}{\includegraphics[width=0.7cm]{../logo-crop.pdf}}
      \hspace*{1.3em}
    \end{minipage}
  }
  \vspace{0.8ex}
}

\usepackage[absolute,overlay]{textpos}

\title{Presentazione RTB}
\author[]{Gruppo 1 - A.A 2025/2026}
\setDate{XX/02/2026}

\begin{document}
{
\setbeamertemplate{footline}{}
\begin{frame}
  \titlepage
\end{frame}
}

\begin{frame}{Obiettivo capitolato}
\begin{itemize}
  \item Capitolato C9 (Vimar View4Life): Piattaforma per la gestione degli impianti Smart nelle residenze protette;
  \item Sistema di gestione allarmi e utenti (OSS e Amministrazione) che possono interagire con gli allarmi (gestione e risoluzione)
  \item Visualizzazione di statistiche sui consumi, con relativi consigli per ridurli, e sugli allarmi.
\end{itemize}
\end{frame}

\begin{frame}{Risultati dell'Analisi dei Requisiti (1/2)} 
    \begin{center}
        \begin{tabular}{|c|c|c|c|c|}
            \hline
            \textbf{Tipologia} & \textbf{Obbligatori} & \textbf{Desiderabili} & \textbf{Opzionali} & \textbf{Totali} \\
            \hline
            Funzionali & 87 & 16 & 8 & 111 \\
            \hline
            Di qualità & X & X & X & X \\
            \hline
            Di vincolo & X & X & X & X \\
            \hline
        \end{tabular}
    \end{center}
\end{frame}

\begin{frame}{Risultati dell'Analisi dei Requisiti (2/2)} 
    
\end{frame}

\begin{frame}{Modifiche migliorative attuate} 
    \begin{itemize}
      \item Definizione tramite \textit{template} della struttura dei documenti;
      \item Processo di verifica e approvazione mediato da \textit{pull request};
      \item Ripianificazione della distribuzione oraria;
      \item Aggiunta nuova \text{Github Action} per controllo automatico di errori ortografici nei documenti;
      \item Slittamento della data di consegna prevista al 5/04/2026;
    \end{itemize}
\end{frame}

\begin{frame}{Modifiche migliorative in corso} 
  In seguito alla revisione della TB:
  \begin{itemize}
    \item modifica dell'Analisi dei Requisiti;
    \item rivalutazione della tecnologia scelta per la codifica della componente \textit{backend}
  \end{itemize}
\end{frame}

\begin{frame}{Autovalutazione}
  \begin{itemize}
    \item Il gruppo \textit{SnakeByte} si ritiene perlopiù soddisfatto del lavoro svolto;
    \item Le criticità maggiori derivano dalla poca esperienza dei membri di fare delle stime orarie corrette;
    \item Numerose ripianificazioni dei ruoli ricoperti durante gli sprint;
    \item Valutiamo molto positivamente la quantità di riunione avvenute con la proponente.
  \end{itemize}
\end{frame}

\begin{frame}[shrink=10]{Consuntivo di periodo attuale}
  \vspace{1cm}
  \begin{center}
        \begin{tabular}{|c|c|c|c|c|c|c|c|}
           \rowcolor{lightgray}
            \hline
            & \textbf{Resp.} & \textbf{Amm.} & \textbf{Ana.} & \textbf{Proge.} & \textbf{Progr.} & \textbf{Ver.} & \textbf{Totale} \\
            \hline
            \cellcolor{lightgray} V. Baleanu & 8 & 16 & 13 & 0 & 0 & 17 & 54\\
            \hline
            \cellcolor{lightgray} G. De Fina & 8 & 20 & 15 & 0 & 4 & 10 & 57 \\
            \hline
            \cellcolor{lightgray} L. Granziero & 9 & 0 & 11 & 8 & 11 & 5 & 44 \\
            \hline 
            \cellcolor{lightgray} C. Libralato & 9 & 18 & 6 & 0 & 11 & 0 & 44\\
            \hline
            \cellcolor{lightgray} F. Pasqual  & 8 & 14 & 5 & 0 & 10 & 18 & 55\\
            \hline
            \cellcolor{lightgray} L. Pellizzon & 8 & 0 & 20 & 12 & 8 & 15 & 63\\
            \hline
            \cellcolor{lightgray} F. Venzo & 6 & 7 & 16 & 0 & 15 & 6 & 50 \\
            \hline
            \cellcolor{lightgray}
            \textbf{Totale}  & 56 & 75 & 86 & 20 & 59 & 71 & 367 \\
            \hline
        \end{tabular}
    \end{center}
\end{frame}

\begin{frame}[shrink=10]{Preventivo a finire} 
  \vspace{1cm}
  \begin{center}
        \begin{tabular}{|c|c|c|}
            \hline
            \rowcolor{lightgray} \textbf{Ruolo} & \textbf{Ore rimanenti} & \textbf{Budget rimanente} \\
            \hline
            Responsabile & 22 & 660 \euro\\
            \hline
            Amministratore & 1 & 20 \euro \\
            \hline
            Analista & -3 & -75 \euro\\
            \hline 
            Progettista & 91 & 2275 \euro\\
            \hline
            Programmatore  & 83 & 1245 \euro\\
            \hline
            Verificatore & 71 & 1065 \euro\\
            \hline
            \cellcolor{lightgray} \textbf{Totale}  & 265  & 5190 \euro\\
            \hline
        \end{tabular}
    \end{center}
\end{frame}

\end{document}