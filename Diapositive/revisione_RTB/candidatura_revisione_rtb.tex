\documentclass{beamer}
\usepackage[italian]{babel}
\usepackage{multicol}

\usepackage{tikz}
\usepackage{graphicx}
\usepackage[table, x11names]{xcolor}
\usepackage{xstring}
\usepackage{etoolbox}
\usepackage{tabularx}
\usepackage[export]{adjustbox}
\usepackage{eurosym}

\newcommand{\Date}{}

\newcommand{\setDate}[1]{\renewcommand{\Date}{#1}}

\useinnertheme{default}
\useoutertheme{default}
\usecolortheme{default}

\definecolor{aziendaPrimary}{RGB}{84, 150, 104}
\setbeamercolor{structure}{fg=aziendaPrimary}
\setbeamercolor{frametitle}{bg=aziendaPrimary!10,fg=aziendaPrimary}

% rimuovere i simboli di navigazione
\setbeamertemplate{navigation symbols}{}


% pagina iniziale
\setbeamertemplate{title page}{
  \vspace{4em}
  \begin{center}
    \hspace{1.5cm}
    \begin{minipage}{0.3\textwidth}
      \centering
      \includegraphics[height=2.5cm, right]{../logo-crop.pdf}
    \end{minipage}
    \begin{minipage}{0.3\textwidth}
      \centering
      \includegraphics[height=2.5cm, left]{img/logounipd.png}
    \end{minipage}
  \end{center}
  \begin{center}
    \vspace{0.5em}
      {\usebeamerfont{title}\color{black}\textbf{\inserttitle}\par}
    \vspace{0.7em}
    {\usebeamerfont{date}\insertauthor\par}
    \vspace{0.5cm}
    {\usebeamerfont{date}\Date\par}
  \end{center}
}

% footer
\setbeamertemplate{footline}{
  \vspace{2ex}
  \hbox{
    \begin{minipage}[t]{0.40\paperwidth}
      \hspace*{0.5em}
      \raggedright\usebeamerfont{footline}{Presentazione RTB 27/02/2026}
    \end{minipage}%
    \begin{minipage}[t]{0.20\paperwidth}
      \centering\usebeamerfont{footline}Slide \insertframenumber{} di \inserttotalframenumber
    \end{minipage}%
    \begin{minipage}[t]{0.40\paperwidth}
      \hfill
      \insertshortauthor
      \hspace*{0.8em}
      \raisebox{-0.25\height}{\includegraphics[width=0.7cm]{../logo-crop.pdf}}
      \hspace*{1.3em}
    \end{minipage}
  }
  \vspace{0.8ex}
}

\usepackage[absolute,overlay]{textpos}

\title{Presentazione RTB}
\author[]{Gruppo 1 - A.A 2025/2026}
\setDate{27/02/2026}

\begin{document}
{
\setbeamertemplate{footline}{}
\begin{frame}
  \titlepage
\end{frame}
}

\begin{frame}{Obiettivo capitolato}
  \begin{itemize}
    \item Capitolato C9 (Vimar View4Life): Piattaforma per la gestione degli impianti \textit{Smart} nelle residenze protette;
          \vspace{0.7em}
    \item Sistema di gestione allarmi e utenti (OSS e Amministrazione) che possono interagire con gli allarmi (gestione e risoluzione);
          \vspace{0.7em}
    \item Visualizzazione di statistiche sui consumi, con relativi consigli per ridurli, e sugli allarmi.
  \end{itemize}
\end{frame}

\begin{frame}{Comprensione degli obiettivi del capitolato}
  \begin{center}
    \footnotesize
    \begin{tabular}{|c|c|c|c|c|}
      \hline
      \cellcolor{aziendaPrimary!20}\textbf{Tipologia} & \cellcolor{aziendaPrimary!20}\textbf{Obbligatori}
                                                      & \cellcolor{aziendaPrimary!20}\textbf{Desiderabili} & \cellcolor{aziendaPrimary!20}\textbf{Opzionali}
                                                      & \cellcolor{aziendaPrimary!20}\textbf{Totali}                                                                                                  \\
      \hline
      Funzionali                                      & 87                                                 & 10                                              & 8  & 105                               \\
      \hline
      Di qualità                                      & 12                                                 & 0                                               & 9  & 21                                \\
      \hline
      Di vincolo                                      & 3                                                  & 5                                               & 2  & 10                                \\
      \hline
      \cellcolor{aziendaPrimary!20}\textbf{Totali}    & 102                                                & 15                                              & 19 & \cellcolor{aziendaPrimary!20} 136 \\
      \hline
    \end{tabular}
  \end{center}
\end{frame}

\begin{frame}{Tecnologie scelte}
  \centering
  \includegraphics[width=1\textwidth]{img/tecnologie.png}
  %   \begin{columns}[c] % [c] = centratura verticale
  %   \column{0.33\textwidth}

  %   \centering
  %   \includegraphics[width=0.70\textwidth]{img/angularlogo.png}\\

  %   \includegraphics[width=0.70\textwidth]{img/nest-js.png}\\

  %   \column{0.33\textwidth}

  %   \centering
  %   \includegraphics[width=0.66\textwidth]{img/postgrelogo.png}\\

  %   \vspace{0.6cm}

  %   \centering
  %   \includegraphics[width=1\textwidth]{img/dockerlogo.png}\\
  %   \vspace{0.6cm}
  % \end{columns}
\end{frame}

\begin{frame}[shrink=5]{Analisi e gestione dei rischi tecnologici}
  \vspace{0.5cm}
  \begin{center}
    \textbf{Conoscenza insufficiente di una tecnologia}
    \begin{itemize}
      \item Probabilità di occorrenza: Alta;
      \item Livello di impatto: Alto;
      \item Mitigazione: Dedicare ore di calendario allo studio individuale di una tecnologia e organizzare dei momenti di allineamento collaborativo.
    \end{itemize}
    \vspace{0.2cm}
    \textbf{\textit{Bug$_{G}$} nel codice}
    \begin{itemize}
      \item Probabilità di occorrenza: Alta;
      \item Livello di impatto: Medio;
      \item Mitigazione: Individuare la parte di codice che genera il \textit{bug} attraverso l'analisi del codice e strumenti come \textit{debugger$_{G}$}.
    \end{itemize}
  \end{center}
\end{frame}

\begin{frame}{Modifiche migliorative attuate}
  \begin{itemize}
    \item Definizione della struttura dei documenti tramite \LaTeX\ \textit{template};
          \vspace{0.6em}
    \item Processo di verifica e approvazione mediato da \textit{pull request$_{G}$} di \textit{GitHub$_{G}$};
          \vspace{0.6em}
    \item Ripianificazione della distribuzione oraria;
          \vspace{0.6em}
    \item Aggiunta nuova \textit{Github Action$_{G}$} per controllo automatico degli errori ortografici tramite \textit{Hunspell$_{G}$}.
  \end{itemize}
\end{frame}

\begin{frame}{Modifiche migliorative in corso}
  In seguito alla revisione "TB":
  \begin{itemize}
    \item Modifica dell'Analisi dei Requisiti;
    \item Rivalutazione della tecnologia scelta per lo sviluppo della componente \textit{backend}.
  \end{itemize}
  \vspace{0.3cm}
  Attività periodiche:
  \begin{itemize}
    \item Aggiornamento Piano di Progetto;
    \item Aggiornamento Piano di Qualifica e relativo Cruscotto di qualità.
  \end{itemize}
\end{frame}

\begin{frame}{Autovalutazione}
  \begin{itemize}
    \item Il gruppo \textit{SnakeByte} si ritiene soddisfatto del lavoro svolto;
          \vspace{0.6em}
    \item Le criticità maggiori derivano dalla poca esperienza dei membri nel fare delle stime orarie corrette;
          \vspace{0.6em}
    \item Numerose ripianificazioni dei ruoli ricoperti durante gli sprint;
          \vspace{0.6em}
    \item Valutiamo molto positivamente la quantità di riunioni avvenute con la Proponente;
          \vspace{0.6em}
    \item Buona collaborazione tra i membri del gruppo.
  \end{itemize}
\end{frame}

\begin{frame}{Esito del colloquio TB}
  \begin{itemize}
    \item Sono state segnalate dal prof. Cardin alcune ambiguità relative agli \textit{use case} nell'Analisi dei Requisiti;
          \vspace{1em}
    \item Sebbene le scelte tecnologiche siano state approvate, è stato richiesto un approfondimento sulla componente di \textit{backend}: \textit{Express}.
  \end{itemize}
\end{frame}

\begin{frame}[shrink=20]{Consuntivo di periodo attuale}
  \vspace{1.75cm}
  \begin{center}
    \small
    \begin{tabular}{|c|c|c|c|c|c|c|c|}
      \hline
                                                 & \cellcolor{aziendaPrimary!20} \textbf{Resp.}  & \cellcolor{aziendaPrimary!20} \textbf{Amm.}
                                                 & \cellcolor{aziendaPrimary!20} \textbf{Ana.}   & \cellcolor{aziendaPrimary!20} \textbf{Proge.}
                                                 & \cellcolor{aziendaPrimary!20} \textbf{Progr.} & \cellcolor{aziendaPrimary!20} \textbf{Ver.}
                                                 & \cellcolor{aziendaPrimary!20} \textbf{Totale}                                                                                                                                                             \\
      \hline
      \cellcolor{aziendaPrimary!20} V. Baleanu   & 8h                                            & 16h                                           & 13h                 & 0h                 & 0h                 & 17h                 & \textbf{54h}        \\
      \hline
      \cellcolor{aziendaPrimary!20} G. De Fina   & 8h                                            & 20h                                           & 15h                 & 0h                 & 4h                 & 10h                 & \textbf{57h}        \\
      \hline
      \cellcolor{aziendaPrimary!20} L. Granziero & 9h                                            & 0h                                            & 11h                 & 8h                 & 11h                & 5h                  & \textbf{44h}        \\
      \hline
      \cellcolor{aziendaPrimary!20} C. Libralato & 9h                                            & 18h                                           & 6h                  & 0h                 & 11h                & 0h                  & \textbf{44h}        \\
      \hline
      \cellcolor{aziendaPrimary!20} F. Pasqual   & 8h                                            & 14h                                           & 5h                  & 0h                 & 10h                & 18h                 & \textbf{55h}        \\
      \hline
      \cellcolor{aziendaPrimary!20} L. Pellizzon & 8h                                            & 0h                                            & 20h                 & 12h                & 8h                 & 15h                 & \textbf{63h}        \\
      \hline
      \cellcolor{aziendaPrimary!20} F. Venzo     & 6h                                            & 7h                                            & 16h                 & 0h                 & 15h                & 6h                  & \textbf{50h}        \\
      \hline
      \cellcolor{aziendaPrimary!20}
      \textbf{Totale}                            & \textbf{56h}                                  & \textbf{75h}                                  & \textbf{89h}        & \textbf{20h}       & \textbf{59h}       & \textbf{71h}        & \textbf{367h}       \\
      \hline
      \cellcolor{aziendaPrimary!20}
      \textbf{Risorse}                           & \textbf{1680 \euro}                           & \textbf{1500 \euro}                           & \textbf{2225 \euro} & \textbf{500 \euro} & \textbf{885 \euro} & \textbf{1065 \euro} & \textbf{7855 \euro} \\
      \hline
    \end{tabular}
  \end{center}

  \begin{itemize}
    \item Il ruolo di Analista ha superato il budget previsto di \euro 75 (3 ore).
  \end{itemize}
\end{frame}

\begin{frame}[shrink=10]{Preventivo a finire}
  \vspace{2.25cm}
  \begin{center}
    \small
    \begin{tabular}{|c|c|c|}
      \hline
      \rowcolor{aziendaPrimary!20} \textbf{Ruolo}   & \textbf{Ore rimanenti} & \textbf{Budget rimanente} \\
      \hline
      Responsabile                                  & 22h                    & 660 \euro                 \\
      \hline
      Amministratore                                & 1h                     & 20 \euro                  \\
      \hline
      Analista                                      & 0h                     & 0 \euro                   \\
      \hline
      Progettista                                   & 91h                    & 2275 \euro                \\
      \hline
      Programmatore                                 & 83h                    & 1245 \euro                \\
      \hline
      Verificatore                                  & 71h                    & 1065 \euro                \\
      \hline
      \cellcolor{aziendaPrimary!20} \textbf{Totale} & \textbf{265h}          & \textbf{5190 \euro}       \\
      \hline
    \end{tabular}
  \end{center}
\end{frame}

\end{document}