\documentclass{beamer}

\usepackage{../tema}

\title{Diario di bordo}
\author[L. Granziero, V. Baleanu  ]{L. Granziero, V. Baleanu}
\setDate{8/01/2026}

\begin{document}

{ 
  \setbeamertemplate{footline}{}
  \begin{frame}
    \titlepage
  \end{frame}
}

\begin{frame}{Finora}
  In ordine di tempo:
  \begin{enumerate}
    \item Realizzata una prima versione del PoC da presentare al prossimo incontro con la Proponente
    \item Riunione interna per allineamento su \textit{PoC} e lo stato dei vari documenti 
    \item Apportate correzioni all'Analisi dei Requisiti 
    \item Aggiornamento del documento di Piano di Progetto fino allo sprint 4 
    \item Aggiunta delle metriche di qualità alle Norme di Progetto 
    \item Prima stesura del Piano di Qualifica
    \item Incontro con la Proponente il 07/01/2026
    
  \end{enumerate}
\end{frame}

\begin{frame}{Cosa faremo}
  In ordine di tempo:
  \begin{enumerate}
    \item Continuare lo sviluppo del PoC
    \item Terminazione dell'Analisi dei Requisiti
    \item Riunione interna di controllo generale in vista dell'RTB
    \item Aggiornamento del Piano di Progetto fino allo sprint 5
    \item Candidatura RTB: data prevista tra il 26 gennaio e il 2 febbraio
    
  \end{enumerate}
\end{frame}


\begin{frame}{Difficoltà e dubbi incontrati}
  \begin{center}
  \begin{itemize}
    \item Dubbio sulla possibilità di inserire un secondo amministratore per far fronte alla grande quantità di lavoro presente nella redazione dei documenti
    \item Il cruscotto di qualità deve considerare, oltre agli sprint passati, anche quelli futuri (dopo RTB)?
  \end{itemize}
  \end{center}
\end{frame}

\end{document}