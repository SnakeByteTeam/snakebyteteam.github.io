\documentclass{beamer}

\usepackage{../tema}


\title{Diario di bordo}
\author[G. De Fina, V. Baleanu]{Giuseppe De Fina, Valeria Baleanu}
\setDate{24/11/2025}

\begin{document}

{
  \setbeamertemplate{footline}{}
  \begin{frame}
    \titlepage
  \end{frame}
}

\begin{frame}{Finora}
  In ordine di tempo:
  \begin{enumerate}
    \item Riunione interna sul miglioramento della modifica dei documenti via \textit{pull request} e \textit{branch};
    \item Incontro di approfondimento su \textit{Docker} e \textit{KNX IoT} con la proponente;
    \item Continuazione dell'Analisi dei Requisiti.
  \end{enumerate}
\end{frame}

\begin{frame}{Cosa faremo}
  In ordine di tempo:
  \begin{enumerate}
    \item Continuare l'Analisi dei Requisiti;
    \item Riunione interna;
    \item Preparazione al SAL con la proponente.
  \end{enumerate}
\end{frame}


\begin{frame}{Difficoltà e dubbi incontrati}
  \begin{center}
  \begin{itemize}
    \item Monitoraggio e tracciamento attività: quali tra gli strumenti (\textit{Jira}, \textit{GitHub}, ...) è in grado di fornire il giusto compromesso tra semplicità di utilizzo ed efficacia?
    \item \textit{Workflow}: stabilire un metodo standard, adeguato ed efficace per una gestione ottima del ciclo di vita dei documenti all'interno del repository.
    \item Fino a quale livello di dettaglio è opportuno descrivere un caso d'uso nell’analisi dei requisiti? 
  \end{itemize}
  \end{center}
\end{frame}

\end{document}