\documentclass{beamer}

\usepackage{../tema}

\title{Diario di bordo}
\author[G. De Fina, V. Baleanu]{Giuseppe De Fina, Valeria Baleanu}
\setDate{01/12/2025}

\begin{document}

{ 
  \setbeamertemplate{footline}{}
  \begin{frame}
    \titlepage
  \end{frame}
}

\begin{frame}{Finora}
  In ordine di tempo:
  \begin{enumerate}
    \item Riunione interna per allineamento su \textit{use case} e migrazione della documentazione;
    \item Completata migrazione della documentazione nel \textit{repository} ufficiale \textit{snakebyteteam.github.io};
    \item Presentazione del SAL con la proponente il 27/11/2025;
    \item Creazione di \textit{draft} di \textit{user experience} per allineamento su \textit{use case}.
  \end{enumerate}
\end{frame}

\begin{frame}{Cosa faremo}
  In ordine di tempo:
  \begin{enumerate}
    \item Completare gli \textit{use case} per amministratore e gestione permessi;
    \item Proseguire con l'Analisi dei Requisiti;
    \item Test dell'\textit{API KNX IoT 3rd-party};
    \item Riunione interna di avanzamento.
  \end{enumerate}
\end{frame}


\begin{frame}{Difficoltà e dubbi incontrati}
  \begin{center}
  \begin{itemize}
    \item Granularità degli \textit{use case}: trovare il giusto livello di astrazione senza scendere nei dettagli implementativi;
    \item Gestione dei requisiti opzionali: quale priorità dare e come integrarli nell'analisi;
    \item Metodo di descrizione \textit{use case} (formato, dettagli).
  \end{itemize}
  \end{center}
\end{frame}

\end{document}
