\documentclass{beamer}
\usepackage[italian]{babel}
\usepackage{multicol}

\usepackage{../tema}

\title{Revisione RTB}
\author[]{}
\setDate{X/02/2026}

\begin{document}

{ 
  \setbeamertemplate{footline}{}
  \begin{frame}
    \titlepage
  \end{frame}
}

\begin{frame}{Obiettivo capitolato}
  \begin{figure}[h]
  \includegraphics[width=\textwidth]{img/schemacapitolato.png}
  \caption{Schema riassuntivo dell'interazione del sistema}
  \end{figure}
\end{frame}

\begin{frame}{Tecnologie}
\begin{columns}[c] % [c] = centratura verticale
  \column{0.33\textwidth}
  \centering
  \includegraphics[width=0.475\textwidth]{img/angularlogo.png}\\
  Angular

  \vspace{0.6cm}

  \includegraphics[width=0.6\textwidth]{img/expressjslogo.png}\\
  Express.js

  \column{0.33\textwidth}
  \centering
  \includegraphics[width=0.75\textwidth]{img/dockerlogo.png}\\
  Docker

  \column{0.33\textwidth}
  \centering
  \includegraphics[width=0.7\textwidth]{img/nodejslogo.png}\\
  Node.js

  \vspace{0.6cm}

  \includegraphics[width=0.385\textwidth]{img/postgrelogo.png}\\
  PostgreSQL
\end{columns}
\end{frame}


\begin{frame}{Funzionalità del PoC}
All'interno del PoC, per verificare la realizzabilità dal punto di vista tecnologico, sono state implementate le seguenti
funzionalità previste dalla specifica dei requisiti:
\begin{itemize}
  \item visualizzazione dispositivi di un impianto;
  \item visualizzazione allarmi attivi;
  \item creazione soglia di un allarme.
\end{itemize}
\end{frame}

% A tal fine è stato necessario implementare:
% una forma base di interazione tra front end, back end, database per visualizzare le informazioni
% il recupero dei dati dei sensori dal vimar cloud tramite KnX IoT 3rd Party API
% il processo di autenticazione tramite OAuth2
% il meccanismo delle Subscription offerto da KnX IoT 3rd Party API

\end{document}