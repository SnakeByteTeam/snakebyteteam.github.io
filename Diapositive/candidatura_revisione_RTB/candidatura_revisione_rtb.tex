\documentclass{beamer}
\usepackage[italian]{babel}
\usepackage{multicol}

\usepackage{tikz}
\usepackage{graphicx}
\usepackage[table, x11names]{xcolor}
\usepackage{xstring}
\usepackage{etoolbox}
\usepackage{tabularx}
\usepackage[export]{adjustbox}

\newcommand{\Date}{}

\newcommand{\setDate}[1]{\renewcommand{\Date}{#1}}

\useinnertheme{default}
\useoutertheme{default}
\usecolortheme{default}

\definecolor{aziendaPrimary}{RGB}{84, 150, 104}
\setbeamercolor{structure}{fg=aziendaPrimary}
\setbeamercolor{frametitle}{bg=aziendaPrimary!10,fg=aziendaPrimary}

% rimuovere i simboli di navigazione
\setbeamertemplate{navigation symbols}{}


% pagina iniziale
\setbeamertemplate{title page}{
  \vspace{4em}
  \begin{center}
    \hspace{1.5cm}
    \begin{minipage}{0.3\textwidth}
      \centering
      \includegraphics[height=2.5cm, right]{../logo-crop.pdf}
    \end{minipage}
    \begin{minipage}{0.3\textwidth}
      \centering
      \includegraphics[height=2.5cm, left]{./img/logounipd.png}
    \end{minipage}
  \end{center}
  \begin{center}
    \vspace{0.5em}
      {\usebeamerfont{title}\color{black}\textbf{\inserttitle}\par}
    \vspace{0.7em}
    {\usebeamerfont{date}\insertauthor\par}
    {\usebeamerfont{subtitle}\textbf{SnakeByte}\par}
    \vspace{0.5cm}
    {\usebeamerfont{date}\Date\par}
  \end{center}
}

% footer
\setbeamertemplate{footline}{
  \vspace{2ex}
  \hbox{
    \begin{minipage}[t]{0.40\paperwidth}
      \hspace*{0.5em}
      \raggedright\usebeamerfont{footline}{Gruppo 1}
    \end{minipage}%
    \begin{minipage}[t]{0.20\paperwidth}
      \centering\usebeamerfont{footline}Slide \insertframenumber{} di \inserttotalframenumber
    \end{minipage}%
    \begin{minipage}[t]{0.40\paperwidth}
      \hfill
      \insertshortauthor
      \hspace*{0.8em}
      \raisebox{-0.25\height}{\includegraphics[width=0.7cm]{../logo-crop.pdf}}
      \hspace*{1.3em}
    \end{minipage}
  }
  \vspace{0.8ex}
}

\usepackage[absolute,overlay]{textpos}

\title{Presentazione tecnologie}
\author[]{Gruppo 1 - A.A 2025/2026}
\setDate{18/02/2026}

\begin{document}
{
\setbeamertemplate{footline}{}
\begin{frame}
  \titlepage
\end{frame}
}

\begin{frame}{Obiettivo capitolato}
  \begin{figure}[h]
  \includegraphics[width=\textwidth]{img/schemacapitolato.png}
  \caption{Schema riassuntivo dell'interazione del sistema}
  \end{figure}
\end{frame}

\begin{frame}{Obiettivo capitolato Vimar View4Life}
\begin{itemize}
  \item Capitolato C9: Piattaforma per la gestione degli impianti Smart nelle residenza protette;
  \item Sistema di gestione allarmi e utenti (OSS e Amministrazione) che possono interagire con gli allarmi (gestione e risoluzione)
  \item Visualizzazione di statistiche sui consumi, con relativi consigli per ridurli, e sugli allarmi.
 
\end{itemize}
\end{frame}

\begin{frame}{Tecnologie}
\begin{columns}[c] % [c] = centratura verticale
  \column{0.33\textwidth}
  \centering
  \includegraphics[width=0.475\textwidth]{img/tslogo.png}\\
  TypeScript

  \vspace{0.6cm}

  \includegraphics[width=0.475\textwidth]{img/angularlogo.png}\\
  Angular

  \column{0.33\textwidth}
  \centering
  \includegraphics[width=0.7\textwidth]{img/nodejslogo.png}\\
  Node.js 

  \vspace{0.6cm}

  \includegraphics[width=0.6\textwidth]{img/expressjslogo.png}\\
  Express %come lo giustifichiamo?

  \vspace{0.6cm}

  \includegraphics[width=0.385\textwidth]{img/postgrelogo.png}\\
  PostgreSQL %Scritture molto più lente delle letture (come lo giustifichiamo?)

  \column{0.33\textwidth}
  \centering
  \includegraphics[width=0.75\textwidth]{img/dockerlogo.png}\\
  Docker. %alternative Podman (è daemonless e rootless di default ma anche docker lo si può far girare rootless) docker è più diffuso 
          %e meglio supportato (Dockerhub con tutte le versioni ecc.)

  
\end{columns}
\end{frame}


\begin{frame}{Casi d'uso implementati dal PoC}
\begin{itemize}
  \item UC28.3.1.2 Visualizzazione elenco dispositivi;
  \item UC31 Creazione allarme
\end{itemize}
\end{frame}

% A tal fine è stato necessario implementare:
% una forma base di interazione tra front end, back end, database per visualizzare le informazioni
% il recupero dei dati dei sensori dal vimar cloud tramite KnX IoT 3rd Party API
% il processo di autenticazione tramite OAuth2
% il meccanismo delle Subscription offerto da KnX IoT 3rd Party API

\end{document}