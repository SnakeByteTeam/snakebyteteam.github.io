\documentclass{beamer}

\usepackage{../tema}


\title{SAL}
\author[F. Pasqual]{Francesco Pasqual}
\setDate{10/12/2025}

\begin{document}

{
  \setbeamertemplate{footline}{}
  \begin{frame}
    \titlepage
  \end{frame}
}


\addMember{Francesco}{Pasqual}{Responsabile}
\addMember{Christian}{Libralato}{Amministratore}
\addMember{Luca}{Granziero}{Progettista}
\addMember{Leonardo}{Pellizzon}{Verificatore}
\addMember{Valeria}{Baleanu}{Analista}
\addMember{Filippo}{Venzo}{Analista}
\addMember{Giuseppe}{De Fina}{Analista}
\makeMembersFrame

\begin{frame}{Avanzamenti settimanali}
    \begin{itemize}
        \item Redatto e inviato per revisione la bozza del documento di \textbf{Analisi dei Requisiti}
        \item Decisione finale sul sistema di tracciamento delle attività di progetto: \textit{GitHub Projects$_G$}
        \item Aggiornato documento \textbf{Norme di Progetto} alla versione 0.2.0 : aggiunta sezione Sviluppo e Analisi dei Requisiti
        \item Redatto il verbale interno del 01/12/2025 con decisioni sulla revisione della struttura degli \textit{use case$_G$} 
                e sul trasferimento di attività non completate tra uno sprint e l'altro
        \item Redatto e inviato per revisione il verbale esterno relativo al SAL del 27/11/2025 
        \item Ricevuto il secondo \textit{kit di impianto portatile Smart$_G$}
        \item Debugging autenticazione \textit{KNX IoT 3rd party API$_G$}
\end{itemize}
\end{frame}

% PRIMO FRAME: Attività terminate dal verbale esterno 
\addPrecedentTodo{ve\_2025\_11\_27.a1}{\#18}{Inviare bozza dell'Analisi dei Requisiti}{C. Libralato, V. Baleanu, G. De Fina, F. Pasqual}{05/12/2025}
\addPrecedentTodo{ve\_2025\_11\_27.a2}{-}{Decisione definitiva sul sistema di tracciamento delle attività di progetto}{SnakeByte}{01/12/2025}
\addPrecedentTodo{ve\_2025\_11\_27.a3}{-}{Correggere l'implementazione dell'autenticazione con \textit{KNX IoT 3rd party API$_G$}}{L. Pellizzon}{07/12/2025}
\addPrecedentTodo{ve\_2025\_11\_27.a4}{-}{Contattare il referente di \textit{Vimar S.p.A.} per la consegna del secondo kit hardware}{F. Pasqual}{28/11/2025}
\makePrecedentTodoFrame

% Reset delle entries per il secondo frame 
\renewcommand{\precedentTodoEntries}{}

%  SECONDO FRAME: Attività terminate dal verbale interno
\addPrecedentTodo{vi\_2025\_12\_01.a1}{\#17}{Stesura e verifica del verbale dell'incontro del 01/12/2025}{F. Pasqual, L. Pellizzon}{05/12/2025}
\addPrecedentTodo{vi\_2025\_12\_01.a2}{\#19}{Stesura, verifica e consegna del verbale esterno dell'incontro \textit{SAL} del 27/11/2025}{F. Pasqual, L. Pellizzon}{05/12/2025}
% \addPrecedentTodo{vi\_2025\_12\_01.a3}{\#22}{Aggiornare il consuntivo di periodo nel Piano di Progetto}{C. Libralato}{08/12/2025}
\addPrecedentTodo{vi\_2025\_12\_01.a4}{-}{Assegnare ai membri le nuove \textit{issue} per il nuovo (corrente) sprint}{F. Pasqual}{02/12/2025}
\addPrecedentTodo{vi\_2025\_12\_01.a5}{-}{Spostare le issue non concluse nel backlog del nuovo (corrente) sprint}{F. Pasqual}{02/12/2025}
\addPrecedentTodo{vi\_2025\_12\_01.a6}{\#5}{Stesura \textit{use case} cruscotto}{F. Venzo}{05/12/2025}
\addPrecedentTodo{vi\_2025\_12\_01.a7}{-}{Aggiornare struttura e forma \textit{use case}}{G. De Fina, V. Baleanu, F. Venzo}{05/12/2025}
\addPrecedentTodo{vi\_2025\_12\_01.a8}{\#6}{Stesura \textit{use case} autenticazione}{F. Venzo}{05/12/2025}
\addPrecedentTodo{vi\_2025\_12\_01.a9}{\#26}{Creazione presentazione per SAL del 10/12/2025}{F. Pasqual}{09/12/2025}
\addPrecedentTodo{vi\_2025\_12\_01.a10}{\#16}{Modifiche NdP: aggiungere sezione Analisi dei Requisiti}{C. Libralato}{08/12/2025}
\makePrecedentTodoFrame


\addManualTodo{vi\_2025\_12\_01.a3}{\#22}{Creazione e aggiornamento del documento di \textit{Piano di Progetto}}{C. Libralato, L. Pellizzon}{14/12/2025}
\addManualTodo{-}{-}{Setup e configurazione del secondo kit hardware}{F. Pasqual}{14/12/2025}
\makeTodoFrame

\begin{frame}{Limiti o blocchi incontrati}
Durante l'attività di Analisi dei Requisiti e di stesura della bozza del documento sono emersi i seguenti dubbi:
    \begin{itemize}
        \item È corretto considerare il Database e i dispositivi (sensori) come attori secondari?
        \item È opportuno mantenere gli \textit{use case} che contengono solo relazioni di tipo \textit{extend}?
        \item È corretto creare uno \textit{use case} distinto per ogni tipo di visualizzazione (Analytics e Dispositivi)?
    \end{itemize}
  
Per chiarire questi aspetti, è stato richiesto un colloquio con il prof. R. Cardin.

\end{frame}

\end{document}