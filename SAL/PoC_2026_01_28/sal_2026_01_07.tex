\documentclass{beamer}

\usepackage{../tema}
\usepackage[absolute,overlay]{textpos}


\title{Progetto View4Life}
\author{Presentazione \textit{Proof of Concept}}
\setDate{28/01/2026}

\begin{document}

{
  \setbeamertemplate{footline}{}
  \begin{frame}
    \titlepage
  \end{frame}
}

\begin{frame}{L'obiettivo del progetto}
  % \begin{textblock*}{2cm}(10.5cm,0.5cm)
  %   \includegraphics[width=2cm]{immagine}
  % \end{textblock*}
  Sviluppare un applicativo \textit{web} per permettere la gestione delle residenze protette da parte del personale sanitario, tramite 
  l'uso dei dispositivi \textit{IoT}.
  In particolare, l'applicativo deve:
    \begin{itemize}
        \item Visualizzare le informazioni dei vari dispositivi \textit{IoT};
        \item gestire gli \textbf{allarmi}, che verrano presi in carico dal personale sanitario;
        \item visualizzare le \textbf{\textit{analytics}} relative alla piattaforma e agli impianti;
        \item visualizzare una \textbf{\textit{dashboard}} contenente le informazioni principali (stato dispositivi, allarmi attivi, ...).
\end{itemize}
\end{frame}

\begin{frame}{Metodo di lavoro}
  \begin{itemize}
    \item Abbiamo scelto una metodologia \textbf{agile}, composta da \textbf{sprint} di 2 settimane ciascuno;
    \item Ad ogni sprint viene effettuata una rotazione dei seguenti ruoli tra i membri del gruppo:
    \item \begin{itemize}
      \item Responsabile
      \item Amministratore
      \item Analista
      \item Progettista
      \item Programmatore
      \item Verificatore
    \end{itemize}
  \end{itemize}
\end{frame}

\begin{frame}{Avanzamento fino ad oggi}
  \centering
  \includegraphics[
    width=\textwidth,
    height=0.9\textheight,
    keepaspectratio
  ]{SAL.drawio.png}
\end{frame}

\begin{frame}{Limiti o blocchi incontrati}
Durante l'ultimo periodo di avanzamento sono emersi alcuni blocchi:
  
  \begin{itemize}
     \item Studio individuale delle tecnologie riguardanti il PoC;
  \end{itemize}

\end{frame}


\end{document}