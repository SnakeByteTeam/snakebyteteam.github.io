\documentclass{beamer}

\usepackage{../tema}
\usepackage[absolute,overlay]{textpos}


\title{Progetto View4Life}
\author{Presentazione del \textit{Proof of Concept}}
\setDate{28/01/2026}

\begin{document}

{
\setbeamertemplate{footline}{}
\begin{frame}
  \titlepage
\end{frame}
}

\begin{frame}{L'obiettivo del progetto}
  % \begin{textblock*}{2cm}(10.5cm,0.5cm)
  %   \includegraphics[width=2cm]{immagine}
  % \end{textblock*}
  Sviluppare un applicativo \textit{web} per permettere la gestione delle residenze protette da parte del personale sanitario
  e dell'amministratore, tramite
  l'uso dei dispositivi \textit{IoT}.
  \\In particolare, l'applicativo deve:
  \begin{itemize}
    \item Visualizzare le informazioni dei vari dispositivi \textit{IoT};
    \item gestire gli \textbf{allarmi}, che verrano presi in carico dal personale sanitario;
    \item visualizzare le \textbf{\textit{analytics}} relative alla piattaforma e agli impianti, con relativi suggerimenti per il risparmio energetico;
    \item visualizzare una \textbf{\textit{dashboard}} contenente le informazioni principali (stato dispositivi, allarmi attivi, ...).
  \end{itemize}
\end{frame}

\begin{frame}{Metodo di lavoro}
  \begin{itemize}
    \item Abbiamo scelto una metodologia \textbf{agile} con framework \textbf{scrum}, composta da \textbf{sprint} di 2 settimane ciascuno;
    \item Issue Tracking System: \textbf{GitHub Projects}.
          \\Le issue non completate in uno sprint vengono spostate e gestite nello sprint successivo;
    \item Ad ogni sprint viene effettuata una rotazione dei seguenti ruoli tra i membri del gruppo:
          \begin{center}
            \begin{tabular}{|l|}
              \hline
              \textbf{Ruoli} \\
              \hline
              Responsabile   \\
              Amministratore \\
              Analista       \\
              Progettista    \\
              Programmatore  \\
              Verificatore   \\
              \hline
            \end{tabular}
          \end{center}
  \end{itemize}
\end{frame}

\begin{frame}{Avanzamento fino ad oggi}
  \centering
  \includegraphics[
    width=\textwidth,
    height=0.9\textheight,
    keepaspectratio
  ]{stato_avanzamento.drawio.png}
\end{frame}

\begin{frame}{Disegno architetturale - Sistema}
  \centering
  \includegraphics[
    width=\textwidth,
    height=0.9\textheight,
    keepaspectratio
  ]{system.drawio.png}
\end{frame}

\begin{frame}{Disegno architetturale - Container}
  \centering
  \includegraphics[
    width=\textwidth,
    height=0.9\textheight,
    keepaspectratio
  ]{container.drawio.png}
\end{frame}

\begin{frame}{Disegno architetturale - Components}
  \centering
  \includegraphics[
    width=\textwidth,
    height=0.9\textheight,
    keepaspectratio
  ]{components.drawio.png}
\end{frame}

\begin{frame}{Tecnologie frontend}
  \centering
  \scriptsize
  \renewcommand{\arraystretch}{1.2}
  \begin{tabularx}{\textwidth}{|
      p{2cm}|
      >{\raggedright\arraybackslash}X|
      >{\raggedright\arraybackslash}X|
      >{\raggedright\arraybackslash}X|}
    \hline
    \textbf{Categoria}
     & \textbf{Angular}
     & \textbf{React}
     & \textbf{Flask}                  \\ \hline

    Tipologia
     & Framework frontend completo
     & Libreria frontend
     & Micro-framework backend         \\ \hline

    Struttura
     & Fortemente strutturato
     & Flessibile, non prescrittivo
     & Minima                          \\ \hline

    Tipizzazione
     & TypeScript obbligatorio
     & TypeScript opzionale
     & Dinamica                        \\ \hline

    Scalabilità
     & Elevata (enterprise)
     & Media-Alta
     & Limitata al backend             \\ \hline

    Velocità di apprendimento
     & Bassa (curva ripida)
     & Alta
     & Alta                            \\ \hline

    Tooling
     & Completo e integrato
     & Ecosistema frammentato
     & Essenziale                      \\ \hline

    Svantaggi
     & Complessità iniziale, verbosità
     & Mancanza di standard nativi
     & Non adatto al frontend          \\ \hline
  \end{tabularx}
\end{frame}

\begin{frame}{Tecnologie backend}
  \centering
  \scriptsize
  \renewcommand{\arraystretch}{1.2}

  \begin{tabularx}{\textwidth}{|
      p{2cm}|
      >{\raggedright\arraybackslash}X|
      >{\raggedright\arraybackslash}X|
      >{\raggedright\arraybackslash}X|}
    \hline

    \textbf{Categoria}
     & \textbf{Node.js + Express}
     & \textbf{Java + Spring}
     & \textbf{Python + Flask / FastAPI} \\ \hline

    Tipologia
     & Runtime JS con framework leggero
     & Framework backend completo
     & Framework leggero                 \\ \hline

    Struttura
     & Poco prescrittiva
     & Fortemente strutturata
     & Minima e flessibile               \\ \hline

    Tipizzazione
     & Dinamica (TypeScript opzionale)
     & Statica (Java)
     & Dinamica (typing opzionale)       \\ \hline

    Prestazioni
     & Ottime per API e I/O
     & Elevate e stabili sotto carico
     & Buone, FastAPI molto performante  \\ \hline

    Tooling
     & Modulare (npm, tool esterni)
     & Completo e integrato
     & Essenziale, estendibile           \\ \hline

    Sicurezza
     & Gestita tramite middleware
     & Forte supporto nativo
     & Supporto base, estendibile        \\ \hline

    Svantaggi
     & Poca struttura nativa
     & Complessità e verbosità
     & Meno standardizzazione            \\ \hline

    \textbf{Ambito ideale}
     & \textbf{API REST, microservizi}
     & \textbf{Applicazioni enterprise}
     & \textbf{API rapide, prototipi}    \\ \hline
  \end{tabularx}
\end{frame}

\begin{frame}{Scelta del Database}
  Le opzioni sono:
  \begin{enumerate}
    \item Database relazionale (per l'applicativo web):
          \\ \begin{itemize}
            \item MySQL
            \item PostgreSQL
          \end{itemize}
          Possibile implementazione di un \textit{Object Relational Mapper} come \textbf{Prisma}
    \item Database non relazionale (per analytics):
          \\ \begin{itemize}
            \item TimescaleDB
            \item InfluxDB
            \item MongoDB
          \end{itemize}
  \end{enumerate}
\end{frame}

\begin{frame}{Backend: Architettura esagonale}
  \textit{In ottica delle successive fasi di progettazione e design.}
  \\Alcuni punti di forza e ragioni della scelta:
  \begin{itemize}
    \item Il più importante, da cui gli altri seguono, \textbf{inversione delle dipendenze};
    \item isolamento del dominio;
    \item ampia possibilità di compatibilità ed estensibilità con sistemi esterni (\textit{adapters});
    \item focus sull'implementazione di interfacce stabili (\textit{ports}).
  \end{itemize}
\end{frame}

\begin{frame}{Architettura esagonale}
  \begin{figure}
    \centering
    \includegraphics[
      width=\textwidth,
      height=0.9\textheight,
      keepaspectratio
    ]{./backend_hex.png}
  \end{figure}
\end{frame}

\begin{frame}{Analytics}
  analisi consumo en, anomalie impianto, rilevamento presenza, presenza prolungata,
  variazione temp, allarmi inviati, allarmi risolti, frequenza allarmi
  \\chart.js - open source
\end{frame}

\begin{frame}{Suggerimenti}
  soglie prefissate e sulla base dei dati delle analytics, vengono mostrati o meno i
  suggerimenti
\end{frame}

\begin{frame}{Gestione degli allarmi}

  \begin{itemize}
    \item E' possibile creare delle \texttt{Threshold}, unità di controllo dei valori rilevati da un \textit{datapoint}. 
    \item Vengono definiti valore di scatto e operatore di confronto;
    \item\texttt{Alarm} è l'evento generato alla violazione della soglia;
    \item I valori che possono far scattare la soglia provengono dalle \textit{subscription}
  \end{itemize}

  \texttt{Ampliamo a voce}

\end{frame}

\begin{frame}{Interfaccia grafica}
  screen
\end{frame}

\begin{frame}{Limiti o blocchi incontrati}
  Durante l'avanzamento del progetto i principali limiti incontrati sono stati:
  \begin{itemize}
    \item Pianificazione iniziale dei ruoli e delle ore per la definizione del costo complessivo;
    \item studio individuale delle tecnologie riguardanti il PoC;
    \item allineamento tra programmatori dopo lo scambio dei ruoli;
    \item tempo limitato a causa di esami o altri progetti universitari;
  \end{itemize}
\end{frame}


\end{document}