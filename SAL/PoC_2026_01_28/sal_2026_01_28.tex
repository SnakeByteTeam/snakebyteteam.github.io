\documentclass{beamer}

\usepackage{tikz}
\usepackage{graphicx}
\usepackage[table, x11names]{xcolor}
\usepackage{xstring}
\usepackage{etoolbox}
\usepackage{tabularx}

\newcommand{\Date}{}

\newcommand{\setDate}[1]{\renewcommand{\Date}{#1}}

\useinnertheme{default}
\useoutertheme{default}
\usecolortheme{default}

\definecolor{aziendaPrimary}{RGB}{84, 150, 104}
\setbeamercolor{structure}{fg=aziendaPrimary}
\setbeamercolor{frametitle}{bg=aziendaPrimary!10,fg=aziendaPrimary}

% rimuovere i simboli di navigazione
\setbeamertemplate{navigation symbols}{}

% pagina iniziale
\setbeamertemplate{title page}{
  \vspace{2em}
  \begin{center}
    \begin{minipage}{0.3\textwidth}
      \centering
      \includegraphics[height=2.1cm]{./logounipd.png}
    \end{minipage}
    \begin{minipage}{0.3\textwidth}
      \centering
      \includegraphics[height=2cm]{../logo-crop.pdf}
    \end{minipage}
    \begin{minipage}{0.3\textwidth}
      \centering
      \includegraphics[height=2cm]{./logoview4life.png}
    \end{minipage}
  \end{center}
  \begin{center}
      {\usebeamerfont{title}\color{black}\textbf{\inserttitle}\par}
    \vspace{0.5em}
    {\usebeamerfont{date}\insertauthor\par}
    {\usebeamerfont{date}\Date\par}
  \end{center}
}

% footer
\setbeamertemplate{footline}{
  \vspace{2ex}
  \hbox{
    \begin{minipage}[t]{0.40\paperwidth}
      \hspace*{0.5em}
      \raggedright\usebeamerfont{footline}\inserttitle
    \end{minipage}%
    \begin{minipage}[t]{0.20\paperwidth}
      \centering\usebeamerfont{footline}Pagina \insertframenumber{} di \inserttotalframenumber
    \end{minipage}%
    \begin{minipage}[t]{0.40\paperwidth}
      \hfill
      \insertshortauthor
      \hspace*{0.8em}
      \raisebox{-0.25\height}{\includegraphics[width=0.7cm]{../logo-crop.pdf}}
      \hspace*{1.3em}
    \end{minipage}
  }
  \vspace{0.8ex}
}

\usepackage[absolute,overlay]{textpos}

\title{Progetto View4Life del corso \\ di Ingegneria del Software}
\author{Presentazione del \textit{Proof of Concept}}
\setDate{28/01/2026}

\begin{document}

{
\setbeamertemplate{footline}{}
\begin{frame}
  \titlepage
\end{frame}
}

\begin{frame}{L'obiettivo del progetto}
  % \begin{textblock*}{2cm}(10.5cm,0.5cm)
  %   \includegraphics[width=2cm]{immagine}
  % \end{textblock*}
  Sviluppare un applicativo \textit{web} per permettere la gestione delle residenze protette da parte del personale sanitario
  e dell'amministratore, tramite l'uso dei dispositivi \textit{IoT} della gamma \textit{View Wireless}. \\
  In particolare, l'applicativo deve:
  \begin{itemize}
    \item Visualizzare le informazioni dei vari \textbf{dispositivi};
    \item gestire gli stati degli \textbf{allarmi}, che verranno presi in carico dal personale sanitario;
    \item visualizzare le \textbf{\textit{analytics}} relative alla piattaforma e agli impianti, con relativi suggerimenti 
    per il \textbf{risparmio energetico};
    \item visualizzare una \textbf{\textit{dashboard}} contenente le informazioni principali (stato dispositivi, allarmi attivi, ...).
  \end{itemize}
\end{frame}

\begin{frame}{Metodo di lavoro}
  \begin{itemize}
    \item Abbiamo scelto una metodologia \textbf{agile} con framework \textbf{scrum}, composta da \textbf{sprint} di 2 settimane ciascuno;
    \item Issue Tracking System: \textbf{GitHub Projects}.
          \\Le issue non completate in uno sprint vengono spostate e gestite nello sprint successivo;
    \item Ad ogni sprint viene effettuata una rotazione dei seguenti ruoli tra i 7 membri del gruppo:
          \begin{center}
            \begin{tabular}{|l|}
              \hline
              \textbf{Ruoli} \\
              \hline
              Responsabile   \\
              Amministratore \\
              Analista       \\
              Progettista    \\
              Programmatore  \\
              Verificatore   \\
              \hline
            \end{tabular}
          \end{center}
  \end{itemize}
\end{frame}

\begin{frame}{Schema riassuntivo dei componenti}
  \centering
  \includegraphics[
    width=\textwidth,
    height=0.9\textheight,
    keepaspectratio
  ]{schema_riassuntivo.png}
\end{frame}

\begin{frame}{Avanzamento fino ad oggi}
  \centering
  \includegraphics[
    width=\textwidth,
    height=0.9\textheight,
    keepaspectratio
  ]{stato_avanzamento.drawio.png}
\end{frame}

\begin{frame}{Disegno architetturale - Sistema}
  \centering
  \includegraphics[
    width=\textwidth,
    height=0.9\textheight,
    keepaspectratio
  ]{system.drawio.png}
\end{frame}

\begin{frame}{Disegno architetturale - Container}
  \centering
  \includegraphics[
    width=\textwidth,
    height=0.9\textheight,
    keepaspectratio
  ]{container.drawio.png}
\end{frame}

% \begin{frame}{Disegno architetturale - Components}
%   \centering
%   \includegraphics[
%     width=\textwidth,
%     height=0.9\textheight,
%     keepaspectratio
%   ]{components.drawio.png}
% \end{frame}

\begin{frame}{Tecnologie frontend}
 \begin{itemize}
  \item \textbf{Angular} è stato scelto per la sua struttura solida e scalabile, adatta ad applicazioni complesse e manutenibili nel tempo in contesti enterprise.

  \item Offre un tooling integrato (Angular CLI, build, testing) che garantisce uno sviluppo coerente, riducendo la necessità di configurazioni aggiuntive rispetto a soluzioni più flessibili come React.

  \item L’uso obbligatorio di \textbf{TypeScript} migliora l’affidabilità del codice grazie alla tipizzazione statica.

  \item Presenta una curva di apprendimento più ripida dovuta alla complessità iniziale e ai concetti architetturali da comprendere. (Angular Docs)

  \item La scelta privilegia solidità e standardizzazione dell’architettura rispetto alla rapidità di sviluppo iniziale.
\end{itemize}

\end{frame}

\begin{frame}{Tecnologie backend}
   \begin{itemize}
  
  \item Il modello event-driven e non bloccante di Node.js garantisce ottime prestazioni nelle operazioni di I/O e nella gestione di richieste concorrenti, rendendolo particolarmente adatto per mantenere attive le subscription richieste dallo standard KNX IoT senza ricorrere a polling.

  \item Express offre una struttura minima e poco prescrittiva, permettendo di modellare l’architettura secondo le esigenze del progetto senza la complessità di framework più pesanti.
 
  \item Il tooling modulare basato su npm consente di integrare facilmente librerie per OAuth2, sicurezza validazione e logging

\item \textbf{Node.js + Express} è stato scelto per la leggerezza e la flessibilità che offre, oltre a essere molto in linea per ciò che il progetto richiede, in particolare per la gestione delle API REST.

\end{itemize}
\end{frame}

\begin{frame}{Scelta del Database}
 \begin{itemize}
  \item I dati dell’applicativo web, come utenti, credenziali e configurazioni, saranno gestiti con un database relazionale (PostgreSQL) per garantire integrità e coerenza.

  \item Le metriche per le analytics saranno memorizzati in un database come TimescaleDB, InfluxDB o MongoDB, ottimizzato per serie temporali e aggiornamenti frequenti.

\end{itemize}
\end{frame}

\begin{frame}{Docker}
\begin{itemize}
\item \textbf{Docker} sarà utilizzato per rendere l’ambiente completamente replicabile e portabile tra diversi cloud provider, come richiesto dal progetto. 

\item Ogni servizio backend, database e componente dell’applicativo sarà containerizzato, consentendo di avviare l’intero stack con un solo comando.

\item Questo approccio rispetta il principio di Infrastructure as Code e facilita la gestione di dipendenze, aggiornamenti e scalabilità dei vari componenti.

\end{itemize}
\end{frame}  


\begin{frame}{Backend: Architettura esagonale}
  \footnotesize
  \setlength{\extrarowheight}{2pt}
  \begin{tabularx}{\textwidth}{|>{\raggedright\arraybackslash}X|>{\raggedright\arraybackslash}X|>{\raggedright\arraybackslash}X|}
    \hline
                 & \textbf{Layered}                                             & \textbf{Esagonale}                                                \\
    \hline
    Dominio      & Dipende da Persistent Logic (mentre per PoC trascuriamo DB)                            & Isolato, guida lo sviluppo. Ci permette di concentriarci sulla modellazione del dominio    \\                              
    \hline
    Dipendenze   & App→Bus→Pers. Dati guidano sviluppo (non sappiamo ancora come rappresentarli)                       & App→Bus←Pers. Domain guida sviluppo (conosciamo le principali entità)                          \\
    \hline
    Testabilità  & Difficile testare senza mocking oneroso o avviare PostGre    & Facile e veloce, in particolare per Domain (Unit Test)            \\
    \hline
    Integrazioni & Chiamate API nei Service. Cambio API = cambio Business Logic & API come Secondary Adapters, interscambiabili rispetto al dominio \\
    \hline
  \end{tabularx}
\end{frame}

\begin{frame}{Backend: Architettura esagonale (continua)}
  \footnotesize
  \setlength{\extrarowheight}{2pt}
  \begin{tabularx}{\textwidth}{|>{\raggedright\arraybackslash}X|>{\raggedright\arraybackslash}X|>{\raggedright\arraybackslash}X|}
    \hline
                               & \textbf{Layered}                                                               & \textbf{Esagonale}                                                                 \\
    \hline
    Database                   & Ruolo chiave, fondamenta. Sistema rigido e fragile                             & Dettaglio implementativo, ampia possibilità di cambiamento                         \\
    \hline
    Struttura e Manutenibilità & Layer nascondono la complessità. Difficoltà orientarsi con progetto che avanza & Indirizza lo sviluppo, mantiene chiarezza e trasparenza nelle integrazioni esterne \\
    \hline
  \end{tabularx}
\end{frame}

\begin{frame}{Backend: Architettura esagonale}
  \textit{In ottica delle successive fasi di progettazione e design.}
  \\Alcuni punti di forza e ragioni della scelta:
  \begin{itemize}
    \item Il più importante, da cui gli altri seguono, \textbf{inversione delle dipendenze};
    \item isolamento del dominio;
    \item ampia possibilità di compatibilità ed estensibilità con sistemi esterni (\textit{adapters});
    \item focus sull'implementazione di interfacce stabili (\textit{ports}).
  \end{itemize}
\end{frame}

\begin{frame}{Backend: Architettura esagonale}
  \begin{figure}
    \centering
    \includegraphics[
      width=\textwidth,
      height=0.9\textheight,
      keepaspectratio
    ]{./backend_hex_final.png}
  \end{figure}
\end{frame}

\begin{frame}{Analytics}
  Nella sezione Analytics saranno visualizzate, attraverso dei \textbf{grafici}, le statistiche specificate nel capitolato, tra cui:
  \begin{itemize}
   \item l'energia consumata dall'illuminazione;
   \item anomalie d'impianto;
   \item rilevamento presenza/assenza/caduta;
   \item rilevamento presenza prolungata;
   \item variazione e cambio di temperatura;
   \item allarmi inviati e risolti giornalmente,
   \item frequenza allarmi e cadute di un periodo a scelta
  \end{itemize}
\end{frame}

\begin{frame}{Analytics}
  Le statistiche risultano utili per:
  \begin{itemize}
    \item fornire una visione d'insieme sugli impianti;
    \item ottenere informazioni sull'efficienza degli impianti;
    \item sviluppare strategie per ottimizzare i consumi;
    \item valutare l'efficacia della gestione degli allarmi.
  \end{itemize}
  
  Per la visualizzazione dei grafici è stato previsto l'utilizzo di una tra le seguenti alternative:
  \begin{itemize}
    \item \textbf{ng2-charts}, wrapper della libreria \textit{Chart.js}
    \item \textbf{ngx-charts}, framework per Angular
  \end{itemize}
\end{frame}

\begin{frame}{Suggerimenti}
  La funzionalità dei suggerimenti per limitare il consumo energetico è basata su:
  \begin{itemize}
    \item una raccolta consigli con struttura fissa;
    \item visualizzazione del suggerimento qualora i dati statistici violino delle soglie (eventualmente parametrizzabili). 
  \end{itemize}

  Alcune proposte esempi di suggerimenti sono:
  \begin{itemize}
    \item Per risparmiare energia, spegni le luci negli impianti esposti a sud dalle 12:00 alle 15:00;
    \item Mantieni una temperatura inferiore 19° dalle 22:00 alle 6:30, risparmi energia e migliora il sonno;
    \item La luce del soggiorno dell'impianto 2 rimane accesa anche senza nessuna presenza rilevata, spegnila per evitare consumi ulteriori.
  \end{itemize}

\end{frame}

\begin{frame}{Interfaccia Analytics (bozzetto grafico)}  
  \centering
  \includegraphics[
    width=\textwidth,
    height=0.9\textheight,
    keepaspectratio
  ]{PreviewAnalytics.png}
\end{frame}


\begin{frame}{Gestione degli allarmi}

  \begin{itemize}
    \item \texttt{Threshold}, unità di controllo dei valori rilevati da un \textit{datapoint} (valore di scatto e operatore di confronto).
    \item\texttt{Alarm} è l'evento generato alla violazione della soglia;
  \end{itemize}

   \begin{figure}
    \centering
    \includegraphics[
      width=\textwidth,
      height=0.9\textheight,
      keepaspectratio
    ]{./AlarmsDiagram.drawio.png}
  \end{figure}
  
\end{frame}

\begin{frame}{Limiti o blocchi incontrati}
  Durante l'avanzamento del progetto i principali limiti incontrati sono stati:
  \begin{itemize}
    \item Pianificazione iniziale dei ruoli e delle ore per la definizione del costo complessivo;
    \item studio individuale delle tecnologie riguardanti il PoC;
    \item allineamento tra programmatori dopo lo scambio dei ruoli;
    \item tempo limitato a causa di esami o altri progetti universitari;
  \end{itemize}
\end{frame}


\end{document}