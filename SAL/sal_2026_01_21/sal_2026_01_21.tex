\documentclass{beamer}

\usepackage{../tema}


\title{SAL}
\author[V. Baleanu, F. Pasqual, C. Libralato]{V. Baleanu, F. Pasqual, C. Libralato}
\setDate{21/01/2026}

\begin{document}

{
  \setbeamertemplate{footline}{}
  \begin{frame}
    \titlepage
  \end{frame}
}

\addMember{Valeria}{Baleanu}{Responsabile}
\addMember{Christian}{Libralato}{Amministratore}
\addMember{Francesco}{Pasqual}{Amministratore}
\addMember{Leonardo}{Pellizzon}{Analista}
\addMember{Giuseppe}{De Fina}{Programmatore}
\addMember{Filippo}{Venzo}{Programmatore}
\addMember{Luca}{Granziero}{Verificatore}
\makeMembersFrame

\begin{frame}{Avanzamenti settimanali}
    \begin{itemize}
        \item Effettuata riunione interna in data 13/01 e redatto il verbale interno ad essa relativo;
        \item Aggiornato il documento di Piano di Qualifica con il cruscotto di qualità;
        \item Incontro con prof. Cardin per finalizzare l'Analisi dei Requisiti in data 20/01;
        \item Allineamento riguardo al PoC tra programmatori dopo il passaggio di ruolo.
\end{itemize}
\end{frame}

% PRIMO FRAME: Attività terminate dal verbale esterno 

% Reset delle entries per il secondo frame 
\renewcommand{\precedentTodoEntries}{}

%  SECONDO FRAME: Attività terminate dal verbale interno
\addPrecedentTodo{vi\_2026\_01\_13.a1}{\noOne}{Richiedere incontro con prof. Cardin}{L. Pellizzon}{15/01/2026}
\addPrecedentTodo{vi\_2026\_01\_13.a3}{\#73}{Scrittura verbale interno del 13/01/2026}{V. Baleanu}{15/01/2026}
\addPrecedentTodo{vi\_2026\_01\_13.a5}{\#75}{Dubbi sul PoC nel SAL del 21/01/2026}{C. Libralato, F. Pasqual, V. Baleanu}{20/01/2026}
\makePrecedentTodoFrame

\addManualTodo{vi\_2026\_01\_13.a2}{\#72}{Iniziare presentazione delle tecnologie per RTB}{C. Libralato, F. Pasqual, V. Baleanu}{27/01/2026}
\addManualTodo{vi\_2026\_01\_13.a4}{\#74}{Aggiornamento cruscotto di qualità nel PdQ}{C. Libralato, F. Pasqual}{25/01/2026}

\makeTodoFrame

\begin{frame}{Limiti o blocchi incontrati}
Durante l'ultimo periodo di avanzamento sono emersi alcuni dubbi:
  \begin{itemize}
      \item Per il \textit{PoC} è necessario esporre un \textit{endpoint} pubblico per ricevere eventi push (webhook / notifiche di subscription) oppure è sufficiente una simulazione locale?
      \item La soglia è un oggetto riutilizzabile e indipendente oppure fa parte della configurazione dell'allarme?
  \end{itemize}
\end{frame}


\end{document}