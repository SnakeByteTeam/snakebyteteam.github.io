\documentclass{beamer}

\usepackage{../tema}


\title{SAL}
\author[L. Granziero, V. Baleanu]{L. Granziero, V. Baleanu}
\setDate{07/01/2026}

\begin{document}

{
  \setbeamertemplate{footline}{}
  \begin{frame}
    \titlepage
  \end{frame}
}

\addMember{Christian}{Libralato}{Programmatore}
\addMember{Giuseppe}{De Fina}{Verificatore}
\addMember{Leonardo}{Pellizzon}{Programmatore}
\addMember{Valeria}{Baleanu}{Amministratore}
\addMember{Luca}{Granziero}{Responsabile}
\addMember{Filippo}{Venzo}{Verificatore}
\addMember{Francesco}{Pasqual}{Programmatore}
\makeMembersFrame

\begin{frame}{Avanzamenti settimanali}
    \begin{itemize}
        \item Effettuata riunione interna in data 29/12 e redatto il verbale interno ad essa relativo;
        \item Aggiornato il documento di Norme di Progetto inserendo metriche di qualità di prodotto e di processo;
        \item Avvio del cruscotto di qualità;
        \item Finalizzazione del documento dell'Analisi dei Requisiti;
        \item Aggiornamento del Piano di Progetto allo sprint 4;
        \item Allineamento tra vecchi e nuovi programmatori del PoC.
\end{itemize}
\end{frame}

% PRIMO FRAME: Attività terminate dal verbale esterno 
\addPrecedentTodo{ve\_2025\_12\_23.a2}{\noOne}{Effettuare revisione della distribuzione ruoli e ore}{SnakeByte}{11/01/2026} 
\makePrecedentTodoFrame

% Reset delle entries per il secondo frame 
\renewcommand{\precedentTodoEntries}{}

%  SECONDO FRAME: Attività terminate dal verbale interno
\addPrecedentTodo{vi\_2025\_12\_15.a2}{\#34}{Compilazione PdP}{G. de Fina}{27/12/2025}
\addPrecedentTodo{vi\_2025\_12\_15.a4}{\#36}{Inizio implementazione PoC}{L. Granziero, F. Venzo}{29/12/2025}
\addPrecedentTodo{vi\_2025\_12\_29.a5}{\#56}{Scrittura verbale interno 29/12/2025}{L. Granziero, F. Venzo}{05/01/2026}
\addPrecedentTodo{vi\_2025\_12\_15.a5}{\#37}{Definizione diagramma Gantt}{G. De Fina}{30/12/2025}
\makePrecedentTodoFrame

\addManualTodo{vi\_2025\_12\_15.a1}{\#33}{Terminazione AdR}{L. Pellizzon}{30/12/2025}
\addManualTodo{vi\_2025\_12\_15.a3}{\#35}{Prima Stesura PdQ}{G. De Fina}{30/12/2025}

\makeTodoFrame

\begin{frame}{Limiti o blocchi incontrati}
Durante l'ultimo periodo di avanzamento sono emersi alcuni dubbi:
  \begin{itemize}
      \item Le metriche di qualità scelte sono adeguate?
      \item Decidere un range di data per la candidatura all'RTB;
      \item Qual è il livello minimo richiesto per il PoC? Quanto è già stato realizzato è sufficiente oppure sono necessarie ulteriori 
      integrazioni?
      \item È concesso l'utilizzo di Prisma, un ORM (Object-Relational Mapping) che, partendo dal modello dei dati, può generare le migrazioni per creare automaticamente le tabelle nel database.
      \item È concesso l'utilizzo del logo Vimar all'interno del PoC?
  \end{itemize}
\end{frame}

\begin{frame}{Limiti o blocchi incontrati}
Durante l'ultimo periodo di avanzamento sono emersi alcuni blocchi:
  
  \begin{itemize}
     \item Studio individuale delle tecnologie riguardanti il PoC;
  \end{itemize}

\end{frame}


\end{document}