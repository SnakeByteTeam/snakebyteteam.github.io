\documentclass[10pt, letterpaper]{article}
\usepackage[nomarginpar, margin=2.75cm, tmargin=3cm, bmargin=1.75cm]{geometry}
\usepackage[
    colorlinks=true,      
    linkcolor=black,      
    urlcolor=blue,       
    citecolor=black       
]{hyperref}
\usepackage{graphicx}
\usepackage{float}
\usepackage[table, x11names]{xcolor}
\usepackage{tabularx}
\renewcommand{\arraystretch}{1.2} % migliora la leggibilità
\renewcommand{\contentsname}{Indice}
\usepackage{fancyhdr}
\usepackage{eurosym}
\pagestyle{fancy}
\fancyhf{}
\fancyhead[L]{SnakeByte} 
\fancyhead[R]{Dichiarazione degli impegni}
\fancyfoot[C]{\thepage}

\begin{document}

\begin{titlepage}
    \begin{center}
        \begin{center}
            \includegraphics[width=0.6\textwidth]{./img/logo.pdf}
        \end{center}
        \vspace{4cm}
        \huge\textbf{Dichiarazione degli impegni}\par
        \vspace{2cm}
        \large \textbf{SnakeByte} (Gruppo 1):\\
        \large Valeria Baleanu, Leonardo Pellizzon, Filippo Venzo, Giuseppe De Fina, \\
         Francesco Pasqual, Christian Libralato, Luca Granziero \\
        (2109911, 2111006, 2113705, 2113187, 2103119, 2101047, 2075512)
        \vfill
        \small
        \begin{center}
            \begin{tabular}{|c|c|c|c|}
                \hline
                \multicolumn{4}{|c|}{\textbf{Informazioni documento}} \\
                \hline
                \rowcolor{lightgray} \textbf{Versione} & \textbf{Data} & \textbf{Stato} & \textbf{Destinatari} \\
                \hline
                1.0.0 & 30/10/2025 & Approvato & SnakeByte, prof. Vardanega Tullio, prof. Cardin Riccardo \\
                \hline
            \end{tabular}
        \end{center}
        \vfill
        \large Contatti: snakebyteteam@gmail.com
    \end{center}
\end{titlepage}

\newpage

\begin{center}
    \begin{tabularx}{\textwidth}{|c|c|c|c|c|X|}
        \hline
        \multicolumn{6}{|c|}{\textbf{Registro delle modifiche}} \\
        \hline
        \rowcolor{lightgray} \textbf{Versione} & \textbf{Data} & \textbf{Autore} & \textbf{Verificatore} & \textbf{Approvatore} & \textbf{Descrizione} \\
        \hline
        1.0.0 & 30/10/2025 & L. Granziero & C. Libralato & L. Pellizzon & Approvazione\\
        \hline
        0.1.1 & 29/10/2025 & L. Granziero & C. Libralato & - & Aggiunta tabella ore individuali per ruolo  e grafici\\
        \hline
        0.1.0 & 28/10/2025 & L. Granziero &  C. Libralato & - & Prima stesura \\
        \hline
    \end{tabularx}
\end{center}

\newpage

\tableofcontents

\newpage

\section{Scopo del documento}{
    Il seguente documento ha l'obiettivo di definire in modo formale gli impegni del gruppo SnakeByte per la realizzazione del progetto \textit{View4Life}, proposto dall'azienda Vimar S.p.A. \\
    In particolare vengono definiti:
    \begin{itemize}
    \item La suddivisione dei ruoli con le rispettive responsabilità;
    \item gli impegni orari di ciascun membro del gruppo SnakeByte;
    \item il preventivo dei costi;
    \item la data prevista per la consegna del prodotto finale.
    \end{itemize}
}

\section{Impegni orari e suddivisione dei ruoli}{
Ogni singolo componente del gruppo si impegna a dedicare al progetto un totale di circa 91 ore produttive. I ruoli all'interno del team saranno ricoperti a rotazione da tutti i componenti, al fine di garantire 
una distribuzione equa del carico di lavoro.
    \subsection{Suddivisione ore e costi}{
    \begin{center}
        \begin{tabular}{|c|c|c|c|c|}
            \hline
            \rowcolor{lightgray} \textbf{Ruolo} & \textbf{Costo orario} & \textbf{Ore totali} & \textbf{Costo totale}
            \\
            \hline
            Responsabile & 30,00 \euro/h & 78 & 2.340,00\euro \\
            \hline
            Amministratore & 20,00\euro/h & 76 & 1520,00\euro\\
            \hline
            Analista & 25,00\euro/h & 86 & 2.150,00\euro \\
            \hline
            Progettista & 25,00\euro/h & 111 & 2.775,00\euro \\
            \hline
            Programmatore & 15,00\euro/h & 142 & 2.130,00\euro\\
            \hline
            Verificatore & 15,00\euro/h & 142 & 2.130,00\euro \\
            \hline
            \cellcolor{lightgray}{\textbf{Totale}} & - & 635 & 13.045,00\euro \\
            \hline
        \end{tabular}
        \end{center}

        \begin{figure}[H]
        \centering
        \includegraphics[width=0.6\textwidth]{./img/Grafico_ruolo_percentuale ore.pdf}
        \caption{Ripartizione percentuale delle ore totali per ruolo}
        \end{figure}
}  
    }
    \subsection{Ruoli e responsabilità}{
        Segue una descrizione dei i ruoli con le relative responsabilità:
        \begin{itemize}
            \item \textbf{Responsabile:} si occupa della coordinazione del gruppo, dei contatti con la proponente e dell'organizzazione delle risorse.
            \item \textbf{Amministratore:} implementa, gestisce e mantiene l'infrastruttura IT di lavoro.
            \item \textbf{Analista:} analizza il contesto del problema e definisce i requisiti funzionali e non funzionali.
            \item \textbf{Progettista:} istanzia l'architettura e definisce le scelte progettuali.
            \item \textbf{Programmatore:} implementa e mantiene il codice secondo le specifiche ricevute.
            \item \textbf{Verificatore:} si dedica alla cura della qualità del prodotto attraverso verifiche e validazioni.
        \end{itemize}
    }

\section{Partizione dei ruoli}{
    La distribuzione delle ore costituisce una pianificazione di riferimento, finalizzata a coprire in modo equilibrato le principali attività di progetto. È prevista la possibilità di apportare adattamenti in corso d'opera, mantenendo 
    invariato sia il monte ore complessivo sia il preventivo economico stabilito.
    \\
    La struttura definita mira a bilanciare responsabilità tecniche e gestionali, garantendo una gestione efficace delle risorse, uno sviluppo coerente con i requisiti e verifiche costanti sulla qualità del prodotto.
    \\
    \\
    In particolare, la ripartizione delle ore è stata definita secondo i seguenti criteri:
    \begin{itemize}
        \item \textbf{Qualità restituita dal Verificatore:} è previsto un impegno significativo per le attività di verifica, al fine di garantire affidabilità e conformità del prodotto;
        \item \textbf{Equilibrio gestionale:} i ruoli di Responsabile e Amministratore sono distribuiti in modo da supportare la pianificazione e il coordinamento durante l'intero ciclo di vita del software, senza togliere attenzione agli aspetti operativi.
        \item \textbf{Progettazione}: è stato assegnata un'importanza adeguata rispetto all'architettura richiesta e alle tecnologie da impiegare.
        \item \textbf{Coerenza tecnica e funzionale}: notevole importanza è stata data ai ruoli di analista e progettista poiché la complessità del progetto richiede qualità nello svolgimento delle relative fasi e nella successiva interazione tra le due, al fine di rendere il lavoro dei programmatori strutturato e guidato.
        
    \end{itemize}
    \subsection{Rotazione dei ruoli}{
        A seguito di varie riunioni è stata presa la decisione di effettuare, all'inizio di ogni sprint, una rotazione dei ruoli assegnati ai membri del gruppo, al fine di garantire la copertura di ogni singolo ruolo in maniera quanto più possibile equa e distribuita. 
    }
}

\section{Preventivo dei costi}{
    Considerando il monte ore complessivo e la distribuzione delle ore tra i vari ruoli, il costo finale previsto del progetto ammonta a \euro 13.045,00.
}
\section{Scandenza prevista}{
    La data di scadenza per la consegna del prodotto finale relativo al capitolato \textit{View4Life} dell'azienda Vimar S.p.A. è improrogabilmente fissata al giorno 18 marzo 2026.
}

\end{document}