\documentclass[10pt, letterpaper]{article}
\usepackage[nomarginpar, margin=2.75cm, tmargin=3cm, bmargin=1.75cm]{geometry}
\usepackage[
    colorlinks=true,      
    linkcolor=black,      
    urlcolor=blue,       
    citecolor=black       
]{hyperref}
\usepackage{graphicx}
\usepackage[table, x11names]{xcolor}
\usepackage{tabularx}
\renewcommand{\arraystretch}{1.2} % migliora la leggibilità
\renewcommand{\contentsname}{Indice}
\usepackage{fancyhdr}
\pagestyle{fancy}
\fancyhf{}
\fancyhead[L]{SnakeByte} 
\fancyhead[R]{Verbale interno 23/10/2025}
\fancyfoot[C]{\thepage}


\begin{document}

\begin{titlepage}
    \begin{center}
        \begin{center}
            \includegraphics[width=0.6\textwidth]{./img/logo.pdf}
        \end{center}
        \vspace{4cm}
        \huge\textbf{Verbale interno 23/10/2025}\par
        \vspace{2cm}
        \large \textbf{SnakeByte} (Gruppo 1):\\
        \large Valeria Baleanu, Leonardo Pellizzon, Filippo Venzo, Giuseppe De Fina, \\
         Francesco Pasqual, Christian Libralato, Luca Granziero \\
        (2109911, 2111006, 2113705, 2113187, 2103119, 2101047, 2075512)
        \vfill
        \small
        \begin{center}
            \begin{tabular}{|c|c|c|c|}
                \hline
                \multicolumn{4}{|c|}{\textbf{Informazioni documento}} \\
                \hline
                \rowcolor{lightgray} \textbf{Versione} & \textbf{Data} & \textbf{Stato} & \textbf{Destinatari} \\
                \hline
                1.0.0 & 30/10/2025 & Approvato & SnakeByte, prof. Tullio Vardanega, prof. Riccardo Cardin \\
                \hline
            \end{tabular}
        \end{center}
        \vfill
        \large Contatti: snakebyteteam@gmail.com
    \end{center}
\end{titlepage}

\newpage

\begin{center}
    \begin{tabularx}{\textwidth}{|c|c|c|c|c|X|}
        \hline
        \multicolumn{6}{|c|}{\textbf{Registro delle modifiche}} \\
        \hline
        \rowcolor{lightgray} \textbf{Versione} & \textbf{Data} & \textbf{Autore} & \textbf{Verificatore} & \textbf{Approvatore} & \textbf{Descrizione} \\
        \hline
        1.0.0 & 30/10/2025 & L. Granziero & C. Libralato & L. Pellizzon & Approvazione \\
        \hline
        0.1.0 & 23/10/2025 & L. Granziero & C. Libralato & - & Prima stesura \\
        \hline
    \end{tabularx}
\end{center}

\newpage

\tableofcontents

\newpage

        
\section{Informazioni}{
    \begin{center}
        \begin{tabular}{|c|c|c|c|}
            \hline
            \rowcolor{lightgray} \textbf{Data} & \textbf{Ora inizio} & \textbf{Ora fine} & \textbf{Modalità} \\
            \hline
            23/10/2025 & 18:00 & 19:30 & via Discord \\
            \hline
        \end{tabular}    
    \end{center}
}

\section{Presenze}{
    \begin{center}
        \begin{tabular}{|c|c|c|c|}
            \hline
            \rowcolor{lightgray} \textbf{Nome} & \textbf{Cognome} & \textbf{Ruolo} & \textbf{Presenza} \\
            \hline
            Valeria & Baleanu & ND & P \\
            \hline
            Leonardo & Pellizzon & Responsabile & P \\
            \hline
            Filippo & Venzo & ND & P \\
            \hline
            Giuseppe & De Fina & ND & A \\
            \hline
            Francesco & Pasqual & ND & P \\
            \hline
            Christian & Libralato & ND & P \\
            \hline
            Luca & Granziero & ND & P \\
            \hline
        \end{tabular}    
    \end{center}
}

\section{Ordine del giorno}{
    \begin{itemize}
        \item Revisione documento Norme di Progetto;
        \item rivisitazione criteri del Documento di valutazione dei capitolati;
        \item gestione avanzamento di versione;
    \end{itemize} 
}

\section{Approfondimento}{
    \subsection*{Revisione documento norme di progetto}{
    Il gruppo ha revisionato il documento "Norme di Progetto", in particolare le voci "Glossario" e "Riferimenti informativi", per poi introdurre delle nuove definizioni relative agli argomenti trattati. \\
    Sono stati successivamente corretti alcuni errori ortografici nel documento.
    }
    \subsection*{Rivisitazione criteri del Documento di valutazione dei capitolati}{
    Il gruppo ha discusso la modalità di valutazione dei capitolati: si è giunti alla conclusione che prevede la definizione di una serie di fattori che descrivono il progetto proposto da vari punti di vista.
    }
    \subsection*{Gestione avanzamento di versioni}{
    Il gruppo ha discusso come gestire e definire l'avanzamento delle versioni.
    Dunque dovrà essere aggiornato il documento di Norme di Progetto riportando quanto definito.
    }
}

\section{Decisioni}
    \begin{center}
        \begin{tabularx}{\textwidth}{|c|X|}
            \hline
            \rowcolor{lightgray} \textbf{Id} & \textbf{Descrizione} \\
            \hline
            vi\_2025\_10\_23.d1 & Metodo di valutazione capitolati \\
            \hline
            vi\_2025\_10\_23.d2 & Convenzioni di avanzamento versione \\
            \hline
        \end{tabularx}
    \end{center}

\section{Attività da svolgere}{
    \begin{center}
        \begin{tabularx}{\textwidth}{|c|X|c|}
            \hline
            \rowcolor{lightgray} \textbf{Id} & \textbf{Descrizione} & \textbf{Id GitHub Issue} \\
            \hline
            vi\_2025\_10\_23.a1 & Bozza documento di valutazione dei capitolati & - \\
            \hline
        \end{tabularx}
    \end{center}
}

\end{document}