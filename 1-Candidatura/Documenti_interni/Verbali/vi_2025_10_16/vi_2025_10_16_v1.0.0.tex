\documentclass[10pt, letterpaper]{article}
\usepackage[nomarginpar, margin=2.75cm, tmargin=3cm, bmargin=1.75cm]{geometry}
\usepackage[
    colorlinks=true,      
    linkcolor=black,      
    urlcolor=blue,       
    citecolor=black       
]{hyperref}
\usepackage{graphicx}
\usepackage[table, x11names]{xcolor}
\usepackage{tabularx}
\renewcommand{\arraystretch}{1.2} % migliora la leggibilità
\renewcommand{\contentsname}{Indice}
\usepackage{fancyhdr}
\pagestyle{fancy}
\fancyhf{}
\fancyhead[L]{SnakeByte} 
\fancyhead[R]{Verbale interno 16/10/2025}
\fancyfoot[C]{\thepage}


\begin{document}

\begin{titlepage}
    \begin{center}
        \begin{center}
            \includegraphics[width=0.6\textwidth]{./img/logo.pdf}
        \end{center}
        \vspace{4cm}
        \huge\textbf{Verbale interno 16/10/2025}\par
        \vspace{2cm}
        \large \textbf{SnakeByte} (Gruppo 1):\\
        \large Valeria Baleanu, Leonardo Pellizzon, Filippo Venzo, Giuseppe De Fina, \\
         Francesco Pasqual, Christian Libralato, Luca Granziero \\
        (2109911, 2111006, 2113705, 2113187, 2103119, 2101047, 2075512)
        \vfill
        \small
        \begin{center}
            \begin{tabular}{|c|c|c|c|}
                \hline
                \multicolumn{4}{|c|}{\textbf{Informazioni documento}} \\
                \hline
                \rowcolor{lightgray} \textbf{Versione} & \textbf{Data} & \textbf{Stato} & \textbf{Destinatari} \\
                \hline
                1.0.0 & 30/10/2025 & Approvato & SnakeByte, prof. Tullio Vardanega, prof. Riccardo Cardin \\
                \hline
            \end{tabular}
        \end{center}
        \vfill
        \large Contatti: snakebyteteam@gmail.com
    \end{center}
\end{titlepage}

\newpage

\begin{center}
    \begin{tabularx}{\textwidth}{|c|c|c|c|c|X|}
        \hline
        \multicolumn{6}{|c|}{\textbf{Registro delle modifiche}} \\
        \hline
        \rowcolor{lightgray} \textbf{Versione} & \textbf{Data} & \textbf{Autore} & \textbf{Verificatore} & \textbf{Approvatore} & \textbf{Descrizione} \\
        \hline
        1.0.0 & 30/10/2025 & L. Pellizzon & C. Libralato & L. Pellizzon & Approvazione \\
        \hline
        0.1.0 & 16/10/2025 & L. Pellizzon & C. Libralato & - & Prima stesura \\
        \hline
    \end{tabularx}
\end{center}

\newpage

\tableofcontents

\newpage

\section{Informazioni}{
    \begin{center}
        \begin{tabular}{|c|c|c|c|}
            \hline
            \rowcolor{lightgray} \textbf{Data} & \textbf{Ora inizio} & \textbf{Ora fine} & \textbf{Modalità} \\
            \hline
            16/10/2025 & 12:15 & 13:15 & in presenza \\
            \hline
        \end{tabular}    
    \end{center}
}

\section{Presenze}{
    \begin{center}
        \begin{tabular}{|c|c|c|c|}
            \hline
            \rowcolor{lightgray} \textbf{Nome} & \textbf{Cognome} & \textbf{Ruolo} & \textbf{Presenza} \\
            \hline
            Valeria & Baleanu & ND & P \\
            \hline
            Leonardo & Pellizzon & Responsabile & P \\
            \hline
            Filippo & Venzo & ND & P \\
            \hline
            Giuseppe & De Fina & ND & P \\
            \hline
            Francesco & Pasqual & ND & P \\
            \hline
            Christian & Libralato & ND & P \\
            \hline
            Luca & Granziero & ND & P \\
            \hline
        \end{tabular}    
    \end{center}
}

\section{Ordine del giorno}{
    \begin{itemize}
        \item Discussione riguardante il nome e il logo del gruppo;
        \item scelta dei canali di comunicazione e tracciamento attività del gruppo;
        \item prima stesura delle Norme di Progetto;
        \item dichiarazione impegno orario settimanale per persona;
        \item discussione incentrata sui capitolati.
    \end{itemize}
}

\section{Approfondimento}{
    \subsection*{Scelta dei canali di comunicazione e tracciamento attività del gruppo}{
        Il gruppo comunicherà e traccerà le attività con l'ausilio di:
        \begin{itemize}
            \item \textit{Discord} per la definire i giorni in cui avvengono le riunioni;
            \item \textit{GitHub} per il tracciamento delle attività, la gestione del codice e della documentazione;
            \item \textit{Google Calendar} per la gestione delle scadenze;
            \item \textit{Google Sheets} per la gestione degli orari.
        \end{itemize}
    }
    \subsection*{Prima stesura delle Norme di Progetto}{
        Il gruppo ha deciso la struttura che il documento "Norme di Progetto" dovrà avere e i primi contenuti che
        riguardano:
        \begin{itemize}
            \item struttura dei documenti;
            \item struttura dei processi organizzativi.
        \end{itemize}
    }
    \subsection*{Dichiarazione impegno orario settimanale per persona}{
        Ogni componente del gruppo ha dichiarato quante ore settimanali
        dedicherà al progetto (potrebbero subire cambiamenti):
        \begin{center}
        \begin{tabular}{|c|c|}
            \hline
            \rowcolor{lightgray} \textbf{Membro} & \textbf{Ore settimanali} \\
            \hline
            Valeria Baleanu & 12 \\
            \hline
            Leonardo Pellizzon & 13 \\
            \hline
            Filippo Venzo & 11 \\
            \hline
            Giuseppe De Fina & 13 \\
            \hline
            Francesco Pasqual & 14 \\
            \hline
            Christian Libralato & 12 \\
            \hline
            Luca Granziero & 13 \\
            \hline
        \end{tabular}
        \end{center}
    }
    \subsection*{Discussione incentrata sui capitolati}{
        Dopo una prima analisi dei capitolati proprosti, il gruppo ha dimostrato 
        maggior interesse per i seguenti:
        \begin{enumerate}
            \item \textit{View4Life} di Vimar S.p.A.;
            \item \textit{Code Guardian} di Var Group S.p.A.;
            \item \textit{Sistema di acquisizione dati da sensori} di M31 S.r.l..
        \end{enumerate}
    }
}

\section{Decisioni}
    \begin{center}
        \begin{tabularx}{\textwidth}{|c|X|}
            \hline
            \rowcolor{lightgray} \textbf{Id} & \textbf{Descrizione} \\
            \hline
            vi\_2025\_10\_16.d1 & Nome, logo e email gruppo \\
            \hline
            vi\_2025\_10\_16.d2 & Canali di comunicazione: \textit{Discord} \\
            \hline
            vi\_2025\_10\_16.d3 & Strumento di gestione documentazione, codice e attività: \textit{GitHub}  \\
            \hline
            vi\_2025\_10\_16.d4 & Strumento di tracciamento orari e scadenze: \textit{Google Sheets} e \textit{Google Calendar}  \\
            \hline
        \end{tabularx}
    \end{center}

\section{Attività da svolgere}{
    \begin{center}
        \begin{tabularx}{\textwidth}{|c|X|c|}
            \hline
            \rowcolor{lightgray} \textbf{Id} & \textbf{Descrizione} & \textbf{Id GitHub Issue} \\
            \hline
            vi\_2025\_10\_16.a1 & Creare l'organizzazione su GitHub & - \\
            \hline
            vi\_2025\_10\_16.a2 & Creare il logo e l'account Gmail del gruppo & - \\
            \hline
            vi\_2025\_10\_16.a3 & Segnare le scadenze dei Diari di bordo e della presentazione della
            candidatura in Google Calendar & - \\
            \hline
            vi\_2025\_10\_16.a4 & Prima stesura Norme di Progetto & - \\
            \hline
        \end{tabularx}
    \end{center}
}
\end{document}