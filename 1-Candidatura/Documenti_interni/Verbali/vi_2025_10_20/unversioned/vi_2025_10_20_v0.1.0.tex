\documentclass[10pt, letterpaper]{article}
\usepackage[nomarginpar, margin=2.75cm, tmargin=3cm, bmargin=1.75cm]{geometry}
\usepackage[
    colorlinks=true,      
    linkcolor=black,      
    urlcolor=blue,       
    citecolor=black       
]{hyperref}
\usepackage{graphicx}
\usepackage[table, x11names]{xcolor}
\usepackage{tabularx}
\renewcommand{\arraystretch}{1.2} 
\renewcommand{\contentsname}{Indice}
\usepackage{fancyhdr}
\pagestyle{fancy}
\fancyhf{}
\fancyhead[L]{SnakeByte} 
\fancyhead[R]{Verbale interno 20/10/2025}
\fancyfoot[C]{\thepage}


\begin{document}

\begin{titlepage}
    \begin{center}
        \begin{center}
            \includegraphics[width=0.6\textwidth]{../img/logo.pdf}
        \end{center}
        \vspace{4cm}
        \huge\textbf{Verbale interno del 20/10/2025}\par
        \vspace{2cm}
        \large \textbf{SnakeByte} (Gruppo 1):\\
        \large Valeria Baleanu, Leonardo Pellizzon, Filippo Venzo, Giuseppe De Fina, \\
         Francesco Pasqual, Christian Libralato, Luca Granziero \\
        (2109911, 2111006, 2113705, 2113187, 2103119, 2101047, 2075512)
        \vfill
        \small
        \begin{center}
            \begin{tabular}{|c|c|c|c|}
                \hline
                \multicolumn{4}{|c|}{\textbf{Informazioni documento}} \\
                \hline
                \rowcolor{lightgray} \textbf{Versione} & \textbf{Data} & \textbf{Stato} & \textbf{Destinatari} \\
                \hline
                0.1.0 & 20/10/2025 & Verificato & SnakeByte, prof. Vardanega Tullio, prof. Cardin Riccardo \\
                \hline
            \end{tabular}
        \end{center}
        \vfill
        \large Contatti: snakebyteteam@gmail.com
    \end{center}
\end{titlepage}

\newpage

\begin{center}
    \begin{tabularx}{\textwidth}{|c|c|c|c|c|X|}
        \hline
        \multicolumn{6}{|c|}{\textbf{Registro delle modifiche}} \\
        \hline
        \rowcolor{lightgray} \textbf{Versione} & \textbf{Data} & \textbf{Autore} & \textbf{Verificatore} & \textbf{Approvatore} & \textbf{Descrizione} \\
        \hline
        0.1.0 & 20/10/2025 & F. Pasqual & F. Venzo & - & Prima stesura \\
        \hline
    \end{tabularx}
\end{center}

\newpage

\tableofcontents

\newpage

        
\section{Informazioni}
    \begin{center}
        \begin{tabular}{|c|c|c|c|}
            \hline
            \rowcolor{lightgray} \textbf{Data} & \textbf{Ora inizio} & \textbf{Ora fine} & \textbf{Modalità} \\
            \hline
            20/10/2025 & 12:15 & 14:00 & in presenza \\
            \hline
        \end{tabular}    
    \end{center}


\section{Presenze}
    \begin{center}
        \begin{tabular}{|c|c|c|c|}
            \hline
            \rowcolor{lightgray} \textbf{Nome} & \textbf{Cognome} & \textbf{Ruolo} & \textbf{Presenza} \\
            \hline
            Valeria & Baleanu & ND & P \\
            \hline
            Leonardo & Pellizzon & Responsabile & P \\
            \hline
            Filippo & Venzo & ND & P \\
            \hline
            Giuseppe & De Fina & ND & P \\
            \hline
            Francesco & Pasqual & ND & P \\
            \hline
            Christian & Libralato & ND & P \\
            \hline
            Luca & Granziero & ND & P \\
            \hline
        \end{tabular}    
    \end{center}


\section{Ordine del giorno}
    \begin{itemize}
        \item Pianificazione di progetto;
        \item stima di alcune scadenze e date di calendario;
        \item creazione del sito web associato al progetto.
        \item organizzazione della repository \textit{GitHub} per la documentazione;
    \end{itemize}

\section{Approfondimento}
    \subsection*{Pianificazione di progetto}
        Il gruppo ha discusso riguardo come e dove fissare gli obiettivi di avanzamento delle attività,
        valutando principalmente strumenti quali:
        \begin{itemize}
            \item la \textit{view Kanban} di \textit{GitHub} per gestire gli stati delle attività di progetto 
            e le \textit{issue} loro associate;
            \item i \textit{diagrammi Gantt} per organizzare temporalmente le attività.
        \end{itemize}

        Il gruppo ha individuato come primo obiettivo signficativo, o \textit{milestone}, la consegna della lettera di candidatura il 31 ottobre. 
    

    \subsection*{Stima di alcune scadenze e date di calendario}
        Il gruppo ha discusso alcune possibili date di calendario e scadenze per l'obiettivo finale di consegna del progetto.
        Le prime previsioni, seppur meramente indicative, prevedono: 
        \begin{itemize}
            \item \textit{MVP} definitivo entro il 2 marzo 2026;
            \item  \textit{slack time} fino al 15 marzo 2026;
            \item consegna definitiva \textit{PB} nella settimana del 16 marzo 2026.
        \end{itemize}
    
    
    \subsection*{Creazione del sito web associato al progetto}
        Il gruppo ha discusso le modalità di creazione e organizzazione del sito web associato al progetto. 
        In questa fase iniziale, si ritiene che tale sito conterrà esclusivamente la documentazione del progetto. 
        Per quanto riguarda la sua creazione, l'unica possibilità emersa è quella di utilizzare il servizio \textit{GitHub Pages}.
        Di conseguenza, vi è la necessità di creare una repository \textit{GitHub} pubblica "snakebyteteam.github.io" per ospitare il sito.


    \subsection*{Organizzazione della repository GitHub per la documentazione}
        Il gruppo ha discusso riguardo alle possibilità di versionamento della documentazione prodotta durante l'intero progetto.
        In particolare, sono state discusse le seguenti modalità:
        \begin{itemize}
            \item creare un'unica repository "documenti" contenente tutta la documentazione.
            Al suo interno, i documenti verranno suddivisi in sottocartelle a seconda della loro categoria.
            Il primo livello di sottocartelle sarà rappresentato dalla divisione tra documenti "interni" ed "esterni";
            \item creare direttamente due repository separate: una per i "documenti interni" e una per i "documenti esterni";
            \item usare la repository del sito "snakebyteteam.github.io" come unico contentitore della documentazione. 
        \end{itemize}
    


\section{Decisioni}
    \begin{center}
        \begin{tabularx}{\textwidth}{|c|X|}
            \hline
            \rowcolor{lightgray} \textbf{Id} & \textbf{Descrizione} \\
            \hline
            vi2\_2025\_10\_20.d1 & \textit{Kanban view} per fissare le attività da svolgere \\
            \hline
            vi2\_2025\_10\_20.d2 & \textit{Diagrammi Gantt} per pianificare e ordinare temporalmente le attività da svolgere \\
            \hline
            vi2\_2025\_10\_20.d3 & Date di calendario: terminazione \textit{MVP} entro 2 marzo e consegna entro il 18 marzo \\
            \hline
            vi2\_2025\_10\_20.d4 & Prima \textit{milestone}: consegna della lettera di candidatura il 31 ottobre \\
            \hline
            vi2\_2025\_10\_20.d5 & Repository snakebyteteam.github.io per contenere la documentazione \\
            \hline
            vi2\_2025\_10\_20.d6 & Repository snakebyteteam.github.io per ospitare il sito web \\
            \hline
        \end{tabularx}
    \end{center}


\section{Attività da svolgere}
    \begin{center}
        \begin{tabularx}{\textwidth}{|c|X|c|}
            \hline
            \rowcolor{lightgray} \textbf{Id} & \textbf{Descrizione} & \textbf{Id GitHub Issue} \\
            \hline
            vi2\_2025\_10\_20.a1 & Creare \textit{GitHub Project} con \textit{view Kanban} & - \\
            \hline
            vi2\_2025\_10\_20.a2 & Creare la \textit{milestone} "Candidatura" & - \\
            \hline  
            vi2\_2025\_10\_20.a3 & Creare repository \textit{GitHub} snakebyteteam.github.io & - \\
            \hline
            vi2\_2025\_10\_20.a4 & Creare il sito web & - \\
            \hline
        \end{tabularx}
    \end{center}


\end{document}