\documentclass[10pt, letterpaper]{article}
\usepackage[nomarginpar, margin=2.75cm, tmargin=3cm, bmargin=1.75cm]{geometry}
\usepackage[
    colorlinks=true,      
    linkcolor=black,      
    urlcolor=blue,       
    citecolor=black       
]{hyperref}
\usepackage{graphicx}
\usepackage[table, x11names]{xcolor}
\usepackage{tabularx}
\renewcommand{\arraystretch}{1.2} % migliora la leggibilità
\renewcommand{\contentsname}{Indice}
\usepackage{fancyhdr}
\pagestyle{fancy}
\fancyhf{}
\fancyhead[L]{SnakeByte} 
\fancyhead[R]{Verbale esterno 21/10/2025}
\fancyfoot[C]{\thepage}


\begin{document}

\begin{titlepage}
    \begin{center}
        \begin{center}
            \includegraphics[width=0.6\textwidth]{./img/logo.pdf}
        \end{center}
        \vspace{4cm}
        \huge\textbf{Verbale esterno 21/10/2025}\par
        \vspace{2cm}
        \large \textbf{SnakeByte} (Gruppo 1):\\
        \large Valeria Baleanu, Leonardo Pellizzon, Filippo Venzo, Giuseppe De Fina, \\
         Francesco Pasqual, Christian Libralato, Luca Granziero \\
        (2109911, 2111006, 2113705, 2113187, 2103119, 2101047, 2075512)
        \vfill
        \small
        \begin{center}
            \begin{tabular}{|c|c|c|c|}
                \hline
                \multicolumn{4}{|c|}{\textbf{Informazioni documento}} \\
                \hline
                \rowcolor{lightgray} \textbf{Versione} & \textbf{Data} & \textbf{Stato} & \textbf{Destinatari} \\
                \hline
                0.1.0 & 21/10/2025 & Verificato & SnakeByte, Prof. Vardanega Tullio e Prof. Cardin Riccardo \\
                \hline
            \end{tabular}
        \end{center}
        \vfill
        \large Contatti: snakebyteteam@gmail.com
    \end{center}
\end{titlepage}

\newpage

\begin{center}
    \begin{tabularx}{\textwidth}{|c|c|c|c|c|X|}
        \hline
        \multicolumn{6}{|c|}{\textbf{Registro delle modifiche}} \\
        \hline
        \rowcolor{lightgray} \textbf{Versione} & \textbf{Data} & \textbf{Autore} & \textbf{Verificatore} & \textbf{Approvatore} & \textbf{Descrizione} \\
        \hline
        0.1.0 & 21/10/2025 & L. Granziero & C. Libralato & L. Pellizzon & Prima stesura \\
        \hline
    \end{tabularx}
\end{center}

\newpage

\tableofcontents

\newpage

\section{Informazioni}{
    \begin{center}
        \begin{tabular}{|c|c|c|c|}
            \hline
            \rowcolor{lightgray} \textbf{Data} & \textbf{Ora inizio} & \textbf{Ora fine} & \textbf{Modalità} \\
            \hline
            21/10/2025 & - & - & via Email \\
            \hline
        \end{tabular}    
    \end{center}
}

\section{Presenze}{
    \begin{center}
        \begin{tabular}{|c|c|c|c|}
            \hline
            \rowcolor{lightgray} \textbf{Nome} & \textbf{Cognome} & \textbf{Ruolo} & \textbf{Presenza} \\
            \hline
            Valeria & Baleanu &  & P \\
            \hline
            Leonardo & Pellizzon & Responsabile & P \\
            \hline
            Filippo & Venzo &  & P \\
            \hline
            Giuseppe & De Fina &  & P \\
            \hline
            Francesco & Pasqual &  & P \\
            \hline
            Christian & Libralato &  & P \\
            \hline
            Luca & Granziero &  & P \\
            \hline
        \end{tabular}    
    \end{center}
}

\section{Ordine del giorno}{
    Il gruppo ha posto alcune domande al proponente \textbf{Vimar S.p.A} riguardo al progetto \textbf{View4Life}.
}

\section{Approfondimento}{
    \subsection{Domande}{
        \begin{enumerate}
            \item \textbf{Domanda}: In caso di attivazione di molteplici allarmi in contemporanea, sarà necessario gestire un ordine di priorità di questi ultimi? \\ \\
            \textbf{Risposta}: La priorità degli allarmi può essere gestita in una modalità a vostra discrezione. Per semplicità, potete pensarla in base al tempo, quindi 
            se due allarmi arrivano nell’arco di pochi secondi, va gestito il primo pendente dei due, con la logica first-come-first-served. \\ \\
            Come nice-to-have, potete valutare anche una priorità per tipologia di allarme, in base all’entità del danno che potrebbe essere arrecato 
            a un ospite della struttura. Ad esempio, un allarme da caduta può avere maggiore priorità rispetto a un allarme di presenza in una stanza non 
            autorizzata, ma potrebbe avere la stessa priorità di una chiamata col pulsante di allarme.
            \item \textbf{Domanda}: All’interno del sistema di accesso utente dell’applicativo, il ruolo di utente amministratore (realizzazione opzionale) avrà qualche 
            tipo di privilegio funzionale, escludendo la gestione degli utenti, rispetto agli utenti standard? In cosa consisterà la gestione degli 
            utenti? Assegnazione di permessi e/o limitazioni e responsabilità? \\ \\
            \textbf{Risposta}: L’utente amministratore è di fatto il direttore della struttura. Pertanto, ci aspettiamo che abbia gli stessi permessi di un operatore 
            sanitario (vedere allarmi, analytics e impianti) ma con una o più sezioni aggiuntive di tipo amministrativo per poter gestire gli utenti 
            e gli allarmi. In particolare:
            \begin{itemize}
                \item Gestione utenti: visualizzare / aggiungere / modificare / eliminare utenti (operazioni chiamate in gergo CRUD: create, read, update, delete)
                A livello utente possono essere definite informazioni di base come nome utente e password, ma a vostra discrezione si possono definire anche 
                altre informazioni come il ruolo da mostrare (es. Infermiere, Medico, ecc.) o il reparto; questo può essere in aiuto per la gestione avanzata 
                degli allarmi, in cui potete decidere di abilitare solo certi tipi di allarmi globalmente, o limitatamente a certe figure (e quindi a un 
                gruppo di utenti).
                \item Gestione avanzata allarmi: come da capitolato (pag. 10) visualizzare i tipi allarmi e i sensori che li scatenano, nonché disattivare 
                globalmente alcuni allarmi. Inoltre, si può gestire il parametro soglia di attivazione dei tipi di allarme (es. l’allarme di presenza di una 
                stanza si attiva solo se l’utente permane per almeno 2 minuti in un ambiente; oppure, se l’utente preme 3 volte il pulsante di allarme 
                nell’arco di 1 minuto, parte un allarme con priorità più alta).
            \end{itemize}
            Per una questione di semplicità e di praticità, consiglio caldamente di implementare la figura amministrativa almeno per la gestione utenti, 
            così da evitare gestioni da parte di una figura tipo DBA che si mette a fare query manuali sul database.
            \item \textbf{Domanda}: Il software dovrà essere in grado di gestire anche casistiche di disconnessioni e/o anomalie dei sensori? \\ \\
            \textbf{Risposta}: Il sistema deve essere in grado di sostenere i casi di malfunzionamento di impianto. Qui è a vostra discrezione di cosa sia opportuno mostrare. \\ \\
            In genere, se il gateway si disconnette, sarebbe opportuno mostrare un banner generale in tutte le sezioni del sito dove si informano operatori 
            sanitari e gestori che c'è un malfunzionamento generale
            \item \textbf{Domanda}: I suggerimenti all’utente per la riduzione del consumo energetico dipenderanno esclusivamente dai dati raccolti o anche da medie generali 
            della struttura? \\ \\
            \textbf{Risposta}: I suggerimenti vanno prodotti a partire dai dati dati grezzi che raccogliete dai sensori. Lo storico dei dati può essere utile nel vostro caso per 
            fare dei suggerimenti più mirati. KNX IoT 3rd party API ha un endpoint timeseries che vi aiuta a reperire lo storico dei dati che mostra gli stati 
            precedenti dei sensori (es. quando la luce è cambiata di stato, quando il sensore è cambiato da assenza / presenza, come la temperatura rilevata è variata). \\ \\
            Potete decidere voi se i suggerimenti devono partire dalle medie generali della struttura (i.e. se ho interpretato bene, media generale di consumo 
            energetico di tutta la residenza protetta) o se volete fare suggerimenti per appartamento / impianto della struttura. \\ \\
            In caso, quando entriamo più nel dettaglio nel progetto, potete farci delle proposte, visto che questa parte è a vostra piena discrezione.            
        \end{enumerate}
    }
}

\section{Attività da svolgere}{
    \begin{center}
        \begin{tabularx}{\textwidth}{|c|X|c|}
            \hline
            \rowcolor{lightgray} \textbf{Id} & \textbf{Descrizione} & \textbf{Id GitHub Issue} \\
            \hline
            ve\_21-10-2025\_Vimar.a1 & Ogni componente del gruppo dovrà analizzare le risposte ricevute al fine di valutare se il capitolato \emph{View4Life} rispecchia il suo interesse & - \\
            \hline
        \end{tabularx}
    \end{center}
}

\end{document}