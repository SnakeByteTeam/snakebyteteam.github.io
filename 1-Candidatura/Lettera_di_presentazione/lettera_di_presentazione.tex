\documentclass[10pt, letterpaper]{article}
\usepackage[nomarginpar, margin=2.75cm, tmargin=3cm, bmargin=1.75cm]{geometry}
\usepackage[
    colorlinks=false,                 % disabilita colore nel testo (usa bordi/underline)
    linkbordercolor={0 0 1},          % colore bordo/underline per link interni (RGB)
    urlbordercolor={0 0 1},           % colore bordo/underline per URL
    citebordercolor={0 0 0},          % colore bordo/underline per citazioni
    pdfborderstyle={/S/U/W 1}          % stile bordo: U = underline, W = spessore
]{hyperref}
\usepackage{graphicx}
\usepackage[table, x11names]{xcolor}
\usepackage{tabularx}
\usepackage[gen]{eurosym}
\renewcommand{\arraystretch}{1.2} 
\renewcommand{\contentsname}{Indice}
\usepackage{fancyhdr}
\pagestyle{fancy}
\fancyhf{}
\fancyhead[L]{SnakeByte} 
\fancyhead[R]{Lettera di Presentazione}
\fancyfoot[C]{\thepage}


\begin{document}

\begin{titlepage}
    \begin{center}
        \begin{center}
            \includegraphics[width=0.6\textwidth]{./img/logo.pdf}
        \end{center}
        \vspace{4cm}
        \huge\textbf{Lettera di Presentazione}\par
        \vspace{2cm}
        \large \textbf{SnakeByte} (Gruppo 1):\\
        \large Valeria Baleanu, Leonardo Pellizzon, Filippo Venzo, Giuseppe De Fina, \\
         Francesco Pasqual, Christian Libralato, Luca Granziero \\
        (2109911, 2111006, 2113705, 2113187, 2103119, 2101047, 2075512)
        \vfill
        \small
        \begin{center}
            \begin{tabular}{|c|c|}
                \hline
                \multicolumn{2}{|c|}{\textbf{Informazioni documento}} \\
                \hline
                \rowcolor{lightgray} \textbf{Data} & \textbf{Destinatari} \\
                \hline
                30/10/2025 & SnakeByte, prof. Vardanega Tullio e prof. Cardin Riccardo \\
                \hline
            \end{tabular}
        \end{center}
        \vfill
        \large Contatti: snakebyteteam@gmail.com
    \end{center}
\end{titlepage}

\newpage

\noindent 
Il presente documento intende comunicare la decisione del gruppo \textit{SnakeByte} di candidarsi e impegnarsi nella realizzazione del capitolato C9 \textit{\textbf{View4Life}}, proposto da \textbf{Vimar S.p.A.}. \\
La documentazione a sostegno di tale decisione, costituita dai verbali interni ed esterni, dalla valutazione dei capitolati e dalla dichiarazione degli impegni,
è disponibile \href{https://snakebyteteam.github.io/index.html}{al sito web ufficiale del gruppo}\footnote{\href{https://snakebyteteam.github.io/index.html}{https://snakebyteteam.github.io/index.html}}.


\paragraph{Motivazione scelta}
La scelta è stata presa dopo un'attenta analisi di tutti i capitolati, consultabile al link \href{https://snakebyteteam.github.io/1-Candidatura/Valutazione_dei_capitolati_v1.0.0.pdf}{valutazione dei capitolati}\footnote{\href{https://snakebyteteam.github.io/1-Candidatura/Valutazione_dei_capitolati_v1.0.0.pdf}{https://snakebyteteam.github.io/1-Candidatura/Valutazione\_dei\_capitolati\_v1.0.0.pdf}}. 
Il progetto proposto ci risulta stimolante, completo e con un elevato valore formativo. 
In particolare, abbiamo valutato che la sua complessità, seppur elevata, è sostenibile dal gruppo, e crediamo che offrirà a tutti 
i membri la possibilità di lavorare su più aspetti dello sviluppo software, offrendo così una visione concreta 
e approfondita dell'intero processo progettuale.  \\
Un elemento che ha inciso in modo determinante sulla scelta è la chiarezza con cui il progetto è presentato: la documentazione 
risulta precisa, ben strutturata e arricchita da materiali che facilitano la comprensione degli obiettivi e delle modalità di realizzazione. 
Anche il supporto offerto dalla proponente si distingue positivamente per disponibilità e organizzazione, garantendo un percorso di lavoro seguito, 
costruttivo e collaborativo. \\
Infine, la presenza di numerose possibilità di estensione e miglioramento rende il progetto particolarmente interessante, 
offrendo ampi margini di creatività e approfondimento personale. 


\paragraph{Conclusioni}
Come indicato nel documento \textit{Dichiarazione degli Impegni}, consultabile al \href{https://snakebyteteam.github.io/1-Candidatura/Dichiarazione_degli_impegni_v1.0.0.pdf}{seguente link}\footnote{\href{https://snakebyteteam.github.io/1-Candidatura/Dichiarazione_degli_impegni_v1.0.0.pdf}{https://snakebyteteam.github.io/1-Candidatura/Dichiarazione\_degli\_impegni\_v1.0.0.pdf}},
il costo complessivo stimato per la realizzazione del prodotto finale è di \EUR{13.045,00}. \\ 
La consegna del prodotto finale avverrà il 18 marzo 2026. \\ \\ 
Rinnoviamo il nostro sincero interesse per il capitolato \textit{View4Life} e la nostra disponibilità a 
collaborare con Vimar S.p.A. per la realizzazione di un progetto di qualità. \\
Il team \textit{SnakeByte}.\\

\begin{center}
    \begin{tabular}{|c|c|}
        \hline
        \rowcolor{lightgray} \textbf{Membro} & \textbf{Matricola} \\
        \hline
        Valeria Baleanu & 2109911 \\
        \hline
        Leonardo Pellizzon & 2111006 \\
        \hline
        Filippo Venzo & 2113705 \\
        \hline
        Giuseppe De Fina & 2113187 \\
        \hline
        Francesco Pasqual & 2103119 \\
        \hline
        Christian Libralato & 2101047 \\
        \hline
        Luca Granziero & 2075512 \\
        \hline
    \end{tabular}
\end{center}


\end{document}