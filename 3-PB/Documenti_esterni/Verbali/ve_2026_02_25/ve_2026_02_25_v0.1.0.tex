\documentclass[10pt, letterpaper]{article}
\usepackage{../../../../template/template}
% ------------------------
% Impostazioni documento
% ------------------------

\setDocTitle{Verbale esterno 25/02/2026}
\setDocVersion{0.1.0}
\setDocData{28/02/2026}
\setDocIndex{ve\_2026\_02\_25}
\setDocState{1}
\setDocRecipient{2}

\begin{document}

\makeTitlePage

\newpage 
\addChangelog{0.1.0}{28/02/2026}{L. Granziero}{C. Libralato}{\noOne}{\firstDraft}
\makeChangelog

\newpage

\tableofcontents

\newpage

\makeMeetingInfoTable{25/02/2026}{15:30}{16:20}{via \textit{Microsoft Teams$_G$}}

% ------------------------
% Partecipanti
% ------------------------
\addParticipant{Filippo}{Venzo}{Progettista}{P}
\addParticipant{Valeria}{Baleanu}{Progettista}{P}
\addParticipant{Luca}{Granziero}{Responsabile}{A}
\addParticipant{Christian}{Libralato}{Verificatore}{P}
\addParticipant{Francesco}{Pasqual}{Progettista}{P}
\addParticipant{Leonardo}{Pellizzon}{Amministratore}{P}
\addParticipant{Giuseppe}{De Fina}{Progettista}{P}
\makeMeetingParticipantsTable

\section{Ordine del giorno}
\begin{itemize}
  \item Aggiornamento avanzamento progetto e pianificazione Sprint$_G$ 9;
  \item Scelte tecnologiche e stato della progettazione del database;
  \item Requisiti di compatibilità browser e supporto mobile;
  \item Preparazione della presentazione per il professor Vardanega;
  \item Discussione architetturale e organizzazione delle attività future.
\end{itemize}

\section{Approfondimento}

\subsection{Avanzamento progetto e pianificazione Sprint 9}
È stato illustrato lo stato di avanzamento del progetto, aggiornando il gruppo sulle attività svolte, sulla pianificazione dello sprint corrente e sulla preparazione della presentazione per il professor Vardanega. In particolare, è stata presentata la composizione del team per lo Sprint 9, evidenziando la presenza di quattro progettisti.
\noindent
È stato inoltre chiarito che, in considerazione della fase progettuale corrente, i programmatori sono stati convertiti al ruolo di progettisti, al fine di garantire maggiore coerenza con gli obiettivi dello sprint e favorire il progresso delle attività di modellazione e definizione dell’architettura.

\subsection{Scelte tecnologiche e stato della progettazione}
È stato presentato l’avvio della fase di progettazione, con particolare attenzione alla modellazione del database mediante diagramma ER. Il gruppo ha illustrato la struttura delle entità e delle relazioni, affrontando in dettaglio la gestione della gerarchia tra unità, appartamenti, stanze e dispositivi e ricevendo suggerimenti per ottimizzare la struttura e garantire l’univocità dei dati.
\noindent
È stata inoltre confermata l’adozione di \textit{NestJS$_G$} in sostituzione di \textit{Express$_G$}, seguendo i suggerimenti ricevuti, e l’utilizzo di \textit{PostgreSQL$_G$} come database relazionale principale, affiancato da \textit{TimescaleDB$_G$} per la gestione efficiente delle serie temporali. In particolare, è stata illustrata la progettazione di una hypertable per la memorizzazione dei dati di consumo energetico, basata sull’associazione tra timestamp e dispositivo.
\noindent
Sono state inoltre discusse le strategie di sincronizzazione con le API esterne e la gestione della struttura dell’impianto, valutando l’utilizzo di un sistema di caching per ridurre il numero di chiamate e migliorare la scalabilità complessiva del sistema. In tale contesto, è stato suggerito di mantenere la struttura principale nel database relazionale, affiancata da una cache per ottimizzare l’accesso ai dati e semplificare la gestione delle modifiche.

\subsection{Gestione allarmi, dashboard e ruoli utente}
Il gruppo ha presentato la progettazione delle tabelle relative alla gestione degli allarmi, distinguendo tra configurazione e istanze attive, al fine di garantire la tracciabilità degli eventi e la conservazione dello storico.
\noindent
È stata inoltre illustrata la progettazione della dashboard, basata su un layout modulare configurabile, con la possibilità di salvare le impostazioni utente e mantenere la coerenza nella gestione dei dati e delle visualizzazioni.
\noindent
Infine, è stata discussa la gestione dei ruoli utente, valutando l’utilizzo di strutture dedicate per garantire chiarezza, univocità e flessibilità nella definizione dei permessi e delle responsabilità.

\subsection{Requisiti di compatibilità browser e supporto mobile}
Sono stati chiariti i requisiti minimi di compatibilità browser dell’applicativo, stabilendo il supporto prioritario per Chrome, Edge e Safari, mentre il supporto per Firefox è stato considerato opzionale. È stato inoltre specificato che non sarà previsto il supporto per browser obsoleti.
\noindent
È stato inoltre suggerito di indicare come requisito minimo la prima versione disponibile nel 2026 per ciascun browser supportato, al fine di garantire chiarezza nella documentazione e coerenza con i requisiti progettuali.
\noindent
Infine, è stata evidenziata l’importanza di garantire il corretto funzionamento dell’applicativo anche su dispositivi mobili, assicurando compatibilità con le principali funzionalità JavaScript e un’esperienza utente priva di anomalie grafiche.

\subsection{Preparazione presentazione e organizzazione attività}
Il gruppo ha discusso la preparazione della presentazione per il professor Vardanega, con particolare attenzione alla gestione documentale, alla pianificazione e alla presentazione delle scelte progettuali effettuate.
\noindent
Infine, è stata sottolineata l’importanza di comunicare tempestivamente eventuali difficoltà, al fine di consentire una gestione efficace delle criticità e garantire il raggiungimento degli obiettivi progettuali.

% ------------------------
% Decisioni
% ------------------------

\addDecision{È stata confermata l’adozione di NestJS come framework principale per lo sviluppo del backend.}

\addDecision{È stato confermato l’utilizzo di PostgreSQL come database relazionale e TimescaleDB per la gestione delle serie temporali.}

\addDecision{Il supporto browser prioritario includerà Chrome, Edge e Safari, con versioni minime corrispondenti alle prime release disponibili nel 2026.}

\addDecision{È stato stabilito che l’applicativo dovrà garantire piena compatibilità anche su dispositivi mobili.}

\addDecision{La struttura dell’impianto sarà gestita tramite database relazionale affiancato da un sistema di caching per migliorare scalabilità ed efficienza.}

\makeDecisionTable

% ------------------------
% TODO
% ------------------------

\addTodo{-}{Aggiornare i requisiti del progetto specificando i browser supportati e le relative versioni minime.}{Gruppo \teamName}{28/02/2026}

\addTodo{-}{Raffinare il modello ER, chiarendo le relazioni tra unità, appartamenti e dispositivi.}{Gruppo \teamName}{28/02/2026}

\addTodo{-}{Definire e presentare lo schema di gestione dei dati per TimescaleDB.}{Gruppo \teamName}{4/03/2026}

\makeTodoTable

\end{document}
