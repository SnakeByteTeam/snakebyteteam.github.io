\documentclass[10pt, letterpaper]{article}

\usepackage{../../../../template/template}

\setDocTitle{Verbale interno 23/02/2026}
\setDocVersion{0.1.0}
\setDocData{24/02/2026}
\setDocIndex{vi\_2026\_02\_23}
\setDocState{1}
\setDocRecipient{1}

\begin{document}

\makeTitlePage

\newpage
\addChangelog{0.1.0}{24/02/2026}{L. Granziero}{C. Libralato}{\noOne}{\firstDraft}
\makeChangelog

\newpage

\tableofcontents

\newpage

% -------------------------------------------------
% Informazioni riunione
% -------------------------------------------------
\makeMeetingInfoTable{23/02/2026}{14:30}{16:30}{via \textit{Discord$_G$}}

\addParticipant{Valeria}{Baleanu}{Progettista}{P}
\addParticipant{Leonardo}{Pellizzon}{Amministratore}{P}
\addParticipant{Luca}{Granziero}{Responsabile}{P}
\addParticipant{Francesco}{Pasqual}{Progettista}{P}
\addParticipant{Giuseppe}{De Fina}{Progettista}{P}
\addParticipant{Christian}{Libralato}{Verificatore}{P}
\addParticipant{Filippo}{Venzo}{Progettista}{P}
\makeMeetingParticipantsTable


\addPrecedentTodo{vi\_2026\_02\_09.a1}{\noOne}{SnakeByte team}{10/02/2026}
\addPrecedentTodo{vi\_2026\_02\_09.a2}{\#106}{G. de Fina}{12/02/2026}
\addPrecedentTodo{vi\_2026\_02\_09.a3}{PR \#103}{F. Pasqual, L. Granziero}{09/02/2026}
\addPrecedentTodo{vi\_2026\_02\_09.a4}{\#105}{F. Venzo}{14/02/2026}
\addPrecedentTodo{vi\_2026\_02\_09.a5}{\#104}{G. de Fina}{12/02/2026}
\addPrecedentTodo{vi\_2026\_02\_09.a6}{\#104}{G. de Fina}{12/02/2026}

\makePrecedentTodoTable


% -------------------------------------------------
% Ordine del giorno
% -------------------------------------------------
\section{Ordine del giorno}
\begin{itemize}
    \item Organizzazione degli \textit{Sprint$_G$} di sviluppo e revisione dei ruoli;
    \item rivalutazione scelta tecnologica del backend;
    \item pianificazione tecnica del database;
    \item organizzazione del lavoro e pianificazione collaborativa.
\end{itemize}
% -------------------------------------------------
% Approfondimento
% -------------------------------------------------
\section{Approfondimento}

\subsection*{Ruoli e organizzazione degli sprint}
È stata discussa l’assegnazione dei ruoli nei vari sprint ed è stato evidenziato come la fase di progettazione debba precedere quella di programmazione, al fine di garantire una corretta definizione dell’architettura e delle funzionalità. Il gruppo ha concordato di assegnare quattro progettisti per accelerare la fase iniziale di progettazione e consentire un avanzamento più efficiente delle attività successive. È stato inoltre stabilito che i ruoli sarebbero stati aggiornati sia per lo sprint corrente sia per quelli futuri e che il budget complessivo sarebbe rimasto invariato, consentendo esclusivamente una riallocazione delle ore tra i diversi sprint senza alcun incremento del totale previsto.

\subsection*{Rivalutazione scelta tecnologica backend}
È stato discusso il possibile impiego di \textit{NestJS$_G$} in sostituzione di \textit{Express$_G$} per lo sviluppo della componente backend, analizzandone le caratteristiche e i vantaggi architetturali. In particolare, NestJS è stato ritenuto più adeguato grazie al supporto nativo per la dependency injection, alla presenza di strumenti integrati per il testing e a una migliore organizzazione modulare del codice. A seguito del confronto tecnico, il gruppo ha deciso all’unanimità di adottare NestJS come framework backend e ha concordato di comunicare tale decisione al prof. Cardin tramite mail.

\subsection*{Database e aspetti tecnici}
È stato confermato l’utilizzo di \textit{PostgreSQL$_G$} come database principale del progetto, in quanto ritenuto adeguato ai requisiti tecnici e funzionali individuati. Nel corso della discussione è stata inoltre valutata positivamente la possibilità di utilizzare \textit{TimescaleDB$_G$} come estensione per la gestione di dati temporali, considerandone i potenziali benefici in termini di organizzazione e prestazioni. Sono stati infine approfonditi alcuni aspetti tecnici relativi alla gestione delle tabelle e alle implicazioni sul comportamento del sistema.

\subsection*{Organizzazione del lavoro}
È stata discussa la modalità di collaborazione tra i progettisti ed è stato concordato di iniziare con una fase di lavoro congiunta, al fine di favorire una migliore comprensione condivisa delle scelte progettuali e dell’architettura del sistema. Successivamente, il gruppo ha stabilito che le attività sarebbero state suddivise tra i membri, in modo da proseguire il lavoro in maniera più efficiente e organizzata.


% -------------------------------------------------
% Decisioni
% -------------------------------------------------

\addDecision{Modifica dei ruoli negli sprint con assegnazione di quattro progettisti}
\addDecision{Adozione di NestJS come framework backend}
\addDecision{Utilizzo di PostgreSQL come database principale}
\makeDecisionTable

% -------------------------------------------------
% Azioni
% -------------------------------------------------
\addTodo{\#141}{Scrittura verbale interno}{L. Granziero}{25/02/2026}
\addTodo{\#143}{Aggiornare il Piano di Progetto allo sprint 8}{L. Pellizzon}{28/02/2026}


\makeTodoTable

\end{document}
 v  