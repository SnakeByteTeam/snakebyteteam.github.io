\documentclass[10pt, letterpaper]{article}


\usepackage{../../../../template/template}

\setDocTitle{Verbale interno 23/02/2026}
\setDocVersion{0.1.0}
\setDocData{24/02/2026}
\setDocIndex{vi\_2026\_02\_23}
\setDocState{1}
\setDocRecipient{1}

\begin{document}

\makeTitlePage

\newpage

\addChangelog{0.1.0}{24/02/2026}{L. Granziero}{C. Libralato}{\noOne}{\firstDraft}
\makeChangelog

\newpage

\tableofcontents

\newpage


\makeMeetingInfoTable{23/02/2026}{14:30}{16:30}{via \textit{Discord$_G$}}

\addParticipant{Valeria}{Baleanu}{Progettista}{P}
\addParticipant{Leonardo}{Pellizzon}{Amministratore}{P}
\addParticipant{Luca}{Granziero}{Responsabile}{P}
\addParticipant{Francesco}{Pasqual}{Progettista}{P}
\addParticipant{Giuseppe}{De Fina}{Progettista}{P}
\addParticipant{Christian}{Libralato}{Verificatore}{P}
\addParticipant{Filippo}{Venzo}{Progettista}{P}
\makeMeetingParticipantsTable


\addPrecedentTodo{vi\_2026\_02\_09.a1}{\noOne}{SnakeByte team}{10/02/2026}
\addPrecedentTodo{vi\_2026\_02\_09.a2}{\#106}{G. de Fina}{12/02/2026}
\addPrecedentTodo{vi\_2026\_02\_09.a3}{PR \#103}{F. Pasqual, L. Granziero}{09/02/2026}
\addPrecedentTodo{vi\_2026\_02\_09.a4}{\#105}{F. Venzo}{14/02/2026}
\addPrecedentTodo{vi\_2026\_02\_09.a5}{\#104}{G. de Fina}{12/02/2026}
\addPrecedentTodo{vi\_2026\_02\_09.a6}{\#105}{G. de Fina}{12/02/2026}

\makePrecedentTodoTable


\section{Ordine del giorno}
\begin{itemize}
    \item organizzazione degli \textit{Sprint$_G$} di sviluppo e revisione dei ruoli;
    \item scelta tecnologica del backend;
    \item pianificazione tecnica del database;
    \item organizzazione del lavoro e pianificazione collaborativa.
\end{itemize}

\section{Approfondimento}

\subsection*{Ruoli e organizzazione degli sprint}
Si discute sull’assegnazione dei ruoli nei vari sprint. Viene evidenziato che la progettazione deve precedere la programmazione.
Il gruppo concorda di assegnare quattro progettisti al fine di accelerare il più possibile la fase iniziale di progettazione.
I ruoli verranno aggiornati sia per lo sprint attuale che per quelli futuri.
Viene inoltre discusso il mantenimento del budget, spostandolo tra sprint senza incrementi.

\subsection*{Scelta tecnologica backend}
Viene discusso il confronto tra \textit{Express$_G$} e \textit{NestJS$_G$}.
NestJS viene ritenuto vantaggioso grazie alla dependency injection automatica, al supporto ai test integrati e all’organizzazione modulare.
Il gruppo decide all’unanimità di adottare NestJS per il backend e di informare il docente tramite mail.

\subsection*{Database e aspetti tecnici}
Viene confermato l’utilizzo di \textit{PostgreSQL$_G$} come database principale.
Viene valutato positivamente l’utilizzo di \textit{TimescaleDB$_G$} per la gestione di dati temporali.
Si discutono inoltre aspetti tecnici relativi alla gestione delle tabelle e delle prestazioni.

\subsection*{Organizzazione del lavoro}
Il gruppo concorda di iniziare con una fase collaborativa tra i progettisti per facilitare la progettazione iniziale.
Successivamente, i compiti verranno suddivisi tra i membri del gruppo.



\addDecision{Modifica dei ruoli negli sprint con assegnazione di quattro progettisti}
\addDecision{Adozione di NestJS come framework backend}
\addDecision{Utilizzo di PostgreSQL come database principale}
\makeDecisionTable

\addTodo{}{Aggiornare il piano sprint con i ruoli corretti}{ }{}
\addTodo{}{Preparare e inviare mail al docente per comunicare il cambio tecnologico}{ }{}
\addTodo{\noOne}{Organizzare un incontro tra i progettisti per avviare la progettazione}{SnakeByte team}{24/02/2026}
\addTodo{}{Approfondire la configurazione del backend e del database}{ }{}
\makeTodoTable

\end{document}
